\documentclass[12pt,titlepage,twoside,cleardoublepage]{article}
\usepackage[ngerman]{babel}
\usepackage[utf8]{inputenc}
\usepackage[a4paper,lmargin={4cm},rmargin={2cm},
tmargin={2.5cm},bmargin = {2.5cm}]{geometry}
\usepackage{amsmath}
\usepackage{amssymb}
\usepackage{pdfpages} 
%\usepackage[pdftex,article]{geometry}
\usepackage{amsthm}
%\usepackage{ngerman,amsthm}
\usepackage{lineno} 
\usepackage{lineno, blindtext} 
\usepackage{cleveref}
\usepackage{enumerate}
\usepackage{float}
\usepackage{thmtools}
\usepackage{tabularx}
\linespread{1.25}
\usepackage{color}
\usepackage{verbatim}
\newcommand{\gelb}{0.550000011920929}
\usepackage{pgf,tikz,pgfplots}
\pgfplotsset{compat=1.15}
\usepackage{mathrsfs}
\usepackage{mathrsfs}
\usetikzlibrary{arrows}
%\numberwithin{equation}{chapter}
%\usepackage{scrheadings}
\pagestyle{headings}
\usepackage{titlesec}                % für Kontrolle der Abschnittüberschriften
\begin{comment}
\makeatother
\theoremstyle{nummermitklammern}
\theorembodyfont{\rmfamily}
\theoremsymbol{\ensuremath{\diamond}}
\newtheorem{temp}{}[section]
\newtheorem{vor}[temp]{Vorüberlegung}
\newtheorem{lemma}[temp]{Lemma}
\newtheorem{folgerung}[temp]{Folgerung}
\newtheorem{bsp}[temp]{Beispiel}
\newtheorem{herleitung}[temp]{Herleitung}
\newtheorem{definition}[temp]{Definition}
\newtheorem{bemerkung}[temp]{Bemerkung}
\newtheorem{satz}[temp]{Satz}
\newtheorem{beweisidee}[temp]{Beweisidee}
\theoremsymbol{\ensuremath{\square}}
\end{comment}
%\begin{comment}
\newtheorem{zahl}{}[section]
%\setcounter{zahl}{1}
%\newtheorem{section}{section}[section]
\newtheorem{definition}[zahl]{Definition}
\newtheorem{vor}[zahl]{Vorüberlegung}
\newtheorem{lemma}[zahl]{Lemma}
\newtheorem{folgerung}[zahl]{Folgerung}
\newtheorem{bsp}[zahl]{Beispiel}
\newtheorem{herleitung}[zahl]{Herleitung}
\newtheorem{bemerkung}[zahl]{Bemerkung}
\newtheorem{satz}[zahl]{Satz}
\newtheorem{beweisidee}[zahl]{Beweisidee}
\numberwithin{equation}{section}


%-----------------------------------------------

%\end{comment}
 %Nummerierung mit Kapitelnummern
%-------------------------
%\newcommand{\secnumbering}[1]{% 
 % \setcounter{chapter}{0}% 
  %\setcounter{section}{0}% 
  %\renewcommand{\thechapter}{\csname #1\endcsname{chapter}.}% nach Duden gehört 
                                  % der Punkt hier hin bei gemischten Zählungen 
%  \renewcommand{\thesection}{\thechapter\csname #1\endcsname{section}}% 
%}
%------------------------------
\begin{document}
\begin{titlepage}
    \begin{center}
      \large
      \textsc{Rheinisch-Westf\"alische Technische Hochschule Aachen}\\
      Lehrstuhl B für Mathematik \\
      Univ.-Prof. Dr.  Alice Niemeyer\\
      \vspace{3 cm}
      \huge  Manipulation diskreter simplizialer Flächen \\
      \vspace{1 cm}
      \large Bachelorarbeit\\
      %\large Bachelor's Thesis\\
      \vspace{2 cm}
       \vspace{1 cm}
      \Large Reymond Oluwaseun Akpanya\\
      \large Matrikelnummer: 357115\\
      %\vspace{2 cm}
      %\large Vorgelegt am: 28.09.2018
      \vspace{3.5 cm}
%      last build:
 %    \today \\[4em]
\begin{flalign*}
&\text{Vorgelegt am:}&\text{28.09.2018}&\\
&\text{Gutachter:}&\text{Prof. Dr. Alice Niemeyer}&\\
&\text{Zweitgutachter:}&\text{Prof. Dr. Wilhelm Plesken}&\\
&\text{Betreuer:}&\text{Jesse Lansdown}&\\
&\text{Betreuer:}&\text{Markus Baumeister}&\\[1em]
\end{flalign*}
    \end{center}
% \begin{flalign*}
 %&\text{ } &\text{ 25.09.2018 }
 %\end{flalign*}
\end{titlepage}
%---------------------------
%\farb
%\section*{Inhaltsverzeichnis}
\newpage 
\thispagestyle{empty}
\quad 
\newpage
\thispagestyle{empty}

\tableofcontents
%\addcontentsline{toc}{section}{Einleitung}
\newpage
\setcounter{page}{1}
%\cleardoubelepage
\section{Einleitung}
Ziel dieser Arbeit ist es, die Manipulation simplizialer Flächen zu untersuchen. Man versucht durch die Anwendung von gewissen Operationen aus einer simplizialen Fläche, bestehend aus Knoten, Kanten und Flächen, eine weitere simpliziale Fläche zu konstruieren. In dieser Arbeit werden nur diese behandelt, die die Kanten einer simplizialen Fläche verändern.\\
Zuerst wird jedoch im erstem Kapitel dieser Arbeit geklärt, was eine simpliziale Fläche überhaupt ist. Zusätzlich werden einige einführende Definitionen und Beispiele, die man dem Skript \emph{Combinatorial Simplicial Surfaces} von Prof. Dr. Wilhelm Plesken entnehmen kann, angeführt, um ein grundlegendes Verständnis von simplizialen Flächen zu schaffen. Im zweitem Kapitel definiert man dann die oben schon erwähnten Operationen, welche die Kanten einer simplizialen Fläche manipulieren. Diese werden dann genutzt, um im darauf folgendem Kapitel eine Operation zu definieren, die Wanderinghole genannt wird. Die Untersuchung dieser Operation ist die Hauptfragestellung dieser Arbeit. Es wird hier Klarheit über Eigenschaften wie zum Beispiel Transitivität dieser Operation gebracht. Dies wird aber später noch näher erläutert. Abschließend wird eine simpliziale Fläche vergrößert. Dies bedeutet, dass die Anzahl der Flächen und damit auch die der Kanten und Knoten vergrößert wird, um dann die Anwendung der Operation Wanderinghole auf diese zu untersuchen. \\
Für die Untersuchung dieser Fragestellungen wurde das Computer-Algebra-System \emph{Gap} zur Hilfe genommen und mithilfe dessen und dem Gap-Packet \emph{SimplicialSurface} von Prof. Dr. Alice Niemeyer und Markus Baumeister zunächst einmal die Operation Wanderinghole implementiert, um so ein erstes Verständnis für diese zu bekommen und dann daraus folgend erste Behauptungen aufstellen zu können. Dies geschah unter der Betreuung von Prof. Dr. Alice Niemeyer, Markus Baumeister und Jesse Lansdown. Der Quellcode hierzu liegt im Anhang bei. 
\newpage
\newpage
\section{Wiederholung} 
Zunächst werden in den nächsten beiden Kapiteln einführende Definitionen und Zusammenhänge dargelegt, die man dem Skript \emph{Combinatorial Simplicial Surfaces} entnehmen kann. Darüber hinaus stammen Sachverhalte und Konzepte nur dann aus dem im Literaturverzeichnis angegeben Quellen, falls dies an den dazugehörigen Stellen gekennzeichnet ist.
\begin{definition}  \label{def1} Sei $X=X_{0} \biguplus X_{1} \biguplus X_{2}$ eine abzählbare Menge mit $X_{i} \ne \emptyset$ für $i=0,1,2$ und $<$ eine transitive Relation auf  ($X_{0}\times X_{1}) \cup (X_{1}\times X_{2})\cup (X_{0}\times X_{2}$). Man nennt $X_{0}$ \emph{die Menge der Knoten}, $X_{1}$ \emph{die Menge der Kanten}, $X_{2}$ \emph{die Menge der Flächen} und $<$ \emph{Inzidenz} einer \emph{simplizialen Fläche} $(X,<)$, falls folgende Eigenschaften erfüllt sind:
 \begin{enumerate}
\item Für jede Kante $e \in X_{1}$ existieren genau zwei Knoten $V_1,V_2 \in X_{0}$ mit $V_1,V_2 < e$. 
\item Für jede Fläche $F\in X_2$ gibt es genau drei Kanten $e_1,e_2,e_3 \in X_{1}$ mit der Eigenschaft $e_1,e_2,e_3 < F$. 
\item Für jede Kante $e \in X_{1}$ gibt es entweder genau zwei Flächen $F_{1},F_{2} \in X_{2}$ mit $e <F_{1},F_2$ oder
genau eine Fläche $F \in X_{2}$ mit $e < F$. Im ersten Fall sind $F_{1}$ und $F_{2}$ \emph{$(e)$-Nachbarn} und $e$ ist eine \emph{innere Kante}, andernfalls ist $e$ eine \emph{Randkante}. 
 \item Für jeden Knoten $V \in X_{0}$ existieren endlich viele Flächen $F\in X_{2}$ mit $V < F$.
  Diese $F_{i}\in X_2$ können in einem Tupel $(F_{1},\ldots,F_{n})$ für ein $n \in \mathbb{N}$ geordnet werden so, dass $e_i<F_{i}$ und $e_i<F_{i+1}$ für $i=1,\ldots,n-1$ ist, wobei $e_i\in X_1$ eine Kante  in $X$ ist, für die $V<e_i$ gilt. 
  Das Tupel $(F_1,\ldots,F_n)$ wird auch \emph{Schirm} genannt. Gibt es auch eine Kante $e\in X_1$ mit $e<F_{1}F_{n}$, so ist $V$ ein \emph{innerer Knoten}. Ist $V$ kein innerer Knoten, so ist er 
 ein Eckknoten.  
 \item Seien $V \in X_0$ ein Knoten in $X$ und $(F_1,\ldots,F_n)$ der zu $V$ gehörige Schirm, wobei $F_i \in X_2$ für $i=1,\ldots ,n$ und $n\in \mathbb{N}$ ist. Dann ist n der \emph{Grad des Knotens} $V$. Für den Grad eines Knotens $V$ in $X$ schreibt man $\deg_X(V)$. Falls $X$ aus dem Kontext heraus klar ist, so schreibt man nur $\deg(V)$ statt $\deg_X(V)$.
 \item Die Menge aller inneren Knoten einer Kante $e \in X_1$ bezeichnet man mit $X_0^0(e).$
\end{enumerate}
\end{definition}
Es sei angemerkt, dass es für einen gegebenen Knoten einer simplizialen Fläche eine endliche Anzahl von Schirmen gibt. Diese sind jedoch alle äquivalent. Das heißt, sie können durch zyklische Permutationen umgeordnet werden.\\ Es ist außerdem möglich, die simpliziale Fläche $(X,<)$ mit der Menge $X$ zu identifizieren.
%\newpage
\begin{bemerkung}  Eine \emph{geschlossene} simpliziale Fläche ist eine simpliziale Fläche, für die alle Kanten innere Kanten sind. Die Anzahl der Flächen einer geschlossenen simplizialen Fläche ist durch $2$ teilbar, da
\[
\vert X_{2} \vert = \frac{2\vert X_{1}\vert}{3}.
\]
Die Anzahl der Kanten ist insbesondere durch 3 teilbar.
\end{bemerkung}
Im Folgendem werden nur endliche simpliziale Flächen betrachtet, das heißt simpliziale Flächen $(X,<)$ mit $\vert X \vert < \infty$.
 \begin{bsp}
 \begin{enumerate}
\item 
 Bis auf Isomorphie gibt es nur eine simpliziale Fläche bestehend aus einer Fläche, welche durch 
\begin{align*}
X_{0}=\{\,V_{1}&,V_{2},V_{3}\,\}, X_{1}=\{\,e_{1},e_{2},e_{3}\,\}, X_{2}=\{\,F_{1}\,\} \text{ und } x<y \Leftrightarrow \\
 (x,y)\in \{\,&(e_{1},F_{1}),(e_{2},F_{1}),(e_{3},F_{1}),(V_{1},e_{2}),(V_{1},e_{3}),(V_{1},F_{1}),(V_{2},e_{1}), (V_{2},e_{3}),\\ &(V_{2},F_{1}),
 (V_{3},e_{1}),(V_{3},e_{2}),(V_{3},F_{1})\,\} 
\end{align*} 
beschrieben wird. Man nennt diese simpliziale Fläche \emph{Dreieck}. 
%--------------------------Bild-------------------------
\begin{figure}[H]
\definecolor{ffffqq}{rgb}{1.,1.,0.}
\definecolor{qqqqff}{rgb}{0.,0.,1.}
\definecolor{xdxdff}{rgb}{0.49019607843137253,0.49019607843137253,1.}
\begin{tikzpicture}[line cap=round,line join=round,>=triangle 45,x=1.0cm,y=1.0cm]
%\begin{axis}
x=1.0cm,y=1.0cm,
axis lines=middle,
ymajorgrids=true,
xmajorgrids=true,
xmin=-2.690233964457656,
xmax=14.560601622891102,
ymin=1,
ymax=4.805217252710678,
xtick={-2.0,-1.0,...,14.0},
ytick={-1.0,-0.0,...,4.0},]
\clip(-7.690233964457656,-0.4) rectangle (8.560601622891102,3.8641016151377553);
\fill[line width=2.pt,color=ffffqq,fill=ffffqq,fill opacity=0.5] (-2.,0.) -- (2.,0.) -- (0.,3.4641016151377553) -- cycle;
\draw [line width=2.pt] (-2.,0.)-- (2.,0.);
\draw [line width=2.pt] (2.,0.)-- (0.,3.4641016151377553);
\draw [line width=2.pt] (0.,3.4641016151377553)-- (-2.,0.);
\begin{scriptsize}
\draw [fill=qqqqff] (-2.,0.) circle (2.5pt);
\draw[color=qqqqff] (-2.082704537251026,0.48500799257463995) node {$V_1$};
\draw [fill=qqqqff] (2.,0.) circle (2.5pt);
\draw[color=qqqqff] (2.147500363298844,0.32) node {$V_2$};
\draw[color=black] (0.03989827632275619,1.2875468655512998) node {$F_1$};
\draw[color=black] (0.07740009281699262,-0.30001889154005933) node {$e_3$};
\draw[color=black] (1.2774582206325584,2.030082832137181) node {$e_1$};
\draw[color=black] (-1.1376587615962677,2.030082832137181) node {$e_2$};
\draw [fill=qqqqff] (0.,3.4641016151377553) circle (2.5pt);
\draw[color=qqqqff] (0.1374029992077709,3.7851678440674466) node {$V_3$};
\end{scriptsize}
%\end{axis}
\end{tikzpicture}
\caption{Dreieck}
\end{figure}

%-------------------------------------------------------
 \item
 Für $n \in \mathbb{N}$ definieren wir die \emph{n-fache Fläche} $n \Delta$ durch $X=\{X_{0},X_{1},X_{2}\}$, wobei
 \begin{align*}
  X_{0}=\{& \,V_{j}^{k}\,\vert\, j=1,2,3 ;k=1,\ldots,n\,\}, X_{1}=\{\,e_{j}^{k}\,\vert\, j=1,2,3 ;k=1,\ldots,n\,\},\\
   X_{2}=\{&F_{1},\ldots,F_{n}\} \text{ und } x<y \Leftrightarrow \\
 (x,y)\in \{\,&(e_{1}^k,F_{k}),(e_{2}^k,F_{k}),(e_{3}^k,F_{k}),(V_{1}^k,e_{2}^k),(V_{1}^k,e_{3}^k),(V_{1}^k,F_{k}), (V_{2}^k,e_{1}^k),\\ &(V_{2}^k,e_{3}^k),(V_{2}^k,F_{k}),(V_{3}^k,e_{1}^k),(V_{3}^k,e_{2}^k),(V_{3}^k,F_{k})\mid k=0,\ldots,n\} 
\end{align*}
 
 \item 
 Der \emph{Janus-Kopf} ist eine geschlossene simpliziale Fläche, die aus zwei Flächen besteht.	Sie besitzt 3 innere Knoten und  3 innere Kanten und wird definiert durch
 \begin{align*}
 X_{0}=\{\,&V_{1},V_{2},V_{3}\,\} ,X_{1}=\{\,e_{1},e_{2},e_{3}\,\},X_{3}=\{\, F_{1},F_{2}\,\}  \text{ und } x<y \Leftrightarrow \\
 (x,y)\in\{&\,(e_{1},F_{1}),(e_{1},F_{2}),(e_{2},F_{1}),(e_{2},F_{2}),(e_{3},F_{1}),(e_{3},F_{2}),(V_{1},e_{2}),(V_{1},e_{3}),\\ &(V_{1},F_{1}),
  (V_{1},F_{2}),(V_{2},e_{1}),(V_{2},e_{3}),(V_{2},F_{1})
 (V_{2},F_{2}), (V_{3},e_{1}), (V_{3},e_{2}),\\&(V_{3},F_{1}),(V_{3},F_{2}) \,\}.
 \end{align*}

 %----bild----------------------------
 \begin{figure}[H]
\definecolor{sqsqsq}{rgb}{0.12549019607843137,0.12549019607843137,0.12549019607843137}
\definecolor{ttqqqq}{rgb}{0.2,0.,0.}
\definecolor{ffffqq}{rgb}{1.,1.,0.}
\definecolor{qqqqff}{rgb}{0.,0.,1.}
\begin{tikzpicture}[line cap=round,line join=round,>=triangle 45,x=1.0cm,y=1.0cm]

x=1.0cm,y=1.0cm,
axis lines=middle,
ymajorgrids=true,
xmajorgrids=true,
xmin=-5.056290110700678,
xmax=5.380866801866215,
ymin=-0.9227448489396118,
ymax=4.359364127681193,
xtick={-5.0,-4.5,...,5.0},
ytick={-0.5,0.0,...,4.0},]
\clip(-7.056290110700678,-0.4) rectangle (5.380866801866215,3.8641016151377553);
\fill[line width=2.pt,color=ffffqq,fill=ffffqq,fill opacity=0.5] (-2.,0.) -- (2.,0.) -- (0.,3.4641016151377553) -- cycle;
\fill[line width=2.pt,color=ffffqq,fill=ffffqq,fill opacity=0.5] (0.,3.4641016151377553) -- (2.,0.) -- (4.,3.464101615137754) -- cycle;
\draw [line width=2.pt] (-2.,0.)-- (2.,0.);
\draw [line width=2.pt] (2.,0.)-- (0.,3.4641016151377553);
\draw [line width=2.pt] (0.,3.4641016151377553)-- (-2.,0.);
\draw [line width=2.pt,color=black] (0.,3.4641016151377553)-- (2.,0.);
\draw [line width=2.pt] (2.,0.)-- (4.,3.464101615137754);
\draw [line width=2.pt,color=black] (4.,3.464101615137754)-- (0.,3.4641016151377553);
\begin{scriptsize}
\draw [fill=qqqqff] (-2.,0.) circle (2.5pt);
\draw[color=black] (-2.347666183295526,0.0934108260899206) node {$V_1$};
\draw [fill=qqqqff] (2.,0.) circle (2.5pt);
\draw[color=black] (2.390727102068595,0.0934108260899206) node {$V_2$};
\draw[color=black] (0.04884098783747624,1.2554444197699985) node {$F_1$};
\draw[color=black] (0.057916776457099625,-0.22407545041275472) node {$e_3$};
\draw[color=black] (1.174238776670776,1.9089012003828816) node {$e_1$};
\draw[color=black] (-1.1673146871920574,1.9089012003828816) node {$e_2$};
\draw [fill=qqqqff] (0.,3.4641016151377553) circle (2.5pt);
\draw[color=black] (0.08514414231596978,3.75145261535057) node {$V_3$};
\draw[color=sqsqsq] (2.0455144841546207,2.417145363081791) node {$F_2$};
\draw[color=black] (3.4162912160860375,1.9089012003828816) node {$e_3$};
%\draw[color=black] (3.4162912160860375,1.9089012003828816) node {$e_3$};
\draw[color=sqsqsq] (2.054590272774244,3.6603620787860503) node {$e_2$};
\draw [fill=qqqqff] (4.,3.464101615137754) circle (2.5pt);
\draw[color=black] (4.087566923569883,3.75145261535057) node {$V_1$};
\end{scriptsize}
\end{tikzpicture}
\caption{Janus-Kopf}
\end{figure}
 %------------------------------------
 \item 
 Der \emph{Open-Bag} ist eine simpliziale Fläche, die aus dem \emph{Janus-Kopf} hervorgeht, wenn man die Kante $e_{2}$ verdoppelt, das heißt sie wird beschrieben durch
%\begin{figure}[h]
 \begin{align*}
  X_{0}=\{\,V_{1},&V_{2},V_{3}\,\},X_{1}=\{\,e_{1},e_{2},e_{3},e_{4} \,\}, X_{2}=\{\,F_{1},F_{2}\,\} \text{ und } x<y \Leftrightarrow\\
 (x,y)\in\{&\,(e_{1},F_{1}),(e_{1},F_{2}),(e_{2},F_{1}),(e_{3},F_{1}),(e_{3},F_{2}),(e_{4},F_{2}),(V_{1},e_{2}),(V_{1},e_{3}),\\ &(V_{1},e_{4}),
  (V_{1},F_{1}),(V_{1},F_{2}),(V_{2},e_{1}),(V_{2},e_{3})
 (V_{2},F_{1}), (V_{2},F_{2}), (V_{3},e_{1}),\\&(V_{3},e_{2}),(V_{3},e_{4}),(V_{3},F_{1}),(V_3,F_2) \,\}.
 \end{align*}
 \end{enumerate}
%--------------------------------------------
\begin{figure}[H]
\definecolor{sqsqsq}{rgb}{0.12549019607843137,0.12549019607843137,0.12549019607843137}
\definecolor{ttqqqq}{rgb}{0.2,0.,0.}
\definecolor{ffffqq}{rgb}{1.,1.,0.}
\definecolor{qqqqff}{rgb}{0.,0.,1.}
\begin{tikzpicture}[line cap=round,line join=round,>=triangle 45,x=1.0cm,y=1.0cm]

x=1.0cm,y=1.0cm,
axis lines=middle,
ymajorgrids=true,
xmajorgrids=true,
xmin=-5.056290110700678,
xmax=5.380866801866215,
ymin=-0.9227448489396118,
ymax=4.359364127681193,
xtick={-5.0,-4.5,...,5.0},
ytick={-0.5,0.0,...,4.0},]
\clip(-7.056290110700678,-0.4) rectangle (5.380866801866215,3.859364127681193);
\fill[line width=2.pt,color=ffffqq,fill=ffffqq,fill opacity=0.5] (-2.,0.) -- (2.,0.) -- (0.,3.4641016151377553) -- cycle;
\fill[line width=2.pt,color=ffffqq,fill=ffffqq,fill opacity=0.5] (0.,3.4641016151377553) -- (2.,0.) -- (4.,3.464101615137754) -- cycle;
\draw [line width=2.pt] (-2.,0.)-- (2.,0.);
\draw [line width=2.pt,color=black] (2.,0.)-- (0.,3.4641016151377553);
\draw [line width=2.pt] (0.,3.4641016151377553)-- (-2.,0.);
\draw [line width=2.pt,color=black] (0.,3.4641016151377553)-- (2.,0.);
\draw [line width=2.pt] (2.,0.)-- (4.,3.464101615137754);
\draw [line width=2.pt,color=black] (4.,3.464101615137754)-- (0.,3.4641016151377553);
\begin{scriptsize}
\draw [fill=qqqqff] (-2.,0.) circle (2.5pt);
\draw[color=black] (-2.347666183295526,0.0934108260899206) node {$V_1$};
\draw [fill=qqqqff] (2.,0.) circle (2.5pt);
\draw[color=black] (2.390727102068595,0.0934108260899206) node {$V_2$};
\draw[color=sqsqsq] (0.04884098783747624,1.2554444197699985) node {$F_1$};
\draw[color=black] (0.057916776457099625,-0.22407545041275472) node {$e_3$};
\draw[color=black] (1.174238776670776,1.9089012003828816) node {$e_1$};
\draw[color=black] (-1.1673146871920574,1.9089012003828816) node {$e_2$};
\draw [fill=qqqqff] (0.,3.4641016151377553) circle (2.5pt);
\draw[color=black] (0.08514414231596978,3.75145261535057) node {$V_3$};
\draw[color=sqsqsq] (2.0455144841546207,2.417145363081791) node {$F_2$};
\draw[color=black] (3.4162912160860375,1.9089012003828816) node {$e_3$};
\draw[color=black] (2.054590272774244,3.6603620787860503) node {$e_4$};
\draw [fill=qqqqff] (4.,3.464101615137754) circle (2.5pt);
\draw[color=black] (4.087566923569883,3.75145261535057) node {$V_1$};
\end{scriptsize}
\end{tikzpicture}
\caption{Open-Bag}
\end{figure}
\end{bsp}
 %------------------------------------
%\newpage
\begin{definition} 
Sei $(X,<)$ eine simpliziale Fläche. Für $i,j \in \{\,0,1,2\,\}$ mit $i \neq j$ und $x \in X_{i}$ definiert man die Menge $X_{j}(x)$ als
\[
X_{j}(x):=\{\,y \in X_{j}\,|\,x < y\,\} \text{, falls $i < j$  }
\]
bzw. 
\[
X_{j}(x):=\{\,y \in X_{i}\,|\,y < x\}, \text{ falls $j<i$}.
\]
Für $S \subseteq X_{i}$ ist 
\[
X_j(S):= \bigcup_{x\in S}X_{j}(x).
\]
\end{definition}
%\newpage
\begin{bemerkung}
Für eine simpliziale Fläche $(X,<)$ können die Bedingungen in \Cref{def1} wie folgt umformuliert werden:
\begin{itemize}
\item $\vert X_{0}(e)\vert=2$ für alle $e \in X_{1}$,
\item $\vert X_{0}(F)\vert=3$ für alle $F \in X_{2}$,
\item $\vert X_{1}(F)\vert=3$ für alle $F \in X_{2}$,
\item $1\leq  \vert X_{2}(e)\vert \leq 2$ für alle $e \in X_{1}$.

\end{itemize}
\end{bemerkung}

\begin{definition} Seien $(X,<)$ und $(Y,\prec)$ simpliziale Flächen.
\begin{enumerate}
 \item Man nennt eine bijektive Abbildung $\alpha: X \to Y$ einen Isomorphismus, falls $A<B$ in $(X,<)$ genau dann gilt, wenn $\alpha(A) \prec \alpha(B)$ in $(Y,\prec)$ gilt. In diesem Fall schreibt man $X \cong Y$.
\item Eine surjektive Abbildung $\alpha: X \to Y$ heißt Überdeckung, falls aus $A<B$ in $(X,<)$ folgt, dass $\alpha(A) \prec \alpha(B)$ in $(Y,\prec)$ gilt. 
\end{enumerate}
\end{definition}
Es sei angemerkt, dass eine Überdeckung $\alpha:X\to Y$ surjektive Abbildungen $X_{i} \to Y_{i}$ und ein Isomorphismus $\beta:X \to Y$ bijektive Abbildungen $X_{i} \to Y_{i}$ für $i=0,1,2$ induziert.

Um simpliziale Flächen vollständig beschreiben zu können, führt man eine Notation ein. Man beachte, dass die hier eingeführte Notation stark von der Nummerierung der Knoten, Kanten und Flächen abhängt. Abgesehen davon ist sie eindeutig.
\begin{definition}
 Sei $(X,<)$ eine simpliziale Fläche, deren Knoten $V_{1},\ldots,V_{n}$, Kanten $e_{1},\ldots,e_{k}$ und Flächen $F_{1},\ldots,F_{m}$ ausgehend von ihrer Nummerierung linear geordnet sind. Das \emph{Symbol} von $(X,<)$ ist definiert durch 
\[
\mu((X,<)):=(n,k,m;(X_{0}(e_{1}),\ldots,X_{0}(e_{k})),(X_{1}(F_{1}),\ldots,X_{1}(F_{m}))).
\]
Man kann im Symbol die Knoten $V_{i}$ durch $i$, die Kanten $e_{j}$ durch $j$ und die Flächen $F_{l}$ durch $l$ ersetzen und nennt dann das resultierende Symbol das \emph{ordinale Symbol} $\omega((X,<))$ von $(X,<)$.
\end{definition}

Im nächsten Abschnitt werden zur Vereinfachung der Konstruktion von simplizialen Flächen Bilder eingeführt, die nur Ausschnitte einer simplizialen Fläche zeigen sollen. Durch das unten eingeführte Bild soll beispielsweise angedeutet werden, dass eine simpliziale Fläche $(X,<)$ betrachtet wird, wobei hier nur $F\in X_2,e_1,e_2,e_3\in X_1$ und $V_1,V_2,V_3 \in X_0$ mit 
\begin{itemize}
 %\item $\vert X_{2}\vert \geq 3$,
 \item $e_{i} < F$ für alle $i \in \{1,2,3\}$,
 \item $V_{i}<e_{j}$ für alle $i \in \{1,2,3\}$ und $j \in \{1,2,3\} \setminus\{i\}$ und
 \item $V_{i} < F$ für alle $i \in \{1,2,3\}$
\end{itemize}  
von Bedeutung sind.

%------------bild2--------------------
\begin{figure}[H] 
\definecolor{qqqqff}{rgb}{0.,0.,1.}
\definecolor{uuuuuu}{rgb}{0.26666666666666666,0.26666666666666666,0.26666666666666666}
\definecolor{ududff}{rgb}{0.30196078431372547,0.30196078431372547,1.}
\definecolor{ffffqq}{rgb}{1.,1.,0.}
\begin{tikzpicture}[line cap=round,line join=round,>=triangle 45,x=1.4cm,y=1.4cm]
%\begin{axis}[
x=1.4cm,y=1.4cm,
axis lines=middle,
ymajorgrids=true,
xmajorgrids=true,
xmin=-5.3,
xmax=4.0600000000000005,
ymin=-0.46,
ymax=4.3,
xtick={-4.0,-3.0,...,7.0},
ytick={-2.0,-1.0,...,6.0},]
\clip(-5.0,-0.1) rectangle (4.06,4.);
\fill[line width=2.pt,color=ffffqq,fill=ffffqq,fill opacity=0.550000011920929] (-2.,0.) -- (2.,0.) -- (2.,4.) -- (-2.,4.) -- cycle;
\fill[line width=2.pt,color=ffffqq,fill=ffffqq,fill opacity=0.15000000596046448] (-1.,1.) -- (1.,1.) -- (0.,2.7320508075688776) -- cycle;
\draw [line width=2.pt,color=uuuuuu] (-1.,1.)-- (1.,1.);
\draw [line width=2.pt,color=uuuuuu] (1.,1.)-- (0.,2.7320508075688776);
\draw [line width=2.pt,color=uuuuuu] (0.,2.7320508075688776)-- (-1.,1.);
\begin{scriptsize}
\draw [fill=ududff] (-1.,1.) circle (2.5pt);
\draw[color=black] (-0.97,0.8) node {$V_1$};
\draw [fill=ududff] (1.,1.) circle (2.5pt);
\draw[color=black] (1.17,0.8) node {$V_2$};
\draw[color=black] (0.06,1.75) node {$F$};
\draw[color=uuuuuu] (0.11,0.8) node {$e_3$};
\draw[color=uuuuuu] (0.81,2.06) node {$e_1$};
\draw[color=uuuuuu] (-0.81,2.06) node {$e_2$};
\draw [fill=qqqqff] (0.,2.7320508075688776) circle (2.5pt);
\draw[color=black] (0.19,3.00) node {$V_3$};
\end{scriptsize}
%\end{axis}
\end{tikzpicture}
\caption{Ausschnitt einer simplizialen Fläche}
\label{abb4}
\end{figure}
Die Fläche $X$ kann aber mehr Knoten, Kanten und Flächen beinhalten, was durch den gelben Hintergrund in Abbildung \ref{abb4} angedeutet werden soll.
%--------------------------------------------------------
%\newpage
 \section{Konstruktion von simplizialen Flächen}
 \subsection{Randkantenpaare}
 Zunächst werden Definitionen eingeführt, die den Zugang zu den unten definierten Operatoren erleichtern sollen.
 
 \begin{definition}\label{mend}
 Sei $(X,<)$ eine simpliziale Fläche.
 \begin{enumerate}
 \item Sei $e \in X_{1}$ eine innere Kante in $(X,<)$. Diese ist vom Typ $i$ mit $i \in \{0,1,2\}$, falls $\vert X_{0}^{0}(e) \vert =i$ gilt.
 \item Man nennt $\{e,f\}$ ein Randkantenpaar, falls $e,f \in X_{1}$ Randkanten sind und $e$ und $f$ zu verschiedenen Flächen gehören. Das heißt 
 \[
 \forall F \in X_2: \{e,f\} \nsubseteq X_1(F).
 \]
 \item \begin{enumerate}
  \item[a)] Man nennt $\{e,f\}$ ein \emph{Randkantenpaar vom Typ 2}, falls $\{e,f\}$ ein Randkantenpaar ist und $X_{0}(e)=X_{0}(f)$ gilt. Das Randkantenpaar ist \emph{mendable}, falls $\vert X_2(e)\vert=\vert X_2(f)\vert =1$ ist.
  %-----------------------------------------------------bild2
  \begin{figure}[H]
  \definecolor{ududff}{rgb}{0.30196078431372547,0.30196078431372547,1.}
\definecolor{ffffqq}{rgb}{1.,1.,0.}
\definecolor{qqqqff}{rgb}{0.,0.,1.}
\definecolor{xdxdff}{rgb}{0.49019607843137253,0.49019607843137253,1.}
\begin{tikzpicture}[line cap=round,line join=round,>=triangle 45,x=1.5cm,y=1.5cm]
%\begin{axis}[
x=1.4cm,y=1.4cm,
axis lines=middle,
ymajorgrids=true,
xmajorgrids=true,
xmin=-4.5,
xmax=7.0600000000000005,
ymin=-1.46,
ymax=3.3,
xtick={-4.0,-3.0,...,7.0},
ytick={-2.0,-1.0,...,6.0},]
\clip(-5.,-1.) rectangle (7.06,3.3);
\fill[line width=2.pt,color=ffffqq,fill=ffffqq,fill opacity=\gelb] (-2.,-1.) -- (2.,-1.) -- (2.,3.) -- (-2.,3.) -- cycle;
\fill[line width=2.pt,color=ffffqq,fill=ffffqq,fill opacity=0.] (0.,2.) -- (0.,0.) -- (1.7320508075688776,1.) -- cycle;
\fill[line width=2.pt,color=ffffqq,fill=ffffqq,fill opacity=0.] (0.,2.) -- (-1.74,0.98) -- (0.013345911860126902,-0.016884202584923735) -- cycle;
\draw [rotate around={90.:(0.,1.)},line width=2.pt,color=black,fill=white,fill opacity=0.90000001192092896] (0.,1.) ellipse (1.50cm and 0.34937972405024542cm);
%\draw [line width=2.pt,color=ffffqq] (-2.,-1.)-- (2.,-1.);
%\draw [line width=2.pt,color=ffffqq] (2.,-1.)-- (2.,3.);
%\draw [line width=2.pt,color=ffffqq] (2.,3.)-- (-2.,3.);
%\draw [line width=2.pt,color=ffffqq] (-2.,3.)-- (-2.,-1.);
%\draw [line width=2.pt,color=ffffqq] (0.,2.)-- (0.,0.);
%\draw [line width=2.pt] (0.,0.)-- (1.7320508075688776,1.);
%\draw [line width=2.pt] (1.7320508075688776,1.)-- (0.,2.);
%\draw [line width=2.pt] (0.,2.)-- (-1.74,0.98);
%\draw [line width=2.pt] (-1.74,0.98)-- (0.013345911860126902,-0.016884202584923735);
\begin{scriptsize}
\draw [fill=xdxdff] (0.,2.) circle (2.5pt);
\draw [fill=qqqqff] (0.,0.) circle (2.0pt);
%\draw [fill=ududff] (2.,-1.) circle (2.5pt);
\draw[color=black] (0.4,1) node {$f$};
\draw[color=black] (-0.4,1) node {$e$};
%\draw [fill=ududff] (1.7320508075688776,1.) circle (2.5pt);
%\draw [fill=ududff] (-1.74,0.98) circle (2.5pt);
\draw [fill=ududff] (0.013345911860126902,-0.016884202584923735) circle (2.5pt);
\end{scriptsize}
%\end{axis}
\end{tikzpicture}
\caption{Randkantenpaar vom Typ 2}
\end{figure}

  %-------------------------------------------------------------
 \item[b)] Man nennt $\{e,f\}$  ein \emph{Randkantenpaar vom Typ 1}, falls $\{e,f\}$ ein Randkantenpaar ist und beide Kanten zu genau einem gemeinsamem Knoten 
 $V_{e,f}\in X_0$ inzident sind. Das heißt, es gilt $V_{e,f}<e$ und $V_{e,f}<f$ in $X$. Die übrigen beiden Knoten, die zu $e$ bzw. $f$ inzident sind, werden mit $V_{e}$,$V_{f}\in X_0$ bezeichnet, wobei $V_{e}<e$ und $V_f<f$ ist. Falls keine Kante $g\in X_1$ mit $X_0(g)=\{V_e,V_f\}$ existiert, dann ist das Randkantenpaar $\{e,f\}$ vom Typ 1 \emph{mendable}.
 %-------
 \begin{figure}[H]
 \definecolor{ffffff}{rgb}{1.,1.,1.}
\definecolor{ududff}{rgb}{0.30196078431372547,0.30196078431372547,1.}
\definecolor{ffffqq}{rgb}{1.,1.,0.}
\begin{tikzpicture}[line cap=round,line join=round,>=triangle 45,x=1.5cm,y=1.5cm]
x=1.5cm,y=1.5cm,
axis lines=middle,
ymajorgrids=true,
xmajorgrids=true,
xmin=-3.3,
xmax=3.0600000000000005,
ymin=-2.46,
ymax=2.3,
xtick={-4.0,-3.0,...,7.0},
ytick={-2.0,-1.0,...,6.0},]
\clip(-5.3,-2.) rectangle (3.06,2.3);
\fill[line width=2.pt,color=ffffqq,fill=ffffqq,fill opacity=0.550000011920929] (-2.,2.) -- (-2.,-2.) -- (2.,-2.) -- (2.,2.) -- cycle;
\fill[line width=2.pt,color=ffffff,fill=ffffff,fill opacity=1.0] (-1.,1.) -- (0.,-0.8) -- (1.05884572681199,0.9660254037844385) -- cycle;
\draw [line width=2.pt] (-1.,1.)-- (0.,-0.8);
\draw [line width=2.pt] (0.,-0.8)-- (1.05884572681199,0.9660254037844385);
\begin{scriptsize}
\draw [fill=ududff] (-1.,1.) circle (2.5pt);
\draw[color=black] (-1.34,1.35) node {$V_{e}$};
\draw [fill=ududff] (0.,-0.8) circle (2.5pt);
\draw[color=black] (0.08,-1.11) node {$V_{e,f}$};
%\draw[color=ffffff] (0.48,0.57) node {$Vieleck2$};
\draw[color=black] (-0.72,0.13) node {e};
\draw[color=black] (0.86,0.09) node {f};
\draw [fill=ududff] (1.05884572681199,0.9660254037844385) circle (2.5pt);
\draw[color=black] (1.2,1.33) node {$V_{f}$};
\end{scriptsize}

\end{tikzpicture}
\caption{mendable Randkantenpaar vom Typ 1}
\end{figure}

 %-----------------------------------------------
 \item[c)] Man nennt $\{e,f\}$  ein \emph{Randkantenpaar vom Typ 0}, falls $\{e,f\}$ ein Randkantenpaar ist und $X_{0}(e) \cap X_{0}(f)= \emptyset$, wobei $X_{0}(e)=\{V_{e},W_{e}\}$ und $X_{0}(f)=\{V_{f},W_{f}\}$ ist. Das Randkantenpaar ist \emph{mendable} bezüglich $V_e$ und $V_f$, falls keine Kante $g \in X_1$ mit $X_0(g)=\{W_e,W_f\}$ oder $X_0(g)=\{V_e,V_f\}$ existiert.\\
 Das Randkantenpaar $\{e,f\}$ heißt mendable, falls es Knoten $V \in \{V_e,W_e\}$ und $W \in \{V_f,W_f\}$ gibt, sodass $\{e,f\}$ mendable bezüglich $V$ und $W$ ist.
 %---------------bidl----------------------
%--------------------------------------------------
%----------------------bild ----------------------------
\begin{figure}[H]
\definecolor{ffffff}{rgb}{1.,1.,1.}
\definecolor{ududff}{rgb}{0.30196078431372547,0.30196078431372547,1.}
\definecolor{ffffqq}{rgb}{1.,1.,0.}
\begin{tikzpicture}[line cap=round,line join=round,>=triangle 45,x=1.5cm,y=1.5cm]
x=1.0cm,y=1.0cm,
axis lines=middle,
ymajorgrids=true,
xmajorgrids=true,
xmin=-2.3,
xmax=2.7,
ymin=-2.34,
ymax=2.3,
xtick={-4.0,-3.0,...,18.0},
ytick={-5.0,-4.0,...,6.0},]
\clip(-5.3,-1.84) rectangle (2.7,2.3);
\fill[line width=2.pt,color=ffffqq,fill=ffffqq,fill opacity=\gelb] (-2.,2.) -- (-2.,-2.) -- (2.,-2.) -- (2.,2.) -- cycle;
\fill[line width=2.pt,color=ffffff,fill=ffffff,fill opacity=1.0] (-1.,1.) -- (-1.,-1.) -- (1.,-1.) -- (1.,1.) -- cycle;
\fill[line width=2.pt,color=black,fill=ffffqq,fill opacity=\gelb] (1.,1.) -- (-1.,-1.013) -- (-0.,-1.013) -- (1.,1) -- cycle;

\draw [line width=2.pt] (-1.,1.)-- (-1.,-1.);
\draw [line width=2.pt] (1.,1.)-- (-1.,-1.);
%\draw [line width=2.pt,color=ffffff] (-1.,-1.)-- (1.,-1.);
\draw [line width=2.pt] (1.,-1.)-- (1.,1.);
%\draw [line width=2.pt,color=ffffff] (1.,1.)-- (-1.,1.);
\begin{scriptsize}
%\draw[color=ffffqq] (0.48,0.17) node {$Vieleck1$};
\draw[color=black] (-1,1.23) node {$V_{e}$};
\draw[color=black] (1,1.23) node {$V_{f}$};
\draw[color=black] (-1,-1.23) node {$W_{e}$};
\draw[color=black] (1,-1.23) node {$W_{f}$};
\draw [fill=ududff] (-1.,1.) circle (2.5pt);
\draw [fill=ududff] (-1.,-1.) circle (2.5pt);
%\draw[color=ffffff] (0.48,0.17) node {$Vieleck2$};
\draw[color=black] (-1.26,0.17) node {e};
\draw[color=black] (1.18,0.17) node {f};
\draw [fill=ududff] (1.,-1.) circle (2.5pt);
\draw [fill=ududff] (1.,1.) circle (2.5pt);
\end{scriptsize}
\end{tikzpicture}
\caption{mendable Randkantenpaar vom Typ 0 bzgl. $V_e$ und $V_f$}
\end{figure}
 %-------------------------------------------------
 \end{enumerate}
 \end{enumerate}
 \end{definition}
 
 \begin{bemerkung}
 Es ist leicht einzusehen, dass die Kanten $e,f \in X_1$ eines Randkantenpaares vom Typ $i$, genau $i$ Knoten gemeinsam haben, wobei $i=0,1,2$ ist. Das heißt, es existieren $V_1,\ldots,V_i \in X_0$ mit 
\[
V_j <e \text{ und } V_j <f \text{ für j=1,\ldots,i}.
\]

 \end{bemerkung}
 \subsection{Mending-Map}
 Ziel dieses Abschnittes ist es, Abbildungen zwischen zwei simplizialen Flächen zu charakterisieren. Von besonderem Interesse sind hierbei jene Abbildungen, in welchen Inzidenzen in der einen simplizialen Fläche Inzidenzen im Bild, also in der zweiten simplizialen Fläche, implizieren.\\
 Zur Erinnerung wird nun der Begriff eines Urbildes einer Menge unter einer Abbildung angeführt.
\begin{bemerkung}
Seien $A,B$ Mengen und $f:A \to B$ eine Funktion. Für $M \subseteq B$ definiert man das \emph{Urbild von M in A} durch 
\[
f^{-1}(M):=\{y\in A \mid f(y)\in M\}.
\]
Für $\{y\} \subseteq B$ schreibt man auch
\[
f^{-1}(y):=f^{-1}(\{y\}).
\]
\end{bemerkung}
  \begin{definition}
  Seien $(X,<)$ und $(Y,\prec)$ simpliziale Flächen.
  \begin{enumerate}
  \item Man nennt eine Überdeckung $\alpha:X \to Y$ eine \emph{Mending Map}, falls sie eine Bijektion $\beta : X_{2}\to Y_{2}$ induziert. Dadurch entsteht die simpliziale Fläche $(X(\alpha),<_{\alpha})$ mit den Knoten
  \[
X(\alpha)_0:=\{\alpha^{-1}(V)\mid V \in Y_0 \},
  \] 
  den Kanten 
  \[
X(\alpha)_1:=\{\alpha^{-1}(e)\mid e \in Y_1 \} 
  \]
 und den Flächen 
  \[
X(\alpha)_2:=X_2  .
  \]
  Für $A,B \in Y$ gilt $\alpha^{-1}(A)<_{\alpha}\alpha^{-1}(B)$ in $X(\alpha)$ genau dann, wenn $A \prec B  $ in $Y$ gilt. Außerdem ist $(X(\alpha),<_{\alpha})$ isomorph zu $(Y,\prec)$. Man nennt $X(\alpha)$ ein \emph{Mending von $X$}.

  \item Die Menge aller Mendings von $(X,<)$ wird definiert durch 
\[
\mathcal{M}(X):=\{  X(\alpha )\mid \text{$(Y,\prec )$ simpl. Fläche so, dass  $\alpha : X \to Y$ Mending Map}\}.
\]
  \end{enumerate}
  \end{definition}
  
  \begin{bsp}
  Sei $J$ der oben definierte Janus-Kopf und $X:= 2\Delta$. Dann ist die Abbildung 
  \[
  \alpha: X \to J, x \mapsto 
  \begin{cases}
e_i & \text{für } x =e_i^j ,i=1,2,3,\,j=1,2\\
V_i &\text{für } x =V_i^j,i=1,2,3,\,j=1,2\\
F_i &\text{für } x=F_i , i=1,2 
\end{cases}
  \]
  nach Konstruktion eine Mending Map. Man erhält dadurch die simpliziale Fläche $(X(\alpha),<_\alpha)$ definiert durch die Knoten
  \[
  X(\alpha)_0=\{\{V_1^1,V_1^2\},\{V_2^1,V_2^2\},\{V_3^1,V_3^2\}\},
  \]
  die Kanten
  \[
  X(\alpha)_1=\{\{e_1^1,e_1^2\},\{e_2^1,e_2^2\},\{e_3^1,e_3^2\}\}
  \]
  und die Flächen 
  \[
X(\alpha)_2=X_2=\{F_1,F_2\} .
  \]
  Es gilt $x<_{\alpha}y$ in $X(\alpha)$ genau dann, wenn
  \begin{align*}
 (x,y) \in \{&(\{V_i^1,V_i^2\}, \{e_j^1,e_j^2\})\mid i=1,2,3 \, j\in\{1,2,3\} \setminus \{i\}\}\, \cup\\
  \{&(\{V_i^1,V_i^2\}, F_j)\mid i=1,2,3 \, j=1,2 \} \,\cup\\
  \{&(\{e_i^1,e_i^2\}, F_j)\mid i=1,2,3 \, j=1,2 \}. 
\end{align*}
 \end{bsp}
   %----bild----------------------------
   \begin{figure}[H]
\definecolor{sqsqsq}{rgb}{0.12549019607843137,0.12549019607843137,0.12549019607843137}
\definecolor{ttqqqq}{rgb}{0.2,0.,0.}
\definecolor{ffffqq}{rgb}{1.,1.,0.}
\definecolor{qqqqff}{rgb}{0.,0.,1.}
\begin{tikzpicture}[line cap=round,line join=round,>=triangle 45,x=1.0cm,y=1.0cm]

x=1.0cm,y=1.0cm,
axis lines=middle,
ymajorgrids=true,
xmajorgrids=true,
xmin=-5.056290110700678,
xmax=5.380866801866215,
ymin=-0.9227448489396118,
ymax=4.359364127681193,
xtick={-5.0,-4.5,...,5.0},
ytick={-0.5,0.0,...,4.0},]
\clip(-7.056290110700678,-0.5227448489396118) rectangle (5.380866801866215,4.359364127681193);
\fill[line width=2.pt,color=ffffqq,fill=ffffqq,fill opacity=0.5] (-2.,0.) -- (2.,0.) -- (0.,3.4641016151377553) -- cycle;
\fill[line width=2.pt,color=ffffqq,fill=ffffqq,fill opacity=0.5] (0.,3.4641016151377553) -- (2.,0.) -- (4.,3.464101615137754) -- cycle;
\draw [line width=2.pt] (-2.,0.)-- (2.,0.);
\draw [line width=2.pt] (2.,0.)-- (0.,3.4641016151377553);
\draw [line width=2.pt] (0.,3.4641016151377553)-- (-2.,0.);
\draw [line width=2.pt,color=ttqqqq] (0.,3.4641016151377553)-- (2.,0.);
\draw [line width=2.pt] (2.,0.)-- (4.,3.464101615137754);
\draw [line width=2.pt,color=sqsqsq] (4.,3.464101615137754)-- (0.,3.4641016151377553);
\begin{scriptsize}
\draw [fill=qqqqff] (-2.,0.) circle (2.5pt);
\draw[color=black] (-2.347666183295526,-0.3234108260899206) node {$\{V_1^1,V_1^2\}$};
\draw [fill=qqqqff] (2.,0.) circle (2.5pt);
\draw[color=black] (2.390727102068595,-0.3234108260899206) node {$\{V_2^1,V_2^2\}$};
\draw[color=sqsqsq] (0.04884098783747624,1.2554444197699985) node {$F_1$};
\draw[color=black] (0.057916776457099625,-0.25407545041275472) node {$\{e_3^1,e_3^2\}$};
\draw[color=black] (1.474238776670776,1.9089012003828816) node {$\{e_1^1,e_1^2\}$};
\draw[color=black] (-1.4673146871920574,1.9089012003828816) node {$\{e_2^1,e_2^2\}$};
\draw [fill=qqqqff] (0.,3.4641016151377553) circle (2.5pt);
\draw[color=black] (0.08514414231596978,3.75145261535057) node {$\{V_3^1,V_3^1\}$};
\draw[color=sqsqsq] (2.0455144841546207,2.417145363081791) node {$F_2$};
\draw[color=black] (3.6162912160860375,1.9089012003828816) node {$\{e_3^1,e_3^2\}$};
\draw[color=sqsqsq] (2.154590272774244,3.6903620787860503) node {$\{e_2^1,e_2^2\}$};
\draw [fill=qqqqff] (4.,3.464101615137754) circle (2.5pt);
\draw[color=black] (4.087566923569883,3.75145261535057) node {$\{V_1^1,V_1^2\}$};
\end{scriptsize}
\end{tikzpicture}
\caption{Mending einer simplizialen Fläche}
\end{figure}
 %------------------------------------

  \begin{bemerkung}
  Sei $(X,<)$ eine simpliziale Fläche.
  \begin{enumerate}
  \item Für ein $Y \in \mathcal{M}(X)$ gilt $Y_2=X_2$.
  \item $X$ bildet ein Mending von sich selbst mit der Identität als Mending Map.
  \item Sei $Y\in \mathcal{M}(X)$ ein Mending von $X$. Das heißt, es gilt $Y=X(\alpha)$ für eine Mending Map $\alpha:X \mapsto Z$, wobei $(Z,\prec)$ eine weitere simpliziale Fläche ist. Dann bildet die Relation 
  \[
A\sim_\alpha B \Leftrightarrow \alpha(A)=\alpha(B),\quad A,B \in X
  \]
  eine Äquivalenzrelation auf X.
  \item Falls $Y\in \mathcal{M}(X)$ und $Z \in \mathcal{M}(Y)$ ist, so gilt auch $Z\in \mathcal{M}(X)$.
   Da $Y\in \mathcal{M}(X)$ und $Z\in \mathcal{M}(Y)$ ist, existieren die Mendings $\alpha_1: X \to Y$ und $\alpha_2:Y \to Z$.
    Dann ist die Abbildung $\alpha :X \to Z,x\mapsto \alpha_2(\alpha_1(x))$ eine Mending Map.
     Die Surjektivität von $\alpha$ folgt aus der Surjektivität der Mending Maps $\alpha_1$ und $\alpha_2$.
      Bleibt also nur noch zu zeigen, dass Inzidenzen in $X$ Inzidenzen in $Z$ implizieren. Seien dazu $x,y\in X$, die $x<y$ erfüllen.
       Da $\alpha_1$ eine Mending Map ist, folgt $\alpha_1(x)<\alpha_1(y)$ in $Y$. 
     Weil $\alpha_2$ eine Mending Map ist, folgt aus $\alpha_1(x)<\alpha_1 (y)$ in $Y$, dass $\alpha_2 ( \alpha_1(x))<\alpha_2(\alpha_1(y))$ in $Z$ gelten muss.
      Damit ist die Abbildung $\alpha$ eine Mending Map.
  \end{enumerate}
  \end{bemerkung}
  
  \begin{definition}
  Für eine simpliziale Fläche $(X,<)$ und $i=0,1,2$ definiert man $I^{i}(X)$ als die Menge der Kanten mit $i$ inneren Knoten, und $BM^{i}(X)$ als die Menge der mendable Randkantenpaare vom Typ $i$. Also sind
  \begin{align*}
  &I^i(X):=\{e \in X_1 \mid \vert X_{0}^{0}(e)\vert=i\}\text{ und}\\
  &BM^{i}(X):=\{\{e,f\}\subseteq X_1 \mid \{e,f\}\text{ ist ein mendable Randkantenpaar vom Typ i}\}.\\
  \end{align*}
  \end{definition}
  
  
 % \newpage
  \subsection{Mender- und Cutter-Operatoren}
  Im Folgenden wird thematisiert, wie aus einer simplizialen Fläche durch Manipulation der Kanten eine weitere simpliziale Fläche konstruiert werden kann. Zu diesem Zweck werden die \emph{Mender}-Operationen, die aus zwei Randkanten eine innere Kante konstruieren  und die \emph{Cutter}-Operationen, die aus einer inneren Kante zwei Randkanten hervorbringen, eingeführt.\\
  Sei dazu $(X,<)$ eine simpliziale Fläche.
 Dann ist der Operator \emph{Cratermender} definiert durch
\[
 C_{e,f}^{m}:\{Y \in \mathcal{M}(X)|\{e,f\} \in BM^{2}(Y) \}   \to 
  \{Z \in \mathcal{M}(X)|\{e,f\} \in I^{2}(Z)\},\
  Y \mapsto Z,
  \]
  wobei $Z_{2}:=Y_{2},Z_{1}:=(Y_{1}-\{e,f\}) \cup \{\{e,f\}\}$ und $Z_0:=Y_{0}$ ist.
  Er setzt die beiden Randkanten $e$ und $f$ zu einer inneren Kante $\{e,f\}$ vom Typ 2 zusammen, um somit die simpliziale Fläche $Z=C^{m}_{e,f}(Y)$ zu erhalten. Den inversen Operator $C^{c}_{\{e,f\}}$ nennt man \emph{Cratercutter}. 
  %---------------------------bild------------------
  \begin{figure}[H]
  \definecolor{ududff}{rgb}{0.30196078431372547,0.30196078431372547,1.}
\definecolor{ffffqq}{rgb}{1.,1.,0.}
\begin{tikzpicture}[line cap=round,line join=round,>=triangle 45,x=1.3cm,y=1.3cm]
%\begin{axis}[
x=1.5cm,y=1.5cm,
axis lines=middle,
ymajorgrids=true,
xmajorgrids=true,
xmin=-5.54,
xmax=5.82,
ymin=-0.6600000000000006,
ymax=4.100000000000001,
xtick={-5.0,-4.0,...,5.0},
ytick={-2.0,-1.0,...,6.0},]
\clip(-6.14,-0.4) rectangle (5.82,4.1);
\fill[line width=2.pt,color=ffffqq,fill=ffffqq,fill opacity=\gelb] (-5.,0.) -- (-1.,0.) -- (-1.,4.) -- (-5.,4.) -- cycle;
\fill[line width=2.pt,color=ffffqq,fill=ffffqq,fill opacity=\gelb] (1.,0.) -- (5.,0.) -- (5.,4.) -- (1.,4.) -- cycle;
\draw [rotate around={90.:(-3.,2.)},line width=2.pt,color=black,fill=white,fill opacity=0.90000001192092896] (-3.,2.) ellipse (1.2586450581159654cm and 0.29400452165887971cm);
\draw [line width=2.pt] (3.,3.)-- (3.,1.);
\draw [line width=1.pt] (.5,1.5)-- (-.7,1.5);
\draw [line width=1.pt] (-0.7,1.5)-- (-.6,1.6);
\draw [line width=1.pt] (-0.7,1.5)-- (-.6,1.4);
\draw [line width=1.pt] (.5,2.5)-- (-.7,2.5);
\draw [line width=1.pt] (.4,2.6)-- (.5,2.5);
\draw [line width=1.pt] (.4,2.4)-- (.5,2.5);
\begin{scriptsize}
\draw [fill=ududff] (-3.,3.) circle (2.5pt);
%\draw[color=ududff] (-2.86,3.37) node {$I$};
\draw [fill=ududff] (-3.,1.) circle (2.5pt);
%\draw[color=ududff] (-2.86,1.37) node {$J$};
\draw[color=black] (-2.52,1.97) node {$\{f\}$};
\draw[color=black] (-3.52,1.97) node {$\{e\}$};
\draw [fill=ududff] (3.,3.) circle (2.5pt);
%\draw[color=ududff] (3.14,3.37) node {$L$};
\draw [fill=ududff] (3.,1.) circle (2.5pt);
%\draw[color=ududff] (3.14,1.37) node {$M$};
\draw[color=black] (3.44,2.17) node {$\{e,f\}$};
\draw[color=black] (0,1.8) node {$C^c_{\{e,f\}}$};
\draw[color=black] (-0.2,2.8) node {$C^m_{e,f}$};
\end{scriptsize}
%\end{axis}
\end{tikzpicture}
\caption{Cratercutter und Cratermender}
\end{figure}
  %---------------------------------------------------
Den Operator \emph{Ripmender}, welcher ein mendable Randkantenpaar $\{e,f\}$ vom Typ 1 zu einer inneren Kante $\{e,f\}$ zusammensetzt, definiert man durch
  \[
R^m_{e,f} :\{Y \in \mathcal{M}(X)|\{e,f\} \in BM^{1}(Y) \}  \to 
  \{Z \in \mathcal{M}(X)|\{e,f\} \in I^{1}(Z)\}, 
  Y \mapsto Z,
  \]
  wobei $Z_{1}:=(Y_1-\{e,f\}) \cup\{\{e,f\}\},Z_0 :=(Y_0 - \{V_{e},V_{f}\}) \cup \{\{V_{e},V_{f}\}\},$ und
  $Z_{2}:=X_{2}$ ist und $V_e, V_f,V_{e,f}\in Y_0$ wie in Definition \ref{mend} definiert sind.
  Die zum Ripmender inverse Operation $R^c_{\{e,f\}}$ nennt man \emph{Ripcutter}. 
  %-----------------------------bild----------------------
 
  \begin{figure}[H]
  \definecolor{sqsqsq}{rgb}{0.12549019607843137,0.12549019607843137,0.12549019607843137}
\definecolor{ffffff}{rgb}{1.,1.,1.}
\definecolor{ududff}{rgb}{0.30196078431372547,0.30196078431372547,1.}
\definecolor{ffffqq}{rgb}{1.,1.,0.}
\begin{tikzpicture}[line cap=round,line join=round,>=triangle 45,x=1.3cm,y=1.3cm]
x=1.0cm,y=1.0cm,
axis lines=middle,
ymajorgrids=true,
xmajorgrids=true,
xmin=-5.36,
xmax=5.24,
ymin=-2.38,
ymax=2.26,
xtick={-5.0,-4.0,...,6.0},
ytick={-5.0,-4.0,...,6.0},]
\clip(-5.86,-2.) rectangle (5.24,2.26);
\fill[line width=2.pt,color=ffffqq,fill=ffffqq,fill opacity=0.6000000238418579] (-5.,2.) -- (-5.,-2.) -- (-1.,-2.) -- (-1.,2.) -- cycle;
\fill[line width=2.pt,color=ffffff,fill=ffffff,fill opacity=1.0] (-3.,-1.) -- (-1.84,1.) -- (-4.152050807568877,1.0045894683899497) -- cycle;
\fill[line width=2.pt,color=ffffqq,fill=ffffqq,fill opacity=0.6000000238418579] (1.,2.) -- (1.,-2.) -- (5.,-2.) -- (5.,2.) -- cycle;
\draw [line width=2.pt] (-3.,-1.)-- (-1.84,1.);
\draw [line width=2.pt,color=ffffff] (-1.84,1.)-- (-4.152050807568877,1.0045894683899497);
\draw [line width=2.pt,color=sqsqsq] (-4.152050807568877,1.0045894683899497)-- (-3.,-1.);
\draw [line width=2.pt] (3.,1.)-- (3.,-1.);
\draw [line width=1.pt,color=black] (-0.7,0.5)-- (.5,0.5);
\draw [line width=1.pt,color=black] (-0.7,-0.5)-- (-0.6,-0.4);
\draw [line width=1.pt,color=black] (-0.7,-0.5)-- (-0.6,-0.6);
\draw [line width=1.pt,color=black] (-0.7,-0.5)-- (.5,-0.5);
\draw [line width=1.pt,color=black] (0.4,0.6)-- (.5,0.5);
\draw [line width=1.pt,color=black] (0.4,0.4)-- (.5,0.5);                                                                                                                                                                                                                                                                                                                                                                                                                                                                                                                                                                                                                                                                                                                                                                                                                                                                                                                                                                                                                                               

\begin{scriptsize}
\draw [fill=ududff] (-3.,-1.) circle (2.5pt);
\draw [fill=ududff] (-1.84,1.) circle (2.5pt);
%\draw[color=ffffff] (-2.52,0.51) node {$Vieleck2$};
\draw[color=black] (-2.1,0.01) node {$\{f\}$};
\draw[color=sqsqsq] (-3.9,0.03) node {$\{e\}$};
\draw [fill=ududff] (-4.152050807568877,1.0045894683899497) circle (2.5pt);
\draw [fill=ududff] (3.,1.) circle (2.5pt);
\draw[color=black] (3.,1.25) node {$\{V_{e},V_{f}\}$};
\draw[color=black] (-1.8,1.25) node {$\{V_{e}\}$};
\draw[color=black] (-4.2,1.25) node {$\{V_{f}\}$};
\draw [fill=ududff] (3.,-1.) circle (2.5pt);
\draw[color=black] (3.,-1.25) node {$\{V_{e,f}\}$};
\draw[color=black] (-3.,-1.25) node {$\{V_{e,f}\}$};
\draw[color=black] (2.6,0.17) node {$\{e,f\}$};
\draw[color=black] (-0.2,0.75) node {$R^m_{e,f}$};
\draw[color=black] (-0.2,-0.25) node {$R^c_{\{e,f\}}$};
\end{scriptsize}

\end{tikzpicture}
\caption{Ripcutter und Ripmender}
\end{figure}

  %--------------------------------------------------------
Seien nun $V,V'$ bzw. $W,W'$ Knoten, die inzident zu $e$ bzw. $f$ sind. Dann ist 
  \begin{align*}
  S^m_{(V,e),(W,f)}:&\{Y \in \mathcal{M}(X)|\{e,f\} \in BM^{0}(Y) \text{ mendable bzgl. V und W} \}  
  \to\\ & \{Z \in \mathcal{M}(X)|\{e,f\} \in I^{0}(Z)\}, 
  Y \mapsto Z, 
   \end{align*}
  der Operator \emph{Splitmender}, wobei $Z_2 := X_2,Z_1 := (Y_1 - \{e,f\}) \cup \{\{e,f\}\}$ und $Z_0:=(Y_0 -(Y_0 (e) \cup Y_0(f))) \cup \{\{V,W\},\{V',W'\}\}$ ist. Dieser setzt zwei disjunkte Kanten, also Kanten, die keinen Knoten gemeinsam haben, zu einer inneren Kante zusammen, um somit die simpliziale Fläche $Z=S^m_{(V,e),(W,f)}(Y)=S^m_{(V',e),(W',f)}(Y)$ zu erhalten. Dieser Operator hat ein eindeutiges Linksinverses, nämlich den Operator \emph{Splitcutter} $S^c_{\{e,f\}}$, wobei dieser ebenfalls ein Linksinverses des Operators  $S^C_{(V',e),(W,f)}=S^C_{(V,e),(W',f)}$ ist, falls $\{e,f\}$ mendable bezüglich $V',W$ bzw. $V,W'$ ist.
  %-------------------bild------------------------------
  %------------------------------------------------------
\begin{bemerkung}
Seien $(X,<)$ eine simpliziale Fläche, $(Y,\prec)\in \mathcal{M}(X)$ ein Mending von $X$ und $e,f \in Y_1$ Kanten in $X$ so, dass $\{e,f\}$ ein Randkantenpaar ist.
 Falls $\{e,f\}$ ein Randkantenpaar vom Typ 1 ist, ist klar, wie die simpliziale Fläche $R^m_{e,f}(Y)$ aus $(Y,\prec)$ hervorgeht. Gleiches gilt auch für die simpliziale Fläche $C^m_{e,f}(Y)$, falls $\{e,f\}$ ein Randkantenpaar vom Typ 2 ist.\\
Der Fall, dass $\{e,f\}$ ein Randkantenpaar vom Typ 0 ist, wird nun näher erläutert.\\
Seien $V_e,W_e \in Y_0$ die zu $e$ und $V_f,W_f \in Y_0$ die zu $f$ zugehörigen Knoten. Das heißt
\[
V_e,W_e \prec e \text{ und } V_f,W_f \prec f.
\]  
Dann werden zwei Fälle unterschieden:
\begin{enumerate}
\item Angenommen für $V_1\in \{V_e,W_e\}$ und $V_2 \in \{V_f,W_f\}$ existiert keine Kante $g \in Y_1$ mit $X_0(g)=\{V_1,V_2\}$. Dann existieren zwei Möglichkeiten, um mit dem Splitmender eine simpliziale Fläche zu konstruieren, nämlich $Z=S^m_{(V_e,e),(V_f,f)}(Y)$ und $T=S^m_{(V_e,e),(W_f,f)}(Y)$, 
wobei für $\{V_e,V_f\},\{W_e,W_f\} \in Z_0$ und $\{e,f\} \in Z_1$ die Relation
\[
\{V_e,V_f\},\{W_e,W_f\}\prec_1 \{e,f\}
\] mit $\prec_1$ als Inzidenz auf $Z$ und für $\{V_e, W_f\},\{V_f,W_e\}\in T_0$ und $\{e,f\}\in T_1$ die Relation
\[
\{V_e,W_f\},\{V_f,W_e\}\prec_2 \{e,f\} 
\]  mit $\prec_2$ als Inzidenz auf $T$ gilt.
 %----------------------------------
\begin{figure}[H] 
 \definecolor{ffffff}{rgb}{1.,1.,1.}
\definecolor{qqqqff}{rgb}{0.,0.,1.}
\definecolor{ffffqq}{rgb}{1.,1.,0.}
\begin{tikzpicture}[line cap=round,line join=round,>=triangle 45,x=1.0cm,y=1.0cm]
%\begin{axis}[
x=1.0cm,y=1.0cm,
axis lines=middle,
ymajorgrids=true,
xmajorgrids=true,
xmin=-6.81112373905725,
xmax=14.391643353227584,
ymin=-8.866642747190788,
ymax=6.9246706856003435,
xtick={-16.0,-14.0,...,14.0},
ytick={-8.0,-6.0,...,6.0},]
\clip(-6.81112373905725,-5.4) rectangle (14.391643353227584,5.2);
\fill[line width=2.pt,color=ffffqq,fill=ffffqq,fill opacity=\gelb] (-5.,-2.) -- (-1.,-2.) -- (-1.,2.) -- (-5.,2.) -- cycle;
\fill[line width=2.pt,color=ffffqq,fill=ffffqq,fill opacity=\gelb] (1.,5.) -- (1.,1.) -- (5.,1.) -- (5.,5.) -- cycle;
\fill[line width=2.pt,color=ffffqq,fill=ffffqq,fill opacity=\gelb] (1.,-1.) -- (1.,-5.) -- (5.,-5.) -- (5.,-1.) -- cycle;
\fill[line width=2.pt,color=ffffff,fill=ffffff,fill opacity=1.0] (-2.,-1.) -- (-2.,1.) -- (-4.,1.) -- (-4.,-1.) -- cycle;
\draw [line width=2.pt] (-2.,-1.)-- (-2.,1.);
\draw [line width=2.pt] (-4.,1.)-- (-4.,-1.);
\draw [line width=2.pt] (3.,-2.)-- (3.,-4.);
\draw [line width=2.pt] (3.,4.)-- (3.,2.);
\draw [line width=1.pt] (-0.6,0.)-- (0.6,0.);
\draw [line width=1.pt] (-0.6,0.)-- (-0.5,0.1);
\draw [line width=1.pt] (-0.6,0.)-- (-0.5,-0.1);
%---
\draw [line width=1.pt] (-0.6,1.5)-- (0.6,1.5);
\draw [line width=1.pt] (0.6,1.5)-- (0.4,1.4);
\draw [line width=1.pt] (0.6,1.5)-- (0.4,1.6);
%----
\draw [line width=1.pt] (-0.6,-1.5)-- (0.6,-1.5);
\draw [line width=1.pt] (0.6,-1.5)-- (0.4,-1.4);
\draw [line width=1.pt] (0.6,-1.5)-- (0.4,-1.6);
\begin{scriptsize}
\draw[color=black] (-2.349319512806975,0.) node {$f$};
\draw[color=black] (-4.3,0.) node {$e$};
%\draw[color=ffffqq] (3.6470383371016757,3.24817073690069) node {$Vieleck2$};
%\draw[color=ffffqq] (3.6470383371016757,-2.7481871130079707) node {$Vieleck3$};
\draw [fill=qqqqff] (-2.,-1.) circle (2.5pt);
\draw[color=black] (-1.9,-1.4) node {$W_f$};
\draw [fill=qqqqff] (-4.,-1.) circle (2.5pt);
%\draw[color=qqqqff] (-3.8144929240968724,-0.49616131417349635) node {$N$};
\draw [fill=qqqqff] (-2.,1.) circle (2.5pt);
\draw[color=black] (-1.9,1.4) node {$V_f$};
%\draw[color=ffffff] (-2.349319512806975,0.2364253914714531) node {$Vieleck4$};
\draw [fill=qqqqff] (-4.,1.) circle (2.5pt);
\draw[color=black] (-4.,1.4) node {$V_e$};
\draw [fill=qqqqff] (-4.,-1.) circle (2.5pt);
\draw[color=black] (-4,-1.4) node {$W_e$};
\draw [fill=qqqqff] (3.,-2.) circle (2.5pt);
\draw[color=black] (3.1,-1.6000764293165751) node {$\{V_e,W_f\}$};
\draw [fill=qqqqff] (3.,-4.) circle (2.5pt);
\draw[color=black] (3.1,-4.4) node {$\{V_f,W_e\}$};
\draw[color=black] (3.4431232219585983,-3.) node {$\{e,f\}$};
\draw [fill=qqqqff] (3.,4.) circle (2.5pt);
\draw[color=black] (3.1,4.4) node {$\{V_e,V_f\}$};
\draw [fill=qqqqff] (3.,2.) circle (2.5pt);
\draw[color=black] (3.1,1.6884511903059274) node {$\{W_e,W_f\}$};
\draw[color=black] (3.4431232219585983,3.1) node {$\{e,f\}$};
\draw[color=black] (0.,0.3) node {$S^c_{\{e,f\}}$};
\draw[color=black] (0.,1.9) node {$S^m_{(V_e,e),(V_f,f)}$};
\draw[color=black] (0.,-1.1) node {$S^m_{(V_e,e),(W_f,f)}$};
\end{scriptsize}
%\end{axis}
\end{tikzpicture}
\caption{Splitmender und Splitcutter}
\end{figure}
%-----------------bild-----------------
%-------------------------------------
%-----------------------------------------
\item Angenommen es existiert eine Kante $g\in Y_1$ mit $V_e,W_f <_{\alpha}g$, aber für $V_f$ und $W_e$ existiert keine Kante $h\in Y_1$ mit $V_f,W_e <_{\alpha} h$. 
%------------------------------------------
\begin{figure}[H]
\definecolor{uuuuuu}{rgb}{0.26666666666666666,0.26666666666666666,0.26666666666666666}
\definecolor{ffffff}{rgb}{1.,1.,1.}
\definecolor{ududff}{rgb}{0.30196078431372547,0.30196078431372547,1.}
\definecolor{ffffqq}{rgb}{1.,1.,0.}
\begin{tikzpicture}[line cap=round,line join=round,>=triangle 45,x=1.2cm,y=1.2cm]
%\begin{axis}[
x=1.0cm,y=1.0cm,
axis lines=middle,
ymajorgrids=true,
xmajorgrids=true,
xmin=-4.3,
xmax=7.0600000000000005,
ymin=-2.46,
ymax=6.3,
xtick={-4.0,-3.0,...,7.0},
ytick={-2.0,-1.0,...,6.0},]
\clip(-6.0,-0.4) rectangle (7.06,3.8);
\fill[line width=2.pt,color=ffffqq,fill=ffffqq,fill opacity=\gelb] (-2.,0.) -- (2.,0.) -- (2.,4.) -- (-2.,4.) -- cycle;
\fill[line width=2.pt,color=ffffff,fill=ffffff,fill opacity=1.0] (-1.,1.) -- (1.,1.) -- (1.,3.) -- (-1.,3.) -- cycle;
\draw [line width=2.pt] (1.,1.)-- (1.,3.);
\draw [line width=2.pt] (-1.,3.)-- (-1.,1.);
\draw [line width=2.pt] (-1.,3.)-- (1.,1.);
\begin{scriptsize}
%\draw[color=ffffqq] (0.48,2.17) node {$Vieleck1$};
\draw [fill=ududff] (-1.,1.) circle (2.5pt);
\draw[color=black] (-0.86,0.77) node {$W_e$};
\draw [fill=ududff] (1.,1.) circle (2.5pt);
\draw[color=black] (1.14,0.77) node {$W_f$};
%\draw[color=ffffff] (0.48,2.17) node {$Vieleck2$};
\draw [fill=ududff] (1.,3.) circle (2.5pt);
\draw[color=black] (1.14,3.27) node {$V_f$};
\draw [fill=ududff] (-1.,3.) circle (2.5pt);
\draw[color=black] (-0.86,3.27) node {$V_e$};
\draw[color=black] (-0.18,1.95) node {$g$};
\draw[color=black] (-1.22,1.95) node {$e$};
\draw[color=black] (1.18,1.95) node {$f$};
\end{scriptsize}
%\end{axis}
\end{tikzpicture}
\caption{mendable Randkantenpaar vom Typ 0 bzgl. $V_e,V_f$}
\end{figure}
%-----------------------------------------------
Dann gibt es genau eine Möglichkeit mit dem Splitmender eine simpliziale Fläche zu konstruieren, nämlich $Z=S^m_{(V_e,e),(V_f,f)}(Y)$, wobei für die Knoten $\{V_e,V_f\},\{W_e,W_f\}\in Z_0$ und $\{e,f\} \in Z_1$ die Relationen
\[
\{V_e,V_f\},\{W_e,W_f\}\prec_Z \{e,f\}
\]und 
\[
\{V_e,V_f\},\{W_e,W_f\}\prec_Z g
\] gelten. Dabei ist $\prec_Z$ die Inzidenz auf $Z$.
%-------------------bild---------------------------
%--------------------------------------------------
%------------bild----------------------------------
\begin{figure}[H]
\definecolor{ffffff}{rgb}{1.,1.,1.}
\definecolor{qqqqff}{rgb}{0.,0.,1.}
\definecolor{ffffqq}{rgb}{1.,1.,0.}
\begin{tikzpicture}[line cap=round,line join=round,>=triangle 45,x=1.5cm,y=1.5cm]
%\begin{axis}[
x=1.5cm,y=1.5cm,
axis lines=middle,
ymajorgrids=true,
xmajorgrids=true,
xmin=-4.3,
xmax=7.0600000000000005,
ymin=-2.46,
ymax=6.3,
xtick={-4.0,-3.0,...,7.0},
ytick={-2.0,-1.0,...,6.0},]
\clip(-4.8,-.46) rectangle (7.06,4.3);
\fill[line width=2.pt,color=ffffqq,fill=ffffqq,fill opacity=\gelb] (-2.,0.) -- (2.,0.) -- (2.,4.) -- (-2.,4.) -- cycle;
%\fill[line width=2.pt,color=ffffqq,fill=ffffqq,fill opacity=0.5] (2.25,2.45) -- (6.25,2.45) -- (6.25,6.45) -- (2.25,6.45) -- cycle;
\draw [line width=2.pt] (0.,1.)-- (0.,3.);
%\draw [line width=2.pt] (4.29,3.45)-- (4.29,5.45);

\draw [shift={(-1.46,2.02)},line width=2.pt,color=ffffff,fill=ffffff,fill opacity=1.0]  plot[domain=-0.6098060014472679:0.5911571672160445,variable=\t]({1.*1.7810109488714547*cos(\t r)+0.*1.7810109488714547*sin(\t r)},{0.*1.7810109488714547*cos(\t r)+1.*1.7810109488714547*sin(\t r)});
\draw [shift={(-1.46,2.02)},line width=2.pt]  plot[domain=-0.6098060014472679:0.5911571672160445,variable=\t]({1.*1.7810109488714547*cos(\t r)+0.*1.7810109488714547*sin(\t r)},{0.*1.7810109488714547*cos(\t r)+1.*1.7810109488714547*sin(\t r)});
%\draw [fill=qqqqff] (4.30,3.45) circle (2.5pt);
%\draw [fill=qqqqff] (4.30,5.45) circle (2.5pt);
\begin{scriptsize}
%\draw[color=ffffqq] (0.48,2.17) node {$Vieleck1$};
\draw [fill=qqqqff] (0.,1.) circle (2.5pt);
\draw[color=black] (0.5,2.) node {$g$};
\draw [fill=qqqqff] (0.,3.) circle (2.5pt);
\draw[color=black] (-0.3,2) node {$\{e,f\}$};
\draw[color=black] (0.,0.7) node {$\{W_e,W_f\}$};
\draw[color=black] (0.,3.3) node {$\{V_e,V_f\}$};
%\draw[color=ffffff] (0.5,2.31) node {$c$};
%\draw[color=black] (0.5,2.31) node {$d$};
\end{scriptsize}
\%end{axis}
\end{tikzpicture}
\caption{simpliziale Fläche nach Anwendung eines Splitmenders}
\end{figure}
%---------------------------------------------------
%\item Der Fall, das eine Kante $g\in Y_1$ mit $V_e,W_f<_{\alpha}$, aber keine Kante $h \in Y_1$ mit $V_f,W_e$ existiert,  erläuft analog zum zuvor behandelten Fall.
\end{enumerate}
\end{bemerkung}

  \begin{bemerkung}
  Sei $(X,<)$ eine simpliziale Fläche mit n Flächen. Dann gilt
  \begin{enumerate}
  \item $X$ ist isomorph zu einer simplizialen Fläche $(Y,\prec) \in \mathcal{M}(n\Delta)$.
  \item Sei $k$ die Anzahl der inneren Kanten in $(X,<)$. Dann ist die Anzahl, der auf $n\Delta$ ausgeführten Mender-Operationen, um $X$ zu erhalten, ebenfalls $k$. Eine analoge Aussage gilt auch für Cutter-Operationen.
  \item Es können genau dann keine Mender-Operationen auf $X$ durchgeführt werden, wenn $X$ geschlossen ist oder isomorph zu einer geschlossenen simplizialen Fläche ist, aus der eine Fläche entfernt wurde.
  \item Es gilt $X=n \cdot \Delta$ genau dann, wenn keine Cutter-Operationen auf $X$ angewendet werden können.
  \end{enumerate}
  \end{bemerkung}
  
%--------------------------------------------Definition HikingHole-----------------------------
\newpage
\subsection{Euler-Charakteristik}
%$\textcolor{red}{weg}$
\begin{definition}
Für eine simpliziale Fläche $(X,<)$ wird die Euler-Charakteristik $\chi(X)$ als
\[
\chi(X) := \vert X_0 \vert - \vert X_1\vert +\vert X_2 \vert
\] definiert.
\end{definition}
\begin{bemerkung}
 Zwei isomorphe simpliziale Flächen $(X,<)$ und $(Y, \prec)$ haben dieselbe Euler-Charakteristik, denn eine bijektive Abbilung $\alpha:X \to Y$ impliziert, wie oben schon erwähnt, bijektive Abbildungen  $X_i \to Y_i$ für $i=0,1,2$. Damit ist $\vert X_i \vert =\vert Y_i \vert $, woraus man
 \[
\chi(X) =\vert X_0 \vert - \vert X_1\vert +\vert X_2 \vert = \vert Y_0 \vert - \vert Y_1\vert +\vert Y_2 \vert =\chi(Y)
 \]
 folgern kann.\\
 Die Umkehrung gilt jedoch nicht, denn beispielsweise für den Janus-Kopf $J$ und das \emph{Tetraeder} $(T,<)$, definiert durch das ordinale Symbol 
 \begin{align*}
 \omega ((T,<)):=(4,6,4;(\{2,3\},\{1,3\},\{2,3\},\{3,4\},\{1,4\},\{2,4\}),\\
 (\{1,2,3\},\{2,4,5\},\{3,5,6\},\{2,3,4\})),
 \end{align*}
 gilt
 \[
\chi(J)=3-3+2=2=4-6+4=\chi(T).
 \]
 Aber wegen $\vert J_i\vert \neq \vert T_i \vert $ für $i=0,1,2$ sind  $J$ und $T$ nicht isomorph zueinander.
 %----bild-------
 \begin{figure}[H]
 \definecolor{ffffqq}{rgb}{1.,1.,0.}
\definecolor{qqqqff}{rgb}{0.,0.,1.}
\begin{tikzpicture}[line cap=round,line join=round,>=triangle 45,x=1.0cm,y=1.0cm]
%\begin{axis}[
x=1.0cm,y=1.0cm,
axis lines=middle,
ymajorgrids=true,
xmajorgrids=true,
xmin=-8.620000000000001,
xmax=14.38,
ymin=-3.72,
ymax=4.32,
xtick={-8.0,-7.0,...,14.0},
ytick={-5.0,-4.0,...,5.0},]
\clip(-7.62,-3.72) rectangle (14.38,4.32);
\fill[line width=2.pt,color=ffffqq,fill=ffffqq,fill opacity=\gelb] (-2.,0.) -- (2.,0.) -- (0.,3.4641016151377553) -- cycle;
\fill[line width=2.pt,color=ffffqq,fill=ffffqq,fill opacity=\gelb] (2.,0.) -- (-2.,0.) -- (0.,-3.4641016151377553) -- cycle;
\fill[line width=2.pt,color=ffffqq,fill=ffffqq,fill opacity=\gelb] (0.,3.4641016151377553) -- (2.,0.) -- (4.,3.464101615137754) -- cycle;
\fill[line width=2.pt,color=ffffqq,fill=ffffqq,fill opacity=\gelb] (-2.,0.) -- (0.,3.4641016151377553) -- (-4.,3.464101615137757) -- cycle;
\draw [line width=2.pt] (-2.,0.)-- (2.,0.);
\draw [line width=2.pt] (2.,0.)-- (0.,3.4641016151377553);
\draw [line width=2.pt] (0.,3.4641016151377553)-- (-2.,0.);
\draw [line width=2.pt] (2.,0.)-- (-2.,0.);
\draw [line width=2.pt] (-2.,0.)-- (0.,-3.4641016151377553);
\draw [line width=2.pt] (0.,-3.4641016151377553)-- (2.,0.);
\draw [line width=2.pt] (0.,3.4641016151377553)-- (2.,0.);
\draw [line width=2.pt] (2.,0.)-- (4.,3.464101615137754);
\draw [line width=2.pt] (4.,3.464101615137754)-- (0.,3.4641016151377553);
\draw [line width=2.pt] (-2.,0.)-- (0.,3.4641016151377553);
\draw [line width=2.pt] (0.,3.4641016151377553)-- (-4.,3.464101615137757);
\draw [line width=2.pt] (-4.,3.464101615137757)-- (-2.,0.);
\begin{scriptsize}
\draw [fill=qqqqff] (-2.,0.) circle (2.5pt);
\draw[color=black] (-2.32,-0.09) node {$V_1$};
\draw [fill=qqqqff] (2.,0.) circle (2.5pt);
\draw[color=black] (2.34,0.) node {$V_2$};
\draw[color=black] (0.,1.33) node {$F_1$};
\draw[color=black] (0.06,-0.25) node {$e_3$};
\draw[color=black] (1.32,2.07) node {$e_1$};
\draw[color=black] (-1.22,2.07) node {$e_2$};
\draw [fill=qqqqff] (0.,3.4641016151377553) circle (2.5pt);
\draw[color=black] (0.14,3.83) node {$V_3$};
\draw[color=black] (0.,-1.33) node {$F_3$};
\draw[color=black] (-1.27,-1.71) node {$e_5$};
\draw[color=black] (1.37,-1.71) node {$e_6$};
\draw [fill=qqqqff] (0.,-3.4641016151377553) circle (2.5pt);
\draw[color=black] (0.49,-3.29) node {$V_4$};
\draw [fill=qqqqff] (-2.,0.) circle (2.5pt);
%\draw[color=qqqqff] (-1.86,0.37) node {$E$};
\draw[color=black] (2.,2.19) node {$F_4$};
\draw[color=black] (3.37,1.75) node {$e_6$};
\draw[color=black] (2.11,3.82) node {$e_4$};
\draw [fill=qqqqff] (4.,3.464101615137754) circle (2.5pt);
\draw[color=black] (4.14,3.83) node {$V_4$};
\draw[color=black] (-2.,2.19) node {$F_2$};
\draw[color=black] (-1.94,3.82) node {$e_4$};
\draw[color=black] (-3.3,1.75) node {$e_5$};
\draw [fill=qqqqff] (-4.,3.464101615137757) circle (2.5pt);
\draw[color=black] (-3.86,3.83) node {$V_4$};
\end{scriptsize}
%\end{axis}
\end{tikzpicture}
 \caption{Tetraeder}
 \label{tetra}
 \end{figure}
 %-------------------------------
\end{bemerkung}

Wenn man die Euler-Charakteristik einer simplizialen Fläche $(X,<)$ unter Anwendung der Mender- und Cutteroperatoren beobachtet, so ergibt sich  
\[
\chi(C^c_{\{e,f\}}(X))=\chi(X)-1,
\]
für $\{e,f\} \in X_1$, falls $\{e,f\} \in I^2(X)$ ist. Denn für die Knoten, Kanten und Flächen der simplizialen Fläche $C^c_{\{e,f\}}(X)$ gilt $\vert C^c_{\{e,f\}}(X)_0 \vert =\vert X_0\vert $, $\vert C^c_{\{e,f\}}(X)_1 \vert =\vert X_1\vert +1$ und $\vert C^c_{\{e,f\}}(X)_2 \vert =\vert X_2\vert $. Somit erhält man 
\begin{align*}
\chi(C^c_{\{e,f\}}(X))&=\vert C^c_{\{e,f\}}(X)_0 \vert-\vert C^c_{\{e,f\}}(X)_1 \vert+\vert C^c_{\{e,f\}}(X)_2 \vert\\
&=\vert X_0\vert-(\vert X_1\vert+1)+\vert X_2\vert\\
&=\vert X_0\vert-\vert X_1\vert+\vert X_2\vert -1\\
&=\chi(X)-1.
\end{align*}
Durch analoge Herangehensweise bei den anderen Cutter- und Menderoperatoren ergibt sich Folgendes:
\begin{itemize}
\item Falls $\{e,f\} \in BM^2(X)$, dann ist 
\[
\chi(C^m_{e,f}(X))=\chi(X)+1.
\]
%----
\item Falls $\{e,f\} \in I^1(X)$, dann ist 
\[
\chi(R^c_{\{e,f\}}(X))=\chi(X).
\]
\item Falls $\{e,f\} \in BM^1(X)$, dann ist 
\[
\chi(R^m_{e,f}(X))=\chi(X).
\]
%----
\item Falls $\{e,f\} \in I^0(X)$, dann ist 
\[
\chi(S^c_{\{e,f\}}(X))=\chi(X)+1.
\]
\item Und falls $\{e,f\} \in BM^0(X)$ mendable bezüglich $V_e,V_f \in X_0$ mit $V_e<e$ und $V_f<f$ ist, dann ist 
\[
\chi(S^m_{(V_e,e),(V_f,f)}(X))=\chi(X)-1.
\]
\end{itemize}
Von besonderem Interesse sind in dieser Arbeit die simplizialen Flächen, die geschlossen sind und deren Euler-Charakteristik 2 ist,  denn für diese simplizialen Flächen kann man unter gewissen Voraussetzungen die Transitivität der Operation Wanderinghole, die im nächsten Kapitel definiert wird, nachweisen.
\newpage
%-------Überarbeitung------
\section{Das wandernde Loch}\label{kwh}
\begin{bemerkung}\label{ident}
Sei $(X,<)$ eine simpliziale Fläche.
Für $x\in X$ und $m \in \mathbb{N}$ schreibt man $x=\{x_1,\ldots,x_m\}$ und dies wird wie folgt interpretiert:\\ Man identifiziert $X$ mit der isomorphen simplizialen Fläche $Y \in \mathcal{M}(n \Delta)$, wobei $n\in \mathbb{N}$ die Anzahl der Flächen in $X$ ist. Es gilt $x_1,\ldots,x_m \in n\Delta$ und $\{x_1,\ldots,x_m\}\in Y$. Außerdem gilt $\beta(\{x_1,\ldots,x_m\})=x$ für einen  Isomorphismus $\beta: Y \mapsto X$. \\
Dabei soll hier weniger im Vordergrund stehen, wie man diese simpliziale Fläche konstruiert oder wie die Urbilder der anderen Elemente der simplizialen Fläche $X$ aussehen. Diese Notation soll lediglich zur Erleichterung der Anwendung der Mender- und Cutteroperatoren auf $X$ dienen. 
\end{bemerkung}
Zum weiteren Verständnis der oben eingeführten Notation führt man folgendes Beispiel an.
\begin{bsp}
Für die simpliziale Fläche $(X,<)$, definiert durch das ordinale Symbol 
\[
\omega((X,<))=(4,5,2;(\{2,3\},\{1,2\},\{1,3\},\{2,4\},\{3,4\}),(\{1,2,3\},\{1,4,5\}),
\]
schreibt man $V_1=\{V_1^1,V_1^2\}$, $e_1=\{e_1^1,e_1^2\}$ für $V_1\in X_0,e_1 \in X_1$ und $V_1^1,V_1^2 \in 2 \Delta_0$, $e_1^1,e_1^2 \in 2 \Delta_1$ und meint damit den Knoten und die Kante der simplizialen Fläche $2\Delta(\alpha)$, wobei die Mending Map $\alpha:2\Delta \mapsto X$ gegeben ist durch
\[
\alpha(x)= 
\begin{cases}
e_i & \text{für } x =e_i^j ,i=1,\,j=1,2\\
e_{i+j} &\text{für }x=e_i^j ,i=2,3 \,j=2\\
e_{5-i} &\text{für }x=e_i^j ,i=2,3 \,j=1\\
V_i &\text{für } x =V_i^j,i=2,3,\,j=1,2\\
V_1 &\text{für }x=V_1^1\\
V_4& \text{für }x=V_1^2\\
F_j & \text{ für } x=F_j, j=1,2
\end{cases}
\]
\begin{figure}[H]
\begin{center}
\definecolor{qqqqff}{rgb}{0.,0.,1.}
\definecolor{ffffqq}{rgb}{1.,1.,0.}
\definecolor{ududff}{rgb}{0.30196078431372547,0.30196078431372547,1.}
\definecolor{xdxdff}{rgb}{0.49019607843137253,0.49019607843137253,1.}
\begin{tikzpicture}[line cap=round,line join=round,>=triangle 45,x=1.5cm,y=1.5cm]
%\begin{axis}[
x=1.0cm,y=1.0cm,
axis lines=middle,
ymajorgrids=true,
xmajorgrids=true,
xmin=-1.5211726268822376,
xmax=6.403358908477344,
ymin=-1.538156922171471,
ymax=4.6112795492675716,
xtick={-1.0,0.0,...,6.0},
ytick={-1.0,0.0,...,4.0},]
\clip(-0.5211726268822376,-0.338156922171471) rectangle (4.403358908477344,2.526112795492675716);
\fill[line width=2.pt,color=ffffqq,fill=ffffqq,fill opacity=\gelb] (2.,0.) -- (2.,2.) -- (0.2679491924311226,1.) -- cycle;
\fill[line width=2.pt,color=ffffqq,fill=ffffqq,fill opacity=\gelb] (2.,2.) -- (2.,0.) -- (3.7320508075688776,1.) -- cycle;
\draw [line width=2.pt] (2.,0.)-- (2.,2.);
\draw [line width=2.pt] (2.,2.)-- (0.2679491924311226,1.);
\draw [line width=2.pt] (0.2679491924311226,1.)-- (2.,0.);
%\draw [line width=2.pt,color=ffffqq] (2.,2.)-- (2.,0.);
\draw [line width=2.pt] (2.,0.)-- (3.7320508075688776,1.);
\draw [line width=2.pt] (3.7320508075688776,1.)-- (2.,2.);
\begin{scriptsize}
\draw [fill=blue] (2.,0.) circle (2.5pt);
\draw[color=black] (2.0712816691474387,-0.19995699458406646) node {$\{V_3^1,V_3^2\}$};
\draw [fill=blue] (2.,2.) circle (2.5pt);
\draw[color=black] (2.0712816691474387,2.2969389414946834) node {$\{V_2^1,V_2^2\}$};
\draw[color=black] (1.30697720713558868,0.980705685914867) node {$F_1$};
\draw[color=black] (1.60697720713558868,1.380705685914867) node {$\{e_1^1,e_1^2\}$};
\draw[color=black] (.959771842857143,1.70690142385852654) node {$\{e_3^1\}$};
\draw[color=black] (0.801414699689985814,0.40799571088334387) node {$\{e_2^1\}$};
\draw [fill=blue] (0.2679491924311226,1.) circle (2.5pt);
\draw[color=black] (0.033845077341547725,1.293164947015802) node {$\{V_1^1\}$};
\draw[color=black] (2.8214706544948127,0.980705685914867) node {$F_2$};
\draw[color=black] (3.164867021027061,0.40799571088334387) node {$\{e_3^2\}$};
\draw[color=black] (3.164867021027061,1.70690142385852654) node {$\{e_2^2\}$};
\draw [fill=blue] (3.7320508075688776,1.) circle (2.5pt);
\draw[color=black] (3.8041125648794005,1.293164947015802) node {$\{V_1^2\}$};
\end{scriptsize}
%\end{axis}
\end{tikzpicture}
\caption{isomorphe simpliziale Fläche in $n\Delta$}
\end{center}
\end{figure}

\end{bsp}
Nun definiert man die in der Einleitung schon angesprochene Operation Wanderinghole, um darauffolgend die Eigenschaften dieser Operation zu untersuchen.\\
 Seien dazu $(X,<)$ eine geschlossene simpliziale Fläche, $F \in X_{2}$ eine Fläche, $e_{i} \in X_{1}$ Kanten und $V_{j} \in X_{0}$ Knoten in $X$ für $i,j \in \{1,2,3\}$ mit folgenden Eigenschaften:
 \begin{itemize}
 \item $\vert X_{2}\vert \geq 4$,
 \item $e_{i} < F$ für alle $i \in \{1,2,3\}$,
 \item $V_{i}<e_{j}$ für alle $i \in \{1,2,3\}$ und $j \in \{1,2,3\} \setminus\{i\}$ ,
 \item $V_{i} < F$ für alle $i \in \{1,2,3\}$.
\end{itemize}  
Zudem seien $f,g \in X_1,V_4 \in X_0,F' \in X_2$ so, dass
\begin{itemize}
\item $e_3<F'$,
\item $f,g <F'$, 
\item $V_1,V_4<f$ und $V_2,V_4<g$ gilt,
\end{itemize} wobei die bis jetzt genannten Knoten, Kanten und Flächen paarweise verschieden sind. 
%--------------bild----------------------
%---------------------------------------------
\begin{figure}[H]
\definecolor{uuuuuu}{rgb}{0.26666666666666666,0.26666666666666666,0.26666666666666666}
\definecolor{ududff}{rgb}{0.30196078431372547,0.30196078431372547,1.}
\definecolor{ffffqq}{rgb}{1.,1.,0.}
\begin{tikzpicture}[line cap=round,line join=round,>=triangle 45,x=1.5cm,y=1.5cm]
%\begin{axis}[
x=1.5cm,y=1.5cm,
axis lines=middle,
ymajorgrids=true,
xmajorgrids=true,
xmin=-4.3,
xmax=7.0600000000000005,
ymin=-2.46,
ymax=6.3,
xtick={-4.0,-3.0,...,7.0},
ytick={-2.0,-1.0,...,6.0},]
\clip(-5.,-0.45) rectangle (3.06,4.3);
\fill[line width=2.pt,color=ffffqq,fill=ffffqq,fill opacity=\gelb] (-2.,0.) -- (2.,0.) -- (2.,4.) -- (-2.,4.) -- cycle;
%\fill[line width=2.pt,color=ffffqq,fill=ffffqq,fill opacity=\gelb] (-1.,2.) -- (1.,2.) -- (0.,3.7320508075688776) -- cycle;
%\fill[line width=2.pt,color=ffffqq,fill=ffffqq,fill opacity=\gelb] (1.,2.) -- (-1.,2.) -- (0.,0.2679491924311226) -- cycle;
\draw [line width=2.pt] (-1.,2.)-- (1.,2.);
\draw [line width=2.pt] (1.,2.)-- (0.,3.7320508075688776);
\draw [line width=2.pt] (0.,3.7320508075688776)-- (-1.,2.);
\draw [line width=2.pt] (1.,2.)-- (-1.,2.);
\draw [line width=2.pt] (-1.,2.)-- (0.,0.2679491924311226);
\draw [line width=2.pt] (0.,0.2679491924311226)-- (1.,2.);
\begin{scriptsize}
\draw [fill=ududff] (-1.,2.) circle (2.5pt);
\draw[color=black] (-1.24,2.07) node {$V_1$};
\draw [fill=ududff] (1.,2.) circle (2.5pt);
\draw[color=black] (1.24,2.07) node {$V_2$};
\draw[color=black] (0.,2.75) node {$F$};
\draw[color=black] (0.06,1.85) node {$e_3$};
\draw[color=black] (0.67,2.96) node {$e_1$};
\draw[color=black] (-0.67,2.96) node {$e_2$};
\draw [fill=ududff] (0.,3.7320508075688776) circle (2.5pt);
\draw[color=black] (0.,3.91) node {$V_3$};
\draw[color=black] (0.,1.31) node {$F'$};
\draw[color=black] (-0.7,1.16) node {$f$};
\draw[color=black] (0.7,1.16) node {$g$};
\draw [fill=ududff] (0.,0.2679491924311226) circle (2.5pt);
\draw[color=black] (0.,0.1) node {$V_4$};
\end{scriptsize}
%\end{axis}
\end{tikzpicture}
\caption{Ausschnitt einer simplizialen Fläche}
\end{figure}
%-----------------------------------------
%$\textcolor{red}{newstuff}$
Für die folgende Konstruktion nutzt man \Cref{ident} und identifiziert $X$ mit der zu $X$ isomorphen simplizialen Fläche $Y \in \mathcal(n\Delta) $, wobei $n=\vert X_2 \vert$ ist. Die simpliziale Fläche $Y$ wird durch die folgenden Identitäten festgesetzt:
\begin{itemize}
\item $e_i=\{e_i^1,e_i^2\}$ für $i=1,2,3$
\item $V_j=\{V_j^1,V_j^2\}$ für $j=2,3$
\item $V_1=\{V_1^1,V_1^2,V_1^3\}$
\item $f=\{f^1,f^2\}$
\item $g=\{g\}$
\item $V_4=\{V_4\}$
\end{itemize}
%--------------------------------------------------
%----------------------------bild------------------
\begin{figure}[H]
\definecolor{uuuuuu}{rgb}{0.26666666666666666,0.26666666666666666,0.26666666666666666}
\definecolor{ududff}{rgb}{0.30196078431372547,0.30196078431372547,1.}
\definecolor{ffffqq}{rgb}{1.,1.,0.}
\definecolor{qqqqff}{rgb}{0.,0.,1.}
\begin{tikzpicture}[line cap=round,line join=round,>=triangle 45,x=1.5cm,y=1.5cm]
%\begin{axis}[
x=1.5cm,y=1.5cm,
axis lines=middle,
ymajorgrids=true,
xmajorgrids=true,
xmin=-4.3,
xmax=7.0600000000000005,
ymin=-2.46,
ymax=6.3,
xtick={-4.0,-3.0,...,7.0},
ytick={-2.0,-1.0,...,6.0},]
\clip(-5.3,-0.46) rectangle (3.06,4.3);
\fill[line width=2.pt,color=ffffqq,fill=ffffqq,fill opacity=\gelb] (-2.2,-0.1) -- (2.,-0.1) -- (2.,4.1) -- (-2.2,4.1) -- cycle;
%\fill[line width=2.pt,color=ffffqq,fill=ffffqq,fill opacity=\gelb] (-1.,2.) -- (1.,2.) -- (0.,3.7320508075688776) -- cycle;
%\fill[line width=2.pt,color=ffffqq,fill=ffffqq,fill opacity=\gelb] (1.,2.) -- (-1.,2.) -- (0.,0.2679491924311226) -- cycle;
\draw [line width=2.pt] (-1.,2.)-- (1.,2.);
\draw [line width=2.pt] (1.,2.)-- (0.,3.7320508075688776);
\draw [line width=2.pt] (0.,3.7320508075688776)-- (-1.,2.);
\draw [line width=2.pt] (1.,2.)-- (-1.,2.);
\draw [line width=2.pt] (-1.,2.)-- (0.,0.2679491924311226);
\draw [line width=2.pt] (0.,0.2679491924311226)-- (1.,2.);
\begin{scriptsize}
\draw [fill=qqqqff] (-1.,2.) circle (2.5pt);
\draw[color=black] (-1.6,2.2) node {$\{V_1^1,V_1^2,V_1^3\}$};
\draw [fill=qqqqff] (1.,2.) circle (2.5pt);
\draw[color=black] (1.54,2.07) node {$\{V_2^1,V_2^2\}$};
\draw[color=black] (0.,2.75) node {$F$};
\draw[color=black] (0.06,1.8) node {$\{e_3^1,e_3^2\}$};
\draw[color=black] (0.87,2.96) node {$\{e_1^1,e_1^2\}$};
\draw[color=black] (-0.87,2.96) node {$\{e_2^1,e_2^2\}$};
\draw [fill=qqqqff] (0.,3.7320508075688776) circle (2.5pt);
\draw[color=black] (0.,3.91) node {$\{V_3^1,V_3^2\}$};
\draw[color=black] (0.,1.31) node {$F'$};
\draw[color=black] (-1.,1.16) node {$\{f^1,f^2\}$};
\draw[color=black] (0.8,1.16) node {$\{g\}$};
\draw [fill=qqqqff] (0.,0.2679491924311226) circle (2.5pt);
\draw[color=black] (0.,0.05) node {$\{V_4\}$};
\end{scriptsize}
%\end{axis}
\end{tikzpicture}
\caption{Ausschnitt eines Mendings einer simplizialen Fläche}\label{abb16}
\end{figure}
%--------------------------------------------------
\begin{bemerkung}
 Durch die Anwendung von Mender- und Cutter-Operationen auf eine simpliziale Fläche $(X,<)$ entsteht eine simpliziale Fläche mit veränderten Knoten, Kanten, Flächen und einer veränderten Inzidenzrelation. Diese sollen in den drei folgenden Prozeduren in Tabellen festgehalten werden, wobei man nur die für die Konstruktion relevanten Knoten, Kanten und Flächen aufführt.  Wegen der Transitivität der Relation reicht es, die Mengen $X_0(e)$ und $X_2(e)$ von den veränderten Kanten $e\in X_1$ aufzustellen.
Beispielsweise stellt man die Inzidenzen der simplizialen Fläche in der Abbildung \ref{abb16} durch folgende Tabellen dar. Es soll zum Beispiel angedeutet werden, dass $\{\{V_2^1,V_2^2\},\{V_3^1,V_3^2\}\}\subset X_0(\{e_1^1,e_1^2\})$ und damit auch $\{\{V_2^1,V_2^2\},\{V_3^1,V_3^2\}\}\subset X_0(F)$ ist.
\begin{figure}[H]
\begin{center}
\begin{tabularx}{\textwidth}{XXX}
\hline
\textbf{e}&\textbf{$X_0(e)$}&\textbf{$X_2(e)$}\\
 \hline
 $\{e_1^1,e_1^2\}$ & $\{V_2^1,V_2^2\},\{V_3^1,V_3^2\}$& $F$\\
  $\{e_2^1,e_2^2\}$&$\{V_1^1,V_1^2,V_1^3\},\{V_3^1,V_3^2\}$ & $F$\\
  $\{e_3^1,e_3^2\}$&$\{V_1^1,V_1^2,V_1^3\},\{V_2^1,V_2^2\}$ & $F,F'$\\ 
   $\{f^1,f^2\}$&$\{V_1^1,V_1^2,V_1^3\},\{V_4\}$& $F'$\\
   $\{g\}$ & $\{V_4\},\{V_2^1,V_2^2\}$ & $F'$ \\

   
 \end{tabularx}
\end{center}
%\caption{Inzidenz nach dem Cratercut}
\end{figure}

\end{bemerkung}
%\newpage
Ziel ist es, wie schon erwähnt, durch die Anwendung der Mender- und Cutteroperatoren aus $(X,<)$ eine simpliziale Fläche $X^H_{(F,f)}$ zu konstruieren. 
Hierzu definiert man die folgenden drei Prozeduren:
%\begin{enumerate} 
%\newpage
\subsection{Prozedur $P^1$}
 Zunächst soll ein sogenanntes \emph{Loch an der Stelle $F$} erzeugt werden, welches entsteht, wenn man $F$ von der simplizialen Fläche trennt:
\begin{enumerate}[(i)]
\item Wende den $Cratercut$ $C^{c}_{\{e_{1}^1,e_{1}^2\}}$ an, um aus der Kante $\{e_{1}^1,e_{1}^2\}$ die Kanten $\{e_1^1\}$ und $\{e_1^2\}$ zu erhalten, wobei
die Inzidenzen durch folgende Tabelle gegeben sind:
\begin{figure}[H]
\begin{center}
\begin{tabularx}{\textwidth}{XXX}
%\caption{rfj}
\hline
\textbf{e}&\textbf{$X_0(e)$}&\textbf{$X_2(e)$}\\
 \hline
 $\{e_1^1\}$ & $\{V_2^1,V_2^2\},\{V_3^1,V_3^2\}$& $F$\\
  $\{e_1^2\}$ & $\{V_2^1,V_2^2\},\{V_3^1,V_3^2\}$&\\ 
  $\{e_2^1,e_2^2\}$&$\{V_1^1,V_1^2,V_1^3\},\{V_3^1,V_3^2\}$ & $F$\\
  $\{e_3^1,e_3^2\}$&$\{V_1^1,V_1^2,V_1^3\},\{V_2^1,V_2^2\}$ & $F,F'$\\ 
   $\{f^1,f^2\}$&$\{V_1^1,V_1^2,V_1^3\},\{V_4\}$& $F'$\\
   $\{g\}$ & $\{V_4\},\{V_2^1,V_2^2\}$ & $F'$ \\

   
 \end{tabularx}
\end{center}
%\caption{Inzidenz nach dem Cratercut}
\end{figure}
%---------------------bild---------------------------
\begin{figure}[H]
\definecolor{ffffff}{rgb}{1.,1.,1.}
\definecolor{qqqqff}{rgb}{0.,0.,1.}
\definecolor{ududff}{rgb}{0.30196078431372547,0.30196078431372547,1.}
\definecolor{ffffqq}{rgb}{1.,1.,0.}
\begin{tikzpicture}[line cap=round,line join=round,>=triangle 45,x=1.4cm,y=1.38cm]
%\begin{axis}[
x=1.0cm,y=1.0cm,
axis lines=middle,
ymajorgrids=true,
xmajorgrids=true,
xmin=-4.3,
xmax=18.7,
ymin=-5.34,
ymax=6.3,
xtick={-4.0,-3.0,...,18.0},
ytick={-5.0,-4.0,...,6.0},]
\clip(-5.5,-0.) rectangle (3.7,4.3);
\fill[line width=2.pt,color=ffffqq,fill=ffffqq,fill opacity=\gelb] (-2.2,0.) -- (2.,0.) -- (2.,4.2) -- (-2.2,4.2) -- cycle;
\fill[line width=2.pt,color=ffffqq,fill=ffffqq,fill opacity=\gelb] (-1.,2.) -- (1.,2.) -- (0.,3.7320508075688776) -- cycle;
\fill[line width=2.pt,color=ffffqq,fill=ffffqq,fill opacity=\gelb] (1.,2.) -- (-1.,2.) -- (0.,0.2679491924311226) -- cycle;
\draw [line width=2.pt] (0.,3.7320508075688776)-- (-1.,2.);
\draw [line width=2.pt] (1.,2.)-- (-1.,2.);
\draw [line width=2.pt] (-1.,2.)-- (0.,0.2679491924311226);
\draw [line width=2.pt] (0.,0.2679491924311226)-- (1.,2.);
\draw [rotate around={-60.:(0.5,2.8660254037844513)},line width=2.pt,color=ffffff,fill=ffffff,fill opacity=1.0] (0.5,2.8660254037844513) ellipse (1.4633824013732526cm and 0.1641459454658895cm);
\draw [rotate around={-60.:(0.5,2.866025403784439)},line width=2.pt] (0.5,2.866025403784439) ellipse (1.463382401373216cm and 0.16414594546590086cm);
\begin{scriptsize}
%\draw[color=black] (0.48,2.17) node {$Vieleck1$};
\draw [fill=qqqqff] (-1.,2.) circle (2.5pt);
\draw[color=black] (-1.56,2.27) node {$\{V_1^1,V_1^2,V_1^3\}$};
\draw [fill=qqqqff] (1.,2.) circle (2.5pt);
\draw[color=black] (1.49,2.12) node {$\{V_2^1,V_2^2\}$};
\draw[color=black] (-0.1,2.37) node {$F$};
\draw[color=black] (-0.,1.8) node {$\{e_3^1,e_3^2\}$};
\draw [fill=qqqqff] (0.,3.7320508075688776) circle (2.5pt);
\draw[color=black] (0.14,4.01) node {$\{V_3^1,V_3^2\}$};
\draw[color=black] (0.06,1.47) node {$F'$};
\draw [fill=qqqqff] (0.,0.2679491924311226) circle (2.5pt);
\draw[color=black] (0.76,1.13) node {$\{g\}$};
\draw[color=black] (-0.94,1.13) node {$\{f^1,f^2\}$};
\draw[color=black] (0.36,0.13) node {$\{V_4\}$};
\draw[color=black] (0.21,2.67) node {$\{e_1^1\}$};
\draw[color=black] (0.84,3.13) node {$\{e_1^2\}$};
\draw[color=black] (-0.86,2.97) node {$\{e_2^1,e_2^2\}$};
\end{scriptsize}
%\end{axis}
\end{tikzpicture}
\caption{simpliziale Fläche nach einem Cratercut}
\end{figure}

%-------------------------------------------------------

\item Wende den $Rip Cut$ $R^{c}_{\{e^1_{2},e^2_{2}\}}$ an, um aus dem Knoten $\{V^1_{3},V^2_{3}\}$ die Knoten $\{V^1_{3}\}$ und $\{V^2_{3}\}$ und aus der Kante $\{e^1_{2},e^2_{2}\}$ die Kanten $\{e^1_{2}\}$ und $\{e^2_{2}\}$ mit den folgenden Inzidenzen zu erhalten. 
 \begin{figure}[H]
\begin{center}
\begin{tabularx}{\textwidth}{XXX}
\hline
\textbf{e}&\textbf{$X_0(e)$}&\textbf{$X_2(e)$}\\
 \hline
 $\{e_1^1\}$ & $\{V_2^1,V_2^2\},\{V_3^1\}$& $F$\\
  $\{e_1^2\}$ & $\{V_2^1,V_2^2\},\{V_3^2\}$&\\ 
  $\{e_2^1\}$&$\{V_1^1,V_1^2,V_1^3\},\{V_3^1\}$ & $F$\\
   $\{e_2^2\}$&$\{V_1^1,V_1^2,V_1^3\},\{V_3^2\}$ & \\
  $\{e_3^1,e_3^2\}$&$\{V_1^1,V_1^2,V_1^3\},\{V_2^1,V_2^2\}$ & $F,F'$\\ 
   $\{f^1,f^2\}$&$\{V_1^1,V_1^2,V_1^3\},\{V_4\}$& $F'$\\
   $\{g\}$ & $\{V_2^1,V_2^2\},\{V_4\}$ & $F'$ \\
 \end{tabularx}
\end{center} 
%\caption{Inzidenzen nach dem Ripcut}\label{tabrc} 
\end{figure}

 %------------------------------bild-------------
  \begin{figure}[H]
\definecolor{xdxdff}{rgb}{0.49019607843137253,0.49019607843137253,1.}
\definecolor{ffffff}{rgb}{1.,1.,1.}
\definecolor{qqqqff}{rgb}{0.,0.,1.}
\definecolor{ffffqq}{rgb}{1.,1.,0.}
\begin{tikzpicture}[line cap=round,line join=round,>=triangle 45,x=1.cm,y=1.cm]

x=1.cm,y=1.cm,
axis lines=middle,
xmin=-4.0,
xmax=14.0,
ymin=-3.3,
ymax=5.34,
xtick={-9.0,-8.0,...,14.0},
ytick={-5.0,-4.0,...,6.0},]
\clip(-8.,-3.0) rectangle (4.,5.3);
\fill[line width=2.pt,color=ffffqq,fill=ffffqq,fill opacity=\gelb] (4.,-3.) -- (4.,5.) -- (-4.,5.) -- (-4.,-3.) -- cycle;   
\fill[line width=2.pt,color=ffffff,fill=ffffff,fill opacity=1.0] (-2.,1.) -- (2.,1.) -- (0.,4.464101615137755) -- cycle;
\fill[line width=2.pt,color=ffffqq,fill=ffffqq,fill opacity=0.1] (-2.,1.) -- (0.,-2.44) -- (1.9791273890184695,1.012050807568877) -- cycle;
\fill[line width=2.pt,color=ffffqq,fill=ffffqq,fill opacity=0.550000011920929] (-2.,1.) -- (0.,3.48) -- (2.,1.) -- cycle;
\draw [line width=2.pt] (-2.,1.)-- (2.,1.);
\draw [line width=2.pt] (2.,1.)-- (0.,4.464101615137755);
\draw [line width=2.pt] (0.,4.464101615137755)-- (-2.,1.);
\draw [line width=2.pt] (-2.,1.)-- (0.,-2.44);
\draw [line width=2.pt] (0.,-2.44)-- (1.9791273890184695,1.012050807568877);
\draw [line width=2.pt] (1.9791273890184695,1.012050807568877)-- (-2.,1.);
\draw [line width=2.pt] (-2.,1.)-- (0.,3.48);
\draw [line width=2.pt] (0.,3.48)-- (2.,1.);
\draw [line width=2.pt] (2.,1.)-- (-2.,1.);
\begin{scriptsize}
\draw[color=black] (0.12,0.617) node {$\{e_3^1,e_3^2\}$};
\draw [fill=qqqqff] (-2.,1.) circle (2.5pt);
\draw[color=black] (-3.01,1.22) node {$\{V_1^1,V_1^2,V_1^3\}$};
\draw [fill=qqqqff] (2.,1.) circle (2.5pt);
\draw[color=black] (2.59,1.42) node {$\{V_2^1,V_2^2\}$};
%\draw[color=ffffff] (0.48,2.33) node {$Vieleck1$};
%\draw[color=black] (0.07,1.5) node {$e_3$};
\draw[color=black] (1.37,3.12) node {$\{e_1^2\}$};
\draw[color=black] (-1.27,3.12) node {$\{e_2^2\}$};
\draw [fill=qqqqff] (0.,4.464101615137755) circle (2.5pt);
\draw[color=black] (0.42,4.8) node {$\{V_3^2\}$};
\draw [fill=qqqqff] (0.,-2.44) circle (2.5pt);
\draw[color=black] (0.09,-2.84) node {$\{V_4\}$};
\draw[color=black] (0.1,-0.13) node {F'};
\draw[color=black] (-1.74,-0.71) node {$\{f^1,f^2\}$};
\draw[color=black] (1.42,-0.69) node {$\{g\}$};
\draw [fill=qqqqff] (0.,3.48) circle (2.5pt);
\draw[color=black] (0.,2.8) node {$\{V_3^1\}$};
\draw[color=black] (0.06,2.01) node {F};
\draw[color=black] (-0.95,1.76) node {$\{e_2^1\}$};
\draw[color=black] (0.85,1.76) node {$\{e_1^1\}$};
\end{scriptsize}

\end{tikzpicture}
\caption{simpliziale Fläche nach einem Ripcut}
\end{figure}

%----------------------------------------------

 \item Wende den $Split Cut$ $ S^{c}_{\{e^1_{3},e^2_{3}\}}$ an, um aus der Kante $\{e^1_{3},e^2_{3}\}$ die Kanten $\{e^1_{3}\}$ und $\{e^2_{3}\}$, aus dem Knoten $\{V_1^1,V_1^2,V_1^3\}$ die Knoten $\{V_1^1\}$ und $\{V_1^2,V_1^3\}$ und aus dem Knoten $\{V_2^1,V_2^2\}$ die Knoten $\{V_2^1\}$ und $\{V_2^2\}$ zu erhalten, wobei für die veränderten Knoten, Kanten und Flächen Folgendes gilt
 \begin{figure}[H]
 \begin{center}
\begin{tabularx}{\textwidth}{XXX}
\hline
\textbf{e}&\textbf{$X_0(e)$}&\textbf{$X_2(e)$}\\
 \hline
 $\{e_1^1\}$ & $\{V_2^1\},\{V_3^1\}$& $F$\\
 
  $\{e_1^2\}$ & $\{V_2^2\},\{V_3^2\}$&\\ 
  
  $\{e_2^1\}$&$\{V_1^1\},\{V_3^1\}$ & $F$\\
  
   $\{e_2^2\}$&$\{V_1^2,V_1^3\},\{V_3^2\}$ & \\
  
  $\{e_3^1\}$&$\{V_1^1\},\{V_2^1\}$ & $F$\\  
  $\{e_3^2\}$&$\{V_1^2,V_1^3\},\{V_2^2\}$ & $F'$\\  
   $\{f^1,f^2\}$&$\{V_1^2,V_1^3\},\{V_4\}$& $F'$\\
   
   $\{g\}$ & $\{V_2^2\},\{V_4\}$ & $F'$ \\
   
 \end{tabularx}
\end{center} 
%\caption{Inzidenzen nach Splitcut} 
\end{figure}

%------------------------bild------------------------
\begin{figure}[H]
\definecolor{qqqqff}{rgb}{0.,0.,1.}
\definecolor{ffffff}{rgb}{1.,1.,1.}
\definecolor{ududff}{rgb}{0.30196078431372547,0.30196078431372547,1.}
\definecolor{ffffqq}{rgb}{1.,1.,0.}
\begin{tikzpicture}[line cap=round,line join=round,>=triangle 45,x=1.cm,y=1.cm]
x=1.0cm,y=1.0cm,
axis lines=middle,
ymajorgrids=true,
xmajorgrids=true,
xmin=-3.5,
xmax=10.0,
ymin=-3.0,
ymax=5.2,
xtick={-9.0,-8.0,...,14.0},
ytick={-5.0,-4.0,...,6.0},]
\clip(-4.,-3.3) rectangle (14.,5.34);
%----------
%\fill[line width=2.pt,color=ffffqq,fill=ffffqq,fill opacity=0.550000011920929] (-4.,-3.) -- (-4.,5.) -- (-3.,5.) -- (-3.,-3.) -- cycle;

%--------------
\fill[line width=2.pt,color=ffffqq,fill=ffffqq,fill opacity=\gelb] (-3.5,-3.) -- (-3.5,5.) -- (3.,5.) -- (3.,-3.) -- cycle;
\fill[line width=2.pt,color=ffffff,fill=ffffff,fill opacity=1.0] (-2.,1.) -- (2.,1.) -- (0.,4.464101615137755) -- cycle;
\fill[line width=2.pt,color=ffffqq,fill=ffffqq,fill opacity=0.20000000298023224] (-2.,1.) -- (0.,-2.44) -- (1.9791273890184695,1.012050807568877) -- cycle;
\fill[line width=2.pt,color=ffffqq,fill=ffffqq,fill opacity=\gelb] (5.,1.) -- (9.,1.) -- (7.,4.464101615137755) -- cycle;
%\draw [line width=2.pt,color=ffffqq] (-3.5,-3.)-- (-3.5,5.);
%\draw [line width=2.pt,color=ffffqq] (-3.,5.)-- (3.,5.);
%\draw [line width=2.pt,color=ffffqq] (3.,5.)-- (3.,-3.);
%\draw [line width=2.pt,color=ffffqq] (3.,-3.)-- (-3.,-3.);
\draw [line width=2.pt] (-2.,1.)-- (2.,1.);
\draw [line width=2.pt] (2.,1.)-- (0.,4.464101615137755);
\draw [line width=2.pt] (0.,4.464101615137755)-- (-2.,1.);
\draw [line width=2.pt] (-2.,1.)-- (0.,-2.44);
\draw [line width=2.pt] (0.,-2.44)-- (1.9791273890184695,1.012050807568877);
\draw [line width=2.pt] (1.9791273890184695,1.012050807568877)-- (-2.,1.);
\draw [line width=2.pt] (5.,1.)-- (9.,1.);
\draw [line width=2.pt] (9.,1.)-- (7.,4.464101615137755);
\draw [line width=2.pt] (7.,4.464101615137755)-- (5.,1.);
\begin{scriptsize}
\draw[color=black] (0,0.7) node {$\{e_3^2\}$};
%\draw[color=black] (7,1.4) node {$e_3^1$};

\draw[color=black] (1.28,2.93) node {$\{e_1^2\}$};
%\draw[color=black] (8.28,2.93) node {$e_1^2$};

\draw[color=black] (-1.28,2.93) node {$\{e_2^2\}$};
\draw[color=black] (5.7,2.93) node {$\{e_2^1\}$};
\draw[color=black] (8.3,2.93) node {$\{e_1^1\}$};

\draw[color=black] (-1.88,-0.6) node {$\{f^1,f^2\}$};
\draw[color=black] (1.68,-0.6) node {$\{g\}$};
\draw [fill=qqqqff] (-2.,1.) circle (2.5pt);
\draw[color=black] (-2.551,1.42) node {$\{V_1^2,V_1^3\}$};
\draw [fill=qqqqff] (2.,1.) circle (2.5pt);
\draw[color=black] (2.29,1.42) node {$\{V_2^2\}$};
\draw[color=ffffff] (0.48,2.33) node {$Vieleck1$};
\draw [fill=qqqqff] (0.,4.464101615137755) circle (2.5pt);
\draw[color=black] (0.43,4.7) node {$\{V_3^2\}$};
\draw [fill=qqqqff] (0.,-2.44) circle (2.5pt);
\draw[color=black] (0.14,-2.75) node {$\{V_4\}$};
\draw[color=black] (0.1,0.03) node {F'};
\draw [fill=qqqqff] (5.,1.) circle (2.5pt);
\draw[color=black] (5.08,0.63) node {$\{V_1^1\}$};
\draw [fill=qqqqff] (9.,1.) circle (2.5pt);
\draw[color=black] (9.1,0.67) node {$\{V_2^1\}$};
\draw[color=black] (7.06,2.33) node {$F$};
\draw[color=black] (7.06,0.70) node {$\{e_3^1\}$};
\draw [fill=qqqqff] (7.,4.464101615137755) circle (2.5pt);
\draw[color=black] (7.14,4.83) node {$\{V_3^1\}$};
\end{scriptsize}

\end{tikzpicture}
\caption{simpliziale Fläche nach einem Splitcut}
\end{figure}
\end{enumerate}
%\newpage
Durch diese Anwendung der Operatoren auf $X$ erhält man eine simpliziale Fläche $(Y,\prec)\in \mathcal{M}(X)$ mit den zwei disjunkten Mengen $X^{1}$ und $X^{2}$, wobei \\$X^{1}=\{F,\{V^1_{1}\},\{V^1_{2}\},\{V^1_{3}\},\{e^1_{1}\},\{e^1_{2}\},\{e^1_ {3}\}\}$ mit den Inzidenzen 
\begin{itemize}
 \item $\{e_{i}^1\} < F$ für alle $i \in \{1,2,3\}$,
 \item $\{V_{i}^1\}<\{e_{j}^1\}$ für alle $i \in \{1,2,3\}$ und $j \in \{1,2,3\} \setminus\{i\}$,
 \item $\{V_{i}^1\} < F$ für alle $i \in \{1,2,3\}$,
\end{itemize}
das Dreieck und $X^{2}$, die Fläche mit fehlender Fläche $F$, beschreibt.\\
Es entstehen also Randkanten $\{e^1_{i}\},\{e^2_{i}\}$ für $i \in \{1,2,3\}$, wobei die Kanten $\{e^2_{i}\}$ zu der Komponente $X^2$ gehören. \\
Man bezeichnet die simpliziale Fläche, die nach der Anwendung der Prozedur $P^1$ ensteht, mit \emph{$P_F^1(X):=(S^c_{\{e_3^1,e_3^2\}}\circ R^c_{\{e_2^1,e_2^2\}}\circ C^c_{\{e_1^1,e_1^2\}})(X)$}.

%-------------------------------------------------------

\subsection{Prozedur $P^2$}
 Nun soll das \emph{Loch an der Stelle F} verschoben werden. 
\begin{enumerate}[(i)]

\item Wende den $Ripcut$ $R^{c}_{\{f^1,f^2\}}$ an, um aus der Kante $\{f^1,f^2\}$ die Kanten $\{f^1\}$ und $\{f^2\}$ und aus dem Knoten $\{V_1^2,V_1^3\}$ die Knoten $\{V_1^2\}$ und $\{V_1^3\}$ zu erhalten, wobei für die veränderten Inzidenzen Folgendes gilt
%--------------------------------
\begin{figure}[H]
\begin{center}
\begin{tabularx}{\textwidth}{XXX}
\hline
\textbf{e}&\textbf{$X_0(e)$}&\textbf{$X_2(e)$}\\
 \hline
 $\{e_1^1\}$ & $\{V_2^1\},\{V_3^1\}$& $F$\\
  $\{e_1^2\}$ & $\{V_2^2\},\{V_3^2\}$&\\ 
  $\{e_2^1\}$&$\{V_1^1\},\{V_3^1\}$ & $F$\\
   $\{e_2^2\}$&$\{V_1^3\},\{V_3^2\}$ & \\
  $\{e_3^1\}$&$\{V_1^1\},\{V_2^1\}$ & $F$\\
  $\{e_3^2\}$&$\{V_1^2\},\{V_2^2\}$ & $F'$\\ 
   $\{f^1\}$&$\{V_1^3\},\{V_4\}$& \\
    $\{f^2\}$&$\{V_1^2\},\{V_4\}$& $F'$\\
   $\{g\}$ & $\{V_2^2\},\{V_4\}$ & $F'$ \\
 \end{tabularx}
\end{center} 
%\caption{Inzidenzen nach Ripcut}
%\end{figure}
%----------------bild------------------------
%\begin{figure}[H]
\definecolor{ududff}{rgb}{0.30196078431372547,0.30196078431372547,1.}
\definecolor{ffffff}{rgb}{1.,1.,1.}
\definecolor{sqsqsq}{rgb}{0.12549019607843137,0.12549019607843137,0.12549019607843137}
\definecolor{ffffqq}{rgb}{1.,1.,0.}
\definecolor{qqqqff}{rgb}{0.,0.,1.}
\begin{tikzpicture}[line cap=round,line join=round,>=triangle 45,x=1.0cm,y=1.0cm]

x=1.0cm,y=1.0cm,
axis lines=middle,
ymajorgrids=true,
xmajorgrids=true,
xmin=-4.5,
xmax=10.0,
ymin=-5.0,
ymax=5.2,
xtick={-9.0,-8.0,...,14.0},
ytick={-5.0,-4.0,...,6.0},]
\clip(-3.664966779911168,-4.336419420914822) rectangle (10.164271214115164,5.25368189432285);
%\fill[line width=2.pt,color=ffffqq,fill=ffffqq,fill opacity=\gelb] (-2.,0.) -- (0.,-3.481320628255737) -- (2.014912102788271,-0.008609506558991509) -- cycle;
\fill[line width=2.pt,color=ffffqq,fill=ffffqq,fill opacity=\gelb] (-3.06633318691027,3.9892123475876073) -- (-3.0412642843067688,-3.8824230699117557) -- (2.6995144118949965,-3.9826986803257602) -- (2.6744455092914956,3.9892123475876073) -- cycle;
\fill[line width=2.pt,color=ffffff,fill=ffffff,fill opacity=1.0] (-2.,0.) -- (2.,0.) -- (0.,3.4641016151377553) -- cycle;
\fill[line width=2.pt,color=ffffff,fill=ffffff,fill opacity=1.0] (0.,-3.481320628255737) -- (-2.,0.) -- (-1.010683173423175,0.028325736234424664) -- cycle;
\fill[line width=2.pt,color=ffffqq,fill=ffffqq,fill opacity=0.44999998807907104] (4.,0.) -- (8.,0.) -- (6.,3.4641016151377553) -- cycle;
\draw [line width=2.pt,color=sqsqsq] (-2.,0.)-- (0.,-3.481320628255737);
\draw [line width=2.pt] (0.,-3.481320628255737)-- (2.014912102788271,-0.008609506558991509);
\draw [line width=2.pt,color=sqsqsq] (2.014912102788271,-0.008609506558991509)-- (-1.,0.);
\draw [line width=2.pt,color=ffffff]  (-3.06633318691027,3.9892123475876073)-- (-3.0412642843067688,-3.8824230699117557);
\draw [line width=2.pt,color=ffffff] (2.6995144118949965,-3.9826986803257602)-- (2.6744455092914956,3.9892123475876073);
\draw [line width=2.pt,color=sqsqsq] (2.,0.)-- (0.,3.4641016151377553);
\draw [line width=2.pt] (0.,3.4641016151377553)-- (-2.,0.);
\draw [line width=2.pt,color=sqsqsq] (0.,-3.481320628255737)-- (-2.,0.);
\draw [line width=2.pt,color=ffffff] (-1.5,0.)-- (-1.010683173423175,0.028325736234424664);
\draw [line width=2.pt,color=sqsqsq] (-1.010683173423175,0.028325736234424664)-- (0.,-3.481320628255737);
\draw [line width=2.pt,color=ffffff] (-2.,0.)-- (-1.010683173423175,0.028325736234424664);
\draw [line width=2.pt,color=sqsqsq] (4.,0.)-- (8.,0.);
\draw [line width=2.pt,color=sqsqsq] (8.,0.)-- (6.,3.4641016151377553);
\draw [line width=2.pt,color=sqsqsq] (6.,3.4641016151377553)-- (4.,0.);
\begin{scriptsize}
\draw [fill=qqqqff] (-2.,0.) circle (2.5pt);
\draw[color=black] (-2.2815938522964825,0.30408366487293736) node {$\{V_1^3\}$};
\draw [fill=qqqqff] (-0.,-3.481320628255737) circle (2.5pt);
\draw[color=black] (0.4555720184676383,-3.6314584334627392) node {$\{V_4\}$};
\draw[color=black] (0.5937265932008994,-0.9368270140003698) node {$F'$};
\draw[color=black] (1.4210003791164376,-1.7139629947089057) node {$\{g\}$};
\draw[color=sqsqsq] (0.3555720184676383,-0.27222548459358455) node {$\{e_3^2\}$};
\draw [fill=qqqqff] (2.014912102788271,-0.108609506558991509) circle (2.5pt);
\draw[color=black] (2.260808616333726,0.3) node {$\{V_2^2\}$};
\draw[color=ffffff] (-3.2295566642470322,0.32915256747643856) node {$\{f^1\}$};
\draw[color=ffffff] (3.21342691526677,0.30408366487293736) node {$h_1$};
\draw[color=ffffff] (0.5937265932008994,1.3695120255217366) node {$Vieleck1$};
\draw[color=sqsqsq] (1.258879343435694,2.209320262739025) node {$\{e_1^2\}$};
\draw[color=black] (-1.199251163777443,2.284526970549529) node {$\{e_2^2\}$};
\draw [fill=qqqqff] (0.,3.4641016151377553) circle (2.5pt);
\draw[color=black] (0.23022750545013249,3.7892123475876073) node {$\{V_3^2\}$};
\draw [fill=ududff] (-1.010683173423175,0.028325736234424664) circle (2.5pt);
\draw[color=black] (-0.7725285986899139,0.3547726909079489) node {$\{V_1^2\}$};
\draw[color=sqsqsq] (-1.0352008551986667,-1.0120337218108733) node {$\{f^2\}$};
\draw[color=ffffff] (-1.4368545176826946,-0.15969103329183398) node {$d$};
\draw[color=sqsqsq] (-1.5363032968546852,-1.5635495790878988) node {$\{f^1\}$};
\draw[color=ffffff] (-1.4368545176826946,-0.15969103329183398) node {$l$};
\draw [fill=qqqqff] (4.,0.) circle (2.5pt);
\draw[color=black] (4.090838506389312,-0.3477078028180926) node {$\{V_1^1\}$};
\draw [fill=qqqqff] (8.,0.) circle (2.5pt);
\draw[color=black] (8.352276338570503,-0.2725010950075892) node {$\{V_2^1\}$};
\draw[color=black] (6.108885165971154,1.4196498307287388) node {$F$};
\draw[color=sqsqsq] (5.96626325083409,-0.22208767938658227) node {$\{e_3^1\}$};
\draw[color=sqsqsq] (7.450347065672471,2.209320262739025) node {$\{e_1^1\}$};
\draw[color=sqsqsq] (4.842905584494346,2.209320262739025) node {$\{e_2^1\}$};
\draw [fill=qqqqff] (6.,3.4641016151377553) circle (2.5pt);
\draw[color=black] (6.321970838100914,3.9892123475876073) node {$\{V_3^1\}$};
\end{scriptsize}

\end{tikzpicture}
\caption{simpliziale Fläche nach einem Ripcut}
\end{figure}
%----------------------------------------------
%\newpage
\item Wende den $Ripmender$ $R^{m}_{\{e^2_{1}\},\{e^2_{3}\}}$ an, um die Kanten $\{e^2_{1}\}$ und $\{e^2_{3}\}$ zu einer Kante $\{e^2_{1},e^2_{3}\}$ und die Knoten $\{V_1^2\}$ und $\{V_3^2\}$ zu dem Knoten $\{V_1^2,V_3^2\}$ zusammenzuführen.

%\centerline{$\textcolor{red}{Bild4}$}
%-----------------------bild---------------------------
\begin{figure}[H]
\begin{center}
\begin{tabularx}{\textwidth}{XXX}
\hline
\textbf{e}&\textbf{$X_0(e)$}&\textbf{$X_2(e)$}\\
 \hline
 $\{e_1^1\}$ & $\{V_2^1\},\{V_3^1\}$& $F$\\
 \hline 
  $\{e_1^2,e_3^2\}$ & $\{V_1^2,V_3^2\},\{V_2^2\}$&$F'$\\ 
  \hline
  $\{e_2^1\}$&$\{V_1^1\},\{V_3^1\}$ & $F$\\
  \hline
   $\{e_2^2\}$&$\{V_1^2,V_3^2\},\{V_1^3\}$ & \\
  \hline
  $\{e_3^1\}$&$\{V_1^1\},\{V_2^1\}$ & $F$\\ \hline  
   $\{f^1\}$&$\{V_1^3\},\{V_4\}$& \\
   \hline
    $\{f^2\}$&$\{V_1^2,V_3^2\},\{V_4\}$& $F'$\\
   \hline
   $\{g\}$ & $\{V_2^2\},\{V_4\}$ & $F'$ \\
   \hline
 \end{tabularx}%\caption{Inzidenzen nach Ripmender}
\end{center} 
\end{figure}
%\newpage
%----------------------------------------------------------
%-----------------------bild----------------------------------
\begin{figure}[H]
\definecolor{ffffff}{rgb}{1.,1.,1.}
\definecolor{qqqqff}{rgb}{0.,0.,1.}
\definecolor{ffffqq}{rgb}{1.,1.,0.}
\begin{tikzpicture}[line cap=round,line join=round,>=triangle 45,x=1.4cm,y=1.4cm]
%\begin{axis}[
x=1.0cm,y=1.0cm,
axis lines=middle,
ymajorgrids=true,
xmajorgrids=true,
xmin=-3.583376623376623,
xmax=16.330043290043285,
ymin=-4.489177489177493,
ymax=5.588744588744593,
xtick={-3.0,-2.0,...,16.0},
ytick={-4.0,-3.0,...,5.0},]
\clip(-3.583376623376623,-0.289177489177493) rectangle (16.330043290043285,4.588744588744593);
\fill[line width=2.pt,color=ffffqq,fill=ffffqq,fill opacity=\gelb] (-2.2,0.) -- (2.2,0.) -- (2.2,4.) -- (-2.2,4.) -- cycle;
\fill[line width=2.pt,color=white,fill=ffffff,fill opacity=1.0] (0.,1.) -- (0.,3.) -- (-1.7320508075688776,2.) -- cycle;
\fill[line width=2.pt,color=ffffqq,fill=ffffqq,fill opacity=0.] (0.,3.) -- (0.,1.) -- (1.7320508075688776,2.) -- cycle;
\fill[line width=2.pt,color=ffffqq,fill=ffffqq,fill opacity=\gelb] (3.,1.) -- (5.594112554112552,0.9696969696969713) -- (4.323299471110349,3.231415856986091) -- cycle;
\draw [line width=2.pt] (0.,1.)-- (0.,3.);
\draw [line width=2.pt] (0.,3.)-- (-1.7320508075688776,2.);
\draw [line width=2.pt] (-1.7320508075688776,2.)-- (0.,1.);
\draw [line width=2.pt] (0.,3.)-- (0.,1.);
\draw [line width=2.pt] (0.,1.)-- (1.7320508075688776,2.);
\draw [line width=2.pt] (1.7320508075688776,2.)-- (0.,3.);
\draw [line width=2.pt] (3.,1.)-- (5.594112554112552,0.9696969696969713);
\draw [line width=2.pt] (5.594112554112552,0.9696969696969713)-- (4.323299471110349,3.231415856986091);
\draw [line width=2.pt] (4.323299471110349,3.231415856986091)-- (3.,1.);
\begin{scriptsize}
\draw[color=black] (0.6993073593073588,2.051515151515153) node {$F'$};
\draw [fill=qqqqff] (0.,1.) circle (2.5pt);
\draw[color=black] (0.12225108225108178,0.7203463203463215) node {$\{V_4\}$};
\draw [fill=qqqqff] (0.,3.) circle (2.5pt);
\draw[color=black] (0.10225108225108178,3.2116883116883145) node {$\{V_1^2,V_3^2\}$};
%\draw[color=ffffff] (-0.17212121212121256,2.151515151515153) node {$Vieleck2$};
\draw [fill=qqqqff] (-1.7320508075688776,2.) circle (2.5pt);
\draw[color=black] (-1.8593506493506495,2.2246753246753267) node {$\{V_1^3\}$};
%\draw[color=ffffqq] (0.9880519480519474,2.151515151515153) node {$Vieleck3$};
\draw [fill=qqqqff] (1.7320508075688776,2.) circle (2.5pt);
\draw[color=black] (1.893852813852813,2.2246753246753267) node {$\{V_2^2\}$};
\draw [fill=qqqqff] (3.,1.) circle (2.5pt);
\draw[color=black] (2.717922077922077,1.1246753246753267) node {$\{V_1^1\}$};
\draw [fill=qqqqff] (5.594112554112552,0.9696969696969713) circle (2.5pt);
\draw[color=black] (5.9153246753246725,1.1246753246753267) node {$\{V_2^1\}$};
\draw[color=black] (4.31099567099567,1.8961038961038983) node {$F$};
\draw [fill=qqqqff] (4.323299471110349,3.231415856986091) circle (2.5pt);
\draw[color=black] (4.45125541125541,3.5584415584415625) node {$\{V_3^1\}$};
\draw[color=black] (4.31125541125541,0.745584415584415625) node {$\{e_3^1\}$};
\draw[color=black] (3.31099567099567,2.1961038961038983) node {$\{e_2^1\}$};
\draw[color=black] (5.31099567099567,2.1961038961038983) node {$\{e_1^1\}$};
\draw[color=black] (-0.26099567099567,2.01961038961038983) node {$\{f^2\}$};
\draw[color=black] (-1.01099567099567,1.2961038961038983) node {$\{f^1\}$};
\draw[color=black] (-1.01099567099567,2.6961038961038983) node {$\{e_2^2\}$};
\draw[color=black] (1.01099567099567,1.2961038961038983) node {$\{g\}$};
\draw[color=black] (1.11099567099567,2.6961038961038983) node {$\{e_1^2,e_3^2\}$};
\end{scriptsize}

%\end{axis}
\end{tikzpicture}
\caption{simpliziale Fläche nach Anwendung eines Ripmernders}
\end{figure}
\end{enumerate}
Für eine nach der Prozedur $P^1$ entstandene simpliziale Fläche $Z=P^1_F(X)$  bezeichnet man die simpliziale Fläche, die nach Anwendung der zweiten Prozedur entsteht, mit \emph{$P^2_f(Z)$}$:=(R^m_{\{e_1^2\},\{e_3^2\}}\circ R^c_{\{f^1\},\{f^2\}})(Z)$. Das heißt, $P^2_f(P^1_F(X))$ ist die Fläche, die nach der Anwendung der beiden Prozeduren $P^1$ und $P^2$ auf die gegebene simpliziale Fläche $X$ entsteht.

%--------------------------------------------------------------
%\newpage
\subsection{Prozedur $P^3$}
 Zuletzt müssen nun die simpliziale Fläche mithilfe der folgenden Operationen wieder zusammengesetzt werden:
\begin{enumerate}[(i)]
%\newpage
\item Wende den $Splitmender$ $S^{m}_{(\{V^1_{1}\},\{e^1_{3}\}),(\{V_{4}\},\{f^2\})}$ an, um die Kanten $\{f^2\}$ und $\{e_3^1\}$ zu der Kante $\{e_3^1,f^2\}$ zusammenzuführen und so ebenfalls aus den Knoten $\{V_1^1\}$ und $\{V_4\}$ den Knoten $\{V_1^1,V_4\}$ und aus den Knoten $\{V_1^2,V_3^2\}$ und $\{V_2^1\}$ den Knoten $\{V_2^1,\{V_1^2,V_3^2\}\}$ zu konstruieren.
%---------------bild----------------------------
%\newpage
\begin{figure}[H]
\begin{center}
\begin{tabularx}{\textwidth}{XXX}
\hline
\textbf{e}&\textbf{$X_0(e)$}&\textbf{$X_2(e)$}\\
 \hline
 $\{e_1^1\}$ & $\{V_2^1,\{V_1^2,V_3^2\}\},\{V_3^1\}$& $F$\\
 \hline 
  $\{e_1^2,e_3^2\}$ & $\{V_2^1,\{V_1^2,V_3^2\}\},\{V_2^2\}$&$F'$\\ 
  \hline
  $\{e_2^1\}$&$\{V_1^1,V_4\},\{V_3^1\}$ & $F$\\
  \hline
   $\{e_2^2\}$&$\{V_1^3\},\{V_2^1,\{V_1^2,V_3^2\}\}$ & \\
  \hline
  $\{e_3^1,f^2\}$&$\{V_1^1,V_4\},\{V_2^1,\{V_1^2,V_3^2\}\}$ & $F,F'$\\ \hline  
   $\{f^1\}$&$\{V_1^1,V_4\},\{V_1^3\}$& \\
   \hline
   $\{g\}$ & $\{V_1^1,V_4\},\{V_2^2\}$ & $F'$ \\
   \hline
 \end{tabularx} %\caption{Inzidenzen nach Splitmender}
\end{center}
\end{figure}
\begin{figure}[H]
%\begin{comment}
\definecolor{ududff}{rgb}{0.30196078431372547,0.30196078431372547,1.}
\definecolor{ffffff}{rgb}{1.,1.,1.}
\definecolor{qqqqff}{rgb}{0.,0.,1.}
\definecolor{ffffqq}{rgb}{1.,1.,0.}
\begin{tikzpicture}[line cap=round,line join=round,>=triangle 45,x=1.4cm,y=1.4cm]
%\begin{axis}[
x=1.0cm,y=1.0cm,
axis lines=middle,
ymajorgrids=true,
xmajorgrids=true,
xmin=-4.3,
xmax=7.0600000000000005,
ymin=-2.46,
ymax=6.3,
xtick={-4.0,-3.0,...,7.0},
ytick={-2.0,-1.0,...,6.0},]
\clip(-5.3,-0.) rectangle (7.06,4.);
\fill[line width=2.pt,color=ffffqq,fill=ffffqq,fill opacity=\gelb] (-2.,0.) -- (2.1,0.) -- (2.1,4.) -- (-2.,4.) -- cycle;
\fill[line width=2.pt,color=ffffqq,fill=ffffqq,fill opacity=0.1499999940395355] (0.,3.) -- (0.,1.) -- (1.7320508075688776,2.) -- cycle;
\fill[line width=2.pt,color=ffffff,fill=ffffff,fill opacity=1.0] (0.,1.) -- (0.,3.) -- (-1.7320508075688776,2.) -- cycle;
\fill[line width=2.pt,color=ffffqq,fill=ffffqq,fill opacity=\gelb] (0.,3.) -- (-1.,2.) -- (0.,1.) -- cycle;
\draw [line width=2.pt] (0.,3.)-- (0.,1.);
\draw [line width=2.pt] (0.,1.)-- (1.7320508075688776,2.);
\draw [line width=2.pt] (1.7320508075688776,2.)-- (0.,3.);
\draw [line width=2.pt] (0.,1.)-- (0.,3.);
\draw [line width=2.pt] (0.,3.)-- (-1.7320508075688776,2.);
\draw [line width=2.pt] (-1.7320508075688776,2.)-- (0.,1.);
\draw [line width=2.pt] (0.,3.)-- (-1.,2.);
\draw [line width=2.pt] (-1.,2.)-- (0.,1.);
\begin{scriptsize}
\draw[color=black] (-1.04,1.37) node {$\{f^1\}$};
\draw [fill=qqqqff] (0.,3.) circle (2.5pt);
\draw[color=black] (0.14,3.17) node {$\{V_2^1,\{V_1^2,V_3^2\}\}$};
\draw[color=black] (0.1,0.77) node {$\{V_1^1,V_4\}$};
\draw [fill=qqqqff] (0.,1.) circle (2.5pt);
\draw[color=black] (0.64,1.97) node {$F'$};
\draw[color=black] (0.34,2.27) node {$\{e_3^1,f^2\}$};
\draw[color=black] (1.04,2.71) node {$\{e_1^2,e_3^2\}$};
\draw[color=black] (0.94,1.31) node {$\{g\}$};
\draw [fill=qqqqff] (1.7320508075688776,2.) circle (2.5pt);
\draw[color=black] (1.88,2.17) node {$\{V_2^2\}$};
\draw[color=black] (-0.3,2.01) node {$F$};
\draw [fill=qqqqff] (-1.7320508075688776,2.) circle (2.5pt);
\draw[color=black] (-1.8,2.22) node {$\{V_1^3\}$};
\draw [fill=ududff] (-1.,2.) circle (2.5pt);
\draw[color=black] (-1.26,2.0) node {$\{V_3^1\}$};
\draw[color=black] (-1.03,2.66) node {$\{e_2^2\}$};
\draw[color=black] (-0.33,2.37) node {$\{e_1^1\}$};
\draw[color=black] (-0.33,1.62) node {$\{e_2^1\}$};
\end{scriptsize}
%\end{axis}
\end{tikzpicture}
%\end{comment}
\caption{simpliziale Fläche nach Anwendung eines Splitmenders}
\end{figure}
%------------------------------------------------
\item Wende den $Ripmender$ $R^m_{\{e^1_{2}\},\{f^1\}}$ an, um die Kanten $\{e^1_{2}\}$ und $\{f^1\}$ zu der Kante $\{e^1_{2}\},\{f^1\}$ und die Knoten $\{ V_1^3\}$ und $\{V_3^1\}$ zu dem Knoten $\{V_1^3,V_3^1\}$ zusammenzuführen.
\begin{figure}[H]
\begin{center}
\begin{tabularx}{\textwidth}{XXX}
\hline
\textbf{e}&\textbf{$X_0(e)$}&\textbf{$X_2(e)$}\\
 \hline
 $\{e_1^1\}$&$\{V_1^3,V_3^1\},\{V_2^1,\{V_1^2,V_3^2\}\}$&F\\
  $\{e_1^2,e_3^2\}$ & $\{V_2^1,\{V_1^2,V_3^2\}\},\{V_2^2\}$&$F'$\\ 
  $\{e_2^1,f_1\}$&$\{V_1^1,V_4\},\{V_1^3,V_3^1\}$ & $F$\\
  $\{e_2^2\}$&$\{V_1^3,V_3^1\},\{V_2^1,\{V_1^2,V_3^2\}\}$&\\
  $\{e_3^1,f^2\}$&$\{V_1^1,V_4\},\{V_2^1,\{V_1^2,V_3^2\}\}$ & $F,F'$\\ 
   $\{g\}$ & $\{V_1^1,V_4\},\{V_2^2\}$ & $F'$ \\
 \end{tabularx} %\caption{Inzidenzen nach Ripmender}
\end{center}
\end{figure}
%-------------------------bild-------------------------
\begin{figure}[H]
\definecolor{ududff}{rgb}{0.30196078431372547,0.30196078431372547,1.}
\definecolor{ffffff}{rgb}{1.,1.,1.}
\definecolor{qqqqff}{rgb}{0.,0.,1.}
\definecolor{ffffqq}{rgb}{1.,1.,0.}
\begin{tikzpicture}[line cap=round,line join=round,>=triangle 45,x=1.5cm,y=1.5cm]
%\begin{axis}[
x=1.5cm,y=1.5cm,
axis lines=middle,
ymajorgrids=true,
xmajorgrids=true,
xmin=-4.3,
xmax=7.0600000000000005,
ymin=-2.46,
ymax=6.3,
xtick={-4.0,-3.0,...,7.0},
ytick={-2.0,-1.0,...,6.0},]
\clip(-5.3,-0.46) rectangle (7.06,4.3);
\fill[line width=2.pt,color=ffffqq,fill=ffffqq,fill opacity=\gelb] (-2.5,0.) -- (2.2,0.) -- (2.2,4.) -- (-2.5,4.) -- cycle;
\fill[line width=2.pt,color=ffffqq,fill=ffffqq,fill opacity=0.] (0.,3.) -- (0.,1.) -- (1.7320508075688776,2.) -- cycle;
\fill[line width=2.pt,color=ffffff,fill=ffffff,fill opacity=1.0] (0.,1.) -- (0.,3.) -- (-1.7320508075688776,2.) -- cycle;
\fill[line width=2.pt,color=ffffqq,fill=ffffqq,fill opacity=0.5] (0.,3.) -- (-1.74,2.) -- (0.,1.) -- cycle;
\draw [line width=2.pt] (0.,3.)-- (0.,1.);
\draw [line width=2.pt] (0.,1.)-- (1.7320508075688776,2.);
\draw [line width=2.pt] (1.7320508075688776,2.)-- (0.,3.);
\draw [line width=2.pt] (0.,1.)-- (0.,3.);
\draw [line width=2.pt] (-1.7320508075688776,2.)-- (0.,1.);
\draw [line width=2.pt] (-1.74,2.)-- (0.,1.);
\draw [rotate around={29.886526940424037:(-0.87,2.5)},line width=2.pt,color=ffffff,fill=ffffff,fill opacity=1.0] (-0.87,2.5) ellipse (1.41989009757162764cm and 0.22745816720870826cm);
\draw [rotate around={29.886526940424037:(-0.87,2.5)},line width=2.pt] (-0.87,2.5) ellipse (1.41989009757163032cm and 0.2274581672087103cm);
\begin{scriptsize}
\draw[color=black] (-0.5,1.9) node {$F$};
\draw [fill=qqqqff] (0.,3.) circle (2.5pt);
\draw[color=black] (0.14,3.23) node {$\{V_2^1,\{V_1^2,V_3^2\}\}$};
\draw [fill=qqqqff] (0.,1.) circle (2.5pt);
\draw[color=black] (0.14,0.75) node {$\{V_1^1,V_4\}$};
\draw[color=black] (1.04,2.75) node {$\{e_1^2,e_3^2\}$};
\draw [fill=qqqqff] (1.7320508075688776,2.) circle (2.5pt);
\draw[color=black] (1.88,2.22) node {$\{V_2^2\}$};
\draw[color=black] (0.8,2.) node {$F'$};
\draw[color=black] (0.4,2.32) node {$\{e_3^1,f^2\}$};
\draw [fill=qqqqff] (-1.7320508075688776,2.) circle (2.5pt);
\draw[color=black] (-2.1,2.22) node {$\{V_1^3,V_3^1\}$};
\draw [fill=qqqqff] (-1.74,2.) circle (2.5pt);
\draw[color=black] (-1.0,2.87) node {$\{e_2^2\}$};
\draw[color=black] (-1.0,1.26) node {$\{e_2^1,f^1\}$};
%\draw[color=black] (-1.16,2.25) node {$c$};
\draw[color=black] (-0.65,2.2) node {$\{e^1_1\}$};
\draw[color=black] (0.94,1.31) node {$\{g\}$};
\end{scriptsize}
%\end{axis}
\end{tikzpicture}
\caption{simpliziale Fläche nach Anwendung eines Ripmenders}
\end{figure}
%-------------------------------------------------------

 \item 
 Wende den $Cratermender$ $R^{m}_{\{e_1^1\},\{e^2_{2}\}}$ an, um die Kanten $\{e_1^1\}$ und $\{e^2_{2}\}$ zu einer Kante $\{e_1^1,e_2^2\}$ zu vereinen.
\begin{figure}[H]
\begin{center}
\begin{tabularx}{\textwidth}{XXX}
\hline
\textbf{e}&\textbf{$X_0(e)$}&\textbf{$X_2(e)$}\\
 \hline
 $\{e_1^1,e_2^2\}$ & $\{V_1^3,V_3^1\},\{V_2^1,\{V_1^2,V_3^2\}\}$& $F$\\
  $\{e_1^2,e_3^2\}$ & $\{V_2^1,\{V_1^2,V_3^2\}\},\{V_2^2\}$&$F'$\\ 
  $\{e_2^1,f_1\}$&$\{V_1^1,V_4\},\{V_1^3,V_3^1\}$ & $F$\\
  $\{e_3^1,f^2\}$&$\{V_1^1,V_4\},\{V_2^1,\{V_1^2,V_3^2\}\}$ & $F,F'$\\ 
   $\{g\}$ & $\{V_1^1,V_4\},\{V_2^2\}$ & $F'$ \\
 \end{tabularx}
\end{center}
\end{figure}
%--------------------bild-------------------------
\begin{figure}[H]
\definecolor{ududff}{rgb}{0.30196078431372547,0.30196078431372547,1.}
\definecolor{ffffff}{rgb}{1.,1.,1.}
\definecolor{qqqqff}{rgb}{0.,0.,1.}
\definecolor{ffffqq}{rgb}{1.,1.,0.}
\begin{tikzpicture}[line cap=round,line join=round,>=triangle 45,x=1.5cm,y=1.5cm]
%\begin{axis}[
x=1.5cm,y=1.5cm,
axis lines=middle,
ymajorgrids=true,
xmajorgrids=true,
xmin=-4.3,
xmax=7.0600000000000005,
ymin=-2.46,
ymax=6.3,
xtick={-4.0,-3.0,...,7.0},
ytick={-2.0,-1.0,...,5.0},]
\clip(-5.3,-0.) rectangle (7.06,4.);
\fill[line width=2.pt,color=ffffqq,fill=ffffqq,fill opacity=\gelb] (-2.5,0.) -- (2.2,0.) -- (2.2,4.) -- (-2.5,4.) -- cycle;
\fill[line width=2.pt,color=ffffqq,fill=ffffqq,fill opacity=0.] (0.,3.) -- (0.,1.) -- (1.7320508075688776,2.) -- cycle;
\fill[line width=2.pt,color=ffffff,fill=ffffff,fill opacity=0.] (0.,1.) -- (0.,3.) -- (-1.7320508075688776,2.) -- cycle;
\fill[line width=2.pt,color=ffffqq,fill=ffffqq,fill opacity=0.] (0.,3.) -- (-1.74,2.) -- (0.,1.) -- cycle;
\draw [line width=2.pt] (0.,3.)-- (0.,1.);
\draw [line width=2.pt] (0.,1.)-- (1.7320508075688776,2.);
\draw [line width=2.pt] (1.7320508075688776,2.)-- (0.,3.);
\draw [line width=2.pt] (0.,1.)-- (0.,3.);
\draw [line width=2.pt] (-1.7320508075688776,2.)-- (0.,1.);
\draw [line width=2.pt] (0.,3.)-- (-1.74,2.);
\draw [line width=2.pt] (-1.74,2.)-- (0.,1.);
\begin{scriptsize}
%\draw[color=ffffqq] (-0.84,1.27) node {$Vieleck1$};
%\draw [fill=qqqqff] (0.,3.) circle (2.5pt);
%\draw[color=qqqqff] (0.14,3.37) node {$E$};
%\draw [fill=qqqqff] (0.,1.) circle (2.5pt);
%\draw[color=qqqqff] (0.14,1.37) node {$F$};
%\draw[color=ffffqq] (1.04,3.11) node {$Vieleck2$};
%\draw [fill=qqqqff] (1.7320508075688776,2.) circle (2.5pt);
%\draw[color=qqqqff] (1.88,2.37) node {$G$};
%\draw[color=ffffff] (-0.1,2.17) node {$Vieleck3$};
%\draw [fill=qqqqff] (-1.7320508075688776,2.) circle (2.5pt);
%\draw[color=qqqqff] (-1.6,2.37) node {$H$};
%\draw [fill=ududff] (-1.74,2.) circle (2.5pt);
%\draw[color=ududff] (-1.6,2.37) node {$I$};
%\draw[color=black] (-0.93,3.) node {$f_1$};
%\draw[color=black] (-0.98,1.41) node {$e$};
%\end{comment}
%-----------------------------
%\begin{figure}
\draw[color=black] (-0.6,2.) node {$F$};
\draw [fill=qqqqff] (0.,3.) circle (2.5pt);
\draw[color=black] (0.14,3.23) node {$\{V_2^1,\{V_1^2,V_3^2\}\}$};
\draw [fill=qqqqff] (0.,1.) circle (2.5pt);
\draw[color=black] (0.14,0.75) node {$\{V_1^1,V_4\}$};
\draw[color=black] (1.04,2.75) node {$\{e_1^2,e_3^2\}$};
\draw [fill=qqqqff] (1.7320508075688776,2.) circle (2.5pt);
\draw[color=black] (1.88,2.22) node {$\{V_2^2\}$};
\draw[color=black] (0.8,2.) node {$F'$};
\draw[color=black] (0.4,2.32) node {$\{e_3^1,f^2\}$};
\draw [fill=qqqqff] (-1.7320508075688776,2.) circle (2.5pt);
\draw[color=black] (-2.1,2.22) node {$\{V_1^3,V_3^1\}$};
\draw [fill=qqqqff] (-1.74,2.) circle (2.5pt);
\draw[color=black] (-1.0,2.77) node {$\{e_1^1,e_2^2\}$};
\draw[color=black] (-1.0,1.26) node {$\{e_2^1,f^1\}$};
%\draw[color=black] (-1.16,2.25) node {$c$};
%\draw[color=black] (-0.65,2.2) node {$\{e^1_1\}$};
\draw[color=black] (0.94,1.31) node {$\{g\}$};
\end{scriptsize}
%\end{axis}
\end{tikzpicture}
\caption{simpliziale Fläche nach einem Cratermender}
\end{figure}
\end{enumerate}
%\newpage
%\section{neuer versuch}
Für eine durch die Prozedur $P^1$ oder $P^2$ entstandene simpliziale Fläche $Z$ bezeichnet man die simpliziale Fläche, die aus der Prozedur $P^3$ hervorgeht, mit \emph{$P^3_F(Z)$}$:=(S^m_{(\{V_1^1\},\{e_3^1\}),(\{V_4\},\{f^2\})} \circ R^m_{\{e_2^1\},\{f^1\}} \circ C^m_{\{e_1^1\},\{e_2^2\}})(Z)$. \\
Somit ist $P^3_F(P^2_f(P^1_F(X)))$ die simpliziale Fläche, die aus $X$ nach der Anwendung der drei Prozeduren hervorgeht.
\begin{bemerkung}
\begin{enumerate}
\item Es gilt $P^3_F(P^1_F(X))\cong X$ für ein $F \in X_2 $.
\item Die Prozeduren hängen nicht nur von $F$ ab, sondern auch von der Wahl der Kanten von $F$.
\item Es gibt für eine nach der ersten Prozedur entfernte Fläche $F \in X_{2}$ genau  drei Möglichkeiten, um eine von den drei entstandenen Randkanten auszuwählen.
 Dann gibt es weiterhin zwei Möglichkeiten eine Kante, wie im erstem Schritt der Prozedur $P^2$ beschrieben, auszuwählen und dann letztlich drei Möglichkeiten, um die in der Prozedur $P^1$ herausgenommene Fläche wieder einzufügen.
  Deshalb ist die nach den drei Prozeduren entstandene simpliziale Fläche nicht eindeutig und aufgrund dessen wird folgende Notation eingeführt: 
\begin{itemize}
\item Man nennt die durch die Anwendung der drei Prozeduren entstandene simpliziale Fläche $X^{H}_{(F,f)}$, falls zunächst ein \emph{Loch an der Stelle F} entsteht und dann in der Prozedur $P^2$ der Operator \emph{RipCutter} auf die Kante $f=\{f^1,f^2\}$ in $X$ angewendet wird. 
%\newpage
\item Für eine geschlossene simpliziale Fläche $(X,<)$ und eine Fläche $F \in X_2$ definiert man die Menge \emph{$\mathcal{W}_F(X)$} als 
\[
\mathcal{W}_F(X):=\{f \in X_1\mid \vert X_{0}(f) \cap X_{0}(F)\vert = 1 \land X_2(f) \cap X_2(X_1(F))\neq \emptyset\} .
\] 
Für den Fall, dass $X$ eine simpliziale Fläche ist, auf die man die Prozedur $P^3$ anwenden kann, so definiert man
\[
\mathcal{W}_F(X):=\mathcal{W}(P^3_F(X)).
\]
%Außerdem ist 
%\[
%H(X):=\bigcup_{F\in X_2} H_F(X).
%\]
Hierbei handelt es sich um die Menge aller Kanten, mit denen man die Prozedur $P^2$ auf $X$
durchführen kann, falls man in Prozedur $P^1$ die Fläche $F$ entfernt hat. Ein $f\in \mathcal{W}_F(X)$ nennt man eine \emph{Wanderkante} in $X$ und $\mathcal{W}_F(X)$ ist die \emph{Menge der Wanderkanten}.
 \end{itemize}
 Sind $F$ und $f$ aus dem Kontext klar, so schreibt man nur $X^H$.
Beispielsweise ist für das Oktaeder $(O,<)$, welches durch das ordinale Symbol  
\begin{align*}
\omega((O,<))=&(6,12,8;(\{ 1, 2 \}, \{ 1, 3 \}, \{ 1, 4 \}, \{ 1, 5 \}, \{ 2, 3 \}, \{ 2, 5 \}, \{ 2, 6 \},\{ 3, 4 \},\\ 
&\{ 3, 6 \},\{ 4, 5 \}, \{ 4, 6 \}, \{ 5, 6 \} ),(\{ 1, 2, 5 \}, \{ 6, 7, 12 \}, \{ 1, 4, 6 \}, \{ 5, 7, 9 \},\\
& \{3, 4, 10 \},\{ 8, 9, 11 \}, \{ 2, 3, 8 \}, \{ 10, 11, 12 \}))\\
\end{align*}
definiert ist und $F_1 \in O_2$, die Menge aller Kanten, mit denen man die Prozedur $P^2$ an der Stelle $F_1$ durchführen kann, gegeben durch die Menge $\mathcal{W}_{F_1}(O)=\{e_3,e_4,e_6,e_7,e_8,e_9\}$.
 \begin{figure}[H]
\begin{center}
\includegraphics[viewport=1cm 22cm 8cm 27cm]{OctaBA1}
\end{center}
\caption{Oktaeder}
\end{figure}
 \end{enumerate}
\end{bemerkung}
%\centerline{$\textcolor{red}{Bild 6}$}
\begin{bemerkung}
\begin{itemize} \label{beminv1}
\item Für eine geschlossene simpliziale Fläche $(X,<)$, eine Fläche $F \in X_2$ und eine Wanderkante $f \in \mathcal{W}(X)$ ist $X^H_{(F,f)}$ wieder eine geschlossene simpliziale Fläche.
\item Mit den Bezeichnungen wie oben gilt $(X^H_{(F,f)})^H_{(F,\{e_1^2,e_3^2\})}\cong X$.
\end{itemize}
%\end{enumerate}
\end{bemerkung}
%\vspace*{}
%---------------------------------
\subsection{Lochwanderungssequenzen}
%-----------------------------------
Es stellt sich nun die Frage, welche und wie viele simpliziale Flächen mithilfe der obigen Prozeduren konstruiert werden können, wenn man nach der ersten Prozedur die Prozedur $P^2$ mehrfach auf eine simpliziale Fläche anwendet, bevor dann darauffolgend mithilfe der Prozedur $P^3$ wieder eine geschlossene simpliziale Fläche entsteht.\\
Hierfür führt man einige Notationen ein.

\begin{definition}
Seien $(X,<)$ eine geschlossene simpliziale Fläche, $F \in X_2$ eine Fläche und $n\in \mathbb{N}$. Sei außerdem $(F,f_1,\ldots,f_n)$ ein Tupel mit
\begin{itemize}
%\item $F \in X_2$
\item $f_1 \in \mathcal{W}_F(P_F^1(X))$,
%\item $f_2 \in \mathcal{W}(P^2_{f_1}(P_F^1(X))))$
\item $f_i \in \mathcal{W}_F((P^2_{f_{i-1}} \circ \ldots \circ P^2_{f_1}\circ P_F^1 )(X)),$ für $i=2,\ldots ,n$.
\end{itemize}
So ist $X_{(F,f_1,\ldots,f_n)}^H$ definiert durch 
\[
X^{H}_{(F,f_1,\ldots,f_n)}:=(P_F^3\circ P^2_{f_n} \circ \ldots \circ P^2_{f_1}\circ P_F^1)(X).
\]
Das Tupel $(F,f_1,\ldots,f_n)$ nennt man eine \emph{Lochwanderung von $X$} und $X^H_{(F,f_1,\ldots,f_n)}$, die durch die Lochwanderung $(F,f_1,\ldots,f_n)$ entstandene simpliziale Fläche.\\
Für eine simpliziale Fläche $(X,<)$ und eine Lochwanderung $\sigma_1$ von $X$ führt man folgende Konstruktion durch:
\begin{itemize}
\item $X^H_{(\sigma_1)}:=X^H_{\sigma_1}$
\item Für eine Lochwanderung $\sigma_2$ von $X^H_{\sigma_1}$ ist 
\[
X^H_{(\sigma_1,\sigma_2)}:=(X_{\sigma_1}^H)^H_{\sigma_2}.
\]
\item Sei nun  $X_{(\sigma_1,\ldots, \sigma_i)}^H$ für $i\in \mathbb{N}\setminus\{1\}$ und Lochwanderungen $\sigma_j$ von $X^H_{(\sigma_1,\ldots,\sigma_{j-1})}$ mit $2 \leq j \leq i$ schon konstruiert und $\sigma_{i+1}$ eine Lochwanderung von $X_{(\sigma_1,\ldots, \sigma_i)}^H$, so ist
\[
X_{(\sigma_1,\ldots, \sigma_{i+1})}^H:=(X_{(\sigma_1,\ldots, \sigma_i)}^H)^H_{\sigma_{i+1}}.
\]
\item Für obige Lochwanderungen $\sigma_i$ und $n \in \mathbb{N}$ nennt man $(\sigma_1, \ldots,\sigma_n)$ eine \emph{Lochwanderungssequenz von $X$} und $X^H_{(\sigma_1,\ldots,\sigma_n)}$ die durch die Lochwanderungssequenz $(\sigma_1,\ldots,\sigma_n)$ entstandene simpliziale Fläche.
\end{itemize}
\end{definition}


\begin{definition}
Für eine simpliziale Fläche $(X,<)$ definiert man die Menge aller simplizialen Flächen, die bis auf Isomorphie durch Anwenden einer Lochwanderungssequenz auf $X$ entstehen können, als
\[
\mathcal{H}_X:=\{[X_{\Sigma}^H] \mid \Sigma \text{ ist eine Lochwanderungssequenz von X}\},
\]
wobei
\[
 [Z] := \{Y \mid Y \text{simpliziale Fläche mit } Y \cong Z\}
\] die \emph{Isomorphieklasse} einer simplizialen Fläche  $Z$ ist. 

\end{definition}
\begin{bemerkung}\label{beminv}
Wegen \Cref{beminv1} ist es leicht einzusehen, dass es zu einer Lochwanderungssequenz $\Sigma$ einer simplizialen Fläche $(X,<)$ eine Lochwanderungssequenz $\Sigma^{-1}$ gibt, sodass ${(X^H_{\Sigma})}^H_{\Sigma^{-1}} \cong X$ ist.
\end{bemerkung}
\subsection{Transitivität der Operation Wanderinghole}
Da die Operation Wanderinghole nun definiert ist, kann man nun die Transitivität dieser untersuchen. Zur Vereinfachung des späteren Beweises wird hier zunächst einmal die Beweisidee skizziert und thematisiert, welche Probleme auftreten können.
\begin{beweisidee}
Seien $(X,<)$ und $(Y,\prec)$ geschlossene simpliziale Flächen mit $\chi(X)=\chi(Y)=2$ und $X_2=Y_2=\{F_1,\ldots, F_n\}$ für ein $n \in \mathbb{N}$.\\
Ziel ist es, $X$, durch wiederholte Anwendung der Operation Wanderinghole, in die simpliziale Fläche $Y$ zu transformieren.
Der Beweis der Transitivität der oben definierten Operation Wanderinghole beruht auf der Idee, die Nachbarschaften der Flächen in $X$ durch die Anwendung von Wanderinghole nach und nach zu verändern, um so die Nachbarschaften der Flächen in $Y$ nachzuahmen. Das heißt, es soll
\[
Z_2(Z_1(F))=Y_2(Y_1(F)) 
\]
für alle $F\in X_2=Y_2$ gelten, wobei $Z=X^H_{\Sigma}$ für eine Lochwanderungssequenz $\Sigma$ ist. Zur Veranschaulichung führt man folgendes Beispiel an. Hierfür wurde das in der Einleitung erwähnte \emph{Gap}-Paket und die implementierte Operation Wanderinghole verwendet.\\
Sei $(X,<)$ eine simpliziale Fläche, die definiert wird durch
\begin{align*}
\omega((X,<))=&(8,18,12;(\{1, 4 \}, \{ 1, 6 \}, \{ 1, 7 \}, \{ 1, 8 \}, \{ 2, 3 \}, \{ 2, 4 \}, \{ 2, 5 \},
  \{ 2, 6 \}, \{ 3, 5 \},\\
   &\{ 3, 6 \}, \{ 3, 7 \}, \{ 4, 5 \}, \{4, 6 \}, \{ 4, 8 \},
  \{ 5, 7 \}, \{ 5, 8 \}, \{ 6, 7 \}, \{ 7, 8 \}), \\
  &(\{ 1, 2, 13 \}, \{ 1, 4, 14 \}, \{ 2, 3, 17 \}, \{ 3, 4, 18 \}, \{ 5, 7, 9 \},\{ 5, 8, 10 \},\\
  & \{ 6, 7, 12 \}, \{ 6, 8, 13 \}, \{ 9, 11, 15 \}, \{ 10, 11, 17 \}, \{ 12, 14, 16 \}, \{ 15, 16, 18 \} )).
\end{align*} 

\begin{figure}[H]
\begin{center}
\includegraphics[viewport=1cm 22cm 8cm 27cm]{surfacey}
\end{center}
\caption{simpliziale Fläche}
\end{figure}

Sei zudem $(Y,\prec)$ eine simpliziale Fläche, welche durch das folgende ordinale Symbol definiert wird.
\begin{align*}
\omega((Y,\prec))=&(8,18,12;(\{1,5\},\{1,6\},\{ 1, 7 \}, \{ 1, 8 \}, \{ 2, 3 \}, \{ 2, 4 \}, \{ 2, 5 \},\{ 2, 6 \}, \{ 3, 5 \}, \\
&\{ 3, 6 \}, \{ 3, 7 \}, \{ 4, 5 \}, \{ 4, 6 \}, \{ 4, 8 \},
  \{ 5, 7 \}, \{ 5, 8 \}, \{ 6, 7 \}, \{ 6, 8 \}) \\
 &(\{ 1, 3, 15 \}, \{ 1, 4, 16 \}, \{ 2, 3, 17 \}, \{ 2, 4, 18 \}, \{ 5, 7, 9 \},  \{ 5, 8, 10 \},\\ 
  &\{ 6, 7, 12 \}, \{ 6, 8, 13 \}, \{ 9, 11, 15 \}, \{ 10, 11, 17 \},
  \{ 12, 14, 16 \}, \{ 13, 14, 18 \} )).
\end{align*}
%----------------------------------------

\begin{figure}[H]
\begin{center}
\includegraphics[viewport=1cm 20.5cm 8cm 27cm]{surfacex}
\end{center}
\caption{simpliziale Fläche}
\end{figure}
Ziel ist es, wie oben schon erwähnt, $X$ durch Lochwanderungen in eine simpliziale Fläche $Z$ umzuformen, die $N_Y(F)=N_Z(F)$ für alle $F \in Y_2$ erfüllt. Man betrachte hierzu zunächst die Fläche $F_1$. Es gilt zunächst einmal $X_2(X_1(F_1))=\{F_1,F_2,F_3,F_8\}$ und $Y_2(Y_1((F_1))=\{F_1,F_2,F_3F_9\}$. Das heißt, statt $F_8$, will man nun $F_9$ als Nachbarfläche von $F_1$ haben. Durch die Operation Wanderinghole kann man sich diesem Ziel, jedoch nur stückweise annähern. Man führt deshalb folgende Anwendungen der Operation Wanderinghole aus:
\begin{enumerate}
\item Durch Anwendung der Lochwanderung $(F_9,e_7)$ auf $X$ erhält man eine simpliziale Fläche $Z^1$, in der $F_7$ und $F_9$ benachbart sind.
\begin{figure}[H]
\begin{center}
\includegraphics[viewport=1cm 22cm 8cm 27cm]{BAWH1}
\end{center}
\caption{Anwendung Wanderinghole auf eine simpliziale Fläche}
\end{figure}

\item Durch die Anwendung der Operation Wanderinghole mithilfe der Lochwanderung $(F_9,e_6)$ erhält man eine simpliziale Fläche $Z^2$, die $F_8$ und $F_9$ als adjazente Flächen enthält.
\begin{figure}[H]
\begin{center}
\includegraphics[viewport=1cm 22cm 8cm 27cm]{BHWH2}
\end{center}
\caption{Anwendung von Wanderinghole auf eine simpliziale Fläche}
\end{figure}

\item Durch die Anwendung der Operation mithilfe der Lochwanderung $(F_9,e_{13})$ ist nun $Z$ eine simpliziale Fläche mit $F_1$ und $F_9$ als benachbarte Flächen.
\begin{figure}[H]
\begin{center}
\includegraphics[viewport=1cm 20cm 8cm 27cm]{BAWH3}
\end{center}
\caption{Anwendung von Wanderinghole auf simpliziale Fläche}
\end{figure}
\end{enumerate}
Also ist $Z'=X_{(F_9,e_7,e_6,e_{13})}^H$ eine wohldefinierte simpliziale Fläche mit der Eigenschaft $Z'_2(Z'_1(F_1))=Y_2(Y_1(F_1))$. Ähnlich soll nun für die restlichen Flächen vorgegangen werden. Das heißt, durch geschickte Anwendung der Operation Wanderinghole entlang eines Pfades, wobei der Begriff eines Pfades in einer simplizialen Fläche in diesem Kapitel noch definiert wird, soll nun nach und nach eine Fläche entstehen, die der simplizialen Fläche immer mehr ähnelt.\\
%-------------------------------------------- 
%% !TEX root= BAkopie.tex
\documentclass{article}

\usepackage[inner=0.5cm,outer=0.5cm,top=1cm,bottom=0.5cm]{geometry}

\pagestyle{empty}
% This document contains the TikZ-header for all our LaTeX-computations.
% It especially contains all global graphic parameters.

\usepackage{amsmath, amssymb, amsfonts} % Standard Math-stuff

\usepackage{ifthen}

\usepackage{tikz}
\usetikzlibrary{calc}
\usetikzlibrary{positioning}
\usetikzlibrary{shapes}
\usetikzlibrary{patterns}


% Sometimes we want to implement different behaviour for the generated 
% HTML-pictures (for example, shading is not supported in HTML).
% For that we define a macro to check whether we run the code with
% htlatex. The code comes from 
% https://tex.stackexchange.com/questions/93852/what-is-the-correct-way-to-check-for-latex-pdflatex-and-html-in-the-same-latex
\makeatletter
\edef\texforht{TT\noexpand\fi
  \@ifpackageloaded{tex4ht}
    {\noexpand\iftrue}
    {\noexpand\iffalse}}
\makeatother


% Define a text=none option for nodes that ignores the given text, from
% https://tex.stackexchange.com/questions/59354/no-text-none-in-tikz
\makeatletter
\newif\iftikz@node@phantom
\tikzset{
  phantom/.is if=tikz@node@phantom,
  text/.code=%
    \edef\tikz@temp{#1}%
    \ifx\tikz@temp\tikz@nonetext
      \tikz@node@phantomtrue
    \else
      \tikz@node@phantomfalse
      \let\tikz@textcolor\tikz@temp
    \fi
}
\usepackage{etoolbox}
\patchcmd\tikz@fig@continue{\tikz@node@transformations}{%
  \iftikz@node@phantom
    \setbox\pgfnodeparttextbox\hbox{}
  \fi\tikz@node@transformations}{}{}
\makeatother

% Find the angle of a given line (within TikZ)
\newcommand{\tikzAngleOfLine}{\tikz@AngleOfLine}
\def\tikz@AngleOfLine(#1)(#2)#3{%
  \pgfmathanglebetweenpoints{%
    \pgfpointanchor{#1}{center}}{%
    \pgfpointanchor{#2}{center}}
  \pgfmathsetmacro{#3}{\pgfmathresult}%
}

% Now we define the global styles
% The global styles are defined nestedly. You have to give your tikzpicture
% the global options [vertexStyle, edgeStyle, faceStyle] to activate them.
% 
% You can disable labels by using the option nolabels, i.e. 
% vertexStyle=nolabels to deactivate vertex labels.
%
% If you want to have a specific style for your picture, you can also use
% this specific meta-style instead of the general style. For example if you
% want to use double edges in one single picture - no matter the style of
% the rest of the document - you can use edgeDouble instead of edgeStyle.
%
% To set the default style, modify the vertexStyle/.default entry.

% Vertex styles
\tikzset{ 
    vertexNodePlain/.style = {fill=#1, shape=circle, inner sep=0pt, minimum size=2pt, text=none},
    vertexNodePlain/.default=gray,
    vertexPlain/labels/.style = {
        vertexNode/.style={vertexNodePlain=##1},
        vertexLabel/.style={gray}
    },
    vertexPlain/nolabels/.style = {
        vertexNode/.style={vertexNodePlain=##1},
        vertexLabel/.style={text=none}
    },
    vertexPlain/.style = vertexPlain/#1,
    vertexPlain/.default=labels
}
\tikzset{
    vertexNodeNormal/.style = {fill=#1, shape=circle, inner sep=0pt, minimum size=4pt, text=none},
    vertexNodeNormal/.default = blue,
    vertexNormal/labels/.style = {
        vertexNode/.style={vertexNodeNormal=##1},
        vertexLabel/.style={blue}
    },
    vertexNormal/nolabels/.style = {
        vertexNode/.style={vertexNodeNormal=##1},
        vertexLabel/.style={text=none}
    },
    vertexNormal/.style = vertexNormal/#1,
    vertexNormal/.default=labels
}
\tikzset{
    vertexNodeBallShading/pdf/.style = {ball color=#1},
    vertexNodeBallShading/svg/.style = {fill=#1},
    vertexNodeBallShading/.code = {% Conditional shading depending whether we want pdf or svg output
        \if\texforht
            \tikzset{vertexNodeBallShading/svg=#1!90!black}
        \else
            \tikzset{vertexNodeBallShading/pdf=#1}
        \fi
    },
    vertexNodeBall/.style = {shape=circle, vertexNodeBallShading=#1, inner sep=2pt, outer sep=0pt, minimum size=3pt, font=\tiny},
    vertexNodeBall/.default = orange,
    vertexBall/labels/.style = {
        vertexNode/.style={vertexNodeBall=##1, text=black},
        vertexLabel/.style={text=none}
    },
    vertexBall/nolabels/.style = {
        vertexNode/.style={vertexNodeBall=##1, text=none},
        vertexLabel/.style={text=none}
    },
    vertexBall/.style = vertexBall/#1,
    vertexBall/.default=labels
}
\tikzset{ 
    vertexStyle/.style={vertexNormal=#1},
    vertexStyle/.default = labels
}


% 1) optional: colour of vertex
% 2) position of the vertex
% 3) relative position of the node
% 4) name of the vertex
\newcommand{\vertexLabelR}[4][]{
    \ifthenelse{ \equal{#1}{} }
        { \node[vertexNode] at (#2) {#4}; }
        { \node[vertexNode=#1] at (#2) {#4}; }
    \node[vertexLabel, #3] at (#2) {#4};
}
% 1) optional: colour of vertex
% 2) position of the vertex
% 3) absolute position of the node
% 4) name of the vertex
\newcommand{\vertexLabelA}[4][]{
    \ifthenelse{ \equal{#1}{} }
        { \node[vertexNode] at (#2) {#4}; }
        { \node[vertexNode=#1] at (#2) {#4}; }
    \node[vertexLabel] at (#3) {#4};
}


% Edge styles
% If you have trouble with the double-lines overlapping, this might (?) help:
% https://tex.stackexchange.com/questions/288159/closing-the-ends-of-double-line-in-tikz
\newcommand{\edgeLabelColor}{blue!20!white}
\tikzset{
    edgeLineNone/.style = {draw=none},
    edgeLineNone/.default=black,
    edgeNone/labels/.style = {
        edge/.style = {edgeLineNone=##1},
        edgeLabel/.style = {fill=\edgeLabelColor,font=\small}
    },
    edgeNone/nolabels/.style = {
        edge/.style = {edgeLineNone=##1},
        edgeLabel/.style = {text=none}
    },
    edgeNone/.style = edgeNone/#1,
    edgeNone/.default = labels
}
\tikzset{
    edgeLinePlain/.style={line join=round, draw=#1},
    edgeLinePlain/.default=black,
    edgePlain/labels/.style = {
        edge/.style={edgeLinePlain=##1},
        edgeLabel/.style={fill=\edgeLabelColor,font=\small}
    },
    edgePlain/nolabels/.style = {
        edge/.style={edgeLinePlain=##1},
        edgeLabel/.style={text=none}
    },
    edgePlain/.style = edgePlain/#1,
    edgePlain/.default = labels
}
\tikzset{
    edgeLineDouble/.style = {very thin, double=#1, double distance=.8pt, line join=round},
    edgeLineDouble/.default=gray!90!white,
    edgeDouble/labels/.style = {
        edge/.style = {edgeLineDouble=##1},
        edgeLabel/.style = {fill=\edgeLabelColor,font=\small}
    },
    edgeDouble/nolabels/.style = {
        edge/.style = {edgeLineDouble=##1},
        edgeLabel/.style = {text=none}
    },
    edgeDouble/.style = edgeDouble/#1,
    edgeDouble/.default = labels
}
\tikzset{
    edgeStyle/.style = {edgePlain=#1},
    edgeStyle/.default = labels
}

% Face styles
% Here we have an exception - the style face is always defined.
% 
\newcommand{\faceColorY}{yellow!60!white}   % yellow
\newcommand{\faceColorB}{blue!60!white}     % blue
\newcommand{\faceColorC}{cyan!60}           % cyan
\newcommand{\faceColorR}{red!60!white}      % red
\newcommand{\faceColorG}{green!60!white}    % green
\newcommand{\faceColorO}{orange!50!yellow!70!white} % orange

% define default face colour (and default swap colour)
\newcommand{\faceColor}{\faceColorY}
\newcommand{\faceColorSwap}{\faceColorC}

% define secondary default colours (to use in a single section)
\newcommand{\faceColorFirst}{green!40!white}
\newcommand{\faceColorSecond}{gray!15!white}
\newcommand{\faceColorThird}{red!17!white}
\newcommand{\faceColorFourth}{olive!20!white}

\tikzset{
    face/.style = {fill=#1},
    face/.default = \faceColor,
    faceY/.style = {face=\faceColorY},
    faceB/.style = {face=\faceColorB},
    faceC/.style = {face=\faceColorC},
    faceR/.style = {face=\faceColorR},
    faceG/.style = {face=\faceColorG},
    faceO/.style = {face=\faceColorO}
}
\tikzset{
    faceStyle/labels/.style = {
        faceLabel/.style = {}
    },
    faceStyle/nolabels/.style = {
        faceLabel/.style = {text=none}
    },
    faceStyle/.style = faceStyle/#1,
    faceStyle/.default = labels
}
\tikzset{ face/.style={fill=#1} }
\tikzset{ faceSwap/.code=
    \ifdefined\swapColors
        \tikzset{face=\faceColorSwap}
    \else
        \tikzset{face=\faceColor}
    \fi
}



\usepackage{hyperref}


\begin{document}



\begin{tikzpicture}[vertexBall, edgeDouble, faceStyle, scale=2]

% Define the coordinates of the vertices
\coordinate (V1_1) at (0, 0);
\coordinate (V2_1) at (1, 0);
\coordinate (V3_1) at (0.5, 0.8660254037844386);
\coordinate (V4_1) at (-0.4999999999999999, 0.8660254037844386);
\coordinate (V5_1) at (0.5, -0.8660254037844386);
\coordinate (V5_2) at (-0.9999999999999998, 1.110223024625157e-16);
\coordinate (V6_1) at (1.5, 0.8660254037844386);
\coordinate (V6_2) at (1.5, -0.8660254037844386);
\coordinate (V6_3) at (5.551115123125783e-17, 1.732050807568877);
\coordinate (V6_4) at (-1.5, 0.8660254037844386);


% Fill in the faces
\fill[face=yellow]  (V2_1) -- (V3_1) -- (V1_1) -- cycle;
\node[faceLabel] at (barycentric cs:V2_1=1,V3_1=1,V1_1=1) {$1$};
\fill[face=yellow]  (V5_1) -- (V6_2) -- (V2_1) -- cycle;
\node[faceLabel] at (barycentric cs:V5_1=1,V6_2=1,V2_1=1) {$2$};
\fill[face=yellow]  (V1_1) -- (V5_1) -- (V2_1) -- cycle;
\node[faceLabel] at (barycentric cs:V1_1=1,V5_1=1,V2_1=1) {$3$};
\fill[face=yellow]  (V2_1) -- (V6_1) -- (V3_1) -- cycle;
\node[faceLabel] at (barycentric cs:V2_1=1,V6_1=1,V3_1=1) {$4$};
\fill[face=yellow]  (V4_1) -- (V5_2) -- (V1_1) -- cycle;
\node[faceLabel] at (barycentric cs:V4_1=1,V5_2=1,V1_1=1) {$5$};
\fill[face=yellow]  (V3_1) -- (V6_3) -- (V4_1) -- cycle;
\node[faceLabel] at (barycentric cs:V3_1=1,V6_3=1,V4_1=1) {$6$};
\fill[face=yellow]  (V3_1) -- (V4_1) -- (V1_1) -- cycle;
\node[faceLabel] at (barycentric cs:V3_1=1,V4_1=1,V1_1=1) {$7$};
\fill[face=yellow]  (V4_1) -- (V6_4) -- (V5_2) -- cycle;
\node[faceLabel] at (barycentric cs:V4_1=1,V6_4=1,V5_2=1) {$8$};


% Draw the edges
\draw[edge] (V2_1) -- node[edgeLabel] {$1$} (V1_1);
\draw[edge] (V1_1) -- node[edgeLabel] {$2$} (V3_1);
\draw[edge] (V1_1) -- node[edgeLabel] {$3$} (V4_1);
\draw[edge] (V5_1) -- node[edgeLabel] {$4$} (V1_1);
\draw[edge] (V1_1) -- node[edgeLabel] {$4$} (V5_2);
\draw[edge] (V3_1) -- node[edgeLabel] {$5$} (V2_1);
\draw[edge] (V2_1) -- node[edgeLabel] {$6$} (V5_1);
\draw[edge] (V6_1) -- node[edgeLabel] {$7$} (V2_1);
\draw[edge] (V2_1) -- node[edgeLabel] {$7$} (V6_2);
\draw[edge] (V4_1) -- node[edgeLabel] {$8$} (V3_1);
\draw[edge] (V3_1) -- node[edgeLabel] {$9$} (V6_1);
\draw[edge] (V6_3) -- node[edgeLabel] {$9$} (V3_1);
\draw[edge] (V5_2) -- node[edgeLabel] {$10$} (V4_1);
\draw[edge] (V4_1) -- node[edgeLabel] {$11$} (V6_3);
\draw[edge] (V6_4) -- node[edgeLabel] {$11$} (V4_1);
\draw[edge] (V6_2) -- node[edgeLabel] {$12$} (V5_1);
\draw[edge] (V5_2) -- node[edgeLabel] {$12$} (V6_4);


% Draw the vertices
\vertexLabelR[blue]{V1_1}{left}{$1$}
\vertexLabelR[blue]{V2_1}{left}{$2$}
\vertexLabelR[blue]{V3_1}{left}{$3$}
\vertexLabelR[blue]{V4_1}{left}{$4$}
\vertexLabelR[blue]{V5_1}{left}{$5$}
\vertexLabelR[blue]{V5_2}{left}{$5$}
\vertexLabelR[blue]{V6_1}{left}{$6$}
\vertexLabelR[blue]{V6_2}{left}{$6$}
\vertexLabelR[blue]{V6_3}{left}{$6$}
\vertexLabelR[blue]{V6_4}{left}{$6$}

\end{tikzpicture}
\end{document}

Dies wirft einige Fragen auf.
\begin{itemize}
\item Ist es immer möglich in $X$ die Nachbarschaften aus der simplizialen Fläche $Y$ nachzuahmen?
\item Können durch das Anwenden von Lochwanderungssequenzen auf $X$ schon erfolgreich konstruierte Nachbarschaften zerstört werden?
\item Ist es immer möglich die Operation Wanderinghole wie oben skizziert entlang eines Pfades anzuwenden? 
\end{itemize}
Falls sich die obigen Fragen bewahrheiten, bleibt immer noch die Frage der Isomorphie der konstruierten simplizialen Fläche $Z$ und der simplizialen Fläche $Y$ zu klären. Diese Fragestellungen sollen im Folgenden behandelt werden, um so die Transitivität der Operation Wanderinghole nachzuweisen. Hierzu ist jedoch noch etwas Vorarbeit nötig.
\end{beweisidee}
Zunächst einmal werden die Begriffe einer stark-zusammenhängenden und einer Jordan-zusammenhängenden Menge eingeführt.

\begin{definition} 

\begin{enumerate}
\item Sei $(X,<)$ eine simpliziale Fläche. Ein \emph{Flächenpfad} von $S\in X_{2}$ nach $T \in X_{2}$ in $X$ ist eine Sequenz $(F_1:=S,F_{2},\ldots,F_{k}:=T)$ für ein $k \in \mathbb{N}$ so, dass $F_{i} $ und $F_{i+1}$ für $i=1,\ldots,k-1$ benachbarte Flächen in $X$ sind.\cite{SS}
\item Man nennt eine Menge $M\subseteq X_2$  \emph{stark-zusammenhängend}, falls für beliebige $S,T \in M$ ein Flächenpfad $(F_{1}:=S,F_{2},\ldots,F_{k}:=T)$ mit $F_i \in M$ für $1\leq i \leq k$ existiert. 
 \item Die stark-zusammenhängenden Mengen $M\subseteq X_2$ nennt man \emph{Zusammenhangskomponenten von $X$}, sofern diese folgende Bedingung erfüllen: Falls ein $M\subseteq M'$ existiert, sodass $M'$ stark-zusammenhängend in $X$ ist, so gilt schon $M'=M$.\cite{SS}
 \item Man nennt die Menge $M \subseteq X_2$ \emph{Jordan-zusammenhängend} in $X$, falls $M=X_2$ stark-zusammenhängend ist oder  $M \subsetneq X_2$ gilt und die Mengen $M$ und $X_2\setminus M$ stark-zusammenhängend sind. Falls $X_2$ stark zusammenhängend ist, so nennt man die simpliziale Fläche $(X,<)$ Jordan-zusammenhängend.
\end {enumerate}
\end{definition}
An dieser Stelle kann man auch den Begriff einer zusammenhängenden Menge einführen. Da jedoch zusammenhängende Mengen für diesen Untersuchungsaspekt nicht von Relevanz sind, bleibt dies an dieser Stelle aus.
\begin{bemerkung}
 Seien $(X,<)$ eine simpliziale Fläche und $S,T \in X_2$ Flächen in $X$.
  Falls ein $S$-$T$-Weg etwa $(S,F_1,\ldots,F_n,T)$ für $F_1,\ldots,F_n \in X_2$ und $n \in \mathbb{N}$ existiert, so existiert auch ein $S$-$T$-Weg ohne \emph{Flächenwiederholung}.
   Das heißt, es existieren paarweise verschiedene $F_1',\ldots F_k'\in X_2$ für $k \in \mathbb{N}$ so, dass $(S,F_1',\ldots,F_k',T)$ ebenfalls ein $S$-$T$-Weg in $X$ ist.
\end{bemerkung}

\begin{bsp}
\begin{itemize}
\item Sei $J$ der Janus-Kopf. Dann ist die Flächenmenge von $J$ gegeben durch $J_2=\{F_1,F_2\}$ und die Mengen $\{F_1\}$,$\{F_2\}$ und $\{F_1,F_2\}$ sind Jordan-zusammenhängende Mengen in $J$.
\item Für eine geschlossene Jordan-zusammenhängende simpliziale Fläche $(X,<)$ mit $\chi(X)=2$ bildet $\{F\}$ für ein $F \in X_2$ stets eine Jordan-zusammenhängende Menge.
\item Sei $T$ das zuvor definierte Tetraeder. Sei $\emptyset \subsetneq M \subseteq T_2$. So bildet $M$ eine Jordan-zusammenhängende Menge.
\item Sei $n\in \mathbb{N}$ und $X=n \Delta$. Dann existiert kein $M \subseteq X_2$, sodass $M$ Jordan-zusammenhängend ist. Denn die Menge $ M = X_2$ ist klarerweise nicht stark-zusammenhängend. Und falls $M \subsetneq X_2$ ist, dann besteht eine der Mengen $M$ und $X_2\setminus M$ aus mindestens zwei Zusammenhangskomponenten. Damit ist $M$ nicht Jordan-zusammenhängend.
\item Seien nun $(X,<)$ eine simpliziale Fläche und $M \subseteq X_2$ eine Jordan- zusammenhängende Menge in $X$. Dann gilt dies auch für $X_2\setminus M$.
\end{itemize}
\end{bsp}
\begin{lemma}\label{lemmajanus}
 Sei $(X,<)$ eine geschlossene Jordan-zusammenhängende simpliziale Fläche und $e\in X_1$ eine Kante in $X$. Falls $\deg(V)=2$ für alle $V \in X_1(e)$ ist, dann ist $X \cong J$, wobei $J$ der Janus-Kopf ist.
\end{lemma}
\begin{proof}
Sei $X_0(e)=\{V_1,V_2\}$ für Knoten $V_1,V_2 \in X_0$. Dann existieren $F_1,F_2 \in X_2$ mit $X_2(V_1)=\{F_1,F_2\}$. Da $\deg(V_1)=2$ ist und $\vert X_2(e) \cap X_2(V_1)\vert =2$ wegen der Geschlossenheit von $X$ gelten muss, folgt $X_2(e)=\{F_1,F_2\}$ und somit auch $X_2(V_2)=\{F_1,F_2\}$. Angenommen es gilt $\deg(V_3)\neq 2$ für den Knoten $V_3 \in X_0(F_1)$. Da $\deg(V_3)=1$ aufgrund der Geschlossenheit der simplizialen Fläche nicht möglich ist, reicht es, den Fall $\deg(V_3)\geq 3$ zu widerlegen. Falls $\deg(V_3)\geq 3$ ist, existiert, da $X$ geschlossen ist, ein $F_i \neq F\in X_2$, sodass $V_3<F$ und $X_0(\tilde{e})=\{V_1,V_3\}$ für $\tilde{e}\in X_1(F)$ und $i=1,2$ ist. Wegen $V_1< \tilde{e}$ und $ \tilde{e}<F_1$ folgt $V_1<F$ und damit auch $\deg (V_1)=\vert X_2(V_1) \vert\geq \vert\{F,F_1,F_2\}\vert=3$. Der Fall $X_0(\tilde{e})=\{V_2,V_3\}$ wird analog behandelt. Also ist $M=\{F_1,F_2\}$ eine Zusammenhangskomponente von $X$ und wegen des starken Zusammenhangs auch die einzige. Also erhält man $X_2=\{F_1,F_2\}$. Da es bis auf Isomorphie nur eine simpliziale Fläche mit 2 Flächen gibt, folgt die Behauptung.
\end{proof}

\begin{folgerung}
Sei $(X,<)$ eine geschlossene Jordan-zusammenhängende simpliziale Fläche mit $\vert X_2\vert > 2$. Dann ist $\mathcal{W}_F(X)\neq \emptyset$ für alle $F \in X_2$.
\end{folgerung}
\begin{proof}
Sei $F\in X_2$ eine beliebige Fläche in $X$. Dann gilt aufgrund der Geschlossenheit von $X$, dass $\deg(V) \geq 2$ für alle $V \in X_0(F)$ ist. Und wegen \Cref{lemmajanus} existiert ein $V_F\in X_0(F)$ mit $\deg(V_F)> 2$. Das heißt, es existieren $F_1,\ldots,F_n$, sodass $(F,F_1,\ldots,F_n)$ für $n \geq 2$ den zu $V_F$ gehörigen Schirm bildet. Somit existieren Kanten $e_1,e_2 \in X_1$, sodass $X_2(e_1)=\{F,F_1\}$ und $X_2(e_2)=\{F_1,F_2\}$ ist. Dann gilt aber schon $e_2 \in \mathcal{W}_F(X)$ und damit ist $\mathcal{W}_F(X) \neq \emptyset$.
\end{proof}
Um die Knoten, Kanten und Flächen einer Jordan-zusammenhängenden Menge von denen des Komplements in $X$ unterscheiden zu können, färbt man den Graphen, wie in folgender Definition beschrieben.

\begin{definition}
Für eine simpliziale Fläche $(X,<)$ und eine Jordan-\\ zusammenhängende Menge $M$definiert man die Abbildung $f_M^X:X \mapsto\{0,1\}$, welche durch
\[
f^X_M(x)=\begin{cases}
1 & \text{falls } x\in M \cup X_0(M)\cup X_1(M)\\
0 &\text{sonst}\\

\end{cases}
\]
gegeben ist.
Die Abbildung $f_M^X$ nennt man eine \emph{Färbung} von $X$ und $(X,<,f_m^X)$ eine gefärbte simpliziale Fläche. Falls die simpliziale Fläche $X$ aus dem Kontext heraus klar ist, so schreibt man nur $f_M$.
\end{definition}

\begin{bsp}
\begin{itemize}
\item Für eine Jordan-zusammenhängende simpliziale Fläche   $\,(X,<)$ ist die Färbung $f_{X_2}$ gegeben durch $f_{X_2}(x)=1$ für alle $x \in X$.

\item Für den Janus-Kopf $J$ bildet $M_J=\{F_1\}$ eine Jordan-zusammenhängende Menge. Damit ist $f_{M_J}$ gegeben durch
\[
f_{M_J}(x)=\begin{cases}
0 &x=F_2\\
1 & \text{falls } x\in X\setminus \{F_2\}.\\
		\end{cases}
\]
\item Für eine geschlossene Jordan-zusammenhängende simpliziale Fläche $(X,<)$ mit $\chi(X)=2$ bildet $M=\{F\}$ für ein $F \in X_2$ eine Jordan-zusammenhängende Menge. Dann ist $f_M(F)=1$ und $f_M(x)=0$ für alle $x \in X\setminus \{F\}$.
\item Für das Tetraeder, $T$ wie in Abbildung \ref{tetra} definiert, bildet $M_T=\{F_3,F_4\}$ eine Jordan-zusammenhängende Menge. Dadurch erhält man die Färbung $f_{M_T}$ gegeben durch
\[
f_{M_T}(x)=\begin{cases}
		0 & \text{ falls } x  \in \{e_2,F_1,F_2\}\\
1 &\text{ falls }x \in X\setminus \{e_2,F_1,F_2\}.\\
		\end{cases}
\]

\end{itemize}
\end{bsp}
%--------------------------------------
\begin{bemerkung}
In Abbildungen wird die Färbung wie folgt realisiert:
Alle Elemente einer simplizialen Fläche, die durch die Färbung auf $1$ abgebildet werden, werden in den folgenden Abbildungen in grau dargestellt und die Urbilder von $0$ bleiben unverändert. Dies wird noch am obigen Beispiel des Tetraeders verdeutlicht.
\end{bemerkung}
\begin{figure}[H]
 \definecolor{ffffqq}{rgb}{1.,1.,0.}
\definecolor{qqqqff}{rgb}{0.,0.,1.}
\definecolor{qqffqq}{rgb}{0.,1.,0.}
\definecolor{yqyqyq}{rgb}{0.5019607843137255,0.5019607843137255,0.5019607843137255}
\begin{tikzpicture}[line cap=round,line join=round,>=triangle 45,x=1.0cm,y=1.0cm]
%\begin{axis}[
x=1.0cm,y=1.0cm,
axis lines=middle,
ymajorgrids=true,
xmajorgrids=true,
xmin=-8.620000000000001,
xmax=14.38,
ymin=-3.72,
ymax=4.32,
xtick={-8.0,-7.0,...,14.0},
ytick={-5.0,-4.0,...,5.0},]
\clip(-7.62,-3.72) rectangle (14.38,4.32);
\fill[line width=2.pt,color=ffffqq,fill=ffffqq,fill opacity=\gelb] (-2.,0.) -- (2.,0.) -- (0.,3.4641016151377553) -- cycle;
\fill[line width=2.pt,color=yqyqyq,fill=yqyqyq,fill opacity=\gelb] (2.,0.) -- (-2.,0.) -- (0.,-3.4641016151377553) -- cycle;
\fill[line width=2.pt,color=yqyqyq,fill=yqyqyq,fill opacity=\gelb] (0.,3.4641016151377553) -- (2.,0.) -- (4.,3.464101615137754) -- cycle;
\fill[line width=2.pt,color=ffffqq,fill=ffffqq,fill opacity=\gelb] (-2.,0.) -- (0.,3.4641016151377553) -- (-4.,3.464101615137757) -- cycle;
\draw [line width=2.pt,color=yqyqyq] (-2.,0.)-- (2.,0.);
\draw [line width=2.pt,color=yqyqyq] (2.,0.)-- (0.,3.4641016151377553);
\draw [line width=2.pt,color=yqyqyq] (0.,3.4641016151377553)-- (-2.,0.);
\draw [line width=2.pt,color=yqyqyq] (2.,0.)-- (-2.,0.);
\draw [line width=2.pt,color=yqyqyq] (-2.,0.)-- (0.,-3.4641016151377553);
\draw [line width=2.pt,,color=yqyqyq] (0.,-3.4641016151377553)-- (2.,0.);
\draw [line width=2.pt,,color=yqyqyq] (0.,3.4641016151377553)-- (2.,0.);
\draw [line width=2.pt,,color=yqyqyq] (2.,0.)-- (4.,3.464101615137754);
\draw [line width=2.pt,color=yqyqyq] (4.,3.464101615137754)-- (0.,3.4641016151377553);
\draw [line width=2.pt] (-2.,0.)-- (0.,3.4641016151377553);%%%%
\draw [line width=2.pt,color=yqyqyq] (0.,3.4641016151377553)-- (-4.,3.464101615137757);
\draw [line width=2.pt,color=yqyqyq] (-4.,3.464101615137757)-- (-2.,0.);
\begin{scriptsize}
\draw [fill=yqyqyq] (-2.,0.) circle (2.5pt);
\draw[color=black] (-2.32,-0.09) node {$V_1$};
\draw [fill=yqyqyq] (2.,0.) circle (2.5pt);
\draw[color=black] (2.34,0.) node {$V_2$};
\draw[color=black] (0.,1.33) node {$F_1$};
\draw[color=black] (0.06,-0.25) node {$e_3$};
\draw[color=black] (1.32,2.07) node {$e_1$};
\draw[color=black] (-1.22,2.07) node {$e_2$};
\draw [fill=yqyqyq] (0.,3.4641016151377553) circle (2.5pt);
\draw[color=black] (0.14,3.83) node {$V_3$};
\draw[color=black] (0.,-1.33) node {$F_3$};
\draw[color=black] (-1.27,-1.71) node {$e_5$};
\draw[color=black] (1.37,-1.71) node {$e_6$};
\draw [fill=yqyqyq] (0.,-3.4641016151377553) circle (2.5pt);
\draw[color=black] (0.49,-3.29) node {$V_4$};
\draw [fill=yqyqyq] (-2.,0.) circle (2.5pt);
%\draw[color=qqqqff] (-1.86,0.37) node {$E$};
\draw[color=black] (2.,2.19) node {$F_4$};
\draw[color=black] (3.37,1.75) node {$e_6$};
\draw[color=black] (2.11,3.82) node {$e_4$};
\draw [fill=yqyqyq] (4.,3.464101615137754) circle (2.5pt);
\draw[color=black] (4.14,3.83) node {$V_4$};
\draw[color=black] (-2.,2.19) node {$F_2$};
\draw[color=black] (-1.94,3.82) node {$e_4$};
\draw[color=black] (-3.3,1.75) node {$e_5$};
\draw [fill=yqyqyq] (-4.,3.464101615137757) circle (2.5pt);
\draw[color=black] (-3.86,3.83) node {$V_4$};
\end{scriptsize}
%\end{axis}
\end{tikzpicture}

 \caption{Tetraeder }
 \label{Tetraeder}
 \end{figure}
%--------------------------------------
\begin{lemma}\label{jordanV}
Seien $(X,<)$ eine geschlossene Jordan-zusammenhängende simpliziale Fläche und $M \subseteq X_2$ eine Jordan-zusammenhängende Teilmenge der Flächen. Seien zusätzlich $V\in X_0$ ein Knoten in $X$ und $(F_1,\ldots,F_n)$ für Flächen $F_1,\ldots,F_n \in X_2$ und $n \in \mathbb{N}$ der zu $V$ gehörige Schirm. Falls $f_M(V)=0$ ist, dann gilt $f_M(F_1)=\ldots=f_M(F_n)=0$.
\end{lemma}
\begin{proof}
Angenommen die obige Behauptung stimmt nicht. Dann ist $f_M(V)=0$, aber es existiert ein $j\in \{1,\ldots,n\}$ mit $f_M(F_j)=1$. Dann gilt aber auch nach Definition der Färbung $f_M(x)=1$ für alle $x \in X_0(F_j)\cup X_1(F_j)$. Da $(F_1,\ldots,F_j,\ldots,F_n)$ der zu $V$ gehörige Schirm ist, ist somit auch $V\in X_0(F_j)$. Dadurch erhält man $f_M(V)=1$ und somit auch den gewünschten Widerspruch.
\end{proof}
\begin{bemerkung}
Es ist nach obigem Lemma leicht einzusehen, dass die Färbung der simplizialen Fläche transitiv im folgendem Sinne ist:\\
Sei $(X,<)$ eine geschlossene Jordan-zusammenhängende simpliziale Fläche und die Menge $M \subseteq X_2$ eine Jordan-zusammenhängende Teilmenge der Flächen. Seien zudem $F\in X_2$ eine Fläche, $e\in X_1(F)$ eine Kante und $V\in X_0(e)$ ein Knoten in $X$. Dann gilt die Implikation 
\[
f_M(V)=0 \Rightarrow f_M(e)=0 \Rightarrow f_M(F)=0.
\]
\end{bemerkung}
Um die Nachbarn einer Fläche in einer simplizialen Fläche besser spezifizieren zu können, führt man folgende Definition ein.
\begin{definition} 
Seien $(X,<)$ eine simpliziale Fläche und $M \subseteq X_2$ eine Menge. Dann ist die Menge $N_M^X(F)$  für $F \in X_2$ definiert durch
\[
N_M^X(F):=\{ F' \in M \mid F'\text{ und F sind adjazent}\}.
\]
Falls $M=X_2$ ist, so definiert man auch $N_X^X(F):=N^X_{X_2}(F)$. Ist $X$ aus dem Kontext klar, so schreibt man $N_M^X(F):=N_M(F)$.
Für $S \subseteq X_2$ ist 
\[
N_M(S):=\bigcup_{F\in S}N_M(F).
\]

\end{definition}
\begin{bemerkung}
Sei $(X,<)$ eine geschlossene Jordan-zusammenhängende simpliziale Fläche. Dann gilt $\vert N_X(F)\vert \in \{2,3\}$ für alle $F\in X_2$.
\end{bemerkung}
\begin{lemma} \label{lemma1} 
Seien $(X,<)$  eine geschlossene Jordan-zusammenhängende simpliziale Fläche und $M \subsetneq X_2$ eine Jordan-zusammenhängende Menge. Dann existiert ein $F\in X_2\setminus M$ so, dass die Menge $M \cup \{F\}$ Jordan-zusammenhängend ist.
\end{lemma}
\begin{proof}
Man betrachte eine Fläche $F \in X_2\setminus M$, Kanten $e_i \in X_1$ und Knoten $V_i\in X_0$ mit $e_i<F$ und $V_i<e_j$ für $i\in \{1,2,3\}$ und $j \in \{1,2,3\} \setminus \{ i \}$. Für den Beweis macht man nun Gebrauch von der Färbung $f_M$. 
\begin{itemize}
\item Falls $f_M(x)=1$ für $x\in \{V_1,V_2,V_3,e_1,e_2,e_3\}$ und $f_M(F)=0$ ist, so gilt schon $M=X_2\setminus \{F\}$. Denn angenommen $\vert X_2 \setminus M \vert > 1$, dann existiert ein $F \neq F' \in X_2 \setminus M $ und somit, da $X_2 \setminus M$ stark-zusammenhängend ist, auch ein $F$-$F'$-Weg etwa $(F,F_1,\ldots,F')$ in $X_2 \setminus M$. Da aber $f_M(e_1)=f_M(e_2)=f_M(e_3)=1$ ist, gilt $f_M(F')=1$ für alle $F'\in N_X(F)$ und ist dann auch $F_1 \in M$. Daraus kann man schon schließen, dass $X_2 \setminus M$ nicht stark-zusammenhängend ist, was den gewünschten Widerspruch liefert. Also ist $M=X_2 \setminus \{F\}$ und $M \cup \{F\}=X_2$ ist dann nach Voraussetzung Jordan-zusammenhängend.
\begin{figure}[H]
\definecolor{ffffff}{rgb}{1.,1.,1.}
\definecolor{qqffqq}{rgb}{0.,1.,0.}
\definecolor{ffffqq}{rgb}{1.,1.,0.}
\definecolor{yqyqyq}{rgb}{0.5019607843137255,0.5019607843137255,0.5019607843137255}
\begin{tikzpicture}[line cap=round,line join=round,>=triangle 45,x=1.2cm,y=1.2cm]
%\begin{axis}[
x=1.2cm,y=1.2cm,
axis lines=middle,
ymajorgrids=true,
xmajorgrids=true,
xmin=-10.24,
xmax=12.76,
ymin=-4.62,
ymax=7.0200000000000005,
xtick={-8.0,-7.0,...,12.0},
ytick={-4.0,-3.0,...,7.0},]
\clip(-6.04,-1.02) rectangle (12.76,4.02);
\fill[line width=2.pt,color=yqyqyq,fill=yqyqyq,fill opacity=0.5] (-3.,-1.) -- (3.,-1.) -- (3.,0.) -- (-3.,0.) -- cycle;
\fill[line width=2.pt,color=yqyqyq,fill=yqyqyq,fill opacity=0.5] (-3.,0.) -- (-3.,4.) -- (3.,4.) -- (3.,0.) -- cycle;
\fill[line width=2.pt,color=ffffff,fill=ffffff,fill opacity=1.0] (-2.,0.) -- (2.,0.) -- (0.,3.4641016151377553) -- cycle;
\fill[line width=2.pt,color=ffffqq,fill=ffffqq,fill opacity=\gelb] (-2.,0.) -- (2.,0.) -- (0.,3.4641016151377553) -- cycle;
\draw [line width=2.pt,color=ffffff] (2.,0.)-- (0.,3.4641016151377553);
\draw [line width=2.pt,color=ffffff] (0.,3.4641016151377553)-- (-2.,0.);
\draw [line width=2.pt,color=yqyqyq] (-2.,0.)-- (2.,0.);
\draw [line width=2.pt,color=yqyqyq] (2.,0.)-- (0.,3.4641016151377553);
\draw [line width=2.pt,color=yqyqyq] (0.,3.4641016151377553)-- (-2.,0.);
\begin{scriptsize}
%\draw[color=ffffff] (0.48,1.33) node {$Vieleck1$};
\draw [fill=yqyqyq] (0.,3.4641016151377553) circle (2.5pt);
%\draw[color=black] (0.14,3.83) node {$I$};
\draw [fill=yqyqyq] (-2.,0.) circle (2.5pt);
\draw[color=black] (-2.16,0.37) node {$V_1$};
\draw [fill=yqyqyq] (2.,0.) circle (2.0pt);
\draw[color=black] (2.14,0.33) node {$V_2$};
\draw[color=black] (0.,1.33) node {$F$};
\draw[color=black] (0.1,-0.2) node {$e_3$};
\draw[color=black] (1.32,2.07) node {$e_1$};
\draw[color=black] (-1.22,2.07) node {$e_2$};
\draw [fill=yqyqyq] (0.,3.4641016151377553) circle (2.5pt);
\draw[color=black] (0.14,3.83) node {$V_3$};
\end{scriptsize}
%\end{axis}
\end{tikzpicture}
\caption{Ausschnitt gefärbte simpliziale Fläche}
\end{figure}
\item Falls $F$ eine Fläche mit der Eigenschaft $f_M(x)=1$ für $x \in \{V_1,V_2,V_3,e_1,e_2\}$ und $f_M(F)=f_M(e_3)=0$ ist, so  lässt sich zeigen, dass $M \cup \{F\}$ Jordan-zusammenhängend ist. Man zeigt zunächst den Zusammenhang von $M\cup \{F\}$. Es reicht zu zeigen, dass es $F$-$T$-Wege in $M\cup \{F\}$ für ein beliebiges $T \in M$ gibt.
Wegen $f(e_1)=1$ existiert ein $F'\in M$ mit $e_1<F'$. Für den Fall $T=F'$ liefert der Weg $(F,F')$ die Behauptung. Betrachte nun $T \neq F,F'$. Da M stark-zusammenhängend ist, existiert ein $F'$-$T$-Weg etwa $(F',F_1,\ldots,F_n,T)$ in $M$, welcher sich zu einem $F$-$T$-Weg etwa $(F,F',F_1,\ldots,F_n,T)$ erweitern lässt. Also ist $M \cup \{F\}$ stark-zusammenhängend. Bleibt nur noch nachzuweisen, dass $X_2\setminus (M\cup \{F\})$ stark-zusammenhängend ist. Sei dazu $(S,F_1,\ldots,F_n,T)$ ein $S$-$T$-Weg in $X_2 \setminus M $ für  beliebige $S,T \in X_2\setminus ( M\cup \{F\})$. Für $F_i \neq F$ mit $1 \leq i \leq n$ ist $(S,F_1,\ldots,F_n,T)$ ebenfalls ein $S$-$T$-Weg in $X_2 \setminus (M \cup \{F\})$. Sei also $F_i=F$ für ein $1< i< n$. Dann gilt wegen $\vert N_{X_2\setminus M}(F) \vert=1$ schon $F_{i-1}=F_{i+1}$ und somit kann man den $S$-$T$-Weg $(S,F_1,\ldots,F_{i-1},F_{i+2},\ldots,F_n,T)$ in $X_2\setminus M$ durch Weglassen von $F=F_i$ und $F_{i+1}$ konstruieren. Dabei setzt man zur Vereinfachung $F_{n+1}:=T$ Da $F_j\neq F$ für $j \in \{1,\ldots , n\}\setminus \{i\}$ gilt, ist $(S,F_1,\ldots,F_{i-1},F_{i+2},\ldots,F_n)$ ein $S-T$-Weg in $X_2 \setminus (M \cup \{F\})$. Also ist $M$ Jordan-zusammenhängend.
Obige Konstruktion eines Weges in $X_2 \setminus (M \cup \{F\})$ kann man auch für $F_1=F$ oder $F_n=F$ durchführen.
\begin{figure}[H]
\definecolor{ffffff}{rgb}{1.,1.,1.}
\definecolor{qqffqq}{rgb}{0.,1.,0.}
\definecolor{ffffqq}{rgb}{1.,1.,0.}
\definecolor{yqyqyq}{rgb}{0.5019607843137255,0.5019607843137255,0.5019607843137255}
\begin{tikzpicture}[line cap=round,line join=round,>=triangle 45,x=1.2cm,y=1.2cm]
%\begin{axis}[
x=1.2cm,y=1.2cm,
axis lines=middle,
ymajorgrids=true,
xmajorgrids=true,
xmin=-10.24,
xmax=12.76,
ymin=-4.62,
ymax=7.0200000000000005,
xtick={-8.0,-7.0,...,12.0},
ytick={-4.0,-3.0,...,7.0},]
\clip(-6.04,-1.02) rectangle (12.76,4.02);
\fill[line width=2.pt,color=ffffqq,fill=ffffqq,fill opacity=\gelb] (-3.,-1.) -- (3.,-1.) -- (3.,0.) -- (-3.,0.) -- cycle;
\fill[line width=2.pt,color=yqyqyq,fill=yqyqyq,fill opacity=0.5] (-3.,0.) -- (-3.,4.) -- (3.,4.) -- (3.,0.) -- cycle;
\fill[line width=2.pt,color=ffffff,fill=ffffff,fill opacity=1.0] (-2.,0.) -- (2.,0.) -- (0.,3.4641016151377553) -- cycle;
\fill[line width=2.pt,color=ffffqq,fill=ffffqq,fill opacity=\gelb] (-2.,0.) -- (2.,0.) -- (0.,3.4641016151377553) -- cycle;
\draw [line width=2.pt,color=ffffff] (2.,0.)-- (0.,3.4641016151377553);
\draw [line width=2.pt,color=ffffff] (0.,3.4641016151377553)-- (-2.,0.);
\draw [line width=2.pt] (-2.,0.)-- (2.,0.);
\draw [line width=2.pt,color=yqyqyq] (2.,0.)-- (0.,3.4641016151377553);
\draw [line width=2.pt,color=yqyqyq] (0.,3.4641016151377553)-- (-2.,0.);
\begin{scriptsize}
%\draw[color=ffffff] (0.48,1.33) node {$Vieleck1$};
\draw [fill=yqyqyq] (0.,3.4641016151377553) circle (2.5pt);
%\draw[color=black] (0.14,3.83) node {$I$};
\draw [fill=yqyqyq] (-2.,0.) circle (2.5pt);
\draw[color=black] (-2.16,0.37) node {$V_1$};
\draw [fill=yqyqyq] (2.,0.) circle (2.0pt);
\draw[color=black] (2.14,0.33) node {$V_2$};
\draw[color=black] (0.,1.33) node {$F$};
\draw[color=black] (0.1,-0.2) node {$e_3$};
\draw[color=black] (1.32,2.07) node {$e_1$};
\draw[color=black] (-1.22,2.07) node {$e_2$};
\draw [fill=yqyqyq] (0.,3.4641016151377553) circle (2.5pt);
\draw[color=black] (0.14,3.83) node {$V_3$};
\end{scriptsize}
%\end{axis}
\end{tikzpicture}
\caption{Ausschnitt gefärbte simpliziale Fläche}
\end{figure}
\item Falls $F$ eine Fläche mit der Eigenschaft $f_M(x)=0$ für $x \in \{F,e_1,e_2,V_3\}$ und $f_M(y)=1$ für $y \in \{V_1,V_2,e_3\}$ ist, so ist die Menge $M \cup \{F\}$ eine Jordan-zusammenhängende Menge. Den starken Zusammenhang von $M \cup \{F\}$ kann man mit analoger Argumentation wie im obigen Fall nachweisen. Es bleibt wieder der Zusammenhang der Menge $X_2\setminus (M \cup \{F\})$ zu zeigen.
 Diese Behauptung gilt, denn wegen $f_M(V_3)=0$ existieren nach \Cref{jordanV} $F_1,\ldots,F_n \in X_2\setminus M$ mit $V_3 <F_i$ für $1 \leq i \leq n \in \mathbb{N}$ so, dass das Tupel $(F,F_1,\ldots,F_n)$ einen Schirm bildet.
  Sei also für $n' \in \mathbb{N}$ das Tupel
   $(S,F_1',\ldots ,F_{n'}',T)$ ein $S$-$T$-Weg in $X_2 \setminus M$, wobei $S,T,F_i' \in X_2\setminus M$ und $S\neq T \neq F$ sind. 
  Falls $F_i' \neq F$ für $1 \leq i \leq n'$ ist, so ist $(S,F_1',\ldots,F_n',T)$ ebenfalls ein $S$-$T$-Weg in $X_2\setminus (M \cup \{F\})$.
   Sei also nun $F_i'=F$ für ein $1 < i < n$. Da $f_M(V_3)=f_M(e_1)=f_M(e_2)=0$ ist, folgt $f_M(F_{i-1})=f_M(F_{i+1})=0$. Außerdem kann man aus $F_{i-1}',F_{i+1}' \in X_2(V_3)$ und der Adjazenz von $F'_{i-1}$ und $F'_{i+1}$ zu der Fläche $F$ schließen, dass $(F'_{i-1},F'_{i+1}) \in \{(F_1,F_1),(F_1,F_n),(F_n,F_1),(F_n,F_n)\}$ gelten muss. Falls $(F'_{i-1},F'_{i+1}) \in \{(F_1,F_1),(F_n,F_n)\}$ ist, so lässt sich dies wie im oben behandelten Fall lösen, indem man im $S$-$T$-Weg die doppelt überlaufene Fläche $F_{i-1}=F_{i+1}$ und die Fläche $F$ weglässt. Falls $(F_{i-1}',F'_{i+1})=(F_1,F_n)$ ist, dann bildet $(S,F_1',\ldots,F_{i-1}'=F_1,F_2, \ldots,F_{i+1}'=F_n\ldots,F_n',T)$ einen $S$-$T$-Weg in $X_2\setminus (M \cup \{F\})$ und für den anderen Fall ist $(S,F_1',\ldots,F_{i-1}'=F_n,F_{n-1}, \ldots,F_{i+1}'=F_1\ldots,F_n',T)$ der gesuchte Weg, der $S$ und $T$ verbindet. Obige Konstruktion eines Weges in $X_2 \setminus (M \cup \{F\})$ kann man auch für $F_1'=F$ oder $F_n'=F$ durchführen. 
   
\end{itemize}
\begin{figure}[H]
\definecolor{ffffff}{rgb}{1.,1.,1.}
\definecolor{qqffqq}{rgb}{0.,1.,0.}
\definecolor{ffffqq}{rgb}{1.,1.,0.}
\definecolor{yqyqyq}{rgb}{0.5019607843137255,0.5019607843137255,0.5019607843137255}
\begin{tikzpicture}[line cap=round,line join=round,>=triangle 45,x=1.1cm,y=1.1cm]
%\begin{axis}[
x=1.2cm,y=1.2cm,
axis lines=middle,
ymajorgrids=true,
xmajorgrids=true,
xmin=-10.24,
xmax=12.76,
ymin=-4.62,
ymax=7.0200000000000005,
xtick={-8.0,-7.0,...,12.0},
ytick={-4.0,-3.0,...,7.0},]
\clip(-6.04,-1.02) rectangle (12.76,4.02);
\fill[line width=2.pt,color=yqyqyq,fill=yqyqyq,fill opacity=\gelb] (-3.,-1.) -- (3.,-1.) -- (3.,0.) -- (-3.,0.) -- cycle;
\fill[line width=2.pt,color=ffffqq,fill=ffffqq,fill opacity=\gelb] (-3.,0.) -- (-3.,4.) -- (3.,4.) -- (3.,0.) -- cycle;
\fill[line width=2.pt,color=ffffff,fill=ffffff,fill opacity=1.0] (-2.,0.) -- (2.,0.) -- (0.,3.4641016151377553) -- cycle;
\fill[line width=2.pt,color=ffffqq,fill=ffffqq,fill opacity=0.5] (-2.,0.) -- (2.,0.) -- (0.,3.4641016151377553) -- cycle;
\draw [line width=2.pt,color=black] (2.,0.)-- (0.,3.4641016151377553);
\draw [line width=2.pt] (0.,3.4641016151377553)-- (-2.,0.);
\draw [line width=2.pt,color=yqyqyq] (-2.,0.)-- (2.,0.);
\draw [line width=2.pt] (2.,0.)-- (0.,3.4641016151377553);
\draw [line width=2.pt] (0.,3.4641016151377553)-- (-2.,0.);
\begin{scriptsize}
%\draw[color=ffffff] (0.48,1.33) node {$Vieleck1$};
\draw [fill=yqyqyq] (0.,3.4641016151377553) circle (2.5pt);
%\draw[color=black] (0.14,3.83) node {$I$};
\draw [fill=yqyqyq] (-2.,0.) circle (2.5pt);
\draw[color=black] (-2.16,0.37) node {$V_1$};
\draw [fill=yqyqyq] (2.,0.) circle (2.0pt);
\draw[color=black] (2.14,0.33) node {$V_2$};
\draw[color=black] (0.,1.33) node {$F$};
\draw[color=black] (0.1,-0.2) node {$e_3$};
\draw[color=black] (1.32,2.07) node {$e_1$};
\draw[color=black] (-1.22,2.07) node {$e_2$};
\draw [fill=blue] (0.,3.4641016151377553) circle (2.5pt);
\draw[color=black] (0.14,3.83) node {$V_3$};
\end{scriptsize}
%\end{axis}
\end{tikzpicture}
\caption{Ausschnitt gefärbte simpliziale Fläche}
\end{figure}
Angenommen keiner der obigen Fälle tritt ein, das heißt, für $n_1:= \vert\{e <F \mid f_M(e)=1\}\vert$ und $n_2:=\vert \{V<F \mid f_M(V)=1\}\vert$ gilt $(n_1,n_2)\notin \{(3,3),(1,2),(2,3)\}$. Da $X_2$ stark-zusammenhängend und $M \subsetneq X_2$ ist, existiert ein $F_1\in X_2\setminus M$ und $f_1 \in X_1$ mit $f_1<F_1$ und $f_M(f_1)=1$. Dann ist $M \cup \{F_1\}$ stark-zusammenhängend. Falls $M \cup \{F_1\}$ Jordan-zusammenhängend ist, so ist die Behauptung gezeigt. Falls dies nicht der Fall ist, dann zerfällt $X_2 \setminus (M \cup \{F_1\})$  ohne Einschränkung in genau zwei  Zusammenhangskomponenten. Seien also $Z^{(1,1)}$ und $Z^{(1,2)}$ die Zusammenangskomponenten von $X\setminus (M \cup \{F_1\})$. Dann gilt $1\leq \vert Z^{(1,1)}\vert,\,\vert Z^{(1,2)}\vert < \vert X_2\setminus M\vert$ und man definiert $N^1:=Z^{(1,1)}$. Klarerweise ist $N^1$ stark-zusammenhängend in $X_2$. Man konstruiert sich nun eine absteigende Kette von stark-zusammenhängenden Teilmengen von $X_2\setminus M$  wie folgt:
Sei $N^i$ für $i \in \mathbb{N}$ schon mithilfe von $F_i\in X_2\setminus M$ konstruiert. Falls  $\vert N^i \vert>2$ ist, so betrachtet man die Menge $N^i\setminus \{F_{i+1}\}$ für ein $F_{i+1}\in N^i$ mit $e_{i+1}<F_{i+1}$ und $f_M(e_{i+1})=1$, wobei $e_{i+1} \in X_1$ ist. 
Falls $N^i \setminus \{F_{i+1}\}$ stark-zusammenhängend ist, dann ist man fertig, da dann $X_2\setminus (M \cup \{F_{i+1}\})$ ebenfalls stark-zusammenhängend ist.
 Falls jedoch $N^i \setminus \{F_{i+1}\}$ in zwei Zusammenhangskomponenten $Z^{(i+1,1)}$ und $Z^{(i+1,2)}$ zerfällt, so definiert man $N^{i+1}=Z^{(i+1,j)}$  für das $j \in \{1,2\}$ mit der Eigenschaft, dass $F_{i} \notin N_{X_2}(F)$ für  alle $F \in Z^{(i+1,j)}$ gilt. Offensichtlich ist $N^{i+1}$ stark-zusammenhängend und es gilt $\vert N^{i}\vert >\vert N^{i+1} \vert$. Dadurch erhält man die Kette $\vert N^1 \vert>\vert  N^2 \vert> \ldots$. Da $\vert X_2\vert < \infty$ ist, muss nach endlich vielen Schritten der Fall $\vert N^k \vert \in \{1,2\}$ für ein $k \in \mathbb{N}$ eintreten.


\begin{figure}[H]\label{37}


\definecolor{qqffqq}{rgb}{0.,1.,0.}
\definecolor{ffffff}{rgb}{1.,1.,1.}
\definecolor{ffffqq}{rgb}{1.,1.,0.}
\definecolor{yqyqyq}{rgb}{0.5019607843137255,0.5019607843137255,0.5019607843137255}
\begin{tikzpicture}[line cap=round,line join=round,>=triangle 45,x=1.0cm,y=1.0cm]
%\begin{axis}[
x=1.0cm,y=1.0cm,
axis lines=middle,
ymajorgrids=true,
xmajorgrids=true,
xmin=-4.3,
xmax=18.7,
ymin=-5.34,
ymax=6.3,
xtick={-1.0,-.0,...,18.0},
ytick={-1.0,-.0,...,4.0},]
\clip(-3.3,-1.) rectangle (10.7,4.3);
\fill[line width=2.pt,color=ffffqq,fill=ffffqq,fill opacity=0.5] (0.,0.) -- (0.,3.) -- (10.,3.) -- (10.,0.) -- cycle;
\fill[line width=2.pt,color=ffffff,fill=ffffff,fill opacity=1.0] (1.,0.) -- (1.,1.) -- (2.,1.) -- (2.,0.) -- cycle;
\fill[line width=2.pt,color=ffffff,fill=ffffff,fill opacity=1.0] (4.,0.) -- (4.,1.) -- (5.,1.) -- (5.,0.) -- cycle;
\fill[line width=2.pt,color=ffffff,fill=ffffff,fill opacity=1.0] (7.,1.) -- (7.,0.) -- (8.,0.) -- (8.,1.) -- cycle;
\fill[line width=2.pt,color=ffffff,fill=ffffff,fill opacity=1.0] (1.,3.) -- (2.,3.) -- (1.48,2.02) -- cycle;
\fill[line width=2.pt,color=ffffff,fill=ffffff,fill opacity=1.0] (4.,3.) -- (5.,3.) -- (4.46,2.02) -- cycle;
\fill[line width=2.pt,color=ffffff,fill=ffffff,fill opacity=1.0] (7.,3.) -- (7.44,2.04) -- (8.,3.) -- cycle;
\fill[line width=2.pt,color=yqyqyq,fill=yqyqyq,fill opacity=0.5] (0.,3.) -- (0.,4.) -- (10.,4.) -- (10.,3.) -- cycle;
\fill[line width=2.pt,color=yqyqyq,fill=yqyqyq,fill opacity=0.5] (0.,0.) -- (0.,-1.) -- (10.,-1.) -- (10.,0.) -- cycle;
%\fill[line width=2.pt,color=ffffqq,fill=ffffqq,fill opacity=\gelb] (1.,1.) -- (2.,1.) -- (1.46,2.02) -- cycle;
%\fill[line width=2.pt,color=ffffqq,fill=ffffqq,fill opacity=\gelb] (5.,1.) -- (4.5,1.98) -- (4.,1.) -- cycle;
%\fill[line width=2.pt,color=ffffqq,fill=ffffqq,fill opacity=\gelb] (7.,1.) -- (8.,1.) -- (7.42,2.) -- cycle;
\fill[line width=2.pt,color=yqyqyq,fill=yqyqyq,fill opacity=0.5] (2.,0.) -- (1.,0.) -- (1.,1.) -- (2.,1.) -- cycle;
\fill[line width=2.pt,color=yqyqyq,fill=yqyqyq,fill opacity=0.5] (4.,1.) -- (4.,0.) -- (5.,0.) -- (5.,1.) -- cycle;
\fill[line width=2.pt,color=yqyqyq,fill=yqyqyq,fill opacity=0.5] (7.,1.) -- (7.,0.) -- (8.,0.) -- (8.,1.) -- cycle;
\fill[line width=2.pt,color=yqyqyq,fill=yqyqyq,fill opacity=0.5] (7.42,2.) -- (8.,3.) -- (7.,3.) -- cycle;
\fill[line width=2.pt,color=yqyqyq,fill=yqyqyq,fill opacity=0.5] (4.5,1.98) -- (5.,3.) -- (4.,3.) -- cycle;
\fill[line width=2.pt,color=yqyqyq,fill=yqyqyq,fill opacity=0.5] (1.46,2.02) -- (2.,3.) -- (1.,3.) -- cycle;
\fill[line width=2.pt,color=yqyqyq,fill=yqyqyq,fill opacity=0.5] (-1.,-1.) -- (0.,-1.) -- (0.,4.) -- (-1.,4.) -- cycle;
%\draw [line width=2.pt,color=yqyqyq] (1.,1.)-- (2.,1.);
%\draw [line width=2.pt] (2.,1.)-- (1.46,2.02);
%\draw [line width=2.pt] (1.46,2.02)-- (1.,1.);
%\draw [line width=2.pt] (5.,1.)-- (4.5,1.98);
%\draw [line width=2.pt] (4.5,1.98)-- (4.,1.);
%\draw [line width=2.pt,color=yqyqyq] (4.,1.)-- (5.,1.);
\draw [line width=2.pt,color=yqyqyq] (7.,1.)-- (8.,1.);
\draw [line width=2.pt] (8.,1.)-- (7.42,2.);
\draw [line width=2.pt] (7.42,2.)-- (7.,1.);
\draw [line width=2.pt,color=yqyqyq] (8.,1.)-- (7.,1.);
\begin{scriptsize}
%\draw [fill=yqyqyq] (1.,1.) circle (2.5pt);
\draw[color=black] (7.5,1.4) node {$F_{1}$};%%%%
\draw[color=black] (4.5,1.5) node {$N^{1}$};
%\draw [fill=yqyqyq] (2.,1.) circle (2.5pt);
%\draw[color=qqffqq] (2.19,1.42) node {$G_1$};
%\draw [fill=yqyqyq] (1.46,2.02) circle (2.5pt);
%\draw[color=qqffqq] (1.65,2.44) node {$H_1$};
%\draw[color=qqffqq] (1.61,0.9) node {$h_1$};
%\draw[color=black] (2.11,1.88) node {$f_1$};
%\draw[color=black] (1.03,1.88) node {$g_1$};
%\draw [fill=yqyqyq] (4.,1.) circle (2.5pt);
%\draw[color=qqffqq] (4.19,1.42) node {$I_1$};
%\draw [fill=yqyqyq] (5.,1.) circle (2.5pt);
%\draw[color=black] (5.19,1.42) node {$J_1$};%%%%%%%
%\draw [fill=yqyqyq] (4.5,1.98) circle (2.5pt);
%\draw[color=qqffqq] (4.69,2.4) node {$K_1$};
%\draw[color=black] (5.13,1.86) node {$i_1$};
%\draw[color=black] (4.07,1.86) node {$j_1$};
%\draw[color=qqffqq] (4.61,0.9) node {$k_1$};
\draw [fill=yqyqyq] (7.,1.) circle (2.5pt);
%\draw[color=qqffqq] (7.19,1.42) node {$L_1$};
\draw [fill=yqyqyq] (8.,1.) circle (2.5pt);
%\draw[color=black] (8.19,1.42) node {$M_1$};%%%%
\draw [fill=yqyqyq] (7.42,2.) circle (2.5pt);
%\draw[color=qqffqq] (7.61,2.42) node {$N_1$};
%\draw[color=ffffqq] (7.61,0.9) node {$n_1$};
%\draw[color=black] (8.09,1.9) node {$l_1$};
%\draw[color=black] (7.01,1.86) node {$m_1$};
%\draw[color=qqffqq] (7.61,1.54) node {$m_2$};
\end{scriptsize}
%\end{axis}
\end{tikzpicture}
%---------------------------------------

\definecolor{qqffqq}{rgb}{0.,1.,0.}
\definecolor{ffffff}{rgb}{1.,1.,1.}
\definecolor{ffffqq}{rgb}{1.,1.,0.}
\definecolor{yqyqyq}{rgb}{0.5019607843137255,0.5019607843137255,0.5019607843137255}
\begin{tikzpicture}[line cap=round,line join=round,>=triangle 45,x=1.0cm,y=1.0cm]
%\begin{axis}[
x=1.0cm,y=1.0cm,
axis lines=middle,
ymajorgrids=true,
xmajorgrids=true,
xmin=-4.3,
xmax=18.7,
ymin=-5.34,
ymax=6.3,
xtick={-1.0,-.0,...,18.0},
ytick={-1.0,-.0,...,4.0},]
\clip(-3.3,-1.) rectangle (10.7,4.3);
\fill[line width=2.pt,color=ffffqq,fill=ffffqq,fill opacity=0.5] (0.,0.) -- (0.,3.) -- (10.,3.) -- (10.,0.) -- cycle;
\fill[line width=2.pt,color=ffffff,fill=ffffff,fill opacity=1.0] (1.,0.) -- (1.,1.) -- (2.,1.) -- (2.,0.) -- cycle;
\fill[line width=2.pt,color=ffffff,fill=ffffff,fill opacity=1.0] (4.,0.) -- (4.,1.) -- (5.,1.) -- (5.,0.) -- cycle;
\fill[line width=2.pt,color=ffffff,fill=ffffff,fill opacity=1.0] (7.,1.) -- (7.,0.) -- (8.,0.) -- (8.,1.) -- cycle;
\fill[line width=2.pt,color=ffffff,fill=ffffff,fill opacity=1.0] (1.,3.) -- (2.,3.) -- (1.48,2.02) -- cycle;
\fill[line width=2.pt,color=ffffff,fill=ffffff,fill opacity=1.0] (4.,3.) -- (5.,3.) -- (4.46,2.02) -- cycle;
\fill[line width=2.pt,color=ffffff,fill=ffffff,fill opacity=1.0] (7.,3.) -- (7.44,2.04) -- (8.,3.) -- cycle;
\fill[line width=2.pt,color=yqyqyq,fill=yqyqyq,fill opacity=0.5] (0.,3.) -- (0.,4.) -- (10.,4.) -- (10.,3.) -- cycle;
\fill[line width=2.pt,color=yqyqyq,fill=yqyqyq,fill opacity=0.5] (0.,0.) -- (0.,-1.) -- (10.,-1.) -- (10.,0.) -- cycle;
%\fill[line width=2.pt,color=ffffqq,fill=ffffqq,fill opacity=\gelb] (1.,1.) -- (2.,1.) -- (1.46,2.02) -- cycle;
%\fill[line width=2.pt,color=ffffqq,fill=ffffqq,fill opacity=\gelb] (5.,1.) -- (4.5,1.98) -- (4.,1.) -- cycle;
%\fill[line width=2.pt,color=ffffqq,fill=ffffqq,fill opacity=\gelb] (7.,1.) -- (8.,1.) -- (7.42,2.) -- cycle;
\fill[line width=2.pt,color=yqyqyq,fill=yqyqyq,fill opacity=0.5] (2.,0.) -- (1.,0.) -- (1.,1.) -- (2.,1.) -- cycle;
\fill[line width=2.pt,color=yqyqyq,fill=yqyqyq,fill opacity=0.5] (4.,1.) -- (4.,0.) -- (5.,0.) -- (5.,1.) -- cycle;
\fill[line width=2.pt,color=yqyqyq,fill=yqyqyq,fill opacity=0.5] (7.,1.) -- (7.,0.) -- (8.,0.) -- (8.,1.) -- cycle;
\fill[line width=2.pt,color=yqyqyq,fill=yqyqyq,fill opacity=0.5] (7.42,2.) -- (8.,3.) -- (7.,3.) -- cycle;
\fill[line width=2.pt,color=yqyqyq,fill=yqyqyq,fill opacity=0.5] (4.5,1.98) -- (5.,3.) -- (4.,3.) -- cycle;
\fill[line width=2.pt,color=yqyqyq,fill=yqyqyq,fill opacity=0.5] (1.46,2.02) -- (2.,3.) -- (1.,3.) -- cycle;
\fill[line width=2.pt,color=yqyqyq,fill=yqyqyq,fill opacity=0.5] (-1.,-1.) -- (0.,-1.) -- (0.,4.) -- (-1.,4.) -- cycle;
%\draw [line width=2.pt,color=yqyqyq] (1.,1.)-- (2.,1.);
%\draw [line width=2.pt] (2.,1.)-- (1.46,2.02);
%\draw [line width=2.pt] (1.46,2.02)-- (1.,1.);
\draw [line width=2.pt] (5.,1.)-- (4.5,1.98);
\draw [line width=2.pt] (4.5,1.98)-- (4.,1.);
\draw [line width=2.pt,color=yqyqyq] (4.,1.)-- (5.,1.);
\draw [line width=2.pt,color=yqyqyq] (7.,1.)-- (8.,1.);
\draw [line width=2.pt] (8.,1.)-- (7.42,2.);
\draw [line width=2.pt] (7.42,2.)-- (7.,1.);
\draw [line width=2.pt,color=yqyqyq] (8.,1.)-- (7.,1.);
\begin{scriptsize}
%\draw [fill=yqyqyq] (1.,1.) circle (2.5pt);
\draw[color=black] (7.5,1.4) node {$F_{1}$};%%%%
\draw[color=black] (4.5,1.4) node {$F_{2}$};
\draw[color=black] (2.5,1.5) node {$N^{2}$};
%\draw [fill=yqyqyq] (2.,1.) circle (2.5pt);
%\draw[color=qqffqq] (2.19,1.42) node {$G_1$};
%\draw [fill=yqyqyq] (1.46,2.02) circle (2.5pt);
%\draw[color=qqffqq] (1.65,2.44) node {$H_1$};
%\draw[color=qqffqq] (1.61,0.9) node {$h_1$};
%\draw[color=black] (2.11,1.88) node {$f_1$};
%\draw[color=black] (1.03,1.88) node {$g_1$};
\draw [fill=yqyqyq] (4.,1.) circle (2.5pt);
%\draw[color=qqffqq] (4.19,1.42) node {$I_1$};
\draw [fill=yqyqyq] (5.,1.) circle (2.5pt);
%\draw[color=black] (5.19,1.42) node {$J_1$};%%%%%%%
\draw [fill=yqyqyq] (4.5,1.98) circle (2.5pt);
%\draw[color=qqffqq] (4.69,2.4) node {$K_1$};
%\draw[color=black] (5.13,1.86) node {$i_1$};
%\draw[color=black] (4.07,1.86) node {$j_1$};
%\draw[color=qqffqq] (4.61,0.9) node {$k_1$};
\draw [fill=yqyqyq] (7.,1.) circle (2.5pt);
%\draw[color=qqffqq] (7.19,1.42) node {$L_1$};
\draw [fill=yqyqyq] (8.,1.) circle (2.5pt);
%\draw[color=black] (8.19,1.42) node {$M_1$};%%%%
\draw [fill=yqyqyq] (7.42,2.) circle (2.5pt);
%\draw[color=qqffqq] (7.61,2.42) node {$N_1$};
%\draw[color=ffffqq] (7.61,0.9) node {$n_1$};
%\draw[color=black] (8.09,1.9) node {$l_1$};
%\draw[color=black] (7.01,1.86) node {$m_1$};
%\draw[color=qqffqq] (7.61,1.54) node {$m_2$};
\end{scriptsize}
%\end{axis}
\end{tikzpicture}
%--------------------------------------
\definecolor{qqffqq}{rgb}{0.,1.,0.}
\definecolor{ffffff}{rgb}{1.,1.,1.}
\definecolor{ffffqq}{rgb}{1.,1.,0.}
\definecolor{yqyqyq}{rgb}{0.5019607843137255,0.5019607843137255,0.5019607843137255}
\begin{tikzpicture}[line cap=round,line join=round,>=triangle 45,x=1.0cm,y=1.0cm]
%\begin{axis}[
x=1.0cm,y=1.0cm,
axis lines=middle,
ymajorgrids=true,
xmajorgrids=true,
xmin=-4.3,
xmax=18.7,
ymin=-5.34,
ymax=6.3,
xtick={-1.0,-.0,...,18.0},
ytick={-1.0,-.0,...,4.0},]
\clip(-3.3,-1.) rectangle (10.7,4.3);
\fill[line width=2.pt,color=ffffqq,fill=ffffqq,fill opacity=0.5] (0.,0.) -- (0.,3.) -- (10.,3.) -- (10.,0.) -- cycle;
\fill[line width=2.pt,color=ffffff,fill=ffffff,fill opacity=1.0] (1.,0.) -- (1.,1.) -- (2.,1.) -- (2.,0.) -- cycle;
\fill[line width=2.pt,color=ffffff,fill=ffffff,fill opacity=1.0] (4.,0.) -- (4.,1.) -- (5.,1.) -- (5.,0.) -- cycle;
\fill[line width=2.pt,color=ffffff,fill=ffffff,fill opacity=1.0] (7.,1.) -- (7.,0.) -- (8.,0.) -- (8.,1.) -- cycle;
\fill[line width=2.pt,color=ffffff,fill=ffffff,fill opacity=1.0] (1.,3.) -- (2.,3.) -- (1.48,2.02) -- cycle;
\fill[line width=2.pt,color=ffffff,fill=ffffff,fill opacity=1.0] (4.,3.) -- (5.,3.) -- (4.46,2.02) -- cycle;
\fill[line width=2.pt,color=ffffff,fill=ffffff,fill opacity=1.0] (7.,3.) -- (7.44,2.04) -- (8.,3.) -- cycle;
\fill[line width=2.pt,color=yqyqyq,fill=yqyqyq,fill opacity=0.5] (0.,3.) -- (0.,4.) -- (10.,4.) -- (10.,3.) -- cycle;
\fill[line width=2.pt,color=yqyqyq,fill=yqyqyq,fill opacity=0.5] (0.,0.) -- (0.,-1.) -- (10.,-1.) -- (10.,0.) -- cycle;
%\fill[line width=2.pt,color=ffffqq,fill=ffffqq,fill opacity=\gelb] (1.,1.) -- (2.,1.) -- (1.46,2.02) -- cycle;
%\fill[line width=2.pt,color=ffffqq,fill=ffffqq,fill opacity=\gelb] (5.,1.) -- (4.5,1.98) -- (4.,1.) -- cycle;
%\fill[line width=2.pt,color=ffffqq,fill=ffffqq,fill opacity=\gelb] (7.,1.) -- (8.,1.) -- (7.42,2.) -- cycle;
\fill[line width=2.pt,color=yqyqyq,fill=yqyqyq,fill opacity=0.5] (2.,0.) -- (1.,0.) -- (1.,1.) -- (2.,1.) -- cycle;
\fill[line width=2.pt,color=yqyqyq,fill=yqyqyq,fill opacity=0.5] (4.,1.) -- (4.,0.) -- (5.,0.) -- (5.,1.) -- cycle;
\fill[line width=2.pt,color=yqyqyq,fill=yqyqyq,fill opacity=0.5] (7.,1.) -- (7.,0.) -- (8.,0.) -- (8.,1.) -- cycle;
\fill[line width=2.pt,color=yqyqyq,fill=yqyqyq,fill opacity=0.5] (7.42,2.) -- (8.,3.) -- (7.,3.) -- cycle;
\fill[line width=2.pt,color=yqyqyq,fill=yqyqyq,fill opacity=0.5] (4.5,1.98) -- (5.,3.) -- (4.,3.) -- cycle;
\fill[line width=2.pt,color=yqyqyq,fill=yqyqyq,fill opacity=0.5] (1.46,2.02) -- (2.,3.) -- (1.,3.) -- cycle;
\fill[line width=2.pt,color=yqyqyq,fill=yqyqyq,fill opacity=0.5] (-1.,-1.) -- (0.,-1.) -- (0.,4.) -- (-1.,4.) -- cycle;
\draw [line width=2.pt,color=yqyqyq] (1.,1.)-- (2.,1.);
\draw [line width=2.pt] (2.,1.)-- (1.46,2.02);
\draw [line width=2.pt] (1.46,2.02)-- (1.,1.);
\draw [line width=2.pt] (5.,1.)-- (4.5,1.98);
\draw [line width=2.pt] (4.5,1.98)-- (4.,1.);
\draw [line width=2.pt,color=yqyqyq] (4.,1.)-- (5.,1.);
\draw [line width=2.pt,color=yqyqyq] (7.,1.)-- (8.,1.);
\draw [line width=2.pt] (8.,1.)-- (7.42,2.);
\draw [line width=2.pt] (7.42,2.)-- (7.,1.);
\draw [line width=2.pt,color=yqyqyq] (8.,1.)-- (7.,1.);
\begin{scriptsize}
\draw [fill=yqyqyq] (1.,1.) circle (2.5pt);
\draw[color=black] (7.5,1.4) node {$F_{1}$};%%%%
\draw[color=black] (4.5,1.4) node {$F_{2}$};
\draw[color=black] (1.5,1.4) node {$F_{3}$};
\draw[color=black] (0.4,1.4) node {$N^{3}$};
\draw [fill=yqyqyq] (2.,1.) circle (2.5pt);
%\draw[color=qqffqq] (2.19,1.42) node {$G_1$};
\draw [fill=yqyqyq] (1.46,2.02) circle (2.5pt);
%\draw[color=qqffqq] (1.65,2.44) node {$H_1$};
%\draw[color=qqffqq] (1.61,0.9) node {$h_1$};
%\draw[color=black] (2.11,1.88) node {$f_1$};
%\draw[color=black] (1.03,1.88) node {$g_1$};
\draw [fill=yqyqyq] (4.,1.) circle (2.5pt);
%\draw[color=qqffqq] (4.19,1.42) node {$I_1$};
\draw [fill=yqyqyq] (5.,1.) circle (2.5pt);
%\draw[color=black] (5.19,1.42) node {$J_1$};%%%%%%%
\draw [fill=yqyqyq] (4.5,1.98) circle (2.5pt);
%\draw[color=qqffqq] (4.69,2.4) node {$K_1$};
%\draw[color=black] (5.13,1.86) node {$i_1$};
%\draw[color=black] (4.07,1.86) node {$j_1$};
%\draw[color=qqffqq] (4.61,0.9) node {$k_1$};
\draw [fill=yqyqyq] (7.,1.) circle (2.5pt);
%\draw[color=qqffqq] (7.19,1.42) node {$L_1$};
\draw [fill=yqyqyq] (8.,1.) circle (2.5pt);
%\draw[color=black] (8.19,1.42) node {$M_1$};%%%%
\draw [fill=yqyqyq] (7.42,2.) circle (2.5pt);
%\draw[color=qqffqq] (7.61,2.42) node {$N_1$};
%\draw[color=ffffqq] (7.61,0.9) node {$n_1$};
%\draw[color=black] (8.09,1.9) node {$l_1$};
%\draw[color=black] (7.01,1.86) node {$m_1$};
%\draw[color=qqffqq] (7.61,1.54) node {$m_2$};
\end{scriptsize}
%\end{axis}
\end{tikzpicture}
\caption{Konstruktion der stark-zusammenhängenden Teilmengen}
\end{figure}
 \begin{itemize}
 \item Falls $\vert N^k \vert=1$ ist, so ist $N_{N^{k-1}}(F)=\{F_{k-1}\}$ für $N^k=\{F\}$. Also ist $\vert \{e<F \mid f_M(e)=1\}\vert =2 $ und damit auch $\vert \{V\in X_0(F) \mid f_M(V)=1\} \vert=3$ und dies ist ein Widerspruch zur oben getroffenen Annahme.
 \item Falls $\vert N^k \vert=2$ ist, so ist $\vert N_{N^{k-1}}(F')\vert =1$ für ein $F' \in N^k$ und damit erhält man wieder den gewünschten Widerspruch.
\end{itemize}
\end{proof}
\begin{bemerkung}
Die Idee des obigen Beweises durch die geschickte Wahl von Flächen in einer simplizialen Fläche, eine stark-zusammenhängende Menge auf stark-zusammenhängende Teilmengen einzuschränken, soll in der Abbildung 36 angedeutet werden.
\end{bemerkung}
\begin{lemma} \label{lemma2} 
Seien $(X,<)$ eine geschlossene Jordan-zusammenhängende simpliziale Fläche und $(S,F_1,F_2,\ldots,F_n,F_{n+1}:=T)$ für $n\geq 1$ ein $S$-$T$-Weg in $X$ ohne Flächenwiederholung, wobei $F_i \in X_2$ und $1 \leq i \leq n$ ist und $S,T\in X_2$ zwei nicht benachbarte Flächen sind. 
Dann existiert eine Lochwanderungssequenz $\Sigma$ so, dass $(S,F_2, \ldots,F_n,T)$ ein $S$-$T$-Weg ohne Flächenwiederholung in $X^H_{\Sigma}$ ist.
\end{lemma}

\begin{proof} 
Für den Beweis dieser Aussage unterscheidet man drei Fälle, denn falls $S$ zu einem Knoten vom Grad zwei gehört, sind Lochwanderungen nur eingeschränkt möglich.
\begin{enumerate}
\item Sei also $(S,F_1,\ldots,F_n,T)$ für $n\geq 1$ ein $S$-$T$-Weg, wie  zuvor beschrieben. Angenommen es existiert kein Knoten $V\in X_0$ mit $X_2(V)=\{S,F_1\}$. Dann ist $\deg(V)\geq 3$ für alle $V\in X_0$ und es existieren Kanten $e_1,e_2 \in X_1$ mit $X_2(e_1)= \{S,F_1\}$ und $X_2(e_2)=\{F_1,F_2\}$, sodass $e_2 \in \mathcal{W}_F(X)$ ist und die Operation Wanderinghole mithilfe des Tupels $(S,e_2)$ anwendbar ist. Denn per Konstruktion ist $e_1 \notin X_1(S) $ und $\vert X_0(e_1) \cap X_0(S)\vert$ = 1. Dadurch erhalten wir die simpliziale Fläche $X^H_{(S,e_1)}$, in welcher $S$ und $F_2$ benachbart sind.
%-----------
\begin{figure}[H]
%\begin{center}
\definecolor{ffffff}{rgb}{1.,1.,1.}
\definecolor{qqqqff}{rgb}{0.,0.,1.}
\definecolor{ududff}{rgb}{0.30196078431372547,0.30196078431372547,1.}
\definecolor{ffffqq}{rgb}{1.,1.,0.}
\begin{tikzpicture}[line cap=round,line join=round,>=triangle 45,x=0.9cm,y=0.9cm]
%\begin{axis}[
x=1.0cm,y=1.0cm,
axis lines=middle,
ymajorgrids=true,
xmajorgrids=true,
xmin=-2.296265478762931,
xmax=13.597363682451801,
ymin=-0.7463544498652792,
ymax=7.297203960419052,
xtick={-2.0,-1.0,...,13.0},
ytick={-0.0,1.0,...,7.0},]
\clip(-3.296265478762931,-0.07463544498652792) rectangle (11.597363682451801,4.297203960419052);
\fill[line width=2.pt,color=ffffqq,fill=ffffqq,fill opacity=\gelb] (0.,0.) -- (4.,0.) -- (4.,4.) -- (0.,4.) -- cycle;
\fill[line width=2.pt,color=ffffqq,fill=ffffqq,fill opacity=\gelb] (7.,0.) -- (11.,0.) -- (11.,4.) -- (7.,4.) -- cycle;
%\fill[line width=2.pt,color=ffffqq,fill=ffffqq,fill opacity=\gelb] (0.62,1.54) -- (1.94,1.54) -- (1.28,2.6831535329954592) -- cycle;
%\fill[line width=2.pt,color=ffffqq,fill=ffffqq,fill opacity=\gelb] (1.94,1.54) -- (3.26,1.54) -- (2.6,2.6831535329954592) -- cycle;
%\fill[line width=2.pt,color=ffffqq,fill=ffffqq,fill opacity=\gelb] (1.28,2.6831535329954592) -- (1.94,1.54) -- (2.6,2.683153532995459) -- cycle;
%\fill[line width=2.pt,color=ffffqq,fill=ffffqq,fill opacity=\gelb] (7.66,1.44) -- (8.98,1.42) -- (8.33732050807569,2.57315353299546) -- cycle;
\fill[line width=2.pt,color=ffffff,fill=ffffff,fill opacity=1.0] (8.98,1.42) -- (10.3,1.44) -- (9.62267949192431,2.573153532995459) -- cycle;
\fill[line width=2.pt,color=ffffqq] (9.62267949192431,2.573153532995459) -- (8.33732050807569,2.57315353299546) -- (8.98,1.46) -- cycle;
\fill[line width=2.pt,color=ffffff,fill=ffffff,fill opacity=1.0] (8.33732050807569,2.57315353299546) -- (9.62267949192431,2.573153532995459) -- (10.3,1.44) -- (8.98,1.42) -- cycle;
\fill[line width=2.pt,color=ffffqq,fill=ffffqq,fill opacity=\gelb] (8.33732050807569,2.57315353299546) -- (10.3,1.44) -- (8.98,1.42) -- cycle;
\fill[line width=2.pt,color=ffffqq,fill=ffffqq,fill opacity=\gelb] (8.33732050807569,2.57315353299546) -- (9.62267949192431,2.573153532995459) -- (10.3,1.44) -- cycle;
\draw [line width=2.pt] (0.62,1.54)-- (1.94,1.54);
\draw [line width=2.pt] (1.94,1.54)-- (1.28,2.6831535329954592);
\draw [line width=2.pt] (1.28,2.6831535329954592)-- (0.62,1.54);
\draw [line width=2.pt] (1.94,1.54)-- (3.26,1.54);
\draw [line width=2.pt] (3.26,1.54)-- (2.6,2.6831535329954592);
\draw [line width=2.pt] (2.6,2.6831535329954592)-- (1.94,1.54);
\draw [line width=2.pt] (1.28,2.6831535329954592)-- (1.94,1.54);
\draw [line width=2.pt] (1.94,1.54)-- (2.6,2.683153532995459);
\draw [line width=2.pt] (2.6,2.683153532995459)-- (1.28,2.6831535329954592);
\draw [line width=2.pt] (5.,2.)-- (6.,2.);
\draw [line width=2.pt] (6.,2.)-- (5.86,2.28);
\draw [line width=2.pt] (5.86,2.28)-- (6.,2.);
\draw [line width=2.pt] (6.,2.)-- (5.86,1.72);
\draw [line width=2.pt] (7.66,1.44)-- (8.98,1.42);
\draw [line width=2.pt] (8.98,1.42)-- (8.33732050807569,2.57315353299546);
\draw [line width=2.pt] (8.33732050807569,2.57315353299546)-- (7.66,1.44);
\draw [line width=2.pt] (8.98,1.42)-- (10.3,1.44);
\draw [line width=2.pt,color=ffffff] (10.3,1.44)-- (9.62267949192431,2.573153532995459);
\draw [line width=2.pt] (9.62267949192431,2.573153532995459)-- (8.33732050807569,2.57315353299546);
\draw [line width=2.pt] (8.33732050807569,2.57315353299546)-- (8.98,1.46);
\draw [line width=2.pt] (8.33732050807569,2.57315353299546)-- (9.62267949192431,2.573153532995459);
\draw [line width=2.pt] (9.62267949192431,2.573153532995459)-- (10.3,1.44);
\draw [line width=2.pt] (10.3,1.44)-- (8.98,1.42);
\draw [line width=2.pt] (8.98,1.42)-- (8.33732050807569,2.57315353299546);
\draw [line width=2.pt] (8.33732050807569,2.57315353299546)-- (10.3,1.44);
\draw [line width=2.pt] (10.3,1.44)-- (8.98,1.42);
\draw [line width=2.pt] (8.98,1.42)-- (8.33732050807569,2.57315353299546);
\draw [line width=2.pt] (8.33732050807569,2.57315353299546)-- (9.62267949192431,2.573153532995459);
\draw [line width=2.pt] (9.62267949192431,2.573153532995459)-- (10.3,1.44);
\draw [line width=2.pt] (10.3,1.44)-- (8.33732050807569,2.57315353299546);
\begin{scriptsize}
\draw[color=black] (1.919797251537696,2.221409072701729) node {$F_1$};
\draw[color=black] (9.326814629568887,1.7121409072701729) node {$S$};
\draw [fill=ududff] (0.62,1.54) circle (2.5pt);
%\draw[color=ududff] (0.716613788319514,1.7897159423807258) node {$I$};
\draw [fill=ududff] (1.94,1.54) circle (2.5pt);
%\draw[color=ududff] (2.0433863096035263,1.7897159423807258) node {$J$};
\draw[color=black] (1.28043849728318963,1.904831165782173) node {$F_2$};
\draw [fill=qqqqff] (1.28,2.6831535329954592) circle (2.5pt);
%\draw[color=qqqqff] (1.3800000489615203,2.936821351407529) node {$K$};
\draw [fill=ududff] (3.26,1.54) circle (2.5pt);
%\draw[color=ududff] (3.3563382837908304,1.7897159423807258) node {$L$};
\draw[color=black] (2.620547586323667,1.904831165782173) node {$S$};
\draw[color=black] (2.55547586323667,2.17604831165782173) node {$e_1$};
\draw[color=black] (1.387586323667,2.19604831165782173) node {$e_2$};
\draw [fill=qqqqff] (2.6,2.6831535329954592) circle (2.5pt);
%\draw[color=qqqqff] (2.692952023148824,2.936821351407529) node {$M$};
%\draw[color=black] (2.8311574941159092,2.9506418985042377) node {$Vieleck5$};
\draw [fill=qqqqff] (2.6,2.683153532995459) circle (2.5pt);
%\draw[color=qqqqff] (2.692952023148824,2.936821351407529) node {$N$};
%\draw[color=black] (5.5399847250707674,1.9002803191543935) node {$d$};
\draw[color=black] (5.3161909344422648,2.3701789204424815) node {$(S,e_2)$};
%\draw[color=black] (5.795664846359874,2.1974220817336256) node {$f_1$};
%\draw[color=black] (5.795664846359874,2.114498799153375) node {$g_1$};
\draw [fill=ududff] (7.66,1.44) circle (2.5pt);
%\draw[color=ududff] (7.7512722605441216,1.6929721127037665) node {$S$};
\draw [fill=ududff] (8.98,1.42) circle (2.5pt);
%\draw[color=ududff] (9.078044781828133,1.679151565607058) node {$T$};
\draw[color=black] (8.349607821830172,1.941741960444519) node {$F_2$};
\draw [fill=qqqqff] (8.33732050807569,2.57315353299546) circle (2.5pt);
%\draw[color=qqqqff] (8.428479068282837,2.826256974633861) node {$U$};
\draw [fill=ududff] (10.3,1.44) circle (2.5pt);
%\draw[color=ududff] (10.390996756015438,1.6929721127037665) node {$V$};
%\draw[color=black] (9.962559796017477,1.941741960444519) node {$Vieleck7$};
\draw [fill=qqqqff] (9.62267949192431,2.573153532995459) circle (2.5pt);
%\draw[color=qqqqff] (9.713789948276723,2.826256974633861) node {$W$};
\draw[color=black] (9.39917353537547,2.328717279152356) node {$F_1$};
\draw [fill=qqqqff] (8.98,1.46) circle (2.5pt);
%\draw[color=qqqqff] (9.078044781828133,1.7206132068971833) node {$Z$};
%\draw[color=black] (9.008942046344592,2.4807432972161494) node {$u$};
%\draw[color=black] (9.810533777953681,2.010844695928061) node {$w$};
%\draw[color=black] (9.672328306986598,1.7758953952840173) node {$v$};
%\draw[color=black] (8.919108490215987,2.266524817217168) node {$t_1$};
%\draw[color=black] (9.278442714730406,1.97629332818629) node {$t_2$};
%\draw[color=black] (9.70687967472837,1.8104467630257883) node {$u_1$};
%\draw[color=black] (8.919108490215987,2.266524817217168) node {$v_1$};
%\draw[color=black] (9.043493414086363,2.5152946649579206) node {$v_2$};
%\draw[color=black] (9.845085145695453,2.0453960636698323) node {$u_2$};
%\draw[color=black] (9.499571468277743,2.3632686468941273) node {$w_1$};
\end{scriptsize}
%\end{axis}
\end{tikzpicture}
\caption{Anwendung von Wanderinghole}
%\end{center}
\end{figure}
%-----------
\item Angenommen es existiert ein Knoten $V\in X_0$, sodass $X_2(V)=\{S,F_1\}$ ist. Dann ist $\deg(V)=2$ und $F_1$ ist adjazent zu $F_2$, deshalb existiert eine Kante $e\in X_1$ mit $X_2(e)=\{F_1,S\}$, sodass mithilfe des Tupels $(F_2,e)$ die Operation Wanderinghole anwendbar ist. In $Y=X^H_{(F_2,e)}$ sind $S$ und $F_2$ adjazent und dann existiert also ein Knoten $V'\in Y_0$ mit $Y_2(V')=\{S,F_1,F_2\}$. Also ist  $X_{\Sigma}^H$ eine simpliziale Fläche, in der $(S,F_2, \ldots,F_n)$ ein $S$-$T$-Weg ohne Flächenwiederholung ist, wobei $\Sigma$ eine Lochwanderungssequenz ist.

\end{enumerate}
\end{proof}

\begin{bemerkung}\label{bemnachbar}
Induktiv kann also aus dem $S$-$T$-Weg $(S,F_1,\ldots,F_n,T)$ durch die Anwendung der Operation Wanderinghole mithilfe einer Lochwanderungssequenz $\Sigma$ eine Fläche $X^H_{\Sigma}$ konstruiert werden, in welcher $S$ und $T$ benachbart sind.
\end{bemerkung}
\begin{vor} \label{Mnachbar}
Seien $(X,<)$ eine geschlossene Jordan-zusammenhängende simpliziale Fläche und $M\subseteq X_2$ eine Jordan-zusammenhängende Menge so, dass Flächen $F_1,F_2 \in X_2\setminus M$, $F_1',F_2' \in M$ und Kanten $e,e_1,e_2\in X_1$ mit $X_2(e)=\{F_1,F_2\},X_2(e_1)=\{F_1,F_1'\}$ und $X_2(e_1)=\{F_2,F_2'\}$ existieren. 
\begin{figure}[H]
\begin{center}
\definecolor{qqqqff}{rgb}{0.,0.,1.}
\definecolor{zzttqq}{rgb}{0.6,0.2,0.}
\definecolor{ffffqq}{rgb}{1.,1.,0.}
\definecolor{xdxdff}{rgb}{0.49019607843137253,0.49019607843137253,1.}
\definecolor{ududff}{rgb}{0.30196078431372547,0.30196078431372547,1.}
\definecolor{yqyqyq}{rgb}{0.5019607843137255,0.5019607843137255,0.5019607843137255}
\begin{tikzpicture}[line cap=round,line join=round,>=triangle 45,x=1.0cm,y=1.0cm]%
%\begin{axis}[
x=1.0cm,y=1.0cm,
axis lines=middle,
ymajorgrids=true,
xmajorgrids=true,
xmin=-4.66,
xmax=6.7,
ymin=-0.47999999999999865,
ymax=4.579999999999999,
xtick={-4.0,-3.0,...,6.0},
ytick={-0.0,1.0,...,4.0},]
\clip(-4.66,-0.48) rectangle (6.7,4.58);
%\fill[line width=2.pt,color=ffffqq,fill=ffffqq,fill opacity=\gelb] (-4.,0.) -- (0.,0.) -- (0.,4.) -- (-4.,4.) -- cycle;
\fill[line width=2.pt,color=yqyqyq,fill=yqyqyq,fill opacity=0.5] (-4.,0.) -- (0.,0.) --(0,2.5026794919243107) --(-0.47115427318801095,2.5026794919243107) -- (-2.04,1.62)--(-3.5888457268119893,2.537320508075689) --(-4,2.537320508075689)-- cycle;
\fill[line width=2.pt,color=ffffqq,fill=ffffqq,fill opacity=\gelb] (-4.,4.) -- (0.,4.) --(0,2.5026794919243107) --(-0.47115427318801095,2.5026794919243107) -- (-2.04,1.62)--(-3.5888457268119893,2.537320508075689) --(-4,2.537320508075689)-- cycle;

\fill[line width=2.pt,color=yqyqyq,fill=yqyqyq,fill opacity=0.5] (-4.+6,0.) -- (0.+6,0.) --(0+6,2.5026794919243107) --(-0.47115427318801095+6,2.5026794919243107) -- (-2.04+6,1.62)--(-3.5888457268119893+6,2.537320508075689) --(-4+6,2.537320508075689)-- cycle;
\fill[line width=2.pt,color=ffffqq,fill=ffffqq,fill opacity=\gelb] (-4.+6,4.) -- (0.+6,4.) --(0+6,2.5026794919243107) --(-0.47115427318801095+6,2.5026794919243107) -- (-2.04+6,1.62)--(-3.5888457268119893+6,2.537320508075689) --(-4+6,2.537320508075689)-- cycle;

%\fill[line width=2.pt,color=ffffqq,fill=ffffqq,fill opacity=\gelb] (2.,0.) -- (6.,0.) -- (6.,4.) -- (2.,4.) -- cycle;
%\fill[line width=2.pt,color=zzttqq,fill=zzttqq,fill opacity=0.10000000149011612] (-2.02,3.42) -- (-2.04,1.62) -- (-0.47115427318801095,2.5026794919243107) -- cycle;
%\fill[line width=2.pt,color=zzttqq,fill=zzttqq,fill opacity=0.10000000149011612] (-2.04,1.62) -- (-2.02,3.42) -- (-3.5888457268119893,2.537320508075689) -- cycle;
%\fill[line width=2.pt,color=zzttqq,fill=zzttqq,fill opacity=0.10000000149011612] (-2.04,1.62) -- (-3.5888457268119893,2.537320508075689) -- (-3.6088457268119893,0.7373205080756889) -- cycle;
%\fill[line width=2.pt,color=zzttqq,fill=zzttqq,fill opacity=0.10000000149011612] (-0.47115427318801095,2.5026794919243107) -- (-2.04,1.62) -- (-0.49115427318801175,0.7026794919243109) -- cycle;
%\draw [line width=2.pt,color=ffffqq] (0.,0.)-- (0.,4.);
%\draw [line width=2.pt,color=ffffqq] (0.,4.)-- (-4.,4.);
%\draw [line width=2.pt,color=ffffqq] (-4.,4.)-- (-4.,0.);
%\draw [line width=2.pt,color=ffffqq] (2.,0.)-- (6.,0.);
%\draw [line width=2.pt,color=ffffqq] (6.,0.)-- (6.,4.);
%\draw [line width=2.pt,color=ffffqq] (6.,4.)-- (2.,4.);
%\draw [line width=2.pt,color=ffffqq] (2.,4.)-- (2.,0.);
\draw [line width=2.pt] (-2.02,3.42)-- (-2.04,1.62);
\draw [line width=2.pt,color=yqyqyq] (-2.04,1.62)-- (-0.47115427318801095,2.5026794919243107);
\draw [line width=2.pt] (-0.47115427318801095,2.5026794919243107)-- (-2.02,3.42);
\draw [line width=2.pt] (-2.04,1.62)-- (-2.02,3.42);
\draw [line width=2.pt] (-2.02,3.42)-- (-3.5888457268119893,2.537320508075689);
\draw [line width=2.pt,color=yqyqyq] (-3.5888457268119893,2.537320508075689)-- (-2.04,1.62);
\draw [line width=2.pt,color=yqyqyq] (-2.04,1.62)-- (-3.5888457268119893,2.537320508075689);
\draw [line width=2.pt,color=yqyqyq] (-3.5888457268119893,2.537320508075689)-- (-3.6088457268119893,0.7373205080756889);
\draw [line width=2.pt,color=yqyqyq] (-3.6088457268119893,0.7373205080756889)-- (-2.04,1.62);
\draw [line width=2.pt] (-0.47115427318801095,2.5026794919243107)-- (-2.04,1.62);
\draw [line width=2.pt,color=yqyqyq] (-2.04,1.62)-- (-0.49115427318801175,0.7026794919243109);
\draw [line width=2.pt,color=yqyqyq] (-0.49115427318801175,0.7026794919243109)-- (-0.47115427318801095,2.5026794919243107);
\draw [line width=2.pt] (0.3,1.96)-- (1.54,1.96);
\draw [line width=2.pt] (1.54,1.96)-- (1.38,2.22);
\draw [line width=2.pt] (1.54,1.96)-- (1.4,1.66);
%-----------------------------------
%\draw [line width=2.pt] (-2.02+6,3.42)-- (-2.04+6,1.62);
\draw [line width=2.pt,color=yqyqyq] (-2.04+6,1.62)-- (-0.47115427318801095+6,2.5026794919243107);
\draw [line width=2.pt] (-0.47115427318801095+6,2.5026794919243107)-- (-2.02+6,3.42);
\draw [line width=2.pt] (-0.47115427318801095+6,2.5026794919243107)-- (-3.5888457268119893+6,2.537320508075689);
%\draw [line width=2.pt] (-2.04+6,1.62)-- (-2.02+6,3.42);
\draw [line width=2.pt] (-2.02+6,3.42)-- (-3.5888457268119893+6,2.537320508075689);
\draw [line width=2.pt] (-3.5888457268119893+6,2.537320508075689)-- (-2.04+6,1.62);
\draw [line width=2.pt,color=yqyqyq] (-2.04+6,1.62)-- (-3.5888457268119893+6,2.537320508075689);
\draw [line width=2.pt,color=yqyqyq] (-3.5888457268119893+6,2.537320508075689)-- (-3.6088457268119893+6,0.7373205080756889);
\draw [line width=2.pt,color=yqyqyq] (-3.6088457268119893+6,0.7373205080756889)-- (-2.04+6,1.62);
\draw [line width=2.pt,color=yqyqyq] (-0.47115427318801095+6,2.5026794919243107)-- (-2.04+6,1.62);
\draw [line width=2.pt,color=yqyqyq] (-2.04+6,1.62)-- (-0.49115427318801175+6,0.7026794919243109);
\draw [line width=2.pt,color=yqyqyq] (-0.49115427318801175+6,0.7026794919243109)-- (-0.47115427318801095+6,2.5026794919243107);
%\draw [line width=2.pt] (0.3+6,1.96)-- (1.54+6,1.96);
%\draw [line width=2.pt] (1.54+6,1.96)-- (1.38+6,2.22);
%\draw [line width=2.pt] (1.54+6,1.96)-- (1.4+6,1.66);

\begin{scriptsize}

%\draw[color=black] (-1.52,2.17) node {$Vieleck1$};
%\draw[color=black] (4.48,2.17) node {$Vieleck2$};
\draw [fill=ududff] (-2.02,3.42) circle (2.5pt);
%\draw[color=black] (-1.88,3.79) node {$I$};
\draw [fill=yqyqyq] (-2.04,1.62) circle (2.5pt);
\draw[color=black] (-1.4,2.5) node {$F_2$};
\draw[color=black] (-1.8,2.1) node {$e_2$};
\draw [fill=yqyqyq] (-0.47115427318801095,2.5026794919243107) circle (2.5pt);
\draw[color=black] (-2.6,2.5) node {$F_1$};
\draw [fill=yqyqyq] (-3.5888457268119893,2.537320508075689) circle (2.5pt);
\draw [fill=yqyqyq] (-3.5888457268119893+6,2.537320508075689) circle (2.5pt);
\draw[color=black] (-3.,1.71) node {$F_1'$};
\draw[color=black] (-3.+6,1.71) node {$F_1'$};
\draw [fill=yqyqyq] (-3.6088457268119893,0.7373205080756889) circle (2.5pt);
\draw [fill=yqyqyq] (-3.6088457268119893+6.,0.7373205080756889) circle (2.5pt);
\draw[color=black] (-1.,1.71) node {$F_2'$};
\draw[color=black] (-1.+6,1.71) node {$F_2'$};
\draw [fill=yqyqyq] (-0.49115427318801175,0.7026794919243109) circle (2.5pt);
\draw [fill=yqyqyq] (-0.49115427318801175+6,0.7026794919243109) circle (2.5pt);
%--------------------------
\draw [fill=ududff] (-2.02+6,3.42) circle (2.5pt);
\draw [fill=yqyqyq] (-2.04+6,1.62) circle (2.5pt);
\draw [fill=yqyqyq] (-0.47115427318801095+6,2.5026794919243107) circle (2.5pt);
\draw[color=black] (-2.0+6,2.81) node {$F_2$};
\draw[color=black] (-2.0+6,2.1) node {$F_1$};
\draw [fill=yqyqyq] (-3.5888457268119893+6,2.537320508075689) circle (2.5pt);
\draw [fill=yqyqyq] (-3.6088457268119893+6.,0.7373205080756889) circle (2.5pt);
\draw [fill=yqyqyq] (-0.49115427318801175+6,0.7026794919243109) circle (2.5pt);

\end{scriptsize}
%\end{axis}
\end{tikzpicture}
\end{center}
\caption{Anwendung von Wanderinghole}
\end{figure}

Damit ist die simpliziale Fläche $Y=X_{(F_1,e_2)}$ wohldefiniert und es gilt $N_M^Y(F_i')=N_M^X(F_i')$ für $i=1,2$.
\\ Hiermit soll angedeutet werden, dass für eine Lochwanderung $(F,f)$, wobei $F\in X_2\setminus M$ und $f\in X_2(\tilde{F})$ für ein $\tilde{F} \in X_2\setminus M$ ist, die Gleichheit $N_M^Z(F')=N_M^X(F')$ für alle $F'\in M$ gilt. Hierbei ist $Z=X^H_{(F,f)}$.
\end{vor}

\begin{lemma}
Seien $(X,<)$ eine geschlossene Jordan-zusammenhängende simpliziale Fläche mit $\chi(X)=2$, $M \subsetneq X_2$ eine Jordan-zusammenhängende Teilmenge der Flächen und $F \in X_2\setminus M$ eine Fläche in $X$. Dann existiert eine Lochwanderungssequenz $\Sigma$ so, dass $M \cup \{F\}$ Jordan-zusammenhängend in $X^H_{\Sigma}$ ist.
\end{lemma}
\begin{proof}
Sei $F_M \in M$ eine Fläche in $M$ so, dass $F_M$ adjazent zu einem $F'\in X_2\setminus M$ ist. Da $X_2\setminus M$ stark-zusammenhängend ist, existiert ein $F$-$F'$-Weg in $X_2\setminus M$ ohne Flächenwiederholung und damit auch ein $F$-$F_M$-Weg in $X$. Wegen \Cref{lemma2} und \Cref{bemnachbar} existiert also eine Lochwanderungssequenz $\Sigma$, sodass $F_M$ und $F$ in $Y=X_{\Sigma}^H$ benachbart sind. 
Also ist $M\cup \{F\}$ stark zusammenhängend in $X$. 
 Deshalb bleibt nur noch der Zusammenhang von $X_2\setminus (M \cup \{F\})$ zu betrachten. 
Falls  stark-zusammenhängend $Y$ ist, so ist die Behauptung gezeigt. Falls dies nicht der Fall ist, so muss nach \Cref{Mnachbar} eine Fläche $\tilde{F}\in M$ existieren, die $X_2\setminus M$ in zwei Zusammenhangskomponenten aufspaltet. Das heißt, $X_2\setminus M \cup \{\tilde{F}\}$ ist nicht mehr stark zusammenhängend. Somit gilt aber $\vert N_{X_2\setminus M}(\tilde{F})\vert=2$. Also existieren zu $\tilde{F}$ adjazente $F_1,F_2 \in X_2$, die $f_M(F_1)=f_M(F_2)=0$ erfüllen. Desweiteren existieren $e_1,e_2 \in Y_1$ mit $X_2(e_1)=\{F_2, \tilde{F}\}$ und $X_2(e_2)=\{F_1, \tilde{F}\}$. 
\begin{figure}[H]
\begin{center}
\definecolor{wqwqwq}{rgb}{0.3764705882352941,0.3764705882352941,0.3764705882352941}
\definecolor{aqaqaq}{rgb}{0.6274509803921569,0.6274509803921569,0.6274509803921569}
\definecolor{cqcqcq}{rgb}{0.7529411764705882,0.7529411764705882,0.7529411764705882}
\definecolor{yqyqyq}{rgb}{0.5019607843137255,0.5019607843137255,0.5019607843137255}
\definecolor{qqqqff}{rgb}{0.,0.,1.}
\definecolor{uququq}{rgb}{0.25098039215686274,0.25098039215686274,0.25098039215686274}
\definecolor{ffffff}{rgb}{1.,1.,1.}
\definecolor{ududff}{rgb}{0.30196078431372547,0.30196078431372547,1.}
\definecolor{ffffqq}{rgb}{1.,1.,0.}
\definecolor{yqyqyq}{rgb}{0.5019607843137255,0.5019607843137255,0.5019607843137255}
\begin{tikzpicture}[line cap=round,line join=round,>=triangle 45,x=1.0cm,y=1.0cm]
%\begin{axis}[
x=1.0cm,y=1.0cm,
axis lines=middle,
ymajorgrids=true,
xmajorgrids=true,
xmin=-4.3,
xmax=18.7,
ymin=-5.34,
ymax=6.3,
xtick={-4.0,-3.0,...,18.0},
ytick={-5.0,-4.0,...,6.0},]
\clip(-1.3,-.34) rectangle (5.3,4.3);
\fill[line width=2.pt,color=ffffqq,fill=ffffqq,fill opacity=0.6000000238418579] (-1.,0.) -- (5.,0.) -- (5.,4.) -- (-1.,4.) -- cycle;
\fill[line width=2.pt,color=ffffff,fill=ffffff,fill opacity=1.0] (1.,1.) -- (3.,1.) -- (2.,2.7320508075688776) -- cycle;
\fill[line width=2.pt,color=ffffff,fill=ffffff,fill opacity=1.0] (-1.,1.) -- (5.,1.) -- (5.,0.) -- (-1.,0.) -- cycle;
\fill[line width=2.pt,color=ffffff,fill=ffffff,fill opacity=1.0] (-1.,4.) -- (2.,2.7320508075688776) -- (5.,4.) -- cycle;
\fill[line width=2.pt,color=yqyqyq,fill=yqyqyq,fill opacity=0.5] (-1.,4.) -- (2.,2.7320508075688776) -- (5.,4.) -- cycle;
\fill[line width=2.pt,color=yqyqyq,fill=yqyqyq,fill opacity=0.5] (1.,1.) -- (2.,2.7320508075688776) -- (3.,1.) -- cycle;
\fill[line width=2.pt,color=yqyqyq,fill=yqyqyq,fill opacity=0.5] (-1.,1.) -- (-1.,0.) -- (5.,0.) -- (5.,1.) -- cycle;
%\fill[line width=2.pt,color=ffffqq,fill=ffffqq,fill opacity=\gelb] (1.,1.) -- (2.,2.7320508075688776) -- (0.,2.7320508075688785) -- cycle;
%\fill[line width=2.pt,color=ffffqq,fill=ffffqq,fill opacity=\gelb] (2.,2.7320508075688776) -- (3.,1.) -- (4.,2.7320508075688763) -- cycle;
%\draw [line width=2.pt,color=ffffff] (1.,1.)-- (3.,1.);
%\draw [line width=2.pt,color=ffffff] (3.,1.)-- (2.,2.7320508075688776);
\draw [line width=2.pt,color=uququq] (2.,2.7320508075688776)-- (1.,1.);
%\draw [line width=2.pt,color=ffffff] (0.,1.)-- (4.,1.);
%\draw [line width=2.pt,color=ffffff] (4.,1.)-- (4.,0.);
%\draw [line width=2.pt,color=ffffff] (4.,0.)-- (0.,0.);
%\draw [line width=2.pt,color=ffffff] (0.,0.)-- (0.,1.);
%\draw [line width=2.pt,color=ffffff] (0.,4.)-- (2.,2.7320508075688776);
%\draw [line width=2.pt,color=ffffff] (2.,2.7320508075688776)-- (4.,4.);
%\draw [line width=2.pt,color=ffffff] (4.,4.)-- (0.,4.);
%\draw [line width=2.pt,color=yqyqyq] (0.,4.)-- (2.,2.7320508075688776);
%\draw [line width=2.pt,color=yqyqyq] (2.,2.7320508075688776)-- (4.,4.);
%\draw [line width=2.pt,color=yqyqyq] (4.,4.)-- (0.,4.);
\draw [line width=2.pt,color=black] (1.,1.)-- (2.,2.7320508075688776);
\draw [line width=2.pt,color=black] (2.,2.7320508075688776)-- (3.,1.);
\draw [line width=2.pt,color=cqcqcq] (3.,1.)-- (1.,1.);
%\draw [line width=2.pt,color=aqaqaq] (0.,1.)-- (0.,0.);
%\draw [line width=2.pt,color=aqaqaq] (0.,0.)-- (4.,0.);
%\draw [line width=2.pt,color=aqaqaq] (4.,0.)-- (4.,1.);
\draw [line width=2.pt,color=yqyqyq] (3.,1.)-- (1.,1.);
\draw [line width=2.pt,color=black] (1.,1.)-- (2.,2.7320508075688776);
\draw [line width=2.pt] (2.,2.7320508075688776)-- (0.,2.7320508075688785);
\draw [line width=2.pt] (0.,2.7320508075688785)-- (1.,1.);
\draw [line width=2.pt,color=black] (2.,2.7320508075688776)-- (3.,1.);
\draw [line width=2.pt] (3.,1.)-- (4.,2.7320508075688763);
\draw [line width=2.pt] (4.,2.7320508075688763)-- (2.,2.7320508075688776);
\begin{scriptsize}
%\draw[color=ffffqq] (2.48,2.17) node {$Vieleck1$};
\draw [fill=yqyqyq] (1.,1.) circle (2.5pt);
%\draw[color=ududff] (1.14,1.37) node {$E$};
\draw [fill=yqyqyq] (3.,1.) circle (2.5pt);
%\draw[color=black] (3.14,1.37) node {$F$};
\draw[color=black] (2.0,1.55) node {$\tilde{F}$};
\draw[color=black] (1.67,1.75) node {$e_1$};
\draw[color=black] (2.4,1.75) node {$e_2$};
\draw [fill=yqyqyq] (2.,2.7320508075688776) circle (2.5pt);
%\draw[color=qqqqff] (2.14,3.11) node {$G$};
%\draw[color=ffffff] (2.11,0.9) node {$h_1$};
%\draw[color=ffffff] (3.79,0.72) node {$i_1$};
%\draw[color=ffffff] (2.06,0.49) node {$b$};
%\draw[color=ffffff] (0.38,0.67) node {$a$};
%\draw[color=ffffff] (0.88,3.27) node {$c$};
%\draw[color=ffffff] (3.22,3.27) node {$d$};
%\draw[color=ffffff] (2.11,4.54) node {$g_1$};
%\draw[color=yqyqyq] (0.93,3.32) node {$c_1$};
%\draw[color=yqyqyq] (3.27,3.32) node {$d_1$};
%\draw[color=yqyqyq] (2.11,4.54) node {$g_2$};
%\draw[color=cqcqcq] (1.87,1.94) node {$f_1$};
%\draw[color=aqaqaq] (2.28,1.89) node {$e$};
%\draw[color=cqcqcq] (2.11,1.54) node {$g_3$};
%\draw[color=aqaqaq] (-0.21,0.72) node {$h_2$};
%\draw[color=aqaqaq] (2.11,-0.1) node {$a_1$};
%\draw[color=aqaqaq] (4.43,0.72) node {$b_1$};
%\draw[color=aqaqaq] (2.11,1.54) node {$i_2$};
\draw[color=black] (1.08,2.) node {$F_2$};
\draw [fill=qqqqff] (0.,2.7320508075688785) circle (2.5pt);
%\draw[color=qqqqff] (0.14,3.11) node {$J$};
\draw[color=black] (3.08,2.) node {$F_1$};
\draw [fill=qqqqff] (4.,2.7320508075688763) circle (2.5pt);
%\draw[color=qqqqff] (4.14,3.11) node {$K$};
\end{scriptsize}
%\end{axis}
\end{tikzpicture}
\end{center}
\caption{trennende Fläche in einer simplizialen Fläche}
\end{figure}
Damit ist $Y_{(F_1,e_1)}$ eine simpliziale Fläche, in der $F_1,F_2$ adjazent zueinander sind und dadurch erhält man den Zusammenhang von $X_2\setminus (M \cup \{F\})$. Nach \\\Cref{Mnachbar} bleibt der Zusammenhang der Menge $M \cup \{F\}$ trotz der letzten Lochwanderung erhalten. 
\end{proof}

\begin{vor} \label{vor2}
Seien $(X,<)$ eine geschlossene Jordan-zusammenhängende simpliziale Fläche, $V\in X_0$ ein Knoten und $F_1,\ldots,F_n \in X_2$ Flächen in $X$ so, dass $(F_1,\ldots,F_n)$ den zu $V$ gehörigen Schirm bildet. Sei außerdem $1 \leq j<k\leq n$ für $j,k \in \mathbb{N}$. Falls $F_i=F_j$ und $F_{i+1}=F_k$ für ein $1 \leq i < n$ oder $F_1=F_j$ und $F_n=F_k$ gilt, so sind $F_j$ und $F_k$ bereits benachbarte Flächen. Andernfalls bildet $(F_j,F_{j+1},\ldots,F_k)$ einen $F_j$-$F_k$-Weg in $X_2$. Für $F_j$ und $F_{j+1}$ existiert eine Kante $f \in X_1$ mit $f<F_{i+1}$ und $f \nless F_j,F_{j+2}$. Damit ist $f \in \mathcal{W}_{F_j}(X)$ und $X^H_{(F_j,f)}$ ist eine simpliziale Fläche mit einem Knoten $V'\in X^H_{(F_j.f)}$ so, dass $(F_1,\ldots,F_j,F_{j+2},\ldots,F_n)$ der zu $V'$ gehörige Schirm ist. Induktiv erhält man also durch eine Lochwanderungssequenz $\Sigma$ eine simpliziale Fläche $X_{\Sigma}^H$, in der es einen Knoten $\tilde{V} \in {(X_{\Sigma}^H)}_0$ gibt, sodass $(F_1,\ldots,F_j,F_k,\ldots,F_n)$ der zu $\tilde{V}$ gehörige Schirm ist.
%\begin{comment}
\begin{figure}[H]
\definecolor{ududff}{rgb}{0.30196078431372547,0.30196078431372547,1.}
\definecolor{xdxdff}{rgb}{0.49019607843137253,0.49019607843137253,1.}
\definecolor{ffffqq}{rgb}{1.,1.,0.}
\definecolor{qqqqff}{rgb}{0.,0.,1.}
\begin{tikzpicture}[line cap=round,line join=round,>=triangle 45,x=.6cm,y=.6cm]
%\begin{axis}[
x=.6cm,y=.6cm,
axis lines=middle,
ymajorgrids=true,
xmajorgrids=true,
xmin=0.528648275564196,
xmax=24.56924864269519,
ymin=-6.103382960803373,
ymax=6.063251311953354,
xtick={2.0,4.0,...,24.0},
ytick={-6.0,-4.0,...,6.0},]
\clip(0.528648275564196,-4.103382960803373) rectangle (22.56924864269519,4.063251311953354);
\fill[line width=2.pt,color=ffffqq,fill=ffffqq,fill opacity=\gelb] (2.,4.) -- (2.,-4.) -- (10.,-4.) -- (10.,4.) -- cycle;
\fill[line width=2.pt,color=ffffqq,fill=ffffqq,fill opacity=\gelb] (14.,4.) -- (14.,-4.) -- (22.,-4.) -- (22.,4.) -- cycle;
\draw [line width=2.pt] (6.,0.)-- (8.94,0.);
\draw [line width=2.pt] (8.94,0.)-- (7.47,2.5461146871262494);
\draw [line width=2.pt] (7.47,2.5461146871262494)-- (6.,0.);
\draw [line width=2.pt] (8.94,0.)-- (6.,0.);
\draw [line width=2.pt] (6.,0.)-- (7.47,-2.5461146871262494);
\draw [line width=2.pt] (7.47,-2.5461146871262494)-- (8.94,0.);
\draw [line width=2.pt] (6.,0.)-- (7.47,-2.5461146871262494);
\draw [line width=2.pt] (7.47,-2.5461146871262494)-- (8.94,0.);
\draw [line width=2.pt] (8.94,0.)-- (6.,0.);
\draw [line width=2.pt] (7.47,-2.5461146871262494)-- (6.,0.);
\draw [line width=2.pt] (6.,0.)-- (4.53,-2.5461146871262486);
\draw [line width=2.pt] (4.53,-2.5461146871262486)-- (7.47,-2.5461146871262494);
\draw [line width=2.pt] (4.53,-2.5461146871262486)-- (6.,0.);
\draw [line width=2.pt] (3.06,0.)-- (4.53,-2.5461146871262486);
\draw [line width=2.pt] (3.06,0.)-- (6.,0.);
\draw [line width=2.pt] (6.,0.)-- (4.53,2.5461146871262517);
\draw [line width=2.pt] (4.53,2.5461146871262517)-- (3.06,0.);
\draw [line width=2.pt] (4.53,2.5461146871262517)-- (6.,0.);
\draw [line width=2.pt] (6.,0.)-- (7.47,2.546114687126247);
\draw [line width=2.pt] (7.47,2.546114687126247)-- (4.53,2.5461146871262517);
\draw [line width=2.pt] (20.,0.)-- (18.369443299859643,2.4121865198142753);
\draw [line width=2.pt] (18.369443299859643,2.4121865198142753)-- (15.571447861035638,1.6068415731818402);
\draw [line width=2.pt] (15.571447861035638,1.6068415731818402)-- (15.472748279615585,-1.3030754963192541);
\draw [line width=2.pt] (15.472748279615585,-1.3030754963192541)-- (18.209744022446607,-2.2961582030819834);
\draw [line width=2.pt] (18.209744022446607,-2.2961582030819834)-- (20.,0.);
\draw [line width=2.pt] (15.571447861035638,1.6068415731818402)-- (17.634879619911473,-0.0057522600151186645);
\draw [line width=2.pt] (17.634879619911473,-0.0057522600151186645)-- (18.369443299859643,2.4121865198142753);
\draw [line width=2.pt] (18.369443299859643,2.4121865198142753)-- (15.571447861035638,1.6068415731818402);
\draw [line width=2.pt] (17.634879619911473,-0.0057522600151186645)-- (20.,0.);
\draw [line width=2.pt] (20.,0.)-- (18.369443299859643,2.4121865198142753);
\draw [line width=2.pt] (18.369443299859643,2.4121865198142753)-- (17.634879619911473,-0.0057522600151186645);
\draw [line width=2.pt] (17.634879619911473,-0.0057522600151186645)-- (18.209744022446607,-2.2961582030819834);
\draw [line width=2.pt] (18.209744022446607,-2.2961582030819834)-- (20.,0.);
\draw [line width=2.pt] (20.,0.)-- (17.634879619911473,-0.0057522600151186645);
\draw [line width=2.pt] (17.634879619911473,-0.0057522600151186645)-- (15.472748279615585,-1.3030754963192541);
\draw [line width=2.pt] (15.472748279615585,-1.3030754963192541)-- (18.209744022446607,-2.2961582030819834);
\draw [line width=2.pt] (18.209744022446607,-2.2961582030819834)-- (17.634879619911473,-0.0057522600151186645);
\draw [line width=2.pt] (17.634879619911473,-0.0057522600151186645)-- (15.571447861035638,1.6068415731818402);
\draw [line width=2.pt] (15.571447861035638,1.6068415731818402)-- (15.472748279615585,-1.3030754963192541);
\draw [line width=2.pt] (15.472748279615585,-1.3030754963192541)-- (17.634879619911473,-0.0057522600151186645);
\draw [line width=2.pt] (18.369443299859643,2.4121865198142753)-- (21.521945759637205,2.687647899794839);
\draw [line width=2.pt] (21.521945759637205,2.687647899794839)-- (20.,0.);
\draw [line width=2.pt] (20.,0.)-- (18.369443299859643,2.4121865198142753);
\draw [line width=2.pt] (11.11562696037146,0.)-- (13.196890720224609,0.);
\draw [line width=2.pt] (13.196890720224609,0.)-- (12.921429340244044,0.4839568599503282);
\draw [line width=2.pt] (13.196890720224609,0.)-- (12.829608880250523,-0.3730340999892038);
\begin{scriptsize}
\draw [fill=qqqqff] (6.,0.) circle (2.5pt);
%\draw[color=qqqqff] (6.152058274484402,0.3875791383219942) node {$A$};
\draw [fill=qqqqff] (8.94,0.) circle (2.5pt);
%\draw[color=qqqqff] (9.078740058309045,0.3875791383219942) node {$B$};
\draw[color=black] (7.370781954432573,1.0356301047403078) node {$S$};
\draw[color=black] (7.370781954432573,-0.6576643559010924) node {$F_4$};
\draw [fill=qqqqff] (7.47,-2.5461146871262494) circle (2.5pt);
%\draw[color=qqqqff] (7.615399166396723,-2.1628149875823373) node {$D$};
%\draw[color=black] (7.970781954432573,-0.6576643559010924) node {$Vieleck3$};
\draw [fill=qqqqff] (8.94,0.) circle (2.5pt);
%\draw[color=qqqqff] (9.078740058309045,0.3875791383219942) node {$G$};
\draw[color=black] (6.,-1.5147640211640234) node {$F_3$};
\draw [fill=qqqqff] (4.53,-2.5461146871262486) circle (2.5pt);
%\draw[color=qqqqff] (4.667812512687619,-2.1628149875823373) node {$H$};
\draw[color=black] (4.7023195300723468,-0.7576643559010924) node {$T$};
\draw [fill=qqqqff] (3.06,0.) circle (2.5pt);
%\draw[color=qqqqff] (3.2044716207752977,0.3875791383219942) node {$I$};
\draw[color=black] (4.723195300723468,1.0356301047403078) node {$F_2$};
\draw [fill=qqqqff] (4.53,2.5461146871262517) circle (2.5pt);
%\draw[color=qqqqff] (4.667812512687619,2.9379732642263257) node {$J$};
\draw[color=black] (6.,1.392729770003239) node {$F_1$};
\draw [fill=qqqqff] (7.47,2.546114687126247) circle (2.5pt);
%\draw[color=qqqqff] (7.615399166396723,2.9379732642263257) node {$K$};
\draw [fill=blue] (20.,0.) circle (2.5pt);
%\draw[color=xdxdff] (20.1374162271893,0.3875791383219942) node {$L$};
\draw [fill=blue] (18.369443299859643,2.4121865198142753) circle (2.5pt);
%\draw[color=ududff] (18.506836376201285,2.7916391750350935) node {$M$};
%\draw[color=ffffqq] (18.026024368858668,0.2621499190152238) node {$Vieleck8$};
\draw [fill=qqqqff] (15.571447861035638,1.6068415731818402) circle (2.5pt);
%\draw[color=qqqqff] (15.726488681567876,1.9972541194255478) node {$N$};
\draw [fill=qqqqff] (15.472748279615585,-1.3030754963192541) circle (2.5pt);
%\draw[color=qqqqff] (15.621964332145566,-0.9085227945146332) node {$O$};
\draw [fill=qqqqff] (18.209744022446607,-2.2961582030819834) circle (2.5pt);
%\draw[color=qqqqff] (18.360502287010053,-1.9119565489687962) node {$P$};
\draw [fill=blue] (17.634879619911473,-0.0057522600151186645) circle (2.5pt);
\draw[color=black] (16.148951643235073,0.12459017384731824) node {$T$};
\draw[color=black] (17.223994350502115,-1.10599244142104) node {$F_3$};

%\draw[color=black] (15.297938848936411,0.3980315732642251) node {$q_5$};

%\draw[color=black] (16.823994350502115,-0.7099265306122466) node {$n_2$};
%\draw[color=black] (20.085154052478146,2.446708821941475) node {$l_4$};
\draw[color=black] (20.0565966059820767,1.4568481157542377) node {$F_1$};
\draw[color=black] (18.5,1.) node {$S$};
\draw[color=black] (18.6,-0.9) node {$F_4$};
\draw[color=black] (17.3,1.5) node {$F_2$};
\draw[color=black] (16.5,1.3) node {$f$};
\draw[color=black] (12,0.4) node {$(S,f)$};
%\draw[color=black] (19.562532305366602,1.6314188964474672) node {$r_1$};
\draw [fill=blue] (21.6,2.7) circle (2.5pt);
%%%%%%%%
%\draw [fill=xdxdff] (11.11562696037146,0.) circle (2.5pt);

%\draw [fill=xdxdff] (13.196890720224609,0.) circle (2.5pt);

%\draw [fill=ududff] (12.921429340244044,0.4839568599503282) circle (2.5pt);
%\draw [fill=ududff] (12.829608880250523,-0.3730340999892038) circle (2.5pt);

\end{scriptsize}
%\end{axis}
\end{tikzpicture} 
\caption{Anwendung von Wanderinghole an Schirm}
\end{figure}
%\end{comment}
\end{vor}
%\end{coment}
%%%%%%%%
%\begin{comment}
\begin{satz} \label{WH}
Sei $(X,<)$ eine geschlossene Jordan-zusammenhängende simpliziale $\,$ Fläche. Dann ist die Anwendung von Wanderinghole auf $X$ transitiv, das heißt, für alle geschlossenen Jordan-zusammenhängenden simplizialen Flächen $(Y,\prec)$, die keine Knoten vom Grad 2 enthalten, wobei $\vert X_2 \vert = \vert Y_2 \vert$ und $\chi(X)=\chi(Y)=2$ ist, existiert eine Lochwanderungssequenz $\Sigma$ mit $X^H_{\Sigma} \cong Y$.
\end{satz}

\begin{proof}
Sei $S:=\{[X^1],\ldots,[X^m]\}$ die Menge aller Isomorphieklassen mit $\vert X^i_2 \vert =n \in \mathbb{N}$ und $\chi(X^i)=\chi(X^j)$ für 
$1 \leq i,j \leq m \in \mathbb{N}$. Das heißt, falls $X$ eine simpliziale Fläche mit $\vert X_2 \vert = n$ und $\chi(X)=\chi(X^1)$ ist, dann folgt schon $X \in [X^j]$ für ein $j \in \{1,\ldots,n\}$. Seien also $X \in [X^i]$ und $Y \in [X^j]$ für $1 \leq i,j \leq m$ zwei Vertreter der Äquivalenzklassen mit $X_2=Y_2=\{F_1,\ldots,F_n\}$.
 Man definiert sich nun die Menge $M_1:=\{F_1\} \subset X_2=Y_2$ und $Z^1:=X$. Nach \Cref{lemma1} existiert nun ein $F \in Y_2\setminus M_1$, sodass $M_2=M_1 \cup \{F\}$ Jordan-zusammenhängend in $Y$ ist. Da $X_2 \setminus M_1$ Jordan-zusammenhängend in $X$ ist, existiert ein $F$-$F_1$-Weg etwa $(F,F_1',\ldots,F_n',F_1)$ ohne Flächenwiederholung in $X$ und wegen \Cref{lemma2} auch eine Lochwanderungssequenz $\Sigma_2$ so, dass $F$ und $F_1$ benachbarte Flächen in $X^H_{\Sigma_2} $ sind. Setze also $Z^2=X^H_{\Sigma_2}$. Seien nun für $1 < i < n$ die Menge $M_i$ und die simpliziale Fläche $Z^i$ schon konstruiert. Nach  \Cref{lemma1} existiert eine Fläche $F \in Y_2 \setminus M_i$ so, dass $M_{i+1}=M_i \cup \{F\}$ Jordan-zusammenhängend in $Y$ ist. \\
  Da $M_i$ Jordan-zusammenhängend in $W={Z^i}$ ist, existiert eine Lochwanderungssequenz $\Sigma'$, sodass $M_{i+1}$ Jordan-zusammenhängend in $W^H_{\Sigma'}$ ist und es nach obiger Vorüberlegung für Knoten $V\in Y_0$ mit zugehörigem Schirm \\$(F_1,\ldots,F_j,F,F_{j+1},\ldots,F_{n_1})$ einen Knoten $V'\in W_0$ mit dem zugehörigem Schirm $(F_1,\ldots,F_j,F,F'_{j+1},\ldots,F'_{n_2})$ gibt, wobei $F_k \in M_i$ für $1\leq k\leq j$, $F_{l_1}\in W_2$ für $k< l_1 \leq n_1$, $F'_{l_2}\in Y_2=$ und $n_1,n_2 \in \mathbb{N}$ gilt. Wegen der \Cref{vor2} findet man gegebenfalls eine Lochwanderungsequenz $\tilde{\Sigma}$, sodass $N_{M_{i+1}}^{\tilde{W}}(F)=N_{M_{i+1}}^Y(F)$ für alle $F \in M_{i+1}$ gilt, wobei $\tilde{W}={(W_{\Sigma'}^H)}^H_{\tilde{\Sigma}}$ ist  und setzt nun $Z^{i+1}=\tilde{W}$.\\
Somit erhalten wir dann schließlich $M_n=X_2$ und eine simpliziale Fläche $X^H_{\Sigma^*}$.\\ Es bleibt nun zu zeigen, dass $Z=X^H_{\Sigma^*}$ und Y isomorph sind. Nach Konstruktion gilt $N_Z(F)=N_Y(F)$ für alle $F\in Z_2=Y_2$. Damit konstruiert man sich nun folgende Inzidenzgraphen: 
Zu $S\in \{Z,Y\}$ definiert man sich den Graphen  $G^S=(V^S,E^S)$, welcher gegeben ist durch die Knoten $V^S=S_2$ und die Kanten $E^S=\{\{F,F'\} \mid \exists e \in S_1 : e<F,F' \}$. 
Dann gilt $G^Y=G^Z$ und da $\vert N_S(F) \vert =3$ für alle $F\in S_2$ gilt,  ist $G^Y=G^Z$ 3-zusammenhängend. Außerdem ist $G^Y=G^Z$ ein planarer Graph. Die Behauptung folgt nun mit dem folgendem Satz aus der Graphentheorie.

\end{proof}
\begin{satz}[Satz von Witney]
Sei $G=(V,E)$ ein 3-fach zusammenhängender planarer Graph. Dann lässt sich G eindeutig in die Sphäre einbetten. \cite{withney}
\end{satz}%\cite{withney}
%\newpage
\begin{folgerung}
Seien $(X,<)$ und $(Y,\prec)$ simpliziale Flächen und $Z$ eine weitere simpliziale Fläche, die keine Knoten vom Grad 2 enthält. Es gilt weiterhin $\chi(X)=\chi(Y)=\chi(Z)=2$ und $\vert X\vert=\vert Y\vert=\vert Z\vert$. Dann existiert eine Lochwanderungssequenz $\Sigma$, sodass $X^H_{\Sigma} \cong Y$ ist.
\begin{proof}
$X$ und $Z$ erfüllen alle Voraussetzungen für den \Cref{WH}. 
Deshalb existiert eine Lochwanderungssequenz $\Sigma_1$, sodass $X_{\Sigma_1}^H \cong Z$ ist.
 Analog findet man ebenfalls eine Lochwanderungssequenz $\Sigma_2$, sodass $Y_{\Sigma_2}^H \cong Z$ ist. 
Mit \Cref{beminv} ist leicht einuzusehen, dass es eine Lochwanderungssquenz ${\Sigma_2}^{-1}$ gibt, sodass ${(Y_{\Sigma_2}^H)}^H_{\Sigma_2^{-1}} \cong Y$ ist.
 Damit ist ${(X^H_{\Sigma_1})}_{\Sigma_2^{*}}^H\cong Y$, wobei hier $\Sigma_2^{*}$ die Lochwanderungssequenz, die aus  $X^H_{\Sigma_1}$ eine zu $Y$ isomorphe simpliziale Fläche konstruiert. Für einen Isomorphismus $\alpha:Y^H_{\Sigma_2} \to X^H_{\Sigma_1}$ besteht $\Sigma^*_2$ aus den Bildern der Flächen und Kanten der Lochwanderungssequenz $\Sigma_2^{-1}$. 
 \end{proof}
\end{folgerung}
%\end{comment}
%---------------------------------
%\newpage
Nun wird die sogenannte Euler-Charakteristik einer simplizialen Fläche genauer betrachtet.
%\newpage
Mit den Bezeichnungen wie in der Beschreibung der obigen drei Prozeduren ist $P_F^1(X)=S^c_{\{e_3^1,e_3^2\}}(R^c_{\{e_2^1,e_2^2\}}(C^c_{\{ e_1^1,e_1^2\} }(X)))$ und somit gilt für die Euler-Charakteristik der simplizialen Fläche $P^1_F(X)$ 
\begin{gather*}
\chi (P^1_F(X))=\chi(S^c_{\{e_3^1,e_3^2\}}(R^c_{\{e_2^1,e_2^2\}}(C^c_{\{ e_1^1,e_1^2\} }(X))))\\
\chi(R^c_{\{e_2^1,e_2^2\}}(C^c_{\{ e_1^1,e_1^2\} }(X))))+1=\chi((C^c_{\{ e_1^1,e_1^2\} }(X)))+1\\
=\chi(X)-1+1=\chi(X).
\end{gather*}
Und durch eine analoge Vorgehensweise bei den Prozeduren $P^2$ und $P^3$ erhält man $\chi(P_f^2(P_F^1(X)))=\chi(X)$ und dann schließlich 
\[
\chi(X^H_{(F,f)})=\chi(P^3_F(P_f^2(P_F^1(X))))=\chi(X).
\]
Also verändert sich bei der Manipulation der simplizialen Flächen durch die drei Prozeduren die Euler-Charakteristik nicht. Wegen $\chi(P^2_f(Y))=\chi(Y)$ für  eine simpliziale Fläche $Y$, auf die man die Prozedur $P^2$ anwenden kann, und $f \in \mathcal{W}_f(Y)$ für ein $F\in Y_2$, kann man, für die durch eine Lochwanderung $\sigma$ entstandene simpliziale Fläche $X^H_{\sigma}$
\[
\chi(X^H_{\sigma})=\chi(X)
\]
und dann schließlich
\[
\chi (X_{(\sigma_1,\ldots, \sigma_n)}^H)=\chi(X)
\]
für eine beliebige Lochwanderungssequenz $(\sigma_1,\ldots, \sigma_n)$ und $n\in \mathbb{N}$ schließen.
\subsection{Knotengrade}\label{kg}
Man interessiert sich ebenfalls dafür, wie sich die Grade der Knoten der simplizialen Fläche $X^H_{(F,f)}$ aus den Graden der Knoten in $X$ zusammensetzen. Hierzu betrachtet man die unten angeführten Tabellen, die die Grade der Knoten in den einzelnen Schritten der drei Prozeduren darstellen. Es werden jedoch nur die Knoten angeführt, deren Kanten durch die Operatoren verändert werden, denn für alle anderen Knoten bleibt der Grad in jedem Schritt der drei Prozeduren unverändert. Hierfür verwendet man die Bezeichnungen der obigen Konstruktion und definiert den Grad der zu betrachteten Knoten wie folgt:
\begin{center}
\begin{tabular}{|c|c|c|c|c|}
\hline
  \textbf{$V$} & $\{V_1^1,V_1^2,V_1^3\}$ & $\{V_2^1,V_2^2\}$& $\{V_3^1,V_3^2\}$& $\{V_4\}$\\ 
  \hline
   \textbf{$\deg(V)$} & $n_1$ & $n_2$ & $n_3$ & $n_4$ \\  
   \hline
 \end{tabular}
 \end{center}
  Hier gilt $n_1,n_2,n_3,n_4 \in \mathbb{N}$.\\
%--------------------------------------
 \textbf{Prozedur $P^1$}
 \begin{itemize}
 \item Nach Anwendung des Cratercutters erhält man folgende Knotengerade.
 \begin{center}
\begin{tabular}{|c|c|c|c|c|}
\hline
  \textbf{$V$} & $\{V_1^1,V_1^2,V_1^3\}$ & $\{V_2^1,V_2^2\}$& $\{V_3^1,V_3^2\}$& $\{V_4\}$\\ 
  \hline
   \textbf{$\deg(V)$} & $n_1$ & $n_2$ & $n_3$ & $n_4$ \\  
   \hline
 \end{tabular}
 \end{center}

%--------------------------------------
\item %\textbf{Prozedur 1 Schritt 2}
Durch Anwendung des Ripcutters ergibt sich:
\begin{center}
\begin{tabular}{|c|c|c|c|c|c|}
\hline
  \textbf{$V$} & $\{V_1^1,V_1^2,V_1^3\}$ & $\{V_2^1,V_2^2\}$& $\{V_3^1\}$ & $\{V_3^2\}$& $\{V_4\}$\\ 
  \hline
   \textbf{$\deg(V)$} & $n_1$ & $n_2$ & 1 & $n_3-1$ & $n_4$ \\  
   \hline
 \end{tabular}
 \end{center}
%---------------------------------------
\item %\textbf{Prozedur 1 Schritt 3}
Nach dem Splitcut ergeben sich folgende Grade.
\begin{center}
\begin{tabular}{|c|c|c|c|c|c|c|c|}
\hline
  \textbf{$V$} & $\{V_1^1\}$ & $\{V_1^2,V_1^3\}$ & $\{V_2^1\}$ & $\{V_2^2\} $ & $\{V_3^1\}$ & $\{V_3^2\}$& $\{V_4\}$\\ 
  \hline
   \textbf{$\deg(V)$} & 1 & $n_1-1$ &1& $n_2-1
   $ & 1 & $n_3-1$ & $n_4$ \\  
   \hline
 \end{tabular}
 \end{center}
 \end{itemize}
%---------------------------------------
\textbf{Prozedur $P^2$}
\begin{itemize}
\item Als Ergebnis des Ripcutters erhält man
\begin{center}
\begin{tabular}{|c|c|c|c|c|c|c|c|c|}
\hline
  \textbf{$V$} & $\{V_1^1\}$ & $\{V_1^2\}$ & $\{V_1^3\}$ & $\{V_2^1\}$ & $\{V_2^2\} $ & $\{V_3^1\}$ & $\{V_3^2\}$& $\{V_4\}$\\ 
  \hline
   \textbf{$\deg(V)$} & 1 & 1 & $n_1-2$ &1& $n_2-1$ & 1 & $n_3-1$ & $n_4$ \\  
   \hline
 \end{tabular}
 \end{center}
%----------------------------------------
%\textbf{Prozedur 2 Schritt 2}
\item Zusammensetzen der Kanten durch den Ripmender erbringt
\begin{center}
\begin{tabular}{|c|c|c|c|c|c|c|c|}
\hline
  \textbf{$V$} & $\{V_1^1\}$ & $\{V_1^2,V_3^2\}$ & $\{V_1^3\}$ & $\{V_2^1\}$ & $\{V_2^2\} $ & $\{V_3^1\}$ & $\{V_4\}$\\ 
  \hline
   \textbf{$\deg(V)$} & $1$ & $n_3$ & $n_1-2$ &1& $n_2-1$ & $1$ & $n_4$ \\  
   \hline
 \end{tabular}
 \end{center}
 \end{itemize}
%-----------------------------------------
\textbf{Prozedur $P^3$}
\begin{itemize}
\item Der Splitmender im ersten Schritt der Prozedur $P^3$ führt zu \begin{center}
\begin{tabular}{|c|c|c|c|c|c|c|}
\hline
  \textbf{$V$} & $\{V_1^1,V_4\}$ & $\{V_2^1,\{V_1^2,V_3^2\}\}$ & $\{V_1^3\}$ & $\{V_2^2\} $  & $\{V_3^1\}$ \\ 
  \hline
   \textbf{$\deg(V)$} & $n_4+1$ & $n_3+1$ & $n_1-2$ & $n_2-1$ & $1$  \\  
   \hline
 \end{tabular}
 \end{center}
%_------------------------------------------
%\textbf{Prozedur 3 Schritt 2}
\item Nach Anwendung der Ripmenders erhält man \begin{center}
\begin{tabular}{|c|c|c|c|c|c|}
\hline
  \textbf{$V$} & $\{V_1^1,V_4\}$ & $\{V_2^1,\{V_1^2,V_3^2\}\}$ & $\{V_1^3,V_3^1\}$ & $\{V_2^2\} $   \\ 
  \hline
   \textbf{$\deg(V)$} & $n_4+1$ & $n_3+1$ & $n_1-1$ & $n_2-1$  \\  
   \hline
 \end{tabular}
 \end{center}
%-----------------------------------------
%\textbf{Prozedur 3 Schritt 3}
\item Die Anwendung des Cratermenders liefert folgende Knotengrade.
\begin{center}
\begin{tabular}{|c|c|c|c|c|c|c|}
\hline
 \textbf{$V$} & $\{V_1^1,V_4\}$ & $\{V_2^1,\{V_1^2,V_3^2\}\}$ & $\{V_1^3,V_3^1\}$ & $\{V_2^2\} $   \\ 
  \hline
   \textbf{$\deg(V)$} & $n_4+1$ & $n_3+1$ & $n_1-1$ & $n_2-1$  \\  
   \hline
 \end{tabular}
 \end{center}
 \end{itemize}
%-----------------------------------------
Für durch Lochwanderungen und Lochwanderungssequenzen entstandene simpliziale Flächen betrachtet man die nach der ersten Prozedur veränderten Grade der Knoten und verändert diese, für jede weitere Kante mit welcher die Prozedur $P^2$ durchgeführt wird, wie oben vorgeschrieben.\\
Hier sei angemerkt, dass alle Knotengrade mindestens zwei sind, da die durch die drei Prozeduren entstandene simpliziale Fläche wieder geschlossen ist. Das heißt, es gilt $n_1,n_2 \geq 3$ und $n_3,n_4 \geq 2$. 
%--------------------------------
%\newpage
\section{Simpliziale Flächen mit 2- und 3-Waists}
\subsection{2-Waists und 3-Waists }
\begin{definition}
Sei $(X,<)$ eine geschlossene simpliziale Fläche und $e_1,\ldots,e_n\in X_1$ mit $ e_i \neq e_j$ für $i \neq j$ und $n \geq 2$.
\begin{enumerate}
\item   Man nennt $(e_1,\ldots,e_n)$ einen \emph{Pfad von Kante $e_1$ zur Kante $e_n$ in $X$}, falls es Knoten $V_1,\ldots,V_{n-1}\in X_0$ mit 
\[
V_i<e_i \text{ und } V_i<e_{i+1}
\] 
gibt.
Hierbei ist $n$ die Länge des Pfades.
\item Falls es in der obigen Definition einen Knoten $V_n\in X_0$ mit $V_n<e_n$ und $V_n<e_1$ gibt, so nennt man den Pfad $(e_1,\ldots,e_n)$ geschlossen.
\item Einen geschlossenen Pfad der Länge 2 nennt man \emph{2-Waist} und einen geschlossenen Pfad der Länge 3 nennt man \emph{3-Waist}, falls 
\[
\vert \{e_1,e_2,e_3\}\cap X_1(F) \vert< 3\text{ für alle } F \in X_2 \text{ gilt},
\]
wobei $(e_1,e_2,e_3)$ ein geschlossener Pfad ist.\cite{SS}
\end{enumerate}
\end{definition}

\begin{bemerkung}
Sei $(X,<)$ eine geschlossene Jordan-zusammenhängende simpliziale Fläche mit $\vert X_2 \vert > 2$ und $V \in X_0$ ein Knoten in $X$ mit $\deg(V)=2$. Dann definiert der Knoten $V$ einen 2-Waist. Sei dazu $X_2(V)=\{F_1,F_2\}$ für $F_1,F_2 \in X_2$. Damit haben $F_1$ und $F_2$ wegen $\Cref{lemmajanus}$ genau zwei gemeinsame Kanten und jeweils eine Kante, die nicht zu der anderen Fläche adjazent ist, da $\vert X_2 \vert > 2$ gilt. Das heißt, es existieren $e_1,e_2\in X_1$, für welche $e_1<F_1,e_1 \nless F_2, e_2\nless F_1$ und $e_2  < F_2$ gilt. Somit ist durch $(e_1,e_2)$ ein 2-Waist gegeben.\\
Falls $\deg(V)=3$ und $\vert X_2 \vert>3$ ist, so definiert V einen 3-Waist. Dies kann man mit analoger Argumentation zeigen.\cite{withney}
\end{bemerkung}
 Die Umkehrung der obigen Bemerkung gilt jedoch nicht, was durch folgende Konstruktion erläutert wird.\\
 Seien $(T^1,<_1)$ und $(T^2,<_2)$ Tetraeder, wie zuvor definiert, wobei die Knoten bzw. Kanten bzw. Flächen von $T^i$ für $i\in \{1,2\}$ mit $V_1^i,V_2^i,V_3^i,V_4^i$, bzw. $e_1^i,e_2^i,e_3^i,e_4^i,e_5^i,e_6^i$ bzw. $F_1^i,F_2^i,F_3^i,F_4^i$ bezeichnet werden. Man nutzt wieder die \Cref{ident} und verwendet die Isomorphie von  $T^1$ zu einer simplizialen Fläche $Y^1\in \mathcal{M}(4 \Delta)$, die durch $e_3^1 =\{e_3^1,e_3^2\}$ und die Isomorphie von $T^2$ zu einer simpliziale Fläche $Y^2\in \mathcal{M}(n\Delta)$ mit $e_3^2=\{{e_3^1}',{e_3^2}'\}$ festgesetzt werden, wobei $V_1^1,V_2^1 \in T^1_0$ die zugehörigen Knoten in $T^1$ und $V_1^2,V_2^2 \in T^2_0$ die zugehörigen Knoten in $T^2$ sind. Mithilfe dieser beiden Mendings konstruiert man sich die simpliziale Fläche $(X,<)$ mit den Knoten $X_0=T^1_0 \cup T^2_0$, Kanten $X_1=T^1_1 \cup T^2_1$, Flächen $X_2=T^1_2 \cup T^2_2$ und $x<y$ in $X$ genau dann, wenn $x<_1y$ in $T^1$ oder $x<_2 y$ in $T^2$ ist. Dann führt man folgende Operationen auf X aus:
 \begin{enumerate}
 \item Wende den Cratercut $C^c_{\{e_{3}^1,e_{3}^2\}}$ an, um die Kanten $\{e_3^1\},\{e_3^2\}$ zu erhalten.
 \item Wende den Cratercut $C^c_{\{{e_{3}^1}',{e_{3}^2}'\}}$ an, um die Kanten $\{{e_3^1}'\},\{{e_3^2}'\}$ zu erhalten.
\item Wende den Splitmender $S^m_{(\{V_1^1\},\{e_3^1\}),(\{V_1^2\},\{{e_3^1}'\})}$ an, um die Kanten $\{e_3^1\}$ und $\{{e_3^1}'\}$ zu der Kante $\{e_1^1,{e_3^1}'\}$ zusammenzuführen und um die Knoten $\{V_1^1,V_1^2\}$ und $\{V_2^1,V_2^2\}$ zu erhalten.
 \item Wende nun den Cratermender $C^m_{\{e_3^2\},\{{e_3^2}'\}}$ an, um die Kante $\{e_3^2,{e_3^2}'\}$ zu erhalten.
 \end{enumerate}
 Die durch die Mender- und Cutteroperationen entstandene simpliziale Fläche $Z$ ist geschlossen, Jordan-zusammenhängend und es gilt
 \[
 \{deg(V)\mid V\in Z_0\}=\{3,6\}.
 \] 
 Damit enthält $Z$ keinen Knoten vom Grad 2, jedoch bildet $(\{e_{3}^1,{e_{3}^1}'\},\{e_{3}^2,{e_{3}^2}'\})$ einen 2-Waist in $Z$. Auf ähnliche Art und Weise kann man zeigen, dass eine simpliziale Fläche einen 3-Waist enthalten kann, ohne einen Knoten vom Grad 3 zu haben.
 \subsection{Prozedur $W^2$}
Im Folgendem will man aus einer simplizialen Fläche, die keinen 2-Waist enthält, eine simpliziale mit zwei zusätzlichen Flächen und einem 2-Waist konstruieren. Hierfür macht man Gebrauch von dem zu Beginn definierten Open-Bag.
\begin{bemerkung}
Zur Erinnerung sei hier angemerkt, dass der Open-Bag $(B,<_B)$ durch das ordinale Symbol 
\[
\omega((B,<_B))=(3,4,2;(\{2,3\},\{1,3\},\{1,2\},\{1,3\}),(\{1,2,3\},\{1,3,4\}))
\] definiert ist.
\end{bemerkung}
 
Seien $(B,<_B)$ der Open-Bag und $(X,<)$ eine geschlossene simpliziale Fläche. Dann definiert man zunächst eine simpliziale Fläche $Z$ durch die Knoten $Z_0=X_0 \cup B_0$, die Kanten $Z_1=X_1 \cup B_1$ und die Flächen $Z_2=X_2 \cup B_2$. Außerdem gilt $x<_Z y$ in $Z$ genau dann, wenn $x<y$ in $X$ oder $x<_B y$ in $B$ gilt.\\
Für eine Kante $f\in X_1 \subseteq Z_1$ und Knoten $V,V'\in X_0\subseteq Z_0$ mit $X_0(f)=\{V,V'\}$ identifiziert man $Z$ mit einem Mending, welches durch $f=\{f_1,f_2\}$ festgesetzt wird.
 Nun führt man folgende Operationen auf $Z$ durch:
\begin{itemize}
\item Man führt den Cratercut $C_{\{f_1,f_2\}}^c$ durch, um die Kanten $\{f_1\}$ und $\{f_2\}$ erhalten.
\item Man wendet den Splitmender $S^m_{(V,{f_1}),(V_1,e_2)}$ an, um die Kanten $\{f_1\}$ und $\{e_2\}$ zusammenzuführen und um somit die Knoten $\{V,V_1\}$ und $\{V',V_2\}$ zu erhalten.
\item Man wendet den Cratermender $C_{\{e_4\},\{f_2\}}^m$ an, um die Kante $\{e_4,f_2\}$ zu erschaffen.
\end{itemize}
Dadurch erhält man eine geschlossene simpliziale Fläche, in welcher man den 2-Waist $(\{e_2,f_1\},\{e_4,f_2\})$ vorfinden kann. Man bezeichnet diese simpliziale Fläche mit \emph{$W^2_f(X)$}, falls durch die obige Prozedur ein 2-Waist an der Stelle $f$ konstruiert wurde.

 Nun stellt sich die Frage, ob es möglich ist, durch Anwendung einer Lochwanderungssequenz auf $W^2_f(X)$ eine simpliziale Fläche zu konstruieren, die keinen 2-Waist mehr enthält. Mithilfe der nachgewiesen Transitivität der Operation Wanderinghole können schon erste Beobachtungen gemacht werden.
 

\begin{folgerung}
Seien $(X,<)$ eine geschlossene Jordan-zusammenhängende simpliziale Fläche, $f\in X_1$ eine Kante in $X$ und $Y=W^2_f(X)$. Seien außerdem $(Z,<_Z)$ eine geschlossene Jordan-zusammenhängende simpliziale Fläche, die keinen Knoten vom Grad zwei enthält, $\vert Y_2\vert=\vert Z_2\vert$ und $\chi(Y)=\chi(Z)=2$. Dann existiert eine Lochwanderungssequenz $\Sigma$ so, dass $Y_{\Sigma}^H \cong Z$ ist.
\end{folgerung}

\subsection{Prozedur $W^3$}
Nun soll aus einer simplizialen Fläche eine weitere simpliziale Fläche konstruiert werden, die zwei zusätzliche Flächen enthält und dazu noch einen 3-Waist besitzt. Hierfür verwendet man die folgende simpliziale Fläche.

\begin{definition}
Man definiert die simpliziale Fläche \emph{Hut} $(H,<_{H})$ durch die Knoten $H_0=\{V_1,V_2,V_3,V_4\}$, die Kanten $H_1=\{e_1,e_2,e_3,e_4,e_5,e_6\}$ und die Flächen $H_2=\{F_1,F_2,F_3\}$. Die Inzidenz $<$ erhält man durch das ordinale Symbol
\begin{align*}
\omega((H,<_H))=(4,6,3;\{2,3\},\{1,3\},\{1,2\},\{3,4\},\\
\{1,4\},\{2,4\},
 \{1,2,3\},\{2,4,5\},\{3,5,6\}).
\end{align*}
\end{definition}
\begin{figure}[H]
\definecolor{qqqqff}{rgb}{0.,0.,1.}
\definecolor{ffffqq}{rgb}{1.,1.,0.}
\definecolor{ududff}{rgb}{0.30196078431372547,0.30196078431372547,1.}
\begin{tikzpicture}[line cap=round,line join=round,>=triangle 45,x=1.2cm,y=1.2cm]
%\begin{axis}[
x=1.0cm,y=1.0cm,
axis lines=middle,
ymajorgrids=true,
xmajorgrids=true,
xmin=-4.3,
xmax=18.7,
ymin=-5.34,
ymax=6.3,
xtick={-4.0,-3.0,...,18.0},
ytick={-5.0,-4.0,...,6.0},]
\clip(-4.3,-.3) rectangle (18.7,3.8);
\fill[line width=2.pt,color=ffffqq,fill=ffffqq,fill opacity=\gelb] (5.,3.) -- (2.,3.) -- (3.5,0.40192378864668354) -- cycle;
\fill[line width=2.pt,color=ffffqq,fill=ffffqq,fill opacity=\gelb] (3.5,0.40192378864668354) -- (2.,3.) -- (0.5,0.4019237886466849) -- cycle;
\fill[line width=2.pt,color=ffffqq,fill=ffffqq,fill opacity=\gelb] (0.5,0.4019237886466849) -- (2.,3.) -- (-1.,3.) -- cycle;
\draw [line width=2.pt,color=black] (5.,3.)-- (2.,3.);
\draw [line width=2.pt,color=black] (2.,3.)-- (3.5,0.40192378864668354);
\draw [line width=2.pt,color=black] (3.5,0.40192378864668354)-- (5.,3.);
\draw [line width=2.pt,color=black] (3.5,0.40192378864668354)-- (2.,3.);
\draw [line width=2.pt,color=black] (2.,3.)-- (0.5,0.4019237886466849);
\draw [line width=2.pt,color=black] (0.5,0.4019237886466849)-- (3.5,0.40192378864668354);
\draw [line width=2.pt,color=black] (0.5,0.4019237886466849)-- (2.,3.);
\draw [line width=2.pt,color=black] (2.,3.)-- (-1.,3.);
\draw [line width=2.pt,color=black] (-1.,3.)-- (0.5,0.4019237886466849);
\begin{scriptsize}
\draw [fill=ududff] (2.,3.) circle (2.5pt);
\draw[color=black] (1.98,3.43) node {$V_1$};
\draw [fill=ududff] (5.,3.) circle (2.5pt);
\draw[color=black] (5.14,3.37) node {$V_2$};
\draw[color=black] (3.6,2.25) node {$F_3$};
\draw [fill=qqqqff] (3.5,0.40192378864668354) circle (2.5pt);
\draw[color=black] (3.56,0.05) node {$V_4$};
\draw[color=black] (2.,1.43) node {$F_2$};
\draw [fill=qqqqff] (0.5,0.4019237886466849) circle (2.5pt);
\draw[color=black] (0.44,0.05) node {$V_3$};
\draw[color=black] (0.56,2.25) node {$F_1$};
\draw [fill=qqqqff] (-1.,3.) circle (2.5pt);
\draw[color=black] (-1.,3.45) node {$V_2$};

\draw[color=black] (0.5,3.25) node {$e_3$};
\draw[color=black] (3.5,3.25) node {$e_3$};
\draw[color=black] (2.,0.1) node {$e_4$};
\draw[color=black] (-0.6,1.8) node {$e_1$};
\draw[color=black] (4.65,1.8) node {$e_6$};
\draw[color=black] (3,2) node {$e_5$};
\draw[color=black] (1,2) node {$e_2$};
\end{scriptsize}
%\end{axis}
\end{tikzpicture}
\caption{Hut}
\end{figure}
Für eine geschlossene simpliziale Fläche $(X,<)$ und $F \in X_2$ kontruiert man zunächst die simpliziale Fläche $(Y,\prec)$, die gegeben ist durch $Y_2=X_2\setminus \{F\}$, $Y_1=X_1$ und $Y_0=X_0$. Es gilt $x\prec y$ in $Y$ genau dann, wenn $y\neq F$ ist und $x<y$ in $X$ gilt. Dadurch entstehen die Randkanten $r_1,r_2,r_3 \in Y_1$ mit zugehörigen Knoten $W_1,W_2,W_3\in Y_0$ so, dass 
\[
W_i < r_j \text{ für } j\in \{1,2,3\},i\in\{1,2,3\}\setminus \{j\}
\] gilt.
\begin{figure}[H]
\begin{center}
\definecolor{ffffff}{rgb}{1.,1.,1.}
\definecolor{ududff}{rgb}{0.30196078431372547,0.30196078431372547,1.}
\definecolor{uuuuuu}{rgb}{0.26666666666666666,0.26666666666666666,0.26666666666666666}
\definecolor{ffffqq}{rgb}{1.,1.,0.}
\begin{tikzpicture}[line cap=round,line join=round,>=triangle 45,x=1.3cm,y=1.3cm]
%\begin{axis}[
x=1.5cm,y=1.5cm,
axis lines=middle,
ymajorgrids=true,
xmajorgrids=true,
xmin=-3.1895867768595068,
xmax=9.207107438016529,
ymin=-1.7595041322314047,
ymax=7.860330578512391,
xtick={-3.0,-2.0,...,9.0},
ytick={-1.0,0.0,...,7.0},]
\clip(-0.1895867768595068,-0.595041322314047) rectangle (4.207107438016529,4.360330578512391);
\fill[line width=2.pt,color=ffffqq,fill=ffffqq,fill opacity=\gelb] (0.,4.) -- (0.,0.) -- (4.,0.) -- (4.,4.) -- cycle;
\fill[line width=2.pt,color=ffffff,fill=ffffff,fill opacity=1.0] (1.,1.) -- (3.,1.) -- (2.,2.7320508075688776) -- cycle;
%\draw [line width=2.pt,color=ffffqq] (0.,4.)-- (0.,0.);
%\draw [line width=2.pt,color=ffffqq] (0.,0.)-- (4.,0.);
%\draw [line width=2.pt,color=ffffqq] (4.,0.)-- (4.,4.);
%\draw [line width=2.pt,color=ffffqq] (4.,4.)-- (0.,4.);
\draw [line width=2.pt] (1.,1.)-- (3.,1.);
\draw [line width=2.pt,color=black] (3.,1.)-- (2.,2.7320508075688776);
\draw [line width=2.pt,color=black] (2.,2.7320508075688776)-- (1.,1.);
\begin{scriptsize}
%\draw[color=ffffqq] (2.380661157024792,2.1495867768595027) node {$Vieleck1$};
%\draw [fill=uuuuuu] (4.,0.) circle (2.5pt);
%\draw[color=uuuuuu] (4.116198347107438,0.2983471074380165) node {$C$};
%\draw [fill=blue] (4.,4.) circle (2.5pt);
%\draw[color=uuuuuu] (4.116198347107438,4.298347107438015) node {$D$};
\draw [fill=blue] (1.,1.) circle (2.5pt);
\draw[color=black] (0.85,1.4066115702479333) node {$W_1$};
\draw [fill=blue] (3.,1.) circle (2.5pt);
\draw[color=black] (3.1579338842975193,1.4066115702479333) node {$W_2$};
%\draw[color=ffffff] (2.380661157024792,1.7198347107438008) node {$Vieleck2$};
\draw [fill=blue] (2.,2.7320508075688776) circle (2.5pt);
\draw[color=black] (2.1161983471074364,3.042148760330577) node {$W_3$};
\draw[color=black] (2.65,2.1) node {$r_1$};
\draw[color=black] (2,0.75) node {$r_3$};
\draw[color=black] (1.3,2.1) node {$r_2$};
\end{scriptsize}
%\end{axis}
\end{tikzpicture}
\end{center}
\caption{Simpliziale Fläche mit fehlender Fläche}
\end{figure}
Diese Fläche nutzt man, um die simpliziale Fläche $Z$ zu erhalten, welche durch die Knoten $Z_0=Y_0 \cup H_0$, die Kanten $Z_1=Y_1 \cup H_1$ und die Flächen $Z_2=Y_2 \cup H_2$ definiert ist. Es gilt $x<_Z y$ in $Z$ genau dann, wenn $x \prec y$ in $Y$ oder $x <_H y$ in $H$ gilt. Nun führt man folgende Operationen auf $Z$ aus:
\begin{enumerate}
\item Führe den Splitmender $S^m_{(\{W_2\},\{r_1\}),(\{V_4,e_1)\}}$ durch, um die Kante $\{e_1,r_1\}$ und Knoten $\{V_4,W_2\}$ und $\{V_1,W_3\}$ zu erhalten.
\item Wende den Ripmender $R^m_{\{e_4\}\{r_3\}}$ an, um die Kante $\{e_4,r_3\}$ und den Knoten $\{V_3,W_1\}$ zu erhalten. 
\item Wende schließlich den Cratermender $C^m_{\{e_6\},\{r_2\}}$ an, um die Kante $\{e_6,r2\}$
\end{enumerate}
Dadurch erhält man eine simpliziale Fläche mit $\vert X_2\vert +2$ Flächen, in welcher man den 3-Waist $(\{e_1,r_1\},\{e_4,r_3\},\{e_6,r_2\})$ vorfindet. Diese bezeichnet man mit \emph{$W_F^3(X)$}, falls durch die Anwendung der obigen Prozedur die Fläche $F$ entfernt und an dieser Stelle der Hut angesetzt wurde, um den 3-Waist zu kontruieren.\\
Hier stellt sich wiederum die Frage, ob es eine Lochwanderungssequenz gibt, sodass man aus $W_F^3(X)$ durch Anwenden dieser eine simpliziale Fläche kreiert, die keinen 2-Waist enthält. Diese Fragestellung wird in der nächsten Folgerung thematisiert.
\begin{folgerung}
Seien $(X,<)$ eine geschlossene joran-zusammenhängende simpliziale Fläche, $F\in X_2$ eine Fläche in $X$ und $Y=W^3_F(X)$. Seien außerdem $(Z,<_Z)$ eine geschlossene Jordan-zusammenhängende simpliziale Fläche, die keinen Knoten vom Grad zwei enthält, $\vert Y_2 \vert = \vert Z_2 \vert$ und $\chi(Y)=\chi(Z)=2$. Dann existiert eine Lochwanderungssequenz $\Sigma$ so, dass $Y_{\Sigma}^H \cong Z$ ist.
\end{folgerung}

Man betrachtet nun die Euler-Charakteristik der oben konstruierten simplizialen Flächen. Seien dazu wieder $(X,<)$ eine geschlossene simpliziale Fläche, $f\in X_1$ eine Kante und $F\in X_2$ eine Fläche.
\begin{itemize}
\item Durch die Prozedur $W^2$ erhält man zunächst die simpliziale Fläche $Z$, welche die disjunkte Vereininung von X und dem Open-Bag $B$ darstellt. Deshalb ist $\vert Z_i\vert= \vert X_i \vert+\vert B_i\vert$ für $i=0,1,2$ und damit auch $\chi(Z)=\chi(X)+\chi(B)=\chi(X)+1.$ Daraufhin geht $W^2_f(X)$ durch Anwendung eines Cratercutter, eines Splitmender und eines Cratermender aus $Z$ hervor. Damit gilt mit den obigen Bezeichnungen 
\begin{align*}
\chi(W^2_f(X))&=\chi(C_{\{e_4\},\{f_2\}}^m(S^m_{(W_2,r_1),(V_4,e_1)}(C_{\{f_1,f_2\}}^c(Z))))\\
&=\chi(S^m_{(W_2,r_1),(V_4,e_1)}(C_{\{f_1,f_2\}}^c(Z)))+1\\
&=\chi(C_{\{f_1,f_2\}}^c(Z))\\
&=\chi(Z)-1\\
&=\chi(X).
\end{align*}

\item Bei der Anwendung der Prozedur $W^3$ erhält man auch zunächst eine simpliziale Fläche $Z$ als disjunkte Vereinung der geschlossenen simplizialen Fläche $X$ und dem oben definierten Hut $H$. Das heißt, es gilt $\vert Z_i\vert= \vert X_i \vert+\vert H_i\vert$ für $i=0,1,2$ und damit auch $\chi(Z)=\chi(X)+\chi(H)=\chi(X)+1.$ Durch Entfernen der Fläche $F$ von $Z$ erhält man die simpliziale Fläche $Y$ mit $\vert Y_0\vert= \vert Z_0 \vert-3,\vert Y_1\vert= \vert Z_1 \vert-3$ und $\vert Y_2\vert= \vert Z_2 \vert-1$. Dadurch kann man 
\begin{align*}
\chi(Y)&=\vert Y_0 \vert-\vert Y_1 \vert+\vert Y_2 \vert\\
&=\vert Z_0 \vert-3-(\vert Z_1 \vert-3)+\vert Z_2 \vert-1\\
&=\chi(Z)-1\\
&=\chi(X).
\end{align*}
schließen. Schließlich erhält man $W^3_F(X)$ durch Anwendung der Operatoren Splitmender, Ripmender und Cratermender auf $Y$. Mit den obigen Bezeichnungen ist dann 
\begin{align*}
\chi(W_F^3)&=\chi(C^m_{\{e_6\},\{r_2\}}(R^m_{\{e_4\},\{r_3\}}(S^m_{(\{W_2\},\{r_1\}),(\{V_4\},\{e_1\})}(Y))))\\
&=\chi(R^m_{\{e_4\},\{r_3\}}(S^m_{(\{W_2\},\{r_1\}),(\{V_4\},\{e_1\})}(Y)))+1\\
&=\chi(S^m_{(\{W_2\},\{r_1\}),(\{V_4\},\{e_1\})}(Y))+1\\
&=\chi(Y)\\
&=\chi(X).
\end{align*}
\end{itemize}
Das heißt die Euler-Charakteristik einer simplizialen Fläche wird durch die Anwendung der Prozeduren $W^2$ und $W^3$ auf diese nicht verändert.
%\newpage
\section{Verallgemeinerte Konstruktionen}
\subsection{Vereinigung Simplizialer Flächen} 
Bisher wurde die Erweiterung einer simplizialen Fläche um den Open-Bag bzw. Tetraeder mit einer fehlenden Fläche skizziert. Dies will man nun verallgemeinern und führt zu diesem Zweck die Vereinigung von simplizialen Flächen ein, die von der Vereinigung simplizialer Flächen aus dem Skript \emph{Combinatorial Simplicial Surfaces} inspiriert wurde.
\begin{definition} Seien $(X^1,<_1), \ldots,(X^n,<_n)$ für ein $n \in \mathbb{N}$ simpliziale Flächen. Zur Vereinfachung sei hier $X^i\cap X^j\neq \emptyset$ für $i \neq j$ angenommen. Dann definiert man die simpliziale Fläche $Z=\biguplus\limits_{i=1}^{n} X^i$ durch
die Knoten 
\[
Z_0= Z=\bigcup\limits_{i=1}^{n} (X^i)_0,
\]
die Kanten 
\[
Z_1=\bigcup\limits_{i=1}^{n} (X^i)_1
\]
und die Flächen 
\[
Z_2=\bigcup\limits_{i=1}^{n} (X^i)_2.
\]
Weiterhin sei $<_Z$ die Inzidenz der simplizialen Fläche $Z$. Dann gilt $x<_{Z} y$ für $x,y \in Z$ genau dann, wenn $x<y$ in $X^i$ gilt, wobei $i \in \{1,\ldots,n\}$ ist. Man nennt diese simpliziale Fläche $Z$ die \emph{aus $X^1,\ldots,X^n$ zusammengesetzte simpliziale Fläche}.
\end{definition}
\begin{bemerkung}
 Seien $(X^1,<_1), \ldots,(X^n,<_n)$ für ein $n \in \mathbb{N}$ simpliziale Flächen. 
%\begin{enumerate}
 Für eine wie in obiger Definition aus $X^1,\ldots,X_n$ zusammengesetzte simpliziale Fläche $Z$ gelten folgende Aussagen
\begin{itemize}
\item Für $i=0,1,2$ ist $\vert Z_i\vert= \sum_{j=1}^n \vert (X^j)_i\vert$. 
\item  Es gilt die Gleichheit $\chi(Z)=\sum_{j=1}^n \chi(X^j)$, denn 
\begin{align*}
&\chi(Z)\\
=&\vert Z_0 \vert-\vert Z_1 \vert +\vert Z_2 \vert \\
=& \vert \bigcup\limits_{i=1}^{n} (X^i)_0\vert-\vert \bigcup\limits_{i=1}^{n} (X^i)_1\vert+\vert \bigcup\limits_{i=1}^{n} (X^i)_2\vert\\
=&\sum_{j=1}^n \vert (X^j)_0\vert-\sum_{j=1}^n \vert (X^j)_1\vert+\sum_{j=1}^n \vert (X^j)_2\vert\\
\end{align*}
\begin{align*}
=&\sum_{j=1}^n \vert (X^j)_0\vert-\vert (X^j)_1\vert+\vert (X^j)_2\vert\\
=&\sum_{j=1}^n \chi(X^j).
\end{align*}
\item Falls $X^1,\ldots,X^n$ geschlossene simpliziale Flächen sind, so ist auch Z geschlossen.
\item $Z$ ist nicht zusammenhängend, denn es gilt 
\begin{align*}
&\vert \{ U \mid \text{U ist Zusammenhangskomponente von Z } \} \vert \\
=\sum_{i=1}^n &\vert \{U' \mid U' \text{ ist Zusammenhangskomponente von $X^i$ } \}\vert \geq  n
\end{align*}
\item Falls $X^1= \ldots =X^n$  gilt, so schreibt man auch $n X^1 :=\biguplus\limits_{i=1}^{n} X^i$.
\end{itemize}
%\end{enumerate}
\end{bemerkung}

Nun sollen die Konstruktionen, die in den Prozeduren $W^2$ und $W^3$ beschrieben wurden, verallgemeinert werden. Hierbei beschränkt man sich aber zunächst auf den Fall, dass man zwei geschlossene simpliziale Flächen mit Euler-Charakteristik 2 durch einen 2-Waist bzw. 3-Waist zu einer geschlossenen simplizialen Fläche verbinden will. Denn der allgemeinere Fall folgt dann induktiv.
\subsection{Verallgemeinerung der Prozedur $W^3$}
Seien deshalb $(X,<)$ und $(Y,\prec)$ geschlossene simpliziale Flächen, $F_X \in X_2$ eine Fläche in $X$ und $F_Y \in Y_2$ eine Fläche in $Y$. Dann bildet man zunächst die simpliziale Fläche $(X',<_X)$, die  gegeben ist durch $X'_0=X_0,\,X'_1=X_1,\,X'_2=X_2\setminus \{F_X\}$, wobei $x<y$ in $X'$ genau dann gilt, wenn $y\neq F_X$ ist und $x<y$ in $X$ gilt. Analog konstruiert man die simpliziale Fläche $(Y',<_Y)$, die definiert ist durch $Y'_0=Y_0,\,Y'_1=Y_1,\,Y'_2=Y_2\setminus \{F_Y\}$, wobei $x<y$ in $Y'$ genau dann gilt, wenn $y\neq F_Y$ und $x<y$ in $Y$ ist.
Dadurch erhält man die aus $X'$ und $Y'$ zusammengesetzte simpliziale Fläche $Z=X' \cup Y'$ mit den Eckknoten $V_i^{X},V_i^{Y}$ und den Randkanten $f_i^{X},f_i^{Y}$ mit den zugehörigen Inzidenzen 
\[
V^S_i < f^S_j \qquad j\in \{1,2,3\},i\in\{1,2,3\}\setminus \{j\}, S\in \{X,Y\}.
\]
Nun führt man folgende Operationen auf $Z$ aus:
\begin{enumerate}
\item Führe den Splitmender $S^m_{(V_1^X,f_3^X),(V_1^Y,f_3^Y)}$ durch, um die Kante $\{f_3^X,f_3^Y\}$ und Knoten $\{V_1^X,V_1^Y\}$ und $\{V_2^X,V_2^Y\}$ zu erhalten.
\item Wende den Ripmender $R^m_{f_2^X, f_2^Y}$ an, um die Kante $\{f_2^X ,f_2^Y\}$ und den Knoten $\{V_3^X,V_3^Y\}$ zu erhalten. 
\item Wende schließlich den Cratermender $C^m_{f_1^X,f_1^Y}$, um die Kante $\{f_1^X,f_1^Y\}$ zu erhalten.
\end{enumerate}
Hierdurch entsteht eine geschlossene simpliziale Fläche, in welcher man den 3-Waist $(\{f_1^X,f_1^Y\},\{f_2^X,f_2^Y\},\{f_3^X,f_3^Y\})$ vorfindet. Man bezeichnet die entstandene simpliziale Fläche mit $W^3_{F_X,F_Y}(X,Y)$. Falls $Y=H$ ist, wobei $H$ den oben definierten Hut beschreibt, so ist $W^3_{F_X,F_Y}(X,Y)=W^3_{F_X}(X)=W^3_{F_X,F_Y}(Y,X)$.
\begin{folgerung} 
Seien $(X,<)$ und $(Y, \prec)$ geschlossene Jordan-zusammenhängende simpliziale Flächen, $F_X\in X_2$ eine Fläche in $X$ ,$F_Y \in Y_2$ eine Fläche in $Y$ und $W=W_{F_X,F_Y}^3(X,Y)$. Seien zudem $(Z,<_Z)$ eine geschlossene Jordan-zusammenhängende simpliziale Fläche, die keinen Knoten vom Grad zwei enthält, $\chi(W)=\chi(Z)=2$ und $\vert W_2 \vert=\vert Z_2\vert $. Dann gibt es eine Lochwanderungssequenz $\Sigma$ so, dass $W_{\Sigma}^H \cong Z$ ist. 
\end{folgerung} 
Bei Verwendung der gleichen Bezeichnungen wie in der obigen Konstruktion ergibt sich für die Euler-Charakteristik Folgendes:
\begin{align*}
&\chi(C^m_{f_1^X,f_1^Y}(R^m_{F_2^X F_2^Y}(S^m_{(V_1^X,f_3^X),(V_1^Y,f_3^Y)}(Z)))\\
=&\chi(R^m_{F_2^X F_2^Y}(S^m_{(V_1^X,f_3^X),(V_1^Y,f_3^Y)}(Z)))+1\\
=&\chi(S^m_{(V_1^X,f_3^X),(V_1^Y,f_3^Y)}(Z)))+1\\
=&\chi(Z)-1+1\\
=&\chi(X')+\chi(Y')\\
=&\chi(P^1_{F_X})+\chi(P^1_{F_Y})-2\\
=&\chi(X)+ \chi(Y)-2.
\end{align*}

  \subsection{Verallgemeinerung der Prozedur $W^2$}
Seien nun wieder $(X,<)$ und $(Y,\prec)$ geschlossene simpliziale Flächen, $f_X\in X_1$ eine Kante in $X$ mit zugehörigen Knoten $V^X,W^X \in X_0$ und $f_Y\in Y$ eine Kante in $Y$ mit zugehörigen Knoten $V^Y,W^Y\in Y_0$. Zunächst führe man $X$ und $Y$ zu der simplizialen $Z=X \uplus Y$ zusammen und nutzt an dieser Stelle wieder die Isomorphie von $Z$ zu einer simplizialen Fläche, die das Anwenden der Mender- und Cutteroperatoren erleichtert, nämlich die Fläche festgesetzt durch $f^X=\{f_1^X,f^X_2\}$ und $f^Y=\{f_1^Y,f^Y_2\}$. Nun führt man folgende Operationen auf $Z$ aus:
\begin{enumerate}
 \item Wende einen Cratercut $C^c_{\{f_1^X,f_2^X\}}$ an, um die Kanten $\{f_1^X\},\{f_2^X\}$ zu erhalten.
 \item Wende den Cratercut $C^c_{\{f_1^Y,f_2^Y\}}$ an, um die Kanten $\{f_1^Y\},\{{f_2^Y}\}$ zu erhalten.
\item Wende den Splitmender $S^m_{(\{V^X\},\{f_1^X\}),(\{V^Y\},\{f_1^Y\})}$ an, um die Kanten $\{f_1^X\}$ und $\{{f_1^X}\}$ zu der Kante $\{f_1^X,f_1^Y\}$ zusammenzuführen und um die Knoten $\{V^X,V^Y\}$ und $\{W^X,W^Y\}$ zu erhalten.
 \item Wende nun den Cratermender $C^m_{\{f_2^X\},\{{f_2^Y}\}}$ an, um die Kante $\{f_2^X,f_2^Y\}$ zu erhalten.
 \end{enumerate}
 
 \begin{folgerung} 
Seien $(X,<)$ und $(Y, \prec)$ geschlossene Jordan-zusammenhängende simpliziale Flächen, $f_X\in X_1$ eine Kante in $X$, $f_Y \in Y_1$ eine Kante in $Y$ und $W=W_{f_X,f_Y}^2(X,Y)$. Seien zudem $(Z,<_Z)$ eine geschlossene Jordan-zusammenhängende simpliziale Fläche, die keinen Knoten vom Grad zwei enthält, $\chi(W)=\chi(Z)=2$ und $\vert W_2 \vert=\vert Z_2 \vert$. Dann gibt es eine Lochwanderungssequenz $\Sigma$ so, dass $W_H^{\Sigma} \cong Z$ ist. 
\end{folgerung}
Beim Betrachten der Euler-Charakteristik der simplizialen Fläche $W_2(X,Y)$, wobei hier die Bezeichnungen wie in der obigen Konstruktion übernommen werden, erkennt man 
\begin{align*}
&\chi(W^2_f(X,Y))\\
=&\chi(C^m_{\{f_2^X\}\{{f_2^Y}\}}(S^m_{(\{V^X\},\{f_1^X\}),(\{V^Y\},\{f_1^Y\})}(C^c_{\{f_1^Y,f_2^Y\}}(C^c_{\{f_1^X,f_2^X\}}(Z)))))\\
=&\chi(S^m_{(\{V^X\},\{f_1^X\}),(\{V^Y\},\{f_1^Y\})}(C^c_{\{f_1^Y,f_2^Y\}}(C^c_{\{f_1^X,f_2^X\}}(Z))))+1\\
=&\chi(C^c_{\{f_1^Y,f_2^Y\}}(C^c_{\{f_1^X,f_2^X\}}(Z)))+1-1\\
=&\chi(C^c_{\{f_1^X,f_2^X\}}(Z))-1\\
=&\chi(Z)-2=\chi(X)+\chi(Y)-2.
\end{align*}
\begin{bemerkung}
An dieser Stelle sei angemerkt, dass die Operation Wanderinghole und die verallgemeinerten Prozeduren $W^2$ und $W^3$ für geschlossene simpliziale Flächen wohldefiniert sind. Da aber durch deren einfache Anwendung nur eine Zusammenhangskomponente der simplizialen Fläche verändert wird, reicht es geschlossene Jordan-zusammenhängende simpliziale Flächen unter der Anwendung der Operation und den Prozeduren zu betrachten, denn der allgemeinere Fall ist dann nur eine einfache Folgerung.
\end{bemerkung}
\newpage
\section{Zusammenfassung}
Ziel dieser Arbeit war es, die Manipulation von simplizialen Flächen durch die Operation Wanderinghole zu untersuchen. Es konnte gezeigt werden, dass das Anwenden der Operation im obigem Sinne transitiv ist. Das heißt, falls $X$ und $Y$ simpliziale Flächen mit den oben angeführten Voraussetzungen sind, so kann man durch die Anwendung einer Lochwanderungssequenz, bis auf Isomorphie $X$ in $Y$ überführen. Die Eigenschaft der Transitivität zu erhalten, ist für spätere Untersuchungen und die Implementierung in \emph{Gap} sehr hilfreich, wie man schon am Beispiel der Prozeduren $W^2$ und $W^3$ erkennen kann. Jedoch kann diese Eigenschaft womöglich noch verstärkt werden.\\
In dieser Arbeit wurde die Transitivität über einen Konstruktionsbeweis nachgewiesen. Also wurde eine Lochwanderungssequenz angegeben, mit der man $X$ in $Y$ überführen kann, jedoch wurde keine Aussage über die simplizialen Flächen getroffen, die durch iteratives Anwenden der Operation entstanden sind. Ein erster Ansatz zur Verstärkung der Aussage ist es, zu zeigen, dass keine dieser simplizialen Flächen, die beim Überführen von $X$ in $Y$ entstehen, zu einem Knoten vom Grad zwei inzident ist. Falls dies möglich ist, ist eine weitere Verbesserung der Aussage, zu zeigen, dass dieses Überführen ebenfalls möglich ist, ohne simpliziale Flächen mit 2- oder 3-Waists zu konstruieren. Dies kann man nutzen, um die Implementierung zu optimieren. Denn derzeit werden bei der Anwendung des Algorithmus zur Bestimmung der Bahn einer simplizialen Fläche unter der  Operation Wanderinghole simpliziale Flächen konstruiert, die nicht von Interesse sind, nämlich jene mit 2- oder 3-Waists. Deshalb liefert der Beweis der stärkeren Aussagen eine Optimierungsmöglichkeit der Implementierung.\\
Falls sich die Aussagen doch nicht zeigen lassen, ist es sinnvoll sich zu überlegen, wie viele simpliziale Flächen es ohne 2- oder 3-Waists und einer vorgegebenen Flächenanzahl gibt. Denn dann ist es möglich, eine Abbruchbedingung anzugeben. Sobald alle simplizialen Flächen gefunden wurden, kann der Algorithmus angehalten werden und es ist nicht nötig abzuwarten, bis jegliche simpliziale Fläche konstruiert wurde. Die stärkeren Aussagen wurden mithilfe des \emph{Gap}-Paketes für Flächen mit Flächenanzahl 24 verifiziert.\\
Es ist ebenfalls interessant zu untersuchen, ob man die Operation sinnvoll auf nicht geschlossene simpliziale Flächen definieren kann, um auf diesen Flächen die Wirkung der Operation zu untersuchen. Dabei kann man für Lochwanderungen, die nur innere Kanten verändern, die Operation Wanderinghole analog zum oben skizzierten Fall einer geschlossenen simplizialen Fläche anwenden. Problematisch sind die Flächen, die zu Randkanten inzident sind. Zur Illustration soll, das hier angeführte Bild, andeuten, wie eine solche Lochwanderung aussehen kann. An dieser Stelle verzichtet man auf eine formale Definition dieser verallgemeinerten Operation Wanderinghole und gibt sich zunächst mit der Abbildung zufrieden.
\begin{figure}[H]
\begin{center}
\definecolor{ffffff}{rgb}{1.,1.,1.}
\definecolor{qqqqff}{rgb}{0.,0.,1.}
\definecolor{ffffqq}{rgb}{1.,1.,0.}
\begin{tikzpicture}[line cap=round,line join=round,>=triangle 45,x=1.0cm,y=1.0cm]
%\begin{axis}[
x=1.0cm,y=1.0cm,
axis lines=middle,
ymajorgrids=true,
xmajorgrids=true,
xmin=-4.3,
xmax=18.7,
ymin=-5.339999999999997,
ymax=6.299999999999999,
xtick={-4.0,-3.0,...,18.0},
ytick={-5.0,-4.0,...,6.0},]
\clip(-0.3,-0.1) rectangle (10.3,4.3);
\fill[line width=2.pt,color=ffffqq,fill=ffffqq,fill opacity=\gelb] (0.,2.) -- (4.,2.) -- (4.,0.) -- (0.,0.) -- cycle;
\fill[line width=2.pt,color=ffffqq,fill=ffffqq,fill opacity=\gelb] (6.,2.) -- (6.,0.) -- (10.,0.) -- (10.,2.) -- cycle;
\fill[line width=2.pt,color=ffffqq,fill=ffffqq,fill opacity=\gelb] (1.,2.) -- (3.,2.) -- (2.,3.7320508075688776) -- cycle;
\fill[line width=2.pt,color=ffffqq,fill=ffffqq,fill opacity=0.10000000149011612] (3.,2.) -- (1.,2.) -- (2.,0.2679491924311226) -- cycle;
\fill[line width=2.pt,color=ffffff,fill=ffffff,fill opacity=1.0] (9.,2.) -- (7.,2.) -- (8.,0.2679491924311226) -- cycle;
\fill[line width=2.pt,color=ffffff,fill=ffffff,fill opacity=1.0] (7.,2.) -- (9.,2.) -- (8.,3.7320508075688776) -- cycle;
\fill[line width=2.pt,color=ffffqq,fill=ffffqq,fill opacity=\gelb] (8.,3.7320508075688776) -- (8.,0.2679491924311226) -- (7.,2.) -- cycle;
\fill[line width=2.pt,color=ffffqq,fill=ffffqq,fill opacity=\gelb] (8.,3.7320508075688776) -- (9.,2.) -- (8.,0.2679491924311226) -- cycle;
\draw [line width=2.pt] (3.,2.)-- (2.,3.7320508075688776);
\draw [line width=2.pt] (2.,3.7320508075688776)-- (1.,2.);
\draw [line width=2.pt] (3.,2.)-- (1.,2.);
\draw [line width=2.pt] (1.,2.)-- (2.,0.2679491924311226);
\draw [line width=2.pt] (2.,0.2679491924311226)-- (3.,2.);
%\draw [line width=2.pt,color=ffffff] (9.,2.)-- (7.,2.);
\draw [line width=2.pt,color=ffffff] (7.,2.)-- (8.,0.2679491924311226);
\draw [line width=2.pt,color=ffffff] (8.,0.2679491924311226)-- (9.,2.);
%\draw [line width=2.pt,color=ffffff] (7.,2.)-- (9.,2.);
\draw [line width=2.pt,color=ffffff] (9.,2.)-- (8.,3.7320508075688776);
\draw [line width=2.pt,color=ffffff] (8.,3.7320508075688776)-- (7.,2.);
\draw [line width=2.pt] (8.,3.7320508075688776)-- (8.,0.2679491924311226);
\draw [line width=2.pt] (8.,0.2679491924311226)-- (7.,2.);
\draw [line width=2.pt] (7.,2.)-- (8.,3.7320508075688776);
\draw [line width=2.pt] (8.,3.7320508075688776)-- (9.,2.);
\draw [line width=2.pt] (9.,2.)-- (8.,0.2679491924311226);
\draw [line width=2.pt] (8.,0.2679491924311226)-- (8.,3.7320508075688776);
\draw [line width=2.pt] (4.4,1.36)-- (5.48,1.36);
\draw [line width=2.pt] (5.48,1.36)-- (5.3,1.66);
\draw [line width=2.pt] (5.48,1.36)-- (5.32,1.1);
\draw [line width=2.pt] (5.54,0.42)-- (4.46,0.42);
\draw [line width=2.pt] (4.46,0.46)-- (4.7,0.78);
\draw [line width=2.pt] (4.46,0.46)-- (4.66,0.2);
\begin{scriptsize}
\draw [fill=qqqqff] (1.,2.) circle (2.5pt);
%\draw[color=qqqqff] (1.14,2.37) node {$I$};
\draw [fill=qqqqff] (3.,2.) circle (2.5pt);
%\draw[color=qqqqff] (3.14,2.37) node {$J$};
%\draw[color=ffffqq] (2.48,2.75) node {$Vieleck1$};
\draw [fill=qqqqff] (2.,3.7320508075688776) circle (2.5pt);
%\draw[color=qqqqff] (2.14,4.11) node {$K$};
%\draw[color=ffffqq] (2.48,1.61) node {$Vieleck2$};
\draw [fill=qqqqff] (2.,0.2679491924311226) circle (2.5pt);
%\draw[color=qqqqff] (2.14,0.63) node {$L$};
\draw [fill=qqqqff] (7.,2.) circle (2.5pt);
%\draw[color=qqqqff] (7.14,2.37) node {$M$};
\draw [fill=qqqqff] (9.,2.) circle (2.5pt);
%\draw[color=qqqqff] (9.14,2.37) node {$N$};
%\draw[color=ffffff] (8.48,1.61) node {$Vieleck4$};
\draw [fill=qqqqff] (8.,0.2679491924311226) circle (2.5pt);
%\draw[color=qqqqff] (8.14,0.63) node {$P$};
%\draw[color=ffffff] (8.48,2.75) node {$Vieleck3$};
\draw [fill=qqqqff] (8.,3.7320508075688776) circle (2.5pt);
%\draw[color=qqqqff] (8.14,4.11) node {$O$};
%\draw[color=black] (7.79,2.22) node {$m_1$};
%\draw[color=black] (7.82,1.47) node {$o$};
%\draw[color=black] (7.87,2.94) node {$p_1$};
%\draw[color=black] (8.33,2.94) node {$p_2$};
%\draw[color=black] (8.33,1.52) node {$o_1$};
%\draw[color=black] (8.43,2.22) node {$n_1$};
%\draw[color=black] (5.05,1.26) node {$g_1$};
%\draw[color=black] (5.77,1.9) node {$h_1$};
%\draw[color=black] (5.23,1.62) node {$i_1$};
%\draw[color=black] (5.11,0.98) node {$j_1$};
%\draw[color=black] (4.93,0.66) node {$k_1$};
%\draw[color=black] (4.41,0.36) node {$l_1$};
\end{scriptsize}
%\end{axis}
\end{tikzpicture}
\end{center}
\caption{Verallgemeinerung der Operation Wanderinghole}
\end{figure}
Für diese Verallgemeinerung ist es sinnvoll, Wanderinghole nicht über den Ansatz der Manipulation der simplizialen Fläche über die Anwendung von Mender- und Cutteroperation zu wählen, sondern sich von der Implementierung in \emph{Gap} inspirieren zu lassen und diese Operation, durch eine Manipulation des ordinalen Symbols, zu realisieren.\\
Abschließend kann man also sagen, dass die Transitivität der Operation Wanderinghole eine sehr nützliche Eigenschaft ist, um so neue simpliziale Flächen zu konstruieren und so die Bibliothek im \emph{Gap}-Paket durch hinzufügen der konstruierten Flächen zu erweitern, wie man am Beispiel der Prozeduren $W^2$ und $W^3$ sehen kann. Diese Eigenschaft entfaltet ihre Ausdrucksstärke jedoch erst mit einem Beweis der Transivität der Operation ohne das Erzeugen von simplizialen Flächen mit 2- und 3-Waists. 
\newpage
\section*{Anhang}
\pagestyle{empty}
\begin{linenumbers}
\begin{verbatim}
#!      @Description
#!      1.Find new simplicial surfaces by using the operation
#!      Wanderinghole.
#!      2. Creating new simplicial surfaces which two/three more  
#!      faces by applying a 2-Waist/3-Waist to a simplicial 
#!      surface

LoadPackage("SimplicialSurfaces");
#Read("wl.g");
Read("Classification.g");
Read("CanonicalRepresentativeOfPolygonalSurface.g");

#!      @Arguments: a simplicial surface, a face and a edge of the
#!      simplicial surface, so that the edge is adjacent to the given 
#!      face
#!      @Returns a simplicial surface py performing the wanderinghole
Wandering:=function(S,Face,Edge)
        local Neighbour,A,B,EdgeNA,EdgeFA,EdgeNB,EdgeFB,VertN,
        VertF,EdgesOfF,VerticesOfE,result,VerticesOfF;

        # the orignal vertices/edges/faces of S which have to be
        #manipulated for the Wanderinghole
        Neighbour:=NeighbourFaceByEdgeNC(S,Face, Edge);
        A:=VerticesOfEdgeNC(S,Edge)[1];
        B:=VerticesOfEdgeNC(S,Edge)[2];
        EdgeFA:=OtherEdgeOfVertexInFaceNC(S,A,Edge,Face);
        EdgeNA:=OtherEdgeOfVertexInFaceNC(S,A,Edge,Neighbour);
        EdgeFB:=OtherEdgeOfVertexInFaceNC(S,B,Edge,Face);
        EdgeNB:=OtherEdgeOfVertexInFaceNC(S,B,Edge,Neighbour);
        VertF:=OtherVertexOfEdgeNC(S,A,EdgeFA);
        VertN:=OtherVertexOfEdgeNC(S,A,EdgeNA);

        #test, whether the oeration can be performed
        if VertF=VertN then
                return fail;
        fi;

        #copy of the List of vertices/edges/faces of S
        VerticesOfE:=ShallowCopy(VerticesOfEdges(S));
        EdgesOfF:=ShallowCopy(EdgesOfFaces(S));
        VerticesOfF:=ShallowCopy(VerticesOfFaces(S));

        #change incidences to perform the Wanderinghole
        VerticesOfE[Edge]:=Set([VertF,VertN]);
        EdgesOfF[Face]:=Set([Edge,EdgeFA,EdgeNA]);
        EdgesOfF[Neighbour]:=Set([Edge,EdgeFB,EdgeNB]);
        VerticesOfF[Face]:=Set([A,VertF,VertN]);
        VerticesOfF[Neighbour]:=Set([B,VertN,VertF]);

        #create new surface
        result:=Objectify(PolygonalComplexType,rec());
        SetVerticesOfEdges(result,VerticesOfE);
        SetEdgesOfFaces(result,EdgesOfF);
        SetVerticesOfFaces(result,VerticesOfF);
        SetVerticesAttributeOfVEFComplex(result,
                        VerticesAttributeOfVEFComplex(S));
        SetEdges(result,Edges(S));
        SetFaces(result,Faces(S));
        SetIsNotEdgeRamified(result,true);
         SetIsNotVertexRamified(result,true);

        SetIsTriangular(result,true);
        return result;
end;

#!      @Description Performing the operation Wanderighole 
#!      iteratively on a simplicial surface To get a list of the 
#!      created simplicial surfaces
#!      @Arguments a simplicial surface
#!      @Returns a list of simplicial surfaces

WanderingIterativ:=function(S)
        local SurfaceStack,edge,surf,FaceR,EdgeR,T;

        #create canonical Representative
        S:=CanonicalRepresentativeOfPolygonalSurface2(S)[1];

        #creating surfacestack
        SurfaceStack:=[S];

        #Now we have to perform every possible Wanderhinghole 
        #operation
        #on a surface. Therefore we create a stack of surfaces and
        #for each surface in the stack wanderinghole is performed
        #with each edge in that surface
        for surf in SurfaceStack do
                for edge in Edges(surf) do
                        T:=Wandering(surf,FacesOfEdge(surf,edge)
                                                       [1],edge);
                        if T <> fail then
                                T:=
                                CanonicalRepresentativeOf
                                PolygonalSurface2(T)[1];
                                #test whether surface is already in 
                                #stack
                                if not(T in SurfaceStack) then
                                        Add(SurfaceStack,T);
                                fi;
                        fi;
                od;
        od;
        return SurfaceStack;
end;

#!      @Description: Performing the operation Wanderinghole 
#!      iteratively on a simplicial surface to get a list of all 
#!      created surfaces. In this case we are only interested in
#!      surfaces without 2-waists and 3-waists
#!      @Arguments: a simplicial surface
#!      @Returns a list of surfaces
WanderingIterativWithoutWaist:=function(S)
        local SurfaceStackW,edgeW,surfW,TW;

        #creating canonical representative of surface S
        S:=CanonicalRepresentativeOfPolygonalSurface2(S)[1];

        #creating surfacestack
        SurfaceStackW:=[S];

        #Now we have to perform every possible Wanderhinghole 
        #operation on a surface. Therefore we create a stack of 
        #surfaces and for each surface in the stack wanderinghole is 
        #performed with each edge in that surface
        for surfW in SurfaceStackW do
                for edgeW in Edges(surfW) do
                        TW:=Wandering(surfW,FacesOfEdge(surfW,edgeW)
                                                          [1],edgeW);

                        #test whether surface has 2-Waists or 
                        #3-Waists and
                        #whether the surface created is already in
                        #surfacestack
                        if TW <> fail then
                                if IsAnomalyFree(TW) and 
                                   HasNoWaist(TW) then
                                        TW:=CanonicalRepresentativeOf
                                            PolygonalSurface2(TW)[1];
                                        if not(TW in SurfaceStackW) 	
                                         then
                                               Add(SurfaceStackW,TW);
                                         fi;
                                 fi;
                        fi;
                od;
        od;
        return SurfaceStackW;
end;

#!      @Description: checks whether a simplicial surface is in a 
#!      given list
#!      @Arguments: a simplicial surface and a list of surfaces
#!      @Returns:true if surface is in the given list otherwise the
#!      function returns false
Found:=function(S,L)
        if S in L then
                return false;
        fi;
        return true;
end;

#!      @Description the function creates iteratively simplicial 
#!      surfaces by performing the wanderinghole and checks whether 
#!      these surfaces are in a given list
#!      @Arguments a simplicial surface and a list of surfaces
#!      @Returns a list of simplicial surfaces and if this list is 
#!      the same as the given list, the function returns true 
#!      otherwise the function returns false

FoundSurfaces:=function(S,Listofsurfaces)
        local counter, ende,surface,edge,i,T,Foundsurf;
        if Listofsurfaces=[] then
        Listofsurfaces:=AllSimplicialSurfaces(NumberOfFaces(S),
                        EulerCharacteristic,2);
        fi;

        #replace all simplicial surfaces in Listofsurfaces which 
        #their canonical representative
        Listofsurfaces:=List(Listofsurfaces,surface->
                        CanonicalRepresentativeOf
                        PolygonalSurface2(surface)[1]);
        S:=CanonicalRepresentativeOfPolygonalSurface2(S)[1];
        Foundsurf:=[S];

        #counting how many created surfaces are in Listofsurfaces
        i:=0;
        counter:=0;
        ende:=Length(Listofsurfaces);
        if S in Listofsurfaces then
                counter:=1;
        fi;
        #Now we have to perform every possible Wanderhinghole 
        #operation on a surface. Therefore we create a stack of 
        #surfaces and for each surface in the stack wanderinghole is 
        #performed with each edge in that surface
        for surface in Foundsurf do
                for edge in Edges(surface) do
                        #if all surfaces in listofsurfaces are found 
                        #we can stop the function
                        if counter=ende then
                                return [true, Foundsurf];
                        fi;
                        T:=Wandering(surface,FacesOfEdge
                                  (surface,edge)[1],edge);
                        i:=i+1;

                        #checking whether created surface is in the
                        #given list and if it is in the given list we 
                        #have to increment the counter
                        if T <> fail and HasNoWaist(T) then  ##
                                T:=CanonicalRepresentativeOf
                                   PolygonalSurface2(T)[1];
                                #Print(Found(T,Listofsurfaces));
                                if Found(T,Foundsurf) then
                                        Add(Foundsurf,T);
                                        counter:=counter+1;
                                fi;
                        fi;
                od;
        od;
        return [false,Foundsurf];

end;

#--------------------------------------------------------------------
#!      @Description the functions returns a list of all tupels which 
#!      can be used to perform the wanderinghole
#!      @Arguments a simplicial surface
#!      @Returns a list of tupels where the first entry is a face and 
#!      the second is a edge
H:=function(S)
        local K,g,f,M,F,i,j,VOF,VOE,Facesofedge,Edgesofface;
        K:=[];
        M:=[];
        #for each face in S we collect the edges that arenot incident 
        #to that face but to a neighbourface and share exactly one 
        #vertex with the face
        for F in Faces(S) do
                VOF:=VerticesOfFace(S,F);
                for f in Edges(S) do
                        VOE:=VerticesOfEdge(S,f);
                        if VOE[1] in VOF and not(VOE[2] in VOF) then
                                Add(K,[F,f]);
                        fi;
                        if VOE[2] in VOF and not(VOE[1] in VOF) then
                                        Add(K,[F,f]);
                        fi;
                od;
        od;
        for g in K do
                Facesofedge:=FacesOfEdge(S,g[2]);
                Edgesofface:=EdgesOfFace(S,g[1]);
                for i in [1,2,3] do
                        for j in [1,2] do

                                if Facesofedge[j]  in
                                FacesOfEdge(S,Edgesofface[i]) then
                                        Add(M,g);
                                fi;
                        od;
                od;
        od;
        return Set( M);
end;

#!      @Description checks whether simplicial surface has no 
#!      vertices with face degree 2 or 3
#!      @Arguments a simplicial surfa
#!      @Returns true if there is a vertex with facedegree 2 or 3
HasNoWaist:=function(S)
        local FaceDegrees;
        FaceDegrees:=FaceDegreesOfVertices(S);
        if 2 in FaceDegrees then or 3 in FaceDegrees then
                return false;
        fi;
        return true;
end;

#!      @Description the function creates a simplicial surface with 
#!      an 2-Waist by adding 2 faces to a given surface and using the 
#!      mender- and cutter-operator
#!      @Arguments a simplicial surface and an edge
#!      Returns simpilical surface with 2-waist

CreatingTwoWaist:=function(S,f)
        local g,VerticesOfEdges,EdgesOfFaces,OpenBag;
        #define surface OpenBag
        VerticesOfEdges:=[[2,3],[1,3],[1,2],[1,2]];
        EdgesOfFaces:=[[1,2,3],[1,2,4]];
        S:=CraterCut(S,f);
        OpenBag:=SimplicialSurfaceByDownwardIncidence
                 (VerticesOfEdges, EdgesOfFaces);                                          
        #creating surface with 2-waist
        S:=DisjointUnion(S,OpenBag,Length(Vertices(S)))[1];
        S:=SplitMend(S,SplitMendableFlagPairs(S)[1]);
        S:=CraterMend(S,CraterMendableEdgePairs(S)[2]);
        return S;
end;

#!      Description the function gets a simplicial surface with n 
#!      faces as argument and creates a simplicial surface with n+2 
#!      faces and a 3-waist
#!      @Arguments simplicial and a face
#!      @Returns a simplicial surface with 3-waist

CreatingThreeWaist:=function(S,F)
        local Hut;
        #define the simplicial surface Hut
        Hut:=RemoveFace(Tetrahedron(),1);

        #remove face F from the surface S
        S:=RemoveFace(S,F);

        #creating surface with 3-waist
        S:=DisjointUnion(S,Hut,Length(Vertices(S)))[1];
        S:=SplitMend(S,SplitMendableFlagPairs(S)[1]);
        S:=RipMend(S,RipMendableEdgePairs(S)[2]);
        S:=CraterMend(S,CraterMendableEdgePairs(S)[2]);
        return S;
end;



#--------------------------------------------------------------------
#!      Profiling
TimingTest:=function()
        local S,R;
        S:=AllSimplicialSurfaces(10,EulerCharacteristic,2)[1];
        R:=Runtime();
        WanderingIterativ(S);
        Print("10: ",Runtime()-R,"\n");

        S:=AllSimplicialSurfaces(12,EulerCharacteristic, 2)[1];

        R:=Runtime();
         WanderingIterativ(S);
         Print("12: ",Runtime()-R,"\n");

end;
#-------------List of Functions----------------------

ListOfFunctions:=[Wandering,NeighbourFaceByEdge,VerticesOfEdge,
        OtherEdgeOfVertexInFace,OtherVertexOfEdge,Append,
        SimplicialSurfaceByDownwardIncidence,Filtered,
        EdgesOfFace,VerticesOfEdge,Add,Faces,WanderingIterativ,
        CanonicalRepresentativeOfPolygonalSurface2];

\end{verbatim}
\end{linenumbers}
\cleardoublepage
 \nocite{*}
 \bibliography{literatur}
\bibliographystyle{plain}
\end{document}

