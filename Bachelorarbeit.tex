 \documentclass[12pt,titlepage]{article}
\usepackage[ngerman]{babel}
\usepackage[utf8]{inputenc}
\usepackage[a4paper,lmargin={4cm},rmargin={2cm},
tmargin={2.5cm},bmargin = {2.5cm}]{geometry}
\usepackage{amsmath}
\usepackage{amssymb}
\usepackage{amsthm}
\usepackage{cleveref}
\usepackage{enumerate}
\usepackage{thmtools}
\linespread{1.25}
\usepackage{color}
\usepackage{verbatim}
\newcommand{\gelb}{0.550000011920929}
\usepackage{pgf,tikz,pgfplots}
\pgfplotsset{compat=1.15}
\usepackage{mathrsfs}
\usetikzlibrary{arrows}
%\usepackage{scrheadings}
\pagestyle{headings}
\usepackage{titlesec}                % für Kontrolle der Abschnittüberschriften
\declaretheorem[name=Lemma]{lemma}
\declaretheorem[name=Folgerung]{folgerung}
\declaretheorem[name=Beispiel]{bsp}
\declaretheorem[name=Herleitung]{herleitung}
\declaretheorem[name=Definition]{definition}
\declaretheorem[name=Bemerkung]{bemerkung}
\declaretheorem[name=Satz]{satz}
\declaretheorem[name=Vorüberlegung]{vor}

\begin{document}
\begin{titlepage}
    \begin{center}
      \large
      \textsc{Rheinisch-Westf\"alische Technische Hochschule Aachen}\\
      Lehrstuhl B für Mathematik \\
      Univ.-Prof. Dr.  Alice Niemeyer\\
      \vspace{3 cm}
      \huge  Manipulation diskreter simplizialer Flächen \\
      \vspace{1 cm}
      \large Bachelorarbeit\\
      %\large Bachelor's Thesis\\
      \vspace{2 cm}
       \vspace{1 cm}
      \Large Reymond Akpanya\\
      \large Matrikelnummer: 357115\\
      %\vspace{2 cm}
      %\large Vorgelegt am: 28.09.2018
      \vspace{3.5 cm}
%      last build:
 %    \today \\[4em]
\begin{flalign*}
&\text{Vorgelegt am:}&\text{28.09.2018}&\\
&\text{Gutachter:}&\text{Prof. Dr. Alice Niemeyer}&\\
&\text{Zweitgutachter:}&\text{Prof. Dr Olaf Wittich}&\\[1em]
\end{flalign*}
    \end{center}
% \begin{flalign*}
 %&\text{ } &\text{ 25.09.2018 }
 %\end{flalign*}
\end{titlepage}
%---------------------------
\begin{comment}
\thispagestyle{empty}
\pagenumbering{arabic}
%\begin{titlepage}
\noindent\rule{\textwidth}{0.5pt}
\centerline{\textbf{\large{Name des Dokuments}}}
\centerline{Reymond Akpanya}
\noindent\rule{\textwidth}{0.5pt}
\newline
\end{comment}
%\farb
%\section*{Inhaltsverzeichnis}

\thispagestyle{empty}
\tableofcontents
%\addcontentsline{toc}{section}{Einleitung}
\newpage
\setcounter{page}{1}
\section{Einleitung}
Ziel dieser Arbeit ist es, wie im Titel schon angedeutet, die Manipulation von simplizialen Flächen zu untersuchen. Man versucht durch die Anwendung von gewissen Operationen aus einer simplizialen Fläche, bestehend aus Knoten, Kanten und Flächen, eine weitere zu konstruieren. In dieser Arbeit werden jedoch nur diese behandelt, die die Kanten einer simplizialen Fläche verändern, hierzu später mehr.\\
Zuallererst wird jedoch im erstem Kapitel dieser Arbeit geklärt, was eine simpliziale Fläche überhaupt ist. Und zusätzlich werden einige einführende Definitionen und Beispiele, die man dem Skript \emph{Combinatorial Simplicial Surfaces} von Wilhelm Plesken entnehmen kann, angeführt. Im zweitem Kapitel definiert man dann die oben schon erwähnten Operationen, die die Kanten einer simplizialen Fläche manipulieren. Diese werden dann genutzt, um im darauf folgendem Kapitel eine Operation zu definieren, die Wanderinghole genannt wird. Die Untersuchung dieser Operation ist die Hauptfragestellung dieser Arbeit. Es wird hier Klarheit über Eigenschaften wie zum Beispiel Transitivität dieser Operation gebracht. Dies wird aber später noch näher erläutert. Und zum Schluss wird eine simpliziale Fläche vergrößert, in diesem Fall bedeutet dies, dass die Anzahl der Flächen und damit auch die der Kanten und Knoten vergrößert wird, um dann die Anwendung der Operation Wanderinghole auf diese zu untersuchen. \\
Für die Untersuchung dieser Fragestellungen wurde das Computer-Algebra-System \emph{Gap} zur Hilfe genommen und mithilfe dessen und dem Gap-Packet \emph{SimplicialSurface} von Markus Baumeister und Alice Niemeyer zunächst einmal die Operation Wanderinghole implementiert, um so ein erstes Verständnis für diese zu bekommen und dann daraus folgend erste Behauptungen aufstellen zu können. Der Quellcode hierzu liegt im Anhang bei. 
\newpage
\section{Wiederholung}
\begin{definition}  \label{def1} Sei $X=X_{0} \biguplus X_{1} \biguplus X_{2}$ eine abzählbare Menge mit $X_{i} \ne \emptyset$ für $i=0,1,2$ und $<$ eine transitive Relation auf  ($X_{0}\times X_{1}) \cup (X_{1}\times X_{2})\cup (X_{0}\times X_{2}$). Man nennt $X_{0}$ \emph{Knoten}, $X_{1}$ \emph{Kanten}, $X_{2}$ \emph{Flächen} und $<$ \emph{Inzidenz} einer \emph{simplizialen Fläche} $(X,<)$, falls folgende Eigenschaften erfüllt sind:
 \begin{enumerate}
\item Für jede Kante $e \in X_{1}$ existieren genau 2 Knoten $V_1,V_2 \in X_{0}$ mit $V_1,V_2 < e$. 
\item Für jede Fläche $F\in X_2$ gibt es genau drei Kanten $e_1,e_2,e_3 \in X_{1}$ mit $e_1,e_2,e_3 < F$. 
\item Für jede Kante $e \in X_{1}$ gibt es entweder genau 2 Flächen $F_{1},F_{2} \in X_{2}$ mit $e <F_{1},F_2$ oder
genau eine Fläche $F \in X_{2}$ mit $e < F$. Im ersten Fall sind $F_{1}$ und $F_{2}$ \emph{$(e)$-Nachbarn} und $e$ ist eine \emph{innere Kante}, andernfalls ist $e$ eine \emph{Randkante}. 
\item Für jeden Knoten $V \in X_{0}$ existieren endlich viele $F_{i}\in X_{2}$ mit $V < F_{i}$. Diese $F_{i}$ können in einem Tupel $(F_{1},\ldots,F_{n})$ geordnet werden so, dass $F_{i}$ und $F_{i+1}$ adjazent sind durch eine Kante $e_{i}$ mit $V <e_{i}$ für $i=1, \ldots, n-1$. Das Tupel $(F_1,\ldots,F_n)$ wird auch \emph{Schirm} genannt. Gilt dies auch für $F_{n}$ und $F_{1}$, so ist $V$ ein \emph{innerer} Knoten. Ist $V$ kein innerer Knoten, so ist er 
 ein Eckknoten. 
 \item Sei $V \in X_0$ und ein Tupel $(F_1,\ldots,F_n)$ so wie zuvor beschrieben, wobei $F_i \in X_2$ für $i=0,\ldots,n$. Dann ist n der \emph{Grad des Knotens} $V$. Für den Grad eines Knotens $V$ schreibt man $\deg(V)$.
 \item Die Menge aller inneren Knoten einer Kante $e \in X_1$ bezeichnet man mit $X_0^0(e).$
\end{enumerate}
\end{definition}
Es sei angemerkt, dass es für einen gegebenen Knoten eine endliche Anzahl von Schirmen gibt. Diese sind jedoch alle äquivalent, dass heißt sie können durch zyklische Permutation umgeordnet werden.\\ Es ist außerdem möglich, die simpliziale Fläche $(X,<)$ mit $X$ zu identifizieren.

\begin{bemerkung}  Eine \emph{geschlossene} simpliziale Fläche ist eine simpliziale Fläche, für die alle Kanten innere Kanten sind. Die Anzahl der Flächen einer geschlossenen simplizialen Fläche ist durch $2$ teilbar, da
\[
\vert X_{2} \vert = \frac{2\vert X_{1}\vert}{3}.
\]
Die Anzahl der Kanten ist insbesondere durch 3 teilbar.\\
\end{bemerkung}
Im Folgendem werden nur endliche simpliziale Flächen betrachtet, das heißt simpliziale Flächen $(X,<)$ mit $\vert X \vert < \infty$.\\

 \begin{bsp}
 \begin{enumerate}
\item 
 Es gibt bis auf Isomorphie nur eine simpliziale Fläche bestehend aus einer Fläche, welche durch 
\begin{align*}
X_{0}=\{\,V_{1}&,V_{2},V_{3}\,\}, X_{1}=\{\,e_{1},e_{2},e_{3}\,\}, X_{2}=\{\,F_{1}\,\} \text{ und } x<y \Leftrightarrow \\
 (x,y)\in \{\,&(e_{1},F_{1}),(e_{2},F_{1}),(e_{3},F_{1}),(V_{1},F_{1}),(V_{1},e_{2}),(V_{1},e_{3}),(V_{2},F_{1}), (V_{2},e_{1}),\\ &(V_{2},e_{3}),
 (V_{3},F_{1}),(V_{3},e_{1}),(V_{3},e_{3})\,\} 
\end{align*} 
beschrieben wird. Man nennt diese simpliziale Fläche \emph{Dreieck}. \\
%--------------------------Bild-------------------------
\begin{figure}[h]
\definecolor{ffffqq}{rgb}{1.,1.,0.}
\definecolor{qqqqff}{rgb}{0.,0.,1.}
\definecolor{xdxdff}{rgb}{0.49019607843137253,0.49019607843137253,1.}
\begin{tikzpicture}[line cap=round,line join=round,>=triangle 45,x=1.0cm,y=1.0cm]
x=1.0cm,y=1.0cm,
axis lines=middle,
ymajorgrids=true,
xmajorgrids=true,
xmin=-2.690233964457656,
xmax=14.560601622891102,
ymin=1,
ymax=4.805217252710678,
xtick={-2.0,-1.0,...,14.0},
ytick={-1.0,-0.0,...,4.0},]
\clip(-7.690233964457656,-0.4) rectangle (8.560601622891102,3.8641016151377553);
\fill[line width=2.pt,color=ffffqq,fill=ffffqq,fill opacity=0.5] (-2.,0.) -- (2.,0.) -- (0.,3.4641016151377553) -- cycle;
\draw [line width=2.pt] (-2.,0.)-- (2.,0.);
\draw [line width=2.pt] (2.,0.)-- (0.,3.4641016151377553);
\draw [line width=2.pt] (0.,3.4641016151377553)-- (-2.,0.);
\begin{scriptsize}
\draw [fill=qqqqff] (-2.,0.) circle (2.5pt);
\draw[color=qqqqff] (-2.082704537251026,0.48500799257463995) node {$V_1$};
\draw [fill=qqqqff] (2.,0.) circle (2.5pt);
\draw[color=qqqqff] (2.147500363298844,0.32) node {$V_2$};
\draw[color=black] (0.03989827632275619,1.2875468655512998) node {$F$};
\draw[color=black] (0.07740009281699262,-0.30001889154005933) node {$e_3$};
\draw[color=black] (1.2774582206325584,2.030082832137181) node {$e_1$};
\draw[color=black] (-1.1376587615962677,2.030082832137181) node {$e_2$};
\draw [fill=qqqqff] (0.,3.4641016151377553) circle (2.5pt);
\draw[color=qqqqff] (0.1374029992077709,3.7851678440674466) node {$V_3$};
\end{scriptsize}
\end{tikzpicture}
\caption{Dreieck}
\end{figure}

%-------------------------------------------------------
 \item
 Für $n \in \mathbb{N}$ definieren wir die \emph{n-fache Fläche} $n \Delta$ durch $X=\{X_{0},X_{1},X_{2}\}$, wobei
 \begin{align*}
  X_{0}=\{& \,V_{j}^{k}\,\vert\, j=1,2,3 ;k=1,\ldots,n\,\}, X_{1}=\{\,e_{j}^{k}\,\vert\, j=1,2,3 ;k=1,\ldots,n\,\},\\
   X_{2}=\{&F_{1},\ldots,F_{n}\} \text{ und } x<y \Leftrightarrow \\
 (x,y)\in \{\,&(e_{1}^k,F_{k}),(e_{2}^k,F_{k}),(e_{3}^k,F_{k}),(V_{1}^k,F_{k}),(V_{1}^k,e_{2}^k),(V_{1}^k,e_{3}^k),(V_{2}^k,F_{k}), (V_{2}^k,e_{1}^k),\\ &(V_{2}^k,e_{3}^k),
 (V_{3}^k,F_{k}),(V_{3}^k,e_{1}^k),(V_{3}^k,e_{3}^k)\mid k=0,\ldots,n\} 
\end{align*}
 
 \newpage
 \item 
 Der \emph{Janus-Kopf} ist eine geschlossene simpliziale Fläche, die aus zwei Flächen besteht.	Sie besitzt 3 innere Knoten und  3 innere Kanten und wird definiert durch
 \begin{align*}
 X_{0}=\{\,&V_{1},V_{2},V_{3}\,\} ,X_{1}=\{\,e_{1},e_{2},e_{3}\,\},X_{3}=\{\, F_{1},F_{2}\,\}  \text{ und } x<y \Leftrightarrow \\
 (x,y)\in\{&\,(e_{1},F_{1}),(e_{1},F_{2}),(e_{2},F_{1}),(e_{2},F_{2}),(e_{3},F_{1}),(e_{3},F_{2}),(V_{1},e_{2}),(V_{1},e_{3}),\\ &(V_{1},F_{1}),
  (V_{1},F_{2}),(V_{2},e_{1}),(V_{2},e_{3}),(V_{2},F_{1})
 (V_{2},F_{2}), (V_{3},e_{1}), (V_{3},e_{2}),\\&(V_{3},F_{1}),(V_{3},F_{2}) \,\}.
 \end{align*}

 %----bild----------------------------
 \begin{figure}[h]
\definecolor{sqsqsq}{rgb}{0.12549019607843137,0.12549019607843137,0.12549019607843137}
\definecolor{ttqqqq}{rgb}{0.2,0.,0.}
\definecolor{ffffqq}{rgb}{1.,1.,0.}
\definecolor{qqqqff}{rgb}{0.,0.,1.}
\begin{tikzpicture}[line cap=round,line join=round,>=triangle 45,x=1.0cm,y=1.0cm]

x=1.0cm,y=1.0cm,
axis lines=middle,
ymajorgrids=true,
xmajorgrids=true,
xmin=-5.056290110700678,
xmax=5.380866801866215,
ymin=-0.9227448489396118,
ymax=4.359364127681193,
xtick={-5.0,-4.5,...,5.0},
ytick={-0.5,0.0,...,4.0},]
\clip(-7.056290110700678,-0.4) rectangle (5.380866801866215,3.8641016151377553);
\fill[line width=2.pt,color=ffffqq,fill=ffffqq,fill opacity=0.5] (-2.,0.) -- (2.,0.) -- (0.,3.4641016151377553) -- cycle;
\fill[line width=2.pt,color=ffffqq,fill=ffffqq,fill opacity=0.5] (0.,3.4641016151377553) -- (2.,0.) -- (4.,3.464101615137754) -- cycle;
\draw [line width=2.pt] (-2.,0.)-- (2.,0.);
\draw [line width=2.pt] (2.,0.)-- (0.,3.4641016151377553);
\draw [line width=2.pt] (0.,3.4641016151377553)-- (-2.,0.);
\draw [line width=2.pt,color=black] (0.,3.4641016151377553)-- (2.,0.);
\draw [line width=2.pt] (2.,0.)-- (4.,3.464101615137754);
\draw [line width=2.pt,color=black] (4.,3.464101615137754)-- (0.,3.4641016151377553);
\begin{scriptsize}
\draw [fill=qqqqff] (-2.,0.) circle (2.5pt);
\draw[color=qqqqff] (-2.347666183295526,0.0934108260899206) node {$V_1$};
\draw [fill=qqqqff] (2.,0.) circle (2.5pt);
\draw[color=qqqqff] (2.390727102068595,0.0934108260899206) node {$V_2$};
\draw[color=black] (0.04884098783747624,1.2554444197699985) node {$F_1$};
\draw[color=black] (0.057916776457099625,-0.22407545041275472) node {$e_3$};
\draw[color=black] (1.174238776670776,1.9089012003828816) node {$e_1$};
\draw[color=black] (-1.1673146871920574,1.9089012003828816) node {$e_2$};
\draw [fill=qqqqff] (0.,3.4641016151377553) circle (2.5pt);
\draw[color=qqqqff] (0.08514414231596978,3.75145261535057) node {$V_3$};
\draw[color=sqsqsq] (2.0455144841546207,2.417145363081791) node {$F_2$};
\draw[color=black] (3.4162912160860375,1.9089012003828816) node {$e_3$};
%\draw[color=black] (3.4162912160860375,1.9089012003828816) node {$e_3$};
\draw[color=sqsqsq] (2.054590272774244,3.6603620787860503) node {$e_2$};
\draw [fill=qqqqff] (4.,3.464101615137754) circle (2.5pt);
\draw[color=qqqqff] (4.087566923569883,3.75145261535057) node {$V_1$};
\end{scriptsize}
\end{tikzpicture}
\caption{Janus-Kopf}
\end{figure}
 %------------------------------------
 \item 
 Die simpliziale Fläche \emph{Open-Bag} ist eine simpliziale Fläche, die aus dem \emph{Janus-Kopf} hervorgeht, wenn man die Kante $e_{2}$ verdoppelt, das heißt sie wird beschrieben durch
%\begin{figure}[h]
 \begin{align*}
  X_{0}=\{\,V_{1},&V_{2},V_{3}\,\},X_{1}=\{\,e_{1},e_{2},e_{3},e_{4} \,\}, X_{2}=\{\,F_{1},F_{2}\,\} \text{ und } x<y \Leftrightarrow\\
 (x,y)\in\{&\,(e_{1},F_{1}),(e_{1},F_{2}),(e_{2},F_{1}),(e_{3},F_{1}),(e_{3},F_{2}),(e_{4},F_{2}),(V_{1},e_{2}),(V_{1},e_{3}),\\ &(V_{1},e_{4}),
  (V_{1},F_{1}),(V_{1},F_{2}),(V_{2},e_{1}),(V_{2},e_{3})
 (V_{2},F_{1}), (V_{2},F_{2}), (V_{3},e_{1}),\\&(V_{3},e_{2}),(V_{3},e_{4}),(V_{3},F_{1}),(V_3,F_2) \,\}.
 \end{align*}
 \end{enumerate}
%--------------------------------------------
\begin{figure}[h]
\definecolor{sqsqsq}{rgb}{0.12549019607843137,0.12549019607843137,0.12549019607843137}
\definecolor{ttqqqq}{rgb}{0.2,0.,0.}
\definecolor{ffffqq}{rgb}{1.,1.,0.}
\definecolor{qqqqff}{rgb}{0.,0.,1.}
\begin{tikzpicture}[line cap=round,line join=round,>=triangle 45,x=1.0cm,y=1.0cm]

x=1.0cm,y=1.0cm,
axis lines=middle,
ymajorgrids=true,
xmajorgrids=true,
xmin=-5.056290110700678,
xmax=5.380866801866215,
ymin=-0.9227448489396118,
ymax=4.359364127681193,
xtick={-5.0,-4.5,...,5.0},
ytick={-0.5,0.0,...,4.0},]
\clip(-7.056290110700678,-0.4) rectangle (5.380866801866215,3.859364127681193);
\fill[line width=2.pt,color=ffffqq,fill=ffffqq,fill opacity=0.5] (-2.,0.) -- (2.,0.) -- (0.,3.4641016151377553) -- cycle;
\fill[line width=2.pt,color=ffffqq,fill=ffffqq,fill opacity=0.5] (0.,3.4641016151377553) -- (2.,0.) -- (4.,3.464101615137754) -- cycle;
\draw [line width=2.pt] (-2.,0.)-- (2.,0.);
\draw [line width=2.pt,color=black] (2.,0.)-- (0.,3.4641016151377553);
\draw [line width=2.pt] (0.,3.4641016151377553)-- (-2.,0.);
\draw [line width=2.pt,color=black] (0.,3.4641016151377553)-- (2.,0.);
\draw [line width=2.pt] (2.,0.)-- (4.,3.464101615137754);
\draw [line width=2.pt,color=black] (4.,3.464101615137754)-- (0.,3.4641016151377553);
\begin{scriptsize}
\draw [fill=qqqqff] (-2.,0.) circle (2.5pt);
\draw[color=qqqqff] (-2.347666183295526,0.0934108260899206) node {$V_1$};
\draw [fill=qqqqff] (2.,0.) circle (2.5pt);
\draw[color=qqqqff] (2.390727102068595,0.0934108260899206) node {$V_2$};
\draw[color=sqsqsq] (0.04884098783747624,1.2554444197699985) node {$F_1$};
\draw[color=black] (0.057916776457099625,-0.22407545041275472) node {$e_3$};
\draw[color=black] (1.174238776670776,1.9089012003828816) node {$e_1$};
\draw[color=black] (-1.1673146871920574,1.9089012003828816) node {$e_2$};
\draw [fill=qqqqff] (0.,3.4641016151377553) circle (2.5pt);
\draw[color=qqqqff] (0.08514414231596978,3.75145261535057) node {$V_3$};
\draw[color=sqsqsq] (2.0455144841546207,2.417145363081791) node {$F_2$};
\draw[color=black] (3.4162912160860375,1.9089012003828816) node {$e_3$};
\draw[color=sqsqsq] (2.054590272774244,3.6603620787860503) node {$e_4$};
\draw [fill=qqqqff] (4.,3.464101615137754) circle (2.5pt);
\draw[color=qqqqff] (4.087566923569883,3.75145261535057) node {$V_1$};
\end{scriptsize}
\end{tikzpicture}
\caption{Open-Bag}
\end{figure}
\end{bsp}
 %------------------------------------
\newpage
\begin{definition} 
Sei $(X,<)$ eine simpliziale Fläche. Für $i,j \in \{\,0,1,2\,\}$ mit $i \neq j$ und $x \in X_{i}$ definiert man die Menge $X_{j}(x)$ als
\[
X_{j}(x):=\{\,y \in X_{j}\,|\,x < y\,\} \text{, falls $i < j$  }
\]
bzw. 
\[
X_{j}(x):=\{\,y \in X_{i}\,|\,y < x\}, \text{ falls $j<i$}.
\]
Für $S \subseteq X_{i}$ ist 
\[
X_j(S):= \bigcup_{x\in S}X_{j}(x).
\]
\end{definition}

\begin{bemerkung}
Für eine simpliziale Fläche $(X,<)$ können die Bedingungen in \Cref{defss} wie folgt umformuliert werden:
\begin{itemize}
\item $\vert X_{0}(e)\vert=2$ für alle $e \in X_{1}$,
\item $\vert X_{0}(F)\vert=3$ für alle $F \in X_{2}$,
\item $\vert X_{1}(F)\vert=3$ für alle $F \in X_{2}$,
\item $1\leq  \vert X_{2}(e)\vert \leq 2$ für alle $e \in X_{1}$.

\end{itemize}
\end{bemerkung}

\begin{definition} Seien $(X,<)$ und $(Y,\prec)$ simpliziale Flächen.
\begin{enumerate}
 \item Man nennt eine bijektive Abbildung $\alpha: X \to Y$ einen Isomorphismus, falls $A<B$ in $(X,<)$ genau dann gilt, wenn $\alpha(A) \prec \alpha(B)$ in $(Y,\prec)$ gilt. In diesem Fall schreibt man $X \cong Y$.
\item Eine surjektive Abbildung $\alpha: X \to Y$ heißt Überdeckung, falls aus $A<B$ in $(X,<)$ folgt, dass $\alpha(A) \prec \alpha(B)$ in $(Y,\prec)$ gilt. 
\end{enumerate}
\end{definition}
Es sei angemerkt, dass eine Überdeckung $\alpha:X\to Y$ surjektive Abbildungen $X_{i} \to Y_{i}$ und ein Isomorphismus $\beta:X \to Y$ bijektive Abbildungen $X_{i} \to Y_{i}$ für $i=0,1,2$ induziert.

Um simpliziale Flächen vollständig beschreiben zu können, führen wir eine Notation ein. Man beachte, dass die hier eingeführte Notation stark von der Nummerierung der Knoten, Kanten und Flächen abhängt, abgesehen davon ist sie eindeutig.
\begin{definition}
 Sei $(X,<)$ eine simpliziale Fläche, deren Knoten $V_{1},\ldots,V_{n}$, Kanten $e_{1},\ldots,e_{k}$ und Flächen $F_{1},\ldots,F_{m}$ ausgehed von ihrer Nummerierung linear geordnet sind. Das \emph{Symbol} von $(X,<)$ ist definiert durch 
\[
\mu((X,<)):=(n,k,m;(X_{0}(e_{1}),\ldots,X_{0}(e_{k})),(X_{1}(F_{1}),\ldots,X_{1}(F_{m}))).
\]
Man kann im Symbol die $V_{i}$ durch $i$, die $e_{j}$ durch $j$ und die $F_{l}$ durch $l$ ersetzen und nennt dann das resultierende Symbol das \emph{ordinale Symbol} $\omega((X,<))$ von $(X,<)$.
\end{definition}

Im nächsten Abschnitt werden zur Vereinfachung der Konstruktion von simplizialen Flächen Bilder eingeführt, die nur Ausschnitte einer simplizialen Fläche zeigen sollen. Durch das unten eingeführte Bild soll beispielsweise angedeutet werden, dass eine simpliziale Fläche $(X,<)$ betrachtet wird, wobei hier nur $F\in X_2,e_1,e_2,e_3\in X_1$ und $V_1,V_2,V_3 \in X_0$ mit 
\begin{itemize}
 %\item $\vert X_{2}\vert \geq 3$,
 \item $e_{i} < F$ für alle $i \in \{1,2,3\}$,
 \item $V_{i}<e_{j}$ für alle $i \in \{1,2,3\}$ und $j \in \{1,2,3\} \setminus\{i\}$ und
 \item $V_{i} < F$ für alle $i \in \{1,2,3\}$
\end{itemize}  
von Bedeutung sind.\\

%------------bild2--------------------
\begin{figure}[h] 
\definecolor{qqqqff}{rgb}{0.,0.,1.}
\definecolor{uuuuuu}{rgb}{0.26666666666666666,0.26666666666666666,0.26666666666666666}
\definecolor{ududff}{rgb}{0.30196078431372547,0.30196078431372547,1.}
\definecolor{ffffqq}{rgb}{1.,1.,0.}
\begin{tikzpicture}[line cap=round,line join=round,>=triangle 45,x=1.4cm,y=1.4cm]
%\begin{axis}[
x=1.4cm,y=1.4cm,
axis lines=middle,
ymajorgrids=true,
xmajorgrids=true,
xmin=-5.3,
xmax=4.0600000000000005,
ymin=-0.46,
ymax=4.3,
xtick={-4.0,-3.0,...,7.0},
ytick={-2.0,-1.0,...,6.0},]
\clip(-5.0,-0.1) rectangle (4.06,4.);
\fill[line width=2.pt,color=ffffqq,fill=ffffqq,fill opacity=0.550000011920929] (-2.,0.) -- (2.,0.) -- (2.,4.) -- (-2.,4.) -- cycle;
\fill[line width=2.pt,color=ffffqq,fill=ffffqq,fill opacity=0.15000000596046448] (-1.,1.) -- (1.,1.) -- (0.,2.7320508075688776) -- cycle;
\draw [line width=2.pt,color=uuuuuu] (-1.,1.)-- (1.,1.);
\draw [line width=2.pt,color=uuuuuu] (1.,1.)-- (0.,2.7320508075688776);
\draw [line width=2.pt,color=uuuuuu] (0.,2.7320508075688776)-- (-1.,1.);
\begin{scriptsize}
\draw [fill=ududff] (-1.,1.) circle (2.5pt);
\draw[color=black] (-0.97,0.8) node {$V_1$};
\draw [fill=ududff] (1.,1.) circle (2.5pt);
\draw[color=black] (1.17,0.8) node {$V_2$};
\draw[color=black] (0.06,1.75) node {$F$};
\draw[color=uuuuuu] (0.11,0.8) node {$e_3$};
\draw[color=uuuuuu] (0.81,2.06) node {$e_1$};
\draw[color=uuuuuu] (-0.81,2.06) node {$e_2$};
\draw [fill=qqqqff] (0.,2.7320508075688776) circle (2.5pt);
\draw[color=black] (0.19,3.00) node {$V_3$};
\end{scriptsize}
%\end{axis}
\end{tikzpicture}
\caption{Ausschnitt einer simplizialen Fläche}
\label{abb4}
\end{figure}
Die Fläche $X$ kann aber mehr Knoten,Kanten und Flächen beinhalten, was durch den gelben Hintergrund in Abbildung \ref{abb4} angedeutet werden soll.
%--------------------------------------------------------
\newpage
 \section{Konstruktion von Simplizialen Flächen}
 \subsection{Randkantenpaare}
 Zunächst werden Definitionen eingeführt, die den Zugang zu den unten definierten Operatoren erleichtern sollen.\\
 
 \begin{definition}
 Sei $(X,<)$ eine simpliziale Fläche.
 \begin{enumerate}
 \item Sei $e \in X_{1}$ eine innere Kante von $(X,<)$. Diese ist vom Typ $i$ mit $i \in \{0,1,2\}$, falls $\vert X_{0}^{0}(e) \vert =i$, falls $e$ inzident zu $i$ Knoten ist.
 \item Man nennt $\{e,f\}$ ein Randkantenpaar, falls $e,f \in X_{1}$ Randkanten sind und $e$ und $f$ zu verschiedenen Flächen gehören, das heißt 
 \[
 \nexists F \in X_2 : e<F \land f<F.
 \]
 \item \begin{enumerate}
  \item[a)] Man nennt $\{e,f\}$ ein \emph{Randkantenpaar vom Typ 2}, falls $\{e,f\}$ ein Randkantenpaar ist und $X_{0}(e)=X_{0}(f)$ gilt. Das Randkantenpaar ist \emph{mendable}, falls $\vert X_2(e)\vert=\vert X_2(f)\vert =1$ ist.
  %-----------------------------------------------------bild2
  \begin{figure}[h]
  \definecolor{ududff}{rgb}{0.30196078431372547,0.30196078431372547,1.}
\definecolor{ffffqq}{rgb}{1.,1.,0.}
\definecolor{qqqqff}{rgb}{0.,0.,1.}
\definecolor{xdxdff}{rgb}{0.49019607843137253,0.49019607843137253,1.}
\begin{tikzpicture}[line cap=round,line join=round,>=triangle 45,x=1.5cm,y=1.5cm]
%\begin{axis}[
x=1.4cm,y=1.4cm,
axis lines=middle,
ymajorgrids=true,
xmajorgrids=true,
xmin=-4.5,
xmax=7.0600000000000005,
ymin=-1.46,
ymax=3.3,
xtick={-4.0,-3.0,...,7.0},
ytick={-2.0,-1.0,...,6.0},]
\clip(-5.,-1.) rectangle (7.06,3.3);
\fill[line width=2.pt,color=ffffqq,fill=ffffqq,fill opacity=\gelb] (-2.,-1.) -- (2.,-1.) -- (2.,3.) -- (-2.,3.) -- cycle;
\fill[line width=2.pt,color=ffffqq,fill=ffffqq,fill opacity=0.] (0.,2.) -- (0.,0.) -- (1.7320508075688776,1.) -- cycle;
\fill[line width=2.pt,color=ffffqq,fill=ffffqq,fill opacity=0.] (0.,2.) -- (-1.74,0.98) -- (0.013345911860126902,-0.016884202584923735) -- cycle;
\draw [rotate around={90.:(0.,1.)},line width=2.pt,color=black,fill=white,fill opacity=0.90000001192092896] (0.,1.) ellipse (1.50cm and 0.34937972405024542cm);
%\draw [line width=2.pt,color=ffffqq] (-2.,-1.)-- (2.,-1.);
%\draw [line width=2.pt,color=ffffqq] (2.,-1.)-- (2.,3.);
%\draw [line width=2.pt,color=ffffqq] (2.,3.)-- (-2.,3.);
%\draw [line width=2.pt,color=ffffqq] (-2.,3.)-- (-2.,-1.);
%\draw [line width=2.pt,color=ffffqq] (0.,2.)-- (0.,0.);
%\draw [line width=2.pt] (0.,0.)-- (1.7320508075688776,1.);
%\draw [line width=2.pt] (1.7320508075688776,1.)-- (0.,2.);
%\draw [line width=2.pt] (0.,2.)-- (-1.74,0.98);
%\draw [line width=2.pt] (-1.74,0.98)-- (0.013345911860126902,-0.016884202584923735);
\begin{scriptsize}
\draw [fill=xdxdff] (0.,2.) circle (2.5pt);
\draw [fill=qqqqff] (0.,0.) circle (2.0pt);
%\draw [fill=ududff] (2.,-1.) circle (2.5pt);
\draw[color=black] (0.4,1) node {$f$};
\draw[color=black] (-0.4,1) node {$e$};
%\draw [fill=ududff] (1.7320508075688776,1.) circle (2.5pt);
%\draw [fill=ududff] (-1.74,0.98) circle (2.5pt);
\draw [fill=ududff] (0.013345911860126902,-0.016884202584923735) circle (2.5pt);
\end{scriptsize}
%\end{axis}
\end{tikzpicture}
\caption{Randkantenpaar vom Typ 2}
\end{figure}

  %-------------------------------------------------------------
 \item[b)] Man nennt $\{e,f\}$  ein \emph{Randkantenpaar vom Typ 1}, falls $\{e,f\}$ ein Randkantenpaar ist und beide Kanten zu genau einem gemeinsamem Knoten 
 $V_{e,f}\in X_0$ inzident sind, das heißt es gilt $V_{e,f}<e$ und $V_{e,f}<f$. Die übrigen beiden Knoten, die zu $e$ bzw. $f$ inzident sind, werden mit $V_{e}$,$V_{f}\in X_0$ bezeichnet, wobei $V_{e}<e$ und $V_f<f$. Falls keine Kante $g\in X_1$ mit $V_e,V_f<g$ existiert, dann ist das Randkantenpaar $\{e,f\}$ vom Typ 1 \emph{mendable}.\\
 %-------
 \begin{figure}[h]
 \definecolor{ffffff}{rgb}{1.,1.,1.}
\definecolor{ududff}{rgb}{0.30196078431372547,0.30196078431372547,1.}
\definecolor{ffffqq}{rgb}{1.,1.,0.}
\begin{tikzpicture}[line cap=round,line join=round,>=triangle 45,x=1.5cm,y=1.5cm]
x=1.5cm,y=1.5cm,
axis lines=middle,
ymajorgrids=true,
xmajorgrids=true,
xmin=-3.3,
xmax=3.0600000000000005,
ymin=-2.46,
ymax=2.3,
xtick={-4.0,-3.0,...,7.0},
ytick={-2.0,-1.0,...,6.0},]
\clip(-5.3,-2.) rectangle (3.06,2.3);
\fill[line width=2.pt,color=ffffqq,fill=ffffqq,fill opacity=0.550000011920929] (-2.,2.) -- (-2.,-2.) -- (2.,-2.) -- (2.,2.) -- cycle;
\fill[line width=2.pt,color=ffffff,fill=ffffff,fill opacity=1.0] (-1.,1.) -- (0.,-0.8) -- (1.05884572681199,0.9660254037844385) -- cycle;
\draw [line width=2.pt] (-1.,1.)-- (0.,-0.8);
\draw [line width=2.pt] (0.,-0.8)-- (1.05884572681199,0.9660254037844385);
\begin{scriptsize}
\draw [fill=ududff] (-1.,1.) circle (2.5pt);
\draw[color=black] (-1.34,1.35) node {$V_{e}$};
\draw [fill=ududff] (0.,-0.8) circle (2.5pt);
\draw[color=black] (0.08,-1.11) node {$V_{e,f}$};
%\draw[color=ffffff] (0.48,0.57) node {$Vieleck2$};
\draw[color=black] (-0.72,0.13) node {e};
\draw[color=black] (0.86,0.09) node {f};
\draw [fill=ududff] (1.05884572681199,0.9660254037844385) circle (2.5pt);
\draw[color=black] (1.2,1.33) node {$V_{f}$};
\end{scriptsize}

\end{tikzpicture}
\caption{mendable Randkantenpaar vom Typ 1}
\end{figure}

 %-----------------------------------------------
 \item[c)] Man nennt $\{e,f\}$  ein \emph{Randkantenpaar vom Typ 0}, falls $\{e,f\}$ ein Randkantenpaar ist und $X_{0}(e) \cap X_{0}(f)= \emptyset$, wobei $X_{0}(e)=\{V_{e},W_{e}\}$ und $X_{0}(f)=\{V_{f},W_{f}\}$. Das Randkantenpaar ist \emph{mendable} bezüglich $V_e$ und $V_f$, falls keine Kante $g \in X_1$ mit $W_e,W_f<g$ oder $V_e,V_f<g$ existiert.\\
 Das Randkantenpaar $\{e,f\}$ heißt mendable, falls es Knoten $V \in \{V_e,W_e\}$ und $W \in \{V_f,W_f\}$ gibt, sodass $\{e,f\}$ mendable bezüglich $V$ und $W$ ist.\\
 %---------------bidl----------------------

%--------------------------------------------------
%----------------------bild ----------------------------
\begin{figure}[h]
\definecolor{ffffff}{rgb}{1.,1.,1.}
\definecolor{ududff}{rgb}{0.30196078431372547,0.30196078431372547,1.}
\definecolor{ffffqq}{rgb}{1.,1.,0.}
\begin{tikzpicture}[line cap=round,line join=round,>=triangle 45,x=1.5cm,y=1.5cm]
x=1.0cm,y=1.0cm,
axis lines=middle,
ymajorgrids=true,
xmajorgrids=true,
xmin=-2.3,
xmax=2.7,
ymin=-2.34,
ymax=2.3,
xtick={-4.0,-3.0,...,18.0},
ytick={-5.0,-4.0,...,6.0},]
\clip(-5.3,-1.84) rectangle (2.7,2.3);
\fill[line width=2.pt,color=ffffqq,fill=ffffqq,fill opacity=\gelb] (-2.,2.) -- (-2.,-2.) -- (2.,-2.) -- (2.,2.) -- cycle;
\fill[line width=2.pt,color=ffffff,fill=ffffff,fill opacity=1.0] (-1.,1.) -- (-1.,-1.) -- (1.,-1.) -- (1.,1.) -- cycle;
\fill[line width=2.pt,color=black,fill=ffffqq,fill opacity=\gelb] (1.,1.) -- (-1.,-1.013) -- (-0.,-1.013) -- (1.,1) -- cycle;

\draw [line width=2.pt] (-1.,1.)-- (-1.,-1.);
\draw [line width=2.pt] (1.,1.)-- (-1.,-1.);
%\draw [line width=2.pt,color=ffffff] (-1.,-1.)-- (1.,-1.);
\draw [line width=2.pt] (1.,-1.)-- (1.,1.);
%\draw [line width=2.pt,color=ffffff] (1.,1.)-- (-1.,1.);
\begin{scriptsize}
%\draw[color=ffffqq] (0.48,0.17) node {$Vieleck1$};
\draw[color=black] (-1,1.23) node {$V_{e}$};
\draw[color=black] (1,1.23) node {$V_{f}$};
\draw[color=black] (-1,-1.23) node {$W_{e}$};
\draw[color=black] (1,-1.23) node {$W_{f}$};
\draw [fill=ududff] (-1.,1.) circle (2.5pt);
\draw [fill=ududff] (-1.,-1.) circle (2.5pt);
%\draw[color=ffffff] (0.48,0.17) node {$Vieleck2$};
\draw[color=black] (-1.26,0.17) node {e};
\draw[color=black] (1.18,0.17) node {f};
\draw [fill=ududff] (1.,-1.) circle (2.5pt);
\draw [fill=ududff] (1.,1.) circle (2.5pt);
\end{scriptsize}
\end{tikzpicture}
\caption{mendable Randkantenpaar vom Typ 0 bzgl. $V_e$ und $V_f$}
\end{figure}
 %-------------------------------------------------
 \end{enumerate}
 \end{enumerate}
 \end{definition}
 
 \begin{bemerkung}
 Es ist leicht einzusehen, dass die Kanten $e,f \in X_1$ eines Randkantenpaares vom Typ $i$, genau $i$ Knoten gemeinsam haben, wobei $i=0,1,2$. Das heißt es existieren $V_1,\ldots,V_i \in X_0$ mit 
\[
V_j <e \text{ und } V_j <f \text{ für j=1,\ldots,i}.
\]

 \end{bemerkung}
 \subsection{Mending-Map}
\begin{bemerkung}
Seien $A,B$ Mengen und $f:A \to B$ eine Funktion. Für $M \subseteq B$ definiert man das \emph{Urbild von M in A} durch 
\[
f^{-1}(M):=\{y\in A \mid f(y)\in M\}.
\]
Für $\{y\} \subset B$ schreibt man auch
\[
f^{-1}(y):=f^{-1}(\{y\}).
\]
\end{bemerkung}
  \begin{definition}
  Seien $(X,<)$ und $(Y,\prec)$ simpliziale Flächen.
  \begin{enumerate}
  \item Man nennt eine Überdeckung $\alpha:X \to Y$ eine \emph{Mending Map}, falls sie eine Bijektion $\beta : X_{2}\to Y_{2}$ induziert. Dadurch entsteht die simpliziale Fläche $(X(\alpha),<_{\alpha})$ mit den Knoten
  \[
X(\alpha)_0:=\{\alpha^{-1}(V)\mid V \in Y_0 \},
  \] 
  den Kanten 
  \[
X(\alpha)_1:=\{\alpha^{-1}(e)\mid e \in Y_1 \} 
  \]
 und den Flächen 
  \[
X(\alpha)_2:=X_2  .
  \]
  Für $A,B \in Y$ ist $\alpha^{-1}(A)<_{\alpha}\alpha^{-1}(B)$ in $X(\alpha)$ genau dann, wenn $A \prec B  $ in $Y$ gilt. Außerdem ist $(X(\alpha),<_{\alpha})$ isomorph zu $(Y,\prec)$. Man nennt $X(\alpha)$ ein \emph{Mending von $X$}.

  \item Die Menge aller Mendings von $(X,<)$ wird definiert durch 
\[
\mathcal{M}(X):=\{X(\alpha )\mid
 \text{$(Y,\prec )$ simpliziale Fläche so, dass  $\alpha : X \to Y$ Mending Map}
 \}  .
\]

  \end{enumerate}
  \end{definition}
  
  \begin{bsp}
  Sei $J$ der oben definierte Janus-Kopf und $X:= 2\Delta$. Dann ist die Abbildung 
  \[
  \alpha: X \to J, x \mapsto 
  \begin{cases}
e_i & \text{für } x =e_i^j ,i=1,2,3,\,j=1,2\\
V_i &\text{für } x =V_i^j,i=1,2,3,\,j=1,2\\
F_i &\text{für } x=F_i , i=1,2 
\end{cases}
  \]
  nach Konstruktion eine Mending Map. Man erhält dadurch die simpliziale Fläche $(X(\alpha),<_\alpha)$ definiert durch die Knoten
  \[
  X(\alpha)_0=\{\{V_1^1,V_1^2\},\{V_2^1,V_2^2\},\{V_3^1,V_3^2\}\},
  \]
  die Kanten
  \[
  X(\alpha)_1=\{\{e_1^1,e_1^2\},\{e_2^1,e_2^2\},\{e_3^1,e_3^2\}\}
  \]
  und die Flächen 
  \[
X(\alpha)_2=X_2=\{F_1,F_2\} .
  \]
  Es gilt $x<_{\alpha}y$ in $X(\alpha)$ genau dann, wenn
  \begin{align*}
 (x,y) \in \{&(\{V_i^1,V_i^2\}, \{e_j^1,e_j^2\})\mid i=1,2,3 \, j\in\{1,2,3\} \setminus \{i\}\}\, \cup\\
  \{&(\{V_i^1,V_i^2\}, F_j)\mid i=1,2,3 \, j=1,2 \} \,\cup\\
  \{&(\{e_i^1,e_i^2\}, F_j)\mid i=1,2,3 \, j=1,2 \}. 
\end{align*}
 \end{bsp}
   %----bild----------------------------
   \begin{figure}[h]
\definecolor{sqsqsq}{rgb}{0.12549019607843137,0.12549019607843137,0.12549019607843137}
\definecolor{ttqqqq}{rgb}{0.2,0.,0.}
\definecolor{ffffqq}{rgb}{1.,1.,0.}
\definecolor{qqqqff}{rgb}{0.,0.,1.}
\begin{tikzpicture}[line cap=round,line join=round,>=triangle 45,x=1.0cm,y=1.0cm]

x=1.0cm,y=1.0cm,
axis lines=middle,
ymajorgrids=true,
xmajorgrids=true,
xmin=-5.056290110700678,
xmax=5.380866801866215,
ymin=-0.9227448489396118,
ymax=4.359364127681193,
xtick={-5.0,-4.5,...,5.0},
ytick={-0.5,0.0,...,4.0},]
\clip(-7.056290110700678,-0.5227448489396118) rectangle (5.380866801866215,4.359364127681193);
\fill[line width=2.pt,color=ffffqq,fill=ffffqq,fill opacity=0.5] (-2.,0.) -- (2.,0.) -- (0.,3.4641016151377553) -- cycle;
\fill[line width=2.pt,color=ffffqq,fill=ffffqq,fill opacity=0.5] (0.,3.4641016151377553) -- (2.,0.) -- (4.,3.464101615137754) -- cycle;
\draw [line width=2.pt] (-2.,0.)-- (2.,0.);
\draw [line width=2.pt] (2.,0.)-- (0.,3.4641016151377553);
\draw [line width=2.pt] (0.,3.4641016151377553)-- (-2.,0.);
\draw [line width=2.pt,color=ttqqqq] (0.,3.4641016151377553)-- (2.,0.);
\draw [line width=2.pt] (2.,0.)-- (4.,3.464101615137754);
\draw [line width=2.pt,color=sqsqsq] (4.,3.464101615137754)-- (0.,3.4641016151377553);
\begin{scriptsize}
\draw [fill=qqqqff] (-2.,0.) circle (2.5pt);
\draw[color=black] (-2.347666183295526,-0.3234108260899206) node {$\{V_1^1,V_1^2\}$};
\draw [fill=qqqqff] (2.,0.) circle (2.5pt);
\draw[color=black] (2.390727102068595,-0.3234108260899206) node {$\{V_2^1,V_2^2\}$};
\draw[color=sqsqsq] (0.04884098783747624,1.2554444197699985) node {$F_1$};
\draw[color=black] (0.057916776457099625,-0.25407545041275472) node {$\{e_3^1,e_3^2\}$};
\draw[color=black] (1.474238776670776,1.9089012003828816) node {$\{e_1^1,e_1^2\}$};
\draw[color=black] (-1.4673146871920574,1.9089012003828816) node {$\{e_2^1,e_2^2\}$};
\draw [fill=qqqqff] (0.,3.4641016151377553) circle (2.5pt);
\draw[color=black] (0.08514414231596978,3.75145261535057) node {$\{V_3^1,V_3^1\}$};
\draw[color=sqsqsq] (2.0455144841546207,2.417145363081791) node {$F_2$};
\draw[color=black] (3.6162912160860375,1.9089012003828816) node {$\{e_3^1,e_3^2\}$};
\draw[color=sqsqsq] (2.154590272774244,3.6903620787860503) node {$\{e_2^1,e_2^2\}$};
\draw [fill=qqqqff] (4.,3.464101615137754) circle (2.5pt);
\draw[color=black] (4.087566923569883,3.75145261535057) node {$\{V_1^1,V_1^2\}$};
\end{scriptsize}
\end{tikzpicture}
\caption{Mending einer simplizialen Fläche}
\end{figure}
 %------------------------------------

  \begin{bemerkung}
  Sei $(X,<)$ eine simpliziale Fläche.
  \begin{enumerate}
  \item Für ein $Y \in \mathcal{M}(X)$ gilt $Y_2=X_2$.
  \item $X$ bildet ein Mending von sich selbst mit der Identität als Mending Map.
  \item Sei $Y\in \mathcal{M}(X)$, das heißt $Y=X(\alpha)$ für eine Mending Map $\alpha:X \mapsto Z$, wobei $(Z,\prec)$ eine weitere simpliziale Fläche ist. Dann bildet die Relation 
  \[
A\sim_\alpha B \Leftrightarrow \alpha(A)=\alpha(B),\text{ für }A,B \in X
  \]
  eine Äquivalenzrelation auf X.
  \item Falls $Y\in \mathcal{M}(X)$ und $Z \in \mathcal{M}(Y)$ gilt, so ist $Z$ auch ein Mending von $X$.
  \end{enumerate}
  \end{bemerkung}
  
  \begin{definition}
  Für eine simpliziale Fläche $(X,<)$ und $i=0,1,2$ definiert man $I^{i}(X)$ als die Menge der Kanten mit $i$ inneren Knoten, und $BM^{i}(X)$ als die Menge der mendable Randkantenpaare vom Typ $i$. Also ist
  \begin{align*}
  &I^i(X):=\{e \in X_1 \mid \vert X_{0}^{0}(e)\vert=i\}\text{ und}\\
  &BM^{i}(X):=\{\{e,f\} \mid e,f \in X_1 \text{ und }\{e,f\}\text{ ist ein mendable Randkantenpaar vom Typ i}\}.\\
  \end{align*}
  \end{definition}
  
  
  \newpage
  \subsection{Mender- und Cutter-Operatoren}
  Im Folgendem wird thematisiert, wie aus einer simplizialen Fläche eine weitere konstruiert werden kann. Zu diesem Zweck werden die \emph{Mender}-Operationen, die aus zwei Randkanten eine innere Kante konstruieren  und \emph{Cutter}-Operationen, die aus einer inneren Kante zwei Randkanten hervorbringen, eingeführt.\\
  Sei dazu $(X,<)$ eine simpliziale Fläche.
\begin{enumerate}
\item Der Operator \emph{Cratermender} ist definiert durch
\[
 C_{e,f}^{m}:\{Y \in \mathcal{M}(X)|\{e,f\} \in BM^{2}(Y) \}  \to 
  \{Z \in \mathcal{M}(X)|\{e,f\} \in I^{2}(Z)\}, 
  Y \mapsto Z,
  \]
  wobei $Z_{2}:=Y_{2},Z_{1}:=(Y_{1}-\{e,f\}) \cup \{\{e,f\}\},Z_0:=Y_{0}$.
  Er setzt die beiden Randkanten $e$ und $f$ zu einer inneren Kante $\{e,f\}$ vom Typ 2 zusammen, um somit die simpliziale Fläche $Z=C^{m}_{e,f}(Y)$ zu erhalten. Den inversen Operator $C^{c}_{\{e,f\}}$ nennt man \emph{Cratercutter}. \\
  %---------------------------bild------------------
  \begin{figure}[h]
  \definecolor{ududff}{rgb}{0.30196078431372547,0.30196078431372547,1.}
\definecolor{ffffqq}{rgb}{1.,1.,0.}
\begin{tikzpicture}[line cap=round,line join=round,>=triangle 45,x=1.3cm,y=1.3cm]
%\begin{axis}[
x=1.5cm,y=1.5cm,
axis lines=middle,
ymajorgrids=true,
xmajorgrids=true,
xmin=-5.54,
xmax=5.82,
ymin=-0.6600000000000006,
ymax=4.100000000000001,
xtick={-5.0,-4.0,...,5.0},
ytick={-2.0,-1.0,...,6.0},]
\clip(-6.14,-0.4) rectangle (5.82,4.1);
\fill[line width=2.pt,color=ffffqq,fill=ffffqq,fill opacity=\gelb] (-5.,0.) -- (-1.,0.) -- (-1.,4.) -- (-5.,4.) -- cycle;
\fill[line width=2.pt,color=ffffqq,fill=ffffqq,fill opacity=\gelb] (1.,0.) -- (5.,0.) -- (5.,4.) -- (1.,4.) -- cycle;
\draw [rotate around={90.:(-3.,2.)},line width=2.pt,color=black,fill=white,fill opacity=0.90000001192092896] (-3.,2.) ellipse (1.2586450581159654cm and 0.29400452165887971cm);
\draw [line width=2.pt] (3.,3.)-- (3.,1.);
\draw [line width=1.pt] (.5,1.5)-- (-.7,1.5);
\draw [line width=1.pt] (-0.7,1.5)-- (-.6,1.6);
\draw [line width=1.pt] (-0.7,1.5)-- (-.6,1.4);
\draw [line width=1.pt] (.5,2.5)-- (-.7,2.5);
\draw [line width=1.pt] (.4,2.6)-- (.5,2.5);
\draw [line width=1.pt] (.4,2.4)-- (.5,2.5);
\begin{scriptsize}
\draw [fill=ududff] (-3.,3.) circle (2.5pt);
%\draw[color=ududff] (-2.86,3.37) node {$I$};
\draw [fill=ududff] (-3.,1.) circle (2.5pt);
%\draw[color=ududff] (-2.86,1.37) node {$J$};
\draw[color=black] (-2.52,1.97) node {$f$};
\draw[color=black] (-3.52,1.97) node {$e$};
\draw [fill=ududff] (3.,3.) circle (2.5pt);
%\draw[color=ududff] (3.14,3.37) node {$L$};
\draw [fill=ududff] (3.,1.) circle (2.5pt);
%\draw[color=ududff] (3.14,1.37) node {$M$};
\draw[color=black] (3.44,2.17) node {$\{e,f\}$};
\draw[color=black] (0,1.8) node {$C^c_{\{e,f\}}$};
\draw[color=black] (-0.2,2.8) node {$C^m_{e,f}$};
\end{scriptsize}
%\end{axis}
\end{tikzpicture}
\caption{Cratercutter und Cratermender}
\end{figure}

  %---------------------------------------------------
  \item Den Operator \emph{Ripmender}, welcher ein mendable Randkantenpaar $\{e,f\}$ vom Typ 1 zu einer inneren Kante $\{e,f\}$ zusammensetzt definiert man durch
  \[
R^m_{e,f} :\{Y \in \mathcal{M}(X)|\{e,f\} \in BM^{1}(Y) \}  \to 
  \{Z \in \mathcal{M}(X)|\{e,f\} \in I^{1}(Z)\}, 
  Y \mapsto Z,
  \]
  wobei $Z_{1}:=(Y_1-\{e,f\}) \cup\{\{e,f\}\},Z_0 :=(Y_0 - \{\sideset{^Y}{}{\mathop{V}}_{e},\sideset{^Y}{}{\mathop{V}}_{f}\}) \cup \{\{\sideset{^Y}{}{\mathop{V}}_{e}, \sideset{^Y}{}{\mathop{V}}_{f}\}\},$\\
  $Z_{2}:=X_{2}$ und $\sideset{^Y}{}{\mathop{V}}_{e}, \sideset{^Y}{}{\mathop{V}}_{f}\in Y_0$ definiert sind wie in Definition 6.
  Die zum Ripmender inverse Operation $R^c_{\{e,f\}}$ nennt man \emph{Ripcutter}. \\
  %-----------------------------bild----------------------
  \newpage
  \begin{figure}[h]
  \definecolor{sqsqsq}{rgb}{0.12549019607843137,0.12549019607843137,0.12549019607843137}
\definecolor{ffffff}{rgb}{1.,1.,1.}
\definecolor{ududff}{rgb}{0.30196078431372547,0.30196078431372547,1.}
\definecolor{ffffqq}{rgb}{1.,1.,0.}
\begin{tikzpicture}[line cap=round,line join=round,>=triangle 45,x=1.3cm,y=1.3cm]
x=1.0cm,y=1.0cm,
axis lines=middle,
ymajorgrids=true,
xmajorgrids=true,
xmin=-5.36,
xmax=5.24,
ymin=-2.38,
ymax=2.26,
xtick={-5.0,-4.0,...,6.0},
ytick={-5.0,-4.0,...,6.0},]
\clip(-5.86,-2.4) rectangle (5.24,2.26);
\fill[line width=2.pt,color=ffffqq,fill=ffffqq,fill opacity=0.6000000238418579] (-5.,2.) -- (-5.,-2.) -- (-1.,-2.) -- (-1.,2.) -- cycle;
\fill[line width=2.pt,color=ffffff,fill=ffffff,fill opacity=1.0] (-3.,-1.) -- (-1.84,1.) -- (-4.152050807568877,1.0045894683899497) -- cycle;
\fill[line width=2.pt,color=ffffqq,fill=ffffqq,fill opacity=0.6000000238418579] (1.,2.) -- (1.,-2.) -- (5.,-2.) -- (5.,2.) -- cycle;
\draw [line width=2.pt] (-3.,-1.)-- (-1.84,1.);
\draw [line width=2.pt,color=ffffff] (-1.84,1.)-- (-4.152050807568877,1.0045894683899497);
\draw [line width=2.pt,color=sqsqsq] (-4.152050807568877,1.0045894683899497)-- (-3.,-1.);
\draw [line width=2.pt] (3.,1.)-- (3.,-1.);
\draw [line width=1.pt,color=black] (-0.7,0.5)-- (.5,0.5);
\draw [line width=1.pt,color=black] (-0.7,-0.5)-- (-0.6,-0.4);
\draw [line width=1.pt,color=black] (-0.7,-0.5)-- (-0.6,-0.6);
\draw [line width=1.pt,color=black] (-0.7,-0.5)-- (.5,-0.5);
\draw [line width=1.pt,color=black] (0.4,0.6)-- (.5,0.5);
\draw [line width=1.pt,color=black] (0.4,0.4)-- (.5,0.5);                                                                                                                                                                                                                                                                                                                                                                                                                                                                                                                                                                                                                                                                                                                                                                                                                                                                                                                                                                                                                                               

\begin{scriptsize}
\draw [fill=ududff] (-3.,-1.) circle (2.5pt);
\draw [fill=ududff] (-1.84,1.) circle (2.5pt);
%\draw[color=ffffff] (-2.52,0.51) node {$Vieleck2$};
\draw[color=black] (-2.1,0.01) node {$f$};
\draw[color=sqsqsq] (-3.8,0.03) node {e};
\draw [fill=ududff] (-4.152050807568877,1.0045894683899497) circle (2.5pt);
\draw [fill=ududff] (3.,1.) circle (2.5pt);
%\draw[color=ududff] (3.14,1.37) node {$L$};
\draw [fill=ududff] (3.,-1.) circle (2.5pt);
%\draw[color=ududff] (3.14,-0.63) node {$M$};
\draw[color=black] (2.6,0.17) node {$\{e,f\}$};
\draw[color=black] (-0.2,0.75) node {$R^m_{e,f}$};
\draw[color=black] (-0.2,-0.25) node {$R^c_{\{e,f\}}$};
\end{scriptsize}

\end{tikzpicture}
\caption{Ripcutter und Ripmender}
\end{figure}

  %--------------------------------------------------------
  \item Seien nun $V,V'$ bzw. $W,W'$ Knoten, die inzident zu $e$ bzw. $f$ sind. Dann ist 
  \begin{align*}
  S^m_{(V,e),(W,f)}:&\{Y \in \mathcal{M}(X)|\{e,f\} \in BM^{0}(Y) \text{ mendable bzgl. V und W} \}  
  \to\\ & \{Z \in \mathcal{M}(X)|\{e,f\} \in I^{0}(Z)\}, 
  Y \mapsto Z, 
   \end{align*}
  der Operator \emph{Splitmender}, wobei $Z_2 := X_2,Z_1 := (Y_1 - \{e,f\}) \cup \{\{e,f\}\},Z_0:=(Y_0 -(Y_0 (e) \cup Y_0(f))) \cup \{\{V,W\},\{V',W'\}\}$. Dieser setzt zwei disjunkte Kanten, also Kanten, die keinen Knoten gemeinsam haben, zu einer inneren Kante zusammen, um somit die simpliziale Fläche $Z=S^m_{(V,e),(W,f)}(X)=S^m_{(V',e),(W',f)}(X)$ zu erhalten. Dieser Operator hat ein eindeutiges Linksinverses, nämlich den Operator \emph{Splitcutter} $S^c_{\{e,f\}}$, wobei dieser ebenfalls ein Linksinverses des Operators  $S^C_{(V',e),(W,f)}=S^C_{(V,e),(W',f)}$ ist, falls $\{e,f\}$ mendable bezüglich $V',W$ bzw. $V,W'$ ist.\\
  %-------------------bild------------------------------
\begin{comment}  
\definecolor{ffffff}{rgb}{1.,1.,1.}
\definecolor{ududff}{rgb}{0.30196078431372547,0.30196078431372547,1.}
\definecolor{ffffqq}{rgb}{1.,1.,0.}
\begin{tikzpicture}[line cap=round,line join=round,>=triangle 45,x=1.3cm,y=1.3cm]

x=1.0cm,y=1.0cm,
axis lines=middle,
ymajorgrids=true,
xmajorgrids=true,
xmin=-5.694797977306429,
xmax=5.862933812537042,
ymin=-2.902772491173462,
ymax=2.50122742333861,
xtick={-8.0,-7.0,...,11.0},
ytick={-3.0,-2.0,...,6.0},]
\clip(-5.694797977306429,-2.902772491173462) rectangle (5.862933812537042,2.50122742333861);
\fill[line width=2.pt,color=ffffqq,fill=ffffqq,fill opacity=0.5] (-5.,2.) -- (-5.,-2.) -- (-1.,-2.) -- (-1.,2.) -- cycle;
\fill[line width=2.pt,color=ffffqq,fill=ffffqq,fill opacity=0.5] (1.,2.) -- (1.,-2.) -- (5.,-2.) -- (5.,2.) -- cycle;
\fill[line width=2.pt,color=ffffff,fill=ffffff,fill opacity=1.0] (-4.,1.) -- (-4.,-1.) -- (-2.,-1.) -- (-2.,1.) -- cycle;
\draw [line width=2.pt] (3.,1.)-- (3.,-1.);
\draw [line width=2.pt] (-4.,1.)-- (-4.,-1.);
\draw [line width=2.pt] (-2.,-1.)-- (-2.,1.);
\begin{scriptsize}
\draw [fill=ududff] (3.,1.) circle (2.5pt);
\draw[color=black] (3.2197483165463314,1.2707326201342032) node {$\{V,V'\}$};
\draw [fill=ududff] (3.,-1.) circle (2.5pt);
\draw[color=black] (3.2108101722898774,-1.2587622044421855) node {$\{W,W'\}$};
\draw[color=black] (2.6711606905446376,0.06408314551295772) node {$\{e,f\}$};
\draw [fill=ududff] (-4.,1.) circle (2.5pt);
\draw[color=black] (-3.823509357539171,1.3707326201342032) node {$V$};
\draw [fill=ududff] (-4.,-1.) circle (2.5pt);
\draw[color=black] (-3.912890800103708,-1.167700348698639) node {$W$};
%\draw[color=ffffff] (-2.5811073058921092,0.16408314551295772) node {$Vieleck3$};
\draw[color=black] (-4.243602137592494,0.16408314551295772) node {$e$};
\draw[color=black] (-1.6694165917338337,0.16408314551295772) node {$f$};
\draw [fill=ududff] (-2.,-1.) circle (2.5pt);
\draw[color=black] (-1.9107464866580832,-1.2213292142373608) node {$W'$};
\draw [fill=ududff] (-2.,1.) circle (2.5pt);
\draw[color=black] (-1.8213650440935463,1.3707326201342032) node {$V'$};
\end{scriptsize}
\end{tikzpicture}
\end{comment}
  %------------------------------------------------------
  
\end{enumerate}
\begin{bemerkung}
Sei $(X,<)$ eine simpliziale Fläche, $(Y,\prec)\in \mathcal{M}(X)$ und $e,f \in Y_1$ so, dass $\{e,f\}$ ein Randkantenpaar ist.
 Falls $\{e,f\}$ ein Randkantenpaar vom Typ 1 ist, ist klar, wie die simpliziale Fläche $R^m_{e,f}(Y)$ aus $(Y,\prec)$ hervorgeht. Gleiches gilt auch für $C^m_{e,f}(Y)$, falls $\{e,f\}$ ein Randkantenpaar vom Typ 2 ist.\\
Der Fall, dass $\{e,f\}$ ein Randkantenpaar vom Typ 0 ist, wird nun näher erläutert.\\
Seien $V_e,W_e \in Y_0$ die zu $e$ und $V_f,W_f \in Y_0$ die zu $f$ zugehörigen Knoten, das heißt
\[
V_e,W_e \prec e \text{ und } V_f,W_f \prec f.
\]  
Dann werden zwei Fälle unterschieden:
\begin{enumerate}
\item Angenommen für $V_1\in \{V_e,W_e\}$ und $V_2 \in \{V_f,W_f\}$ existiert keine Kante $g \in Y_1$ mit $V_1,V_2 <g$. Dann existieren zwei Möglichkeiten, um mit dem Splitmender eine simpliziale Fläche zu konstruieren, nämlich $Z=S^m_{(V_e,e),(V_f,f)}(Y)$ und $T=S^m_{(V_e,e),(W_f,f)}(Y)$, 
wobei für $\{V_e,V_f\},\{W_e,W_f\} \in Z_0$ und $\{e,f\} \in Z_1$
\[
\{V_e,V_f\},\{W_e,W_f\}\prec_1 \{e,f\}
\] gilt mit $\prec_1$ als Inzidenz auf $Z$ und für $\{V_e, W_f\},\{V_f,W_e\}\in T_0$ und $\{e,f\}\in T_1$
\[
\{V_e,W_f\},\{V_f,W_e\}\prec_2 \{e,f\} 
\] gilt mit $\prec_2$ als Inzidenz auf $T$.
 \\
 %----------------------------------
\begin{figure}[h] 
 \definecolor{ffffff}{rgb}{1.,1.,1.}
\definecolor{qqqqff}{rgb}{0.,0.,1.}
\definecolor{ffffqq}{rgb}{1.,1.,0.}
\begin{tikzpicture}[line cap=round,line join=round,>=triangle 45,x=1.0cm,y=1.0cm]
%\begin{axis}[
x=1.0cm,y=1.0cm,
axis lines=middle,
ymajorgrids=true,
xmajorgrids=true,
xmin=-6.81112373905725,
xmax=14.391643353227584,
ymin=-8.866642747190788,
ymax=6.9246706856003435,
xtick={-16.0,-14.0,...,14.0},
ytick={-8.0,-6.0,...,6.0},]
\clip(-6.81112373905725,-5.4) rectangle (14.391643353227584,5.2);
\fill[line width=2.pt,color=ffffqq,fill=ffffqq,fill opacity=\gelb] (-5.,-2.) -- (-1.,-2.) -- (-1.,2.) -- (-5.,2.) -- cycle;
\fill[line width=2.pt,color=ffffqq,fill=ffffqq,fill opacity=\gelb] (1.,5.) -- (1.,1.) -- (5.,1.) -- (5.,5.) -- cycle;
\fill[line width=2.pt,color=ffffqq,fill=ffffqq,fill opacity=\gelb] (1.,-1.) -- (1.,-5.) -- (5.,-5.) -- (5.,-1.) -- cycle;
\fill[line width=2.pt,color=ffffff,fill=ffffff,fill opacity=1.0] (-2.,-1.) -- (-2.,1.) -- (-4.,1.) -- (-4.,-1.) -- cycle;
\draw [line width=2.pt] (-2.,-1.)-- (-2.,1.);
\draw [line width=2.pt] (-4.,1.)-- (-4.,-1.);
\draw [line width=2.pt] (3.,-2.)-- (3.,-4.);
\draw [line width=2.pt] (3.,4.)-- (3.,2.);
\draw [line width=1.pt] (-0.6,0.)-- (0.6,0.);
\draw [line width=1.pt] (-0.6,0.)-- (-0.5,0.1);
\draw [line width=1.pt] (-0.6,0.)-- (-0.5,-0.1);
%---
\draw [line width=1.pt] (-0.6,1.5)-- (0.6,1.5);
\draw [line width=1.pt] (0.6,1.5)-- (0.4,1.4);
\draw [line width=1.pt] (0.6,1.5)-- (0.4,1.6);
%----
\draw [line width=1.pt] (-0.6,-1.5)-- (0.6,-1.5);
\draw [line width=1.pt] (0.6,-1.5)-- (0.4,-1.4);
\draw [line width=1.pt] (0.6,-1.5)-- (0.4,-1.6);
\begin{scriptsize}
\draw[color=black] (-2.349319512806975,0.) node {$f$};
\draw[color=black] (-4.3,0.) node {$e$};
%\draw[color=ffffqq] (3.6470383371016757,3.24817073690069) node {$Vieleck2$};
%\draw[color=ffffqq] (3.6470383371016757,-2.7481871130079707) node {$Vieleck3$};
\draw [fill=qqqqff] (-2.,-1.) circle (2.5pt);
\draw[color=black] (-1.9,-1.4) node {$W_f$};
\draw [fill=qqqqff] (-4.,-1.) circle (2.5pt);
%\draw[color=qqqqff] (-3.8144929240968724,-0.49616131417349635) node {$N$};
\draw [fill=qqqqff] (-2.,1.) circle (2.5pt);
\draw[color=black] (-1.9,1.4) node {$V_f$};
%\draw[color=ffffff] (-2.349319512806975,0.2364253914714531) node {$Vieleck4$};
\draw [fill=qqqqff] (-4.,1.) circle (2.5pt);
\draw[color=black] (-4.,1.4) node {$V_e$};
\draw [fill=qqqqff] (-4.,-1.) circle (2.5pt);
\draw[color=black] (-4,-1.4) node {$W_e$};
\draw [fill=qqqqff] (3.,-2.) circle (2.5pt);
\draw[color=black] (3.1,-1.6000764293165751) node {$\{V_e,W_f\}$};
\draw [fill=qqqqff] (3.,-4.) circle (2.5pt);
\draw[color=black] (3.1,-4.4) node {$\{V_f,W_e\}$};
\draw[color=black] (3.4431232219585983,-3.) node {$\{e,f\}$};
\draw [fill=qqqqff] (3.,4.) circle (2.5pt);
\draw[color=black] (3.1,4.4) node {$\{V_e,V_f\}$};
\draw [fill=qqqqff] (3.,2.) circle (2.5pt);
\draw[color=black] (3.1,1.6884511903059274) node {$\{W_e,W_f\}$};
\draw[color=black] (3.4431232219585983,3.1) node {$\{e,f\}$};
\draw[color=black] (0.,0.3) node {$S^c_{\{e,f\}}$};
\draw[color=black] (0.,1.9) node {$S^m_{(V_e,e),(V_f,f)}$};
\draw[color=black] (0.,-1.1) node {$S^m_{(V_e,e),(W_f,f)}$};
\end{scriptsize}
%\end{axis}
\end{tikzpicture}
\caption{Splitmender und Splitcutter}
\end{figure}
%-----------------bild-----------------
\begin{comment}
\begin{figure}[h]
\definecolor{qqqqff}{rgb}{0.,0.,1.}
\definecolor{ffffff}{rgb}{1.,1.,1.}
\definecolor{ududff}{rgb}{0.30196078431372547,0.30196078431372547,1.}
\definecolor{ffffqq}{rgb}{1.,1.,0.}
\begin{tikzpicture}[line cap=round,line join=round,>=triangle 45,x=1.5cm,y=1.5cm]
%\begin{axis}[
x=1.0cm,y=1.0cm,
axis lines=middle,
ymajorgrids=true,
xmajorgrids=true,
xmin=-5.598043973330023,
xmax=8.94915637797606,
ymin=-0.5015304882422273,
ymax=4.354530906940496,
xtick={-4.0,-3.0,...,8.0},
ytick={-2.0,-1.0,...,4.0},]
\clip(-5.598043973330023,-0.4)rectangle (8.94915637797606,4.354530906940496);
\fill[line width=2.pt,color=ffffqq,fill=ffffqq,fill opacity=0.550000011920929] (-2.,0.) -- (2.,0.) -- (2.,4.) -- (-2.,4.) -- cycle;
\fill[line width=2.pt,color=ffffff,fill=ffffff,fill opacity=1.0] (-1.,3.) -- (-1.,1.) -- (1.,1.) -- (1.,3.) -- cycle;
\draw [line width=2.pt] (-1.,3.)-- (-1.,1.);
\draw [line width=2.pt] (1.,1.)-- (1.,3.);
\begin{scriptsize}
%\draw[color=ffffqq] (0.2789481531401666,2.1104077183111) node {$Vieleck1$};
\draw [fill=ududff] (-1.,3.) circle (2.5pt);
\draw[color=black] (-0.8931791816032729,3.3471945303989567) node {$V_e$};
\draw [fill=ududff] (-1.,1.) circle (2.5pt);
\draw[color=black] (-0.928519704258854,0.6911596866935535) node {$W_e$};
%\draw[color=ffffff] (0.2789481531401666,2.1104077183111) node {$Vieleck2$};
\draw[color=black] (-1.2582331015201308,2.1104077183111) node {e};
\draw[color=black] (1.321362090622329,2.1104077183111) node {f};
\draw [fill=qqqqff] (1.,1.) circle (2.5pt);
\draw[color=black] (1.085890087109268,0.6440389898194454) node {$W_f$};
\draw [fill=qqqqff] (1.,3.) circle (2.5pt);
\draw[color=black] (1.109450435546322,3.3471945303989567) node {$V_f$};
\end{scriptsize}
%\end{axis}
\end{tikzpicture}
\caption{mendable Randkantenpaar vom Typ 0}
\end{figure}
\end{comment}

%-------------------------------------
\begin{comment}
\definecolor{xdxdff}{rgb}{0.49019607843137253,0.49019607843137253,1.}
\definecolor{ffffqq}{rgb}{1.,1.,0.}
\begin{tikzpicture}[line cap=round,line join=round,>=triangle 45,x=1.5cm,y=1.5cm]
%\begin{axis}[
x=1.5cm,y=1.5cm,
axis lines=middle,
ymajorgrids=true,
xmajorgrids=true,
xmin=-4.3,
xmax=7.0600000000000005,
ymin=-2.46,
ymax=6.3,
xtick={-4.0,-3.0,...,7.0},
ytick={-2.0,-1.0,...,6.0},]
\clip(-2.3,-0.46) rectangle (3.06,4.3);
\fill[line width=2.pt,color=ffffqq,fill=ffffqq,fill opacity=0.550000011920929] (-2.,0.) -- (2.,0.) -- (2.,4.) -- (-2.,4.) -- cycle;
\draw [line width=2.pt] (0.,1.)-- (0.,3.);
\begin{scriptsize}
\draw [fill=xdxdff] (0.,1.) circle (2.5pt);
\draw[color=black] (0.,0.77) node {$\{W_e,W_f\}$};
\draw [fill=xdxdff] (0.,3.) circle (2.5pt);
\draw[color=black] (0.,3.27) node {$\{V_e,V_f\}$};
\draw[color=black] (0.38,2.17) node {$\{e,f\}$};
\end{scriptsize}
%\end{axis}
\end{tikzpicture}

%------------------------------------------
\definecolor{xdxdff}{rgb}{0.49019607843137253,0.49019607843137253,1.}
\definecolor{ffffqq}{rgb}{1.,1.,0.}
\begin{tikzpicture}[line cap=round,line join=round,>=triangle 45,x=1.5cm,y=1.5cm]
%\begin{axis}[
x=1.5cm,y=1.5cm,
axis lines=middle,
ymajorgrids=true,
xmajorgrids=true,
xmin=-4.3,
xmax=7.0600000000000005,
ymin=-2.46,
ymax=6.3,
xtick={-4.0,-3.0,...,7.0},
ytick={-2.0,-1.0,...,6.0},]
\clip(-2.3,-0.46) rectangle (3.06,4.3);
\fill[line width=2.pt,color=ffffqq,fill=ffffqq,fill opacity=0.550000011920929] (-2.,0.) -- (2.,0.) -- (2.,4.) -- (-2.,4.) -- cycle;
\draw [line width=2.pt] (0.,1.)-- (0.,3.);
\begin{scriptsize}
\draw [fill=xdxdff] (0.,1.) circle (2.5pt);
\draw[color=black] (0.,0.77) node {$\{V_f,W_e\}$};
\draw [fill=xdxdff] (0.,3.) circle (2.5pt);
\draw[color=black] (0.,3.27) node {$\{V_e,W_f\}$};
\draw[color=black] (0.38,2.17) node {$\{e,f\}$};
\end{scriptsize}
%\end{axis}
\end{tikzpicture}

%-----------------------------------------
\definecolor{ududff}{rgb}{0.30196078431372547,0.30196078431372547,1.}
\definecolor{xdxdff}{rgb}{0.49019607843137253,0.49019607843137253,1.}
\definecolor{ffffqq}{rgb}{1.,1.,0.}
\begin{tikzpicture}[line cap=round,line join=round,>=triangle 45,x=1.4cm,y=1.4cm]
%\begin{axis}[
x=1.4cm,y=1.4cm,
axis lines=middle,
ymajorgrids=true,
xmajorgrids=true,
xmin=-2.9600000000000013,
xmax=7.400000000000004,
ymin=-0.5400000000000005,
ymax=4.2200000000000015,
xtick={-2.0,-1.0,...,8.0},
ytick={-2.0,-1.0,...,6.0},]
\clip(-2.96,-0.54) rectangle (7.4,4.22);
\fill[line width=2.pt,color=ffffqq,fill=ffffqq,fill opacity=0.550000011920929] (-2.,0.) -- (2.,0.) -- (2.,4.) -- (-2.,4.) -- cycle;
\fill[line width=2.pt,color=ffffqq,fill=ffffqq,fill opacity=0.550000011920929] (3.,0.) -- (7.,0.) -- (7.,4.) -- (3.,4.) -- cycle;
\draw [line width=2.pt] (0.,1.)-- (0.,3.);
%\draw [line width=2.pt,color=ffffqq] (3.,0.)-- (7.,0.);
%\draw [line width=2.pt,color=ffffqq] (7.,0.)-- (7.,4.);
%\draw [line width=2.pt,color=ffffqq] (7.,4.)-- (3.,4.);
%\draw [line width=2.pt,color=ffffqq] (3.,4.)-- (3.,0.);
\draw [line width=2.pt] (5.,3.)-- (5.,1.);
\begin{scriptsize}
\draw [fill=xdxdff] (0.,1.) circle (2.5pt);
\draw[color=black] (0.,0.77) node {$\{W_e,W_f\}$};
\draw [fill=xdxdff] (0.,3.) circle (2.5pt);
\draw[color=black] (0.14,3.37) node {$\{V_e,V_f\}$};
\draw[color=black] (0.38,2.17) node {$\{e,f\}$};
\draw [fill=ududff] (5.,3.) circle (2.5pt);
\draw[color=black] (5.14,3.37) node {$\{V_e,W_f\}$};
\draw [fill=ududff] (5.,1.) circle (2.5pt);
\draw[color=black] (5.,0.77) node {$\{V_f,W_e\}$};
\draw[color=black] (5.38,2.17) node {$\{e,f\}$};
\end{scriptsize}
%\end{axis}
\end{tikzpicture}
\end{comment}
%-----------------------------------------
\item Angenommen es existiert eine Kante $g\in Y_1$ mit $V_e,W_f <_{\alpha}g$, aber für $V_f$ und $W_e$ existiert keine Kante $h\in Y_1$ mit $V_f,W_e <_{\alpha} h$. \\
%------------------------------------------
\begin{figure}[h]
\definecolor{uuuuuu}{rgb}{0.26666666666666666,0.26666666666666666,0.26666666666666666}
\definecolor{ffffff}{rgb}{1.,1.,1.}
\definecolor{ududff}{rgb}{0.30196078431372547,0.30196078431372547,1.}
\definecolor{ffffqq}{rgb}{1.,1.,0.}
\begin{tikzpicture}[line cap=round,line join=round,>=triangle 45,x=1.2cm,y=1.2cm]
%\begin{axis}[
x=1.0cm,y=1.0cm,
axis lines=middle,
ymajorgrids=true,
xmajorgrids=true,
xmin=-4.3,
xmax=7.0600000000000005,
ymin=-2.46,
ymax=6.3,
xtick={-4.0,-3.0,...,7.0},
ytick={-2.0,-1.0,...,6.0},]
\clip(-6.0,-0.4) rectangle (7.06,3.8);
\fill[line width=2.pt,color=ffffqq,fill=ffffqq,fill opacity=\gelb] (-2.,0.) -- (2.,0.) -- (2.,4.) -- (-2.,4.) -- cycle;
\fill[line width=2.pt,color=ffffff,fill=ffffff,fill opacity=1.0] (-1.,1.) -- (1.,1.) -- (1.,3.) -- (-1.,3.) -- cycle;
\draw [line width=2.pt] (1.,1.)-- (1.,3.);
\draw [line width=2.pt] (-1.,3.)-- (-1.,1.);
\draw [line width=2.pt] (-1.,3.)-- (1.,1.);
\begin{scriptsize}
%\draw[color=ffffqq] (0.48,2.17) node {$Vieleck1$};
\draw [fill=ududff] (-1.,1.) circle (2.5pt);
\draw[color=black] (-0.86,0.77) node {$W_e$};
\draw [fill=ududff] (1.,1.) circle (2.5pt);
\draw[color=black] (1.14,0.77) node {$W_f$};
%\draw[color=ffffff] (0.48,2.17) node {$Vieleck2$};
\draw [fill=ududff] (1.,3.) circle (2.5pt);
\draw[color=black] (1.14,3.27) node {$V_f$};
\draw [fill=ududff] (-1.,3.) circle (2.5pt);
\draw[color=black] (-0.86,3.27) node {$V_e$};
\draw[color=black] (-0.18,1.95) node {$h$};
\draw[color=black] (-1.22,1.95) node {$e$};
\draw[color=black] (1.18,1.95) node {$f$};
\end{scriptsize}
%\end{axis}
\end{tikzpicture}
\caption{mendable Randkantenpaar vom Typ 0 bzgl. $V_e,V_f$}
\end{figure}

%-----------------------------------------------
Dann gibt es nur eine Möglichkeit mit dem Splitmender eine simpliziale Fläche zu konstruieren, nämlich $Z=S^m_{(V_e,e),(V_f,f)}(Y)$, wobei für $\{V_e,V_f\},\{W_e,W_f\}\in Z_0$ und $\{e,f\} \in Z_1$
\[
\{V_e,V_f\},\{W_e,W_f\}\prec \{e,f\}
\]und 
\[
\{V_e,V_f\},\{W_e,W_f\}\prec g
\] gilt. Dabei ist $\prec$ die Inzidenz auf $Z$.\\
%-------------------bild---------------------------
\begin{comment}
\definecolor{ududff}{rgb}{0.30196078431372547,0.30196078431372547,1.}
\definecolor{xdxdff}{rgb}{0.49019607843137253,0.49019607843137253,1.}
\definecolor{ffffqq}{rgb}{1.,1.,0.}
\begin{tikzpicture}[line cap=round,line join=round,>=triangle 45,x=1.0cm,y=1.0cm]
\begin{axis}[
x=1.0cm,y=1.0cm,
axis lines=middle,
ymajorgrids=true,
xmajorgrids=true,
xmin=-4.3,
xmax=7.0600000000000005,
ymin=-2.46,
ymax=6.3,
xtick={-4.0,-3.0,...,7.0},
ytick={-2.0,-1.0,...,6.0},]
\clip(-4.3,-2.46) rectangle (7.06,6.3);
\fill[line width=2.pt,color=ffffqq,fill=ffffqq,fill opacity=0.5] (-2.,0.) -- (2.,0.) -- (2.,4.) -- (-2.,4.) -- cycle;
\draw [shift={(-0.9,2.06)},line width=2.pt,fill=black,fill opacity=0.8999999761581421]  plot[domain=-0.9505468408120752:0.9302106616363873,variable=\t]({1.*1.3763720427268205*cos(\t r)+0.*1.3763720427268205*sin(\t r)},{0.*1.3763720427268205*cos(\t r)+1.*1.3763720427268205*sin(\t r)});
\draw [line width=2.pt] (0.02,3.16)-- (0.,0.94);
\begin{scriptsize}
\draw[color=ffffqq] (0.48,2.17) node {$Vieleck1$};
\draw [fill=xdxdff] (0.,0.94) circle (2.5pt);
\draw[color=xdxdff] (0.14,1.31) node {$F$};
\draw [fill=ududff] (0.02,3.16) circle (2.5pt);
\draw[color=ududff] (0.16,3.53) node {$G$};
\draw[color=black] (0.74,2.35) node {$c$};
\draw[color=black] (-0.26,2.23) node {$j$};
\end{scriptsize}
\end{axis}
\end{tikzpicture}
\end{comment}
%--------------------------------------------------
%------------bild----------------------------------
\begin{figure}[h]
\definecolor{ffffff}{rgb}{1.,1.,1.}
\definecolor{qqqqff}{rgb}{0.,0.,1.}
\definecolor{ffffqq}{rgb}{1.,1.,0.}
\begin{tikzpicture}[line cap=round,line join=round,>=triangle 45,x=1.5cm,y=1.5cm]
%\begin{axis}[
x=1.5cm,y=1.5cm,
axis lines=middle,
ymajorgrids=true,
xmajorgrids=true,
xmin=-4.3,
xmax=7.0600000000000005,
ymin=-2.46,
ymax=6.3,
xtick={-4.0,-3.0,...,7.0},
ytick={-2.0,-1.0,...,6.0},]
\clip(-4.8,-.46) rectangle (7.06,4.3);
\fill[line width=2.pt,color=ffffqq,fill=ffffqq,fill opacity=\gelb] (-2.,0.) -- (2.,0.) -- (2.,4.) -- (-2.,4.) -- cycle;
%\fill[line width=2.pt,color=ffffqq,fill=ffffqq,fill opacity=0.5] (2.25,2.45) -- (6.25,2.45) -- (6.25,6.45) -- (2.25,6.45) -- cycle;
\draw [line width=2.pt] (0.,1.)-- (0.,3.);
%\draw [line width=2.pt] (4.29,3.45)-- (4.29,5.45);

\draw [shift={(-1.46,2.02)},line width=2.pt,color=ffffff,fill=ffffff,fill opacity=1.0]  plot[domain=-0.6098060014472679:0.5911571672160445,variable=\t]({1.*1.7810109488714547*cos(\t r)+0.*1.7810109488714547*sin(\t r)},{0.*1.7810109488714547*cos(\t r)+1.*1.7810109488714547*sin(\t r)});
\draw [shift={(-1.46,2.02)},line width=2.pt]  plot[domain=-0.6098060014472679:0.5911571672160445,variable=\t]({1.*1.7810109488714547*cos(\t r)+0.*1.7810109488714547*sin(\t r)},{0.*1.7810109488714547*cos(\t r)+1.*1.7810109488714547*sin(\t r)});
%\draw [fill=qqqqff] (4.30,3.45) circle (2.5pt);
%\draw [fill=qqqqff] (4.30,5.45) circle (2.5pt);
\begin{scriptsize}
%\draw[color=ffffqq] (0.48,2.17) node {$Vieleck1$};
\draw [fill=qqqqff] (0.,1.) circle (2.5pt);
\draw[color=black] (0.5,2.) node {$g$};
\draw [fill=qqqqff] (0.,3.) circle (2.5pt);
\draw[color=black] (-0.3,2) node {$\{e,f\}$};
\draw[color=black] (0.,0.7) node {$\{W_e,W_f\}$};
\draw[color=black] (0.,3.3) node {$\{V_e,V_f\}$};
%\draw[color=ffffff] (0.5,2.31) node {$c$};
%\draw[color=black] (0.5,2.31) node {$d$};
\end{scriptsize}
\%end{axis}
\end{tikzpicture}
\caption{simpliziale Fläche nach Anwenden eines Splitmenders}
\end{figure}

%---------------------------------------------------
%\item Der Fall, das eine Kante $g\in Y_1$ mit $V_e,W_f<_{\alpha}$, aber keine Kante $h \in Y_1$ mit $V_f,W_e$ existiert,  erläuft analog zum zuvor behandelten Fall.
\end{enumerate}
\end{bemerkung}

  \begin{bemerkung}
  Sei $(X,<)$ eine simpliziale Fläche mit n Flächen. Dann gilt: 
  \begin{enumerate}
  \item $X$ ist isomorph zu einer simplizialen Fläche $(Y,\prec) \in \mathcal{M}(n\Delta)$.
  \item Sei $k$ die Anzahl der inneren Kanten von $(X,<)$. Dann ist die Anzahl, der auf $n\Delta$ ausgeführten Mender-Operationen, um $X$ zu erhalten, ebenfalls $k$. Eine analoge Aussage gilt auch für Cutter-Operationen.
  \item Es können genau dann keine Mender-Operationen auf $X$ durchgeführt werden, wenn $X$ abgeschlossen ist oder isomorph zu einer abgeschlossenen simplizialen Fläche ist, aus der eine Fläche entfernt wurde.
  \item $X=n \cdot \Delta$ genau dann wenn keine Cutter-Operationen auf $X$ angewendet können.
  \end{enumerate}
  \end{bemerkung}
  
%--------------------------------------------Definition HikingHole-----------------------------
\subsection{Euler-Charakteristik}
%$\textcolor{red}{weg}$
\begin{definition}
Für eine simpliziale Fläche $(X,<)$ definiert man die Euler-Charakteristik $\chi(X)$ als
\[
\chi(X) := \vert X_0 \vert - \vert X_1\vert +\vert X_2 \vert.
\]
\end{definition}
\begin{bemerkung}
 Zwei isomorphe simpliziale Flächen $(X,<)$ und $(Y, \prec)$ haben dieselbe Euler-Charakteristik, denn eine bijektive Abbilung $\alpha:X \to Y$ impliziert, wie oben schon erwähnt, bijektive Abbildungen,  $X_i \to Y_i$ für $i=0,1,2$, damit ist $\vert X_i \vert =\vert Y_i \vert $, woraus man
 \[
\chi(X) =\vert X_0 \vert - \vert X_1\vert +\vert X_2 \vert = \vert Y_0 \vert - \vert Y_1\vert +\vert Y_2 \vert =\chi(Y)
 \]
 folgern kann.\\
 Die Umkehrung gilt jedoch nicht, denn beispielsweise für den Janus-Kopf $J$ und den \emph{Tetraeder} $(T,<)$ definiert durch das ordinale Symbol 
 \begin{align*}
 \mu ((T,<)):=(4,6,4;(\{2,3\},\{1,3\},\{2,3\},\{3,4\},\{1,4\},\{2,4\}),\\
 (\{1,2,3\},\{2,4,5\},\{3,5,6\},\{2,3,4\}))
 \end{align*}
 gilt
 \[
\chi(J)=3-3+2=2=4-6+4=\chi(T).
 \]
 Aber wegen $\vert J_i\vert \neq \vert T_i \vert $ für $i=0,1,2$ sind  $J$ und $T$ nicht isomorph zueinander.
 %----bild-------
 \begin{figure}[h]
 \definecolor{ffffqq}{rgb}{1.,1.,0.}
\definecolor{qqqqff}{rgb}{0.,0.,1.}
\begin{tikzpicture}[line cap=round,line join=round,>=triangle 45,x=1.0cm,y=1.0cm]
%\begin{axis}[
x=1.0cm,y=1.0cm,
axis lines=middle,
ymajorgrids=true,
xmajorgrids=true,
xmin=-8.620000000000001,
xmax=14.38,
ymin=-3.72,
ymax=4.32,
xtick={-8.0,-7.0,...,14.0},
ytick={-5.0,-4.0,...,5.0},]
\clip(-7.62,-3.72) rectangle (14.38,4.32);
\fill[line width=2.pt,color=ffffqq,fill=ffffqq,fill opacity=\gelb] (-2.,0.) -- (2.,0.) -- (0.,3.4641016151377553) -- cycle;
\fill[line width=2.pt,color=ffffqq,fill=ffffqq,fill opacity=\gelb] (2.,0.) -- (-2.,0.) -- (0.,-3.4641016151377553) -- cycle;
\fill[line width=2.pt,color=ffffqq,fill=ffffqq,fill opacity=\gelb] (0.,3.4641016151377553) -- (2.,0.) -- (4.,3.464101615137754) -- cycle;
\fill[line width=2.pt,color=ffffqq,fill=ffffqq,fill opacity=\gelb] (-2.,0.) -- (0.,3.4641016151377553) -- (-4.,3.464101615137757) -- cycle;
\draw [line width=2.pt] (-2.,0.)-- (2.,0.);
\draw [line width=2.pt] (2.,0.)-- (0.,3.4641016151377553);
\draw [line width=2.pt] (0.,3.4641016151377553)-- (-2.,0.);
\draw [line width=2.pt] (2.,0.)-- (-2.,0.);
\draw [line width=2.pt] (-2.,0.)-- (0.,-3.4641016151377553);
\draw [line width=2.pt] (0.,-3.4641016151377553)-- (2.,0.);
\draw [line width=2.pt] (0.,3.4641016151377553)-- (2.,0.);
\draw [line width=2.pt] (2.,0.)-- (4.,3.464101615137754);
\draw [line width=2.pt] (4.,3.464101615137754)-- (0.,3.4641016151377553);
\draw [line width=2.pt] (-2.,0.)-- (0.,3.4641016151377553);
\draw [line width=2.pt] (0.,3.4641016151377553)-- (-4.,3.464101615137757);
\draw [line width=2.pt] (-4.,3.464101615137757)-- (-2.,0.);
\begin{scriptsize}
\draw [fill=qqqqff] (-2.,0.) circle (2.5pt);
\draw[color=black] (-2.32,-0.09) node {$V_1$};
\draw [fill=qqqqff] (2.,0.) circle (2.5pt);
\draw[color=black] (2.34,0.) node {$V_2$};
\draw[color=black] (0.,1.33) node {$F_1$};
\draw[color=black] (0.06,-0.25) node {$e_3$};
\draw[color=black] (1.32,2.07) node {$e_1$};
\draw[color=black] (-1.22,2.07) node {$e_2$};
\draw [fill=qqqqff] (0.,3.4641016151377553) circle (2.5pt);
\draw[color=black] (0.14,3.83) node {$V_3$};
\draw[color=black] (0.,-1.33) node {$F_3$};
\draw[color=black] (-1.27,-1.71) node {$e_5$};
\draw[color=black] (1.37,-1.71) node {$e_6$};
\draw [fill=qqqqff] (0.,-3.4641016151377553) circle (2.5pt);
\draw[color=black] (0.49,-3.29) node {$V_4$};
\draw [fill=qqqqff] (-2.,0.) circle (2.5pt);
%\draw[color=qqqqff] (-1.86,0.37) node {$E$};
\draw[color=black] (2.,2.19) node {$F_4$};
\draw[color=black] (3.37,1.75) node {$e_6$};
\draw[color=black] (2.11,3.82) node {$e_4$};
\draw [fill=qqqqff] (4.,3.464101615137754) circle (2.5pt);
\draw[color=black] (4.14,3.83) node {$V_4$};
\draw[color=black] (-2.,2.19) node {$F_2$};
\draw[color=black] (-1.94,3.82) node {$e_4$};
\draw[color=black] (-3.3,1.75) node {$e_5$};
\draw [fill=qqqqff] (-4.,3.464101615137757) circle (2.5pt);
\draw[color=black] (-3.86,3.83) node {$V_4$};
\end{scriptsize}
%\end{axis}
\end{tikzpicture}

 \caption{Tetraeder}
 \label{tetra}
 \end{figure}
 %-------------------------------
\end{bemerkung}

Wenn man die Euler-Charakteristik einer simplizialen Fläche $(X,<)$ unter Anwendung der Mender- und Cutteroperatoren beobachtet, so ergibt sich für $\{e,f\} \in X_1$, 
\begin{itemize}
\item falls $\{e,f\} \in I^2(X)$,  
\[
\chi(C^c_{\{e,f\}}(X))=\chi(X)-1,
\]
denn für die Knoten, Flächen und Kanten von der simplizialen Fläche $C^c_{\{e,f\}}(X)$ gilt $\vert C^c_{\{e,f\}}(X)_0 \vert =\vert X_0\vert $, $\vert C^c_{\{e,f\}}(X)_1 \vert =\vert X_1\vert +1$ und $\vert C^c_{\{e,f\}}(X)_2 \vert =\vert X_2\vert $. Somit erhält man 
\begin{align*}
\chi(C^c_{\{e,f\}}(X))&=\vert C^c_{\{e,f\}}(X)_0 \vert-\vert C^c_{\{e,f\}}(X)_1 \vert+\vert C^c_{\{e,f\}}(X)_2 \vert\\
&=\vert X_0\vert-(\vert X_1\vert+1)+\vert X_2\vert\\
&=\vert X_0\vert-\vert X_1\vert+\vert X_2\vert -1\\
&=\chi(X)-1.
\end{align*}
\end{itemize}
Durch analoge Herangehensweise bei den anderen Operatoren ergibt sich folgendes:
\begin{itemize}
\item Falls $\{e,f\} \in BM^2(X)$, dann ist 
\[
\chi(C^m_{e,f}(X))=\chi(X)+1.
\]
%----
\item Falls $\{e,f\} \in I^1(X)$, dann ist 
\[
\chi(R^c_{\{e,f\}}(X))=\chi(X).
\]
\item Falls $\{e,f\} \in BM^1(X)$, dann ist 
\[
\chi(R^m_{e,f}(X))=\chi(X).
\]
%----
\item Falls $\{e,f\} \in I^0(X)$, dann ist 
\[
\chi(S^c_{\{e,f\}}(X))=\chi(X)+1.
\]
\item Und falls $\{e,f\} \in BM^0(X)$ bezüglich $V_e,V_f \in X_0$ mit $V_e<e$ und $V_f<f$, dann ist 
\[
\chi(S^m_{(V_e,e),(V_f,f)}(X))=\chi(X)-1.
\]
\end{itemize}
\newpage
%-------Überarbeitung------
\section{Das wandernde Loch}
 Seien $(X,<)$ eine geschlossene simpliziale Fläche, $F \in X_{2}$, $e_{i} \in X_{1}$ und $V_{j} \in X_{0}$ für $i \in \{1,2,3\},j \in \{1,2,3\}$ mit folgenden Eigenschaften:
 \begin{itemize}
 \item $\vert X_{2}\vert \geq 3$,
 \item $e_{i} < F$ für alle $i \in \{1,2,3\}$,
 \item $V_{i}<e_{j}$ für alle $i \in \{1,2,3\}$ und $j \in \{1,2,3\} \setminus\{i\}$ ,
 \item $V_{i} < F$ für alle $i \in \{1,2,3\}$.
\end{itemize}  
Zudem seien $f,g \in X_1,V_4 \in X_0,F' \in X_2$ so, dass
\begin{itemize}
\item $e_3<F'$ und $e_3<F$,
\item $f,g <F'$, 
\item $V_1,V_4<f$ und $V_2,V_4<g$.
\end{itemize}
%--------------bild----------------------
\begin{comment}
\definecolor{ttqqqq}{rgb}{0.2,0.,0.}
\definecolor{sqsqsq}{rgb}{0.12549019607843137,0.12549019607843137,0.12549019607843137}
\definecolor{ffffqq}{rgb}{1.,1.,0.}
\definecolor{qqqqff}{rgb}{0.,0.,1.}
\begin{tikzpicture}[line cap=round,line join=round,>=triangle 45,x=1.0cm,y=1.0cm]
x=1.0cm,y=1.0cm,
axis lines=middle,
ymajorgrids=true,
xmajorgrids=true,
xmin=-5.056290110700678,
xmax=5.380866801866215,
ymin=-0.9227448489396118,
ymax=4.359364127681193,
xtick={-5.0,-4.5,...,5.0},
ytick={-0.5,0.0,...,4.0},]
\clip(-5.056290110700678,-0.9227448489396118) rectangle (5.380866801866215,4.359364127681193);
\fill[line width=2.pt,color=ffffqq,fill=ffffqq,fill opacity=0.5] (-2.,0.) -- (2.,0.) -- (0.,3.4641016151377553) -- cycle;
\fill[line width=2.pt,color=ffffqq,fill=ffffqq,fill opacity=0.5] (0.,3.4641016151377553) -- (2.,0.) -- (4.,3.464101615137754) -- cycle;
\draw [line width=2.pt,color=sqsqsq] (-2.,0.)-- (2.,0.);
\draw [line width=2.pt] (2.,0.)-- (0.,3.4641016151377553);
\draw [line width=2.pt] (0.,3.4641016151377553)-- (-2.,0.);
\draw [line width=2.pt,color=ttqqqq] (0.,3.4641016151377553)-- (2.,0.);
\draw [line width=2.pt] (2.,0.)-- (4.,3.464101615137754);
\draw [line width=2.pt,color=sqsqsq] (4.,3.464101615137754)-- (0.,3.4641016151377553);
\begin{scriptsize}
\draw [fill=qqqqff] (-2.,0.) circle (2.5pt);
\draw[color=qqqqff] (-2.047666183295526,-0.2934108260899206) node {$V_4$};
\draw [fill=qqqqff] (2.,0.) circle (2.5pt);
\draw[color=qqqqff] (2.190727102068595,-0.2934108260899206) node {$V_2$};
\draw[color=black] (0.05337888214728793,1.23275494822094) node {$F'$};
\draw[color=sqsqsq] (0.03522730490804116,-0.24676492196181318) node {$g$};
\draw[color=black] (1.174238776670776,1.9089012003828816) node {$e_3$};
\draw[color=black] (-1.1900041587411159,1.7862117288338232) node {$f$};
\draw [fill=qqqqff] (0.,3.4641016151377553) circle (2.5pt);
\draw[color=qqqqff] (0.09514414231596978,3.75145261535057) node {$V_1$};
\draw[color=black] (2.0228250126055625,2.3944558915327323) node {$F$};
\draw[color=black] (3.3162912160860375,1.7636885824689075) node {$e_1$};
\draw[color=sqsqsq] (2.054590272774244,3.7603620787860503) node {$e_2$};
\draw [fill=qqqqff] (4.,3.464101615137754) circle (2.5pt);
\draw[color=qqqqff] (4.087566923569883,3.75145261535057) node {$V_3$};
\end{scriptsize}
\end{tikzpicture}
\end{comment}
%---------------------------------------------
\begin{figure}[h]
\definecolor{uuuuuu}{rgb}{0.26666666666666666,0.26666666666666666,0.26666666666666666}
\definecolor{ududff}{rgb}{0.30196078431372547,0.30196078431372547,1.}
\definecolor{ffffqq}{rgb}{1.,1.,0.}
\begin{tikzpicture}[line cap=round,line join=round,>=triangle 45,x=1.5cm,y=1.5cm]
%\begin{axis}[
x=1.5cm,y=1.5cm,
axis lines=middle,
ymajorgrids=true,
xmajorgrids=true,
xmin=-4.3,
xmax=7.0600000000000005,
ymin=-2.46,
ymax=6.3,
xtick={-4.0,-3.0,...,7.0},
ytick={-2.0,-1.0,...,6.0},]
\clip(-5.,-0.45) rectangle (3.06,4.3);
\fill[line width=2.pt,color=ffffqq,fill=ffffqq,fill opacity=\gelb] (-2.,0.) -- (2.,0.) -- (2.,4.) -- (-2.,4.) -- cycle;
%\fill[line width=2.pt,color=ffffqq,fill=ffffqq,fill opacity=\gelb] (-1.,2.) -- (1.,2.) -- (0.,3.7320508075688776) -- cycle;
%\fill[line width=2.pt,color=ffffqq,fill=ffffqq,fill opacity=\gelb] (1.,2.) -- (-1.,2.) -- (0.,0.2679491924311226) -- cycle;
\draw [line width=2.pt] (-1.,2.)-- (1.,2.);
\draw [line width=2.pt] (1.,2.)-- (0.,3.7320508075688776);
\draw [line width=2.pt] (0.,3.7320508075688776)-- (-1.,2.);
\draw [line width=2.pt] (1.,2.)-- (-1.,2.);
\draw [line width=2.pt] (-1.,2.)-- (0.,0.2679491924311226);
\draw [line width=2.pt] (0.,0.2679491924311226)-- (1.,2.);
\begin{scriptsize}
\draw [fill=ududff] (-1.,2.) circle (2.5pt);
\draw[color=black] (-1.24,2.07) node {$V_1$};
\draw [fill=ududff] (1.,2.) circle (2.5pt);
\draw[color=black] (1.24,2.07) node {$V_2$};
\draw[color=black] (0.,2.75) node {$F$};
\draw[color=black] (0.06,1.85) node {$e_3$};
\draw[color=black] (0.67,2.96) node {$e_1$};
\draw[color=black] (-0.67,2.96) node {$e_2$};
\draw [fill=ududff] (0.,3.7320508075688776) circle (2.5pt);
\draw[color=black] (0.,3.91) node {$V_3$};
\draw[color=black] (0.,1.31) node {$F'$};
\draw[color=black] (-0.7,1.16) node {$f$};
\draw[color=black] (0.7,1.16) node {$g$};
\draw [fill=ududff] (0.,0.2679491924311226) circle (2.5pt);
\draw[color=black] (0.,0.1) node {$V_4$};
\end{scriptsize}
%\end{axis}
\end{tikzpicture}
\caption{Ausschnitt einer simplizialen Fläche}
\end{figure}
%-----------------------------------------
Sei außerdem $(Y,\prec)$ eine simpliziale Fläche mit
\begin{itemize}
\item $Y_{2}=X_{2}$,
\item $Y_{1}=(X_{1} \setminus\{e_{1},e_2,e_3,f\} )\cup \{e_{1}^1,e_1^{2},e_{2}^1,e_2^{2},e_{3}^1,e_3^{2},f^1,f^2\}$,
\item $Y_{0}=(X_{0} \setminus\{V_{1},V_2,V_3\} )\cup \{V_{1}^1,V_1^{2},V_1^{3},V_{2}^1,V_2^{2},V_{3}^1,V_3^{2}\}$

\end{itemize}
und zugehöriger Inzidenz $\prec$ 
so, dass durch die Abbildung 
\[
\alpha: Y \to X ,x \mapsto 
\begin{cases}
x & \text{für } x\in X \cap Y\\
e_i & \text{für } x =e_i^j ,i=1,2,3,\,j=1,2\\
V_i &\text{für } x =V_i^j,i=1,2,3,\,j=1,2\\
V_1 &\text{für } x=V_1^3\\
f &\text{für } x=f^1,f^2 
\end{cases}
\]
eine Mending Map definiert wird. Die Existenz einer solchen Fläche folgt mit der unten beschriebenen Konstruktion. Somit ist $X$ isomorph zu der simplizialen Fläche $Y(\alpha)$ und deshalb ist es möglich, $X$ für die unten skizzierte Konstruktion mit $Y(\alpha)$ und so die oben beschrieben Knoten und Kanten  mit ihren Urbildern in $Y$ zu identifizieren. \\
%--------------------------------------------------
\begin{comment}
\definecolor{ttqqqq}{rgb}{0.2,0.,0.}
\definecolor{sqsqsq}{rgb}{0.12549019607843137,0.12549019607843137,0.12549019607843137}
\definecolor{ffffqq}{rgb}{1.,1.,0.}
\definecolor{qqqqff}{rgb}{0.,0.,1.}
\begin{tikzpicture}[line cap=round,line join=round,>=triangle 45,x=1.0cm,y=1.0cm]
x=1.0cm,y=1.0cm,
axis lines=middle,
ymajorgrids=true,
xmajorgrids=true,
xmin=-5.056290110700678,
xmax=5.380866801866215,
ymin=-0.9227448489396118,
ymax=4.359364127681193,
xtick={-5.0,-4.5,...,5.0},
ytick={-0.5,0.0,...,4.0},]
\clip(-5.056290110700678,-0.9227448489396118) rectangle (5.380866801866215,4.359364127681193);
\fill[line width=2.pt,color=ffffqq,fill=ffffqq,fill opacity=0.5] (-2.,0.) -- (2.,0.) -- (0.,3.4641016151377553) -- cycle;
\fill[line width=2.pt,color=ffffqq,fill=ffffqq,fill opacity=0.5] (0.,3.4641016151377553) -- (2.,0.) -- (4.,3.464101615137754) -- cycle;
\draw [line width=2.pt,color=sqsqsq] (-2.,0.)-- (2.,0.);
\draw [line width=2.pt] (2.,0.)-- (0.,3.4641016151377553);
\draw [line width=2.pt] (0.,3.4641016151377553)-- (-2.,0.);
\draw [line width=2.pt,color=ttqqqq] (0.,3.4641016151377553)-- (2.,0.);
\draw [line width=2.pt] (2.,0.)-- (4.,3.464101615137754);
\draw [line width=2.pt,color=sqsqsq] (4.,3.464101615137754)-- (0.,3.4641016151377553);
\begin{scriptsize}
\draw [fill=qqqqff] (-2.,0.) circle (2.5pt);
\draw[color=qqqqff] (-2.0204388174366557,-0.2934108260899206) node {$\{V_4\}$};
\draw [fill=qqqqff] (2.,0.) circle (2.5pt);
\draw[color=qqqqff] (2.217954467927465,-0.2934108260899206) node {$\{V_2\}$};
\draw[color=black] (0.05337888214728793,1.23275494822094) node {$F'$};
\draw[color=sqsqsq] (0.062454670766911316,-0.24676492196181318) node {\{g\}};
\draw[color=black] (1.4829819149221139,1.9089012003828816) node {$\{e_3^1,e_3^2\}$};
\draw[color=black] (-1.4837010042626223,1.9678938264104335) node {$\{f^1,f^2\}$};
\draw [fill=qqqqff] (0.,3.4641016151377553) circle (2.5pt);
\draw[color=qqqqff] (0.11237150817483993,3.75145261535057) node {$\{V_1\}$};
\draw[color=black] (2.0228250126055625,2.3944558915327323) node {F};
\draw[color=black] (3.6250343543373756,1.7636885824689075) node {$\{e_1^1,e_1^2\}$};
\draw[color=sqsqsq] (2.2633334110255823,3.7603620787860503) node {$\{e_2^1,e_2^2\}$};
\draw [fill=qqqqff] (4.,3.464101615137754) circle (2.5pt);
\draw[color=qqqqff] (4.114794289428753,3.75145261535057) node {$\{V_3\}$};
\end{scriptsize}

\end{tikzpicture}
\end{comment}
%----------------------------bild------------------
\begin{figure}[h]
\definecolor{uuuuuu}{rgb}{0.26666666666666666,0.26666666666666666,0.26666666666666666}
\definecolor{ududff}{rgb}{0.30196078431372547,0.30196078431372547,1.}
\definecolor{ffffqq}{rgb}{1.,1.,0.}
\begin{tikzpicture}[line cap=round,line join=round,>=triangle 45,x=1.5cm,y=1.5cm]
%\begin{axis}[
x=1.5cm,y=1.5cm,
axis lines=middle,
ymajorgrids=true,
xmajorgrids=true,
xmin=-4.3,
xmax=7.0600000000000005,
ymin=-2.46,
ymax=6.3,
xtick={-4.0,-3.0,...,7.0},
ytick={-2.0,-1.0,...,6.0},]
\clip(-5.3,-0.46) rectangle (3.06,4.3);
\fill[line width=2.pt,color=ffffqq,fill=ffffqq,fill opacity=\gelb] (-2.2,-0.1) -- (2.,-0.1) -- (2.,4.1) -- (-2.2,4.1) -- cycle;
%\fill[line width=2.pt,color=ffffqq,fill=ffffqq,fill opacity=\gelb] (-1.,2.) -- (1.,2.) -- (0.,3.7320508075688776) -- cycle;
%\fill[line width=2.pt,color=ffffqq,fill=ffffqq,fill opacity=\gelb] (1.,2.) -- (-1.,2.) -- (0.,0.2679491924311226) -- cycle;
\draw [line width=2.pt] (-1.,2.)-- (1.,2.);
\draw [line width=2.pt] (1.,2.)-- (0.,3.7320508075688776);
\draw [line width=2.pt] (0.,3.7320508075688776)-- (-1.,2.);
\draw [line width=2.pt] (1.,2.)-- (-1.,2.);
\draw [line width=2.pt] (-1.,2.)-- (0.,0.2679491924311226);
\draw [line width=2.pt] (0.,0.2679491924311226)-- (1.,2.);
\begin{scriptsize}
\draw [fill=ududff] (-1.,2.) circle (2.5pt);
\draw[color=black] (-1.6,2.2) node {$\{V_1^1,V_1^2,V_1^3\}$};
\draw [fill=ududff] (1.,2.) circle (2.5pt);
\draw[color=black] (1.54,2.07) node {$\{V_2^1,V_2^2\}$};
\draw[color=black] (0.,2.75) node {$F$};
\draw[color=black] (0.06,1.85) node {$\{e_3^1,e_3^2\}$};
\draw[color=black] (0.87,2.96) node {$\{e_1^1,e_1^2\}$};
\draw[color=black] (-0.87,2.96) node {$\{e_2^1,e_2^2\}$};
\draw [fill=ududff] (0.,3.7320508075688776) circle (2.5pt);
\draw[color=black] (0.,3.91) node {$\{V_3^1,V_3^2\}$};
\draw[color=black] (0.,1.31) node {$F'$};
\draw[color=black] (-1.,1.16) node {$\{f^1,f^2\}$};
\draw[color=black] (0.8,1.16) node {$\{g\}$};
\draw [fill=ududff] (0.,0.2679491924311226) circle (2.5pt);
\draw[color=black] (0.,0.1) node {$\{V_4\}$};
\end{scriptsize}
%\end{axis}
\end{tikzpicture}
\caption{Ausschnitt eines Mendings einer simplizialen Fläche}
\end{figure}
%--------------------------------------------------
Ziel ist es, wie schon erwähnt, durch Anwenden der Mender- und Cutteroperatoren aus $(X,<)$ eine simpliziale Fläche $X^H_{(F,f)}$ zu konstruieren. 
Hierzu definiert man die folgenden drei Prozeduren:\\
%\begin{enumerate} 
\newpage
\subsection{Prozedur $P^1$}
 Zunächst soll ein sogenanntes \emph{Loch an der Stelle $F$} erzeugt werden, welches entsteht, wenn man $F$ von der simplizialen Fläche trennt:
\begin{enumerate}[(i)]
\item Wende einen $Crater Cut$ $C^{c}_{\{e_{1}^1,e_{1}^2\}}$ an, um aus der Kante $\{e_{1}^1,e_{1}^2\}$ die Kanten $\{e_1^1\}$ und $\{e_1^2\}$ zu erhalten, wobei für $\{e_1^1\}$
\[
\{e_1^1\} <F
\]
und für $\{e_1^2\}$
\[
\{e_1^2\} \nless F
\]
gilt.\\
%---------------------bild---------------------------
\begin{figure}[h]
\definecolor{ffffff}{rgb}{1.,1.,1.}
\definecolor{qqqqff}{rgb}{0.,0.,1.}
\definecolor{ududff}{rgb}{0.30196078431372547,0.30196078431372547,1.}
\definecolor{ffffqq}{rgb}{1.,1.,0.}
\begin{tikzpicture}[line cap=round,line join=round,>=triangle 45,x=1.4cm,y=1.4cm]
%\begin{axis}[
x=1.0cm,y=1.0cm,
axis lines=middle,
ymajorgrids=true,
xmajorgrids=true,
xmin=-4.3,
xmax=18.7,
ymin=-5.34,
ymax=6.3,
xtick={-4.0,-3.0,...,18.0},
ytick={-5.0,-4.0,...,6.0},]
\clip(-5.5,-0.34) rectangle (3.7,4.3);
\fill[line width=2.pt,color=ffffqq,fill=ffffqq,fill opacity=\gelb] (-2.2,0.) -- (2.,0.) -- (2.,4.2) -- (-2.2,4.2) -- cycle;
\fill[line width=2.pt,color=ffffqq,fill=ffffqq,fill opacity=0.10000000149011612] (-1.,2.) -- (1.,2.) -- (0.,3.7320508075688776) -- cycle;
\fill[line width=2.pt,color=ffffqq,fill=ffffqq,fill opacity=0.10000000149011612] (1.,2.) -- (-1.,2.) -- (0.,0.2679491924311226) -- cycle;
\draw [line width=2.pt] (0.,3.7320508075688776)-- (-1.,2.);
\draw [line width=2.pt] (1.,2.)-- (-1.,2.);
\draw [line width=2.pt] (-1.,2.)-- (0.,0.2679491924311226);
\draw [line width=2.pt] (0.,0.2679491924311226)-- (1.,2.);
\draw [rotate around={-60.:(0.5,2.8660254037844513)},line width=2.pt,color=ffffff,fill=ffffff,fill opacity=1.0] (0.5,2.8660254037844513) ellipse (1.4633824013732526cm and 0.1641459454658895cm);
\draw [rotate around={-60.:(0.5,2.866025403784439)},line width=2.pt] (0.5,2.866025403784439) ellipse (1.463382401373216cm and 0.16414594546590086cm);
\begin{scriptsize}
%\draw[color=black] (0.48,2.17) node {$Vieleck1$};
\draw [fill=ududff] (-1.,2.) circle (2.5pt);
\draw[color=black] (-1.56,2.27) node {$\{V_1^1,V_1^2,V_1^3\}$};
\draw [fill=ududff] (1.,2.) circle (2.5pt);
\draw[color=black] (1.459,2.12) node {$\{V_2^1,V_2^2\}$};
\draw[color=black] (-0.1,2.37) node {$F$};
\draw [fill=qqqqff] (0.,3.7320508075688776) circle (2.5pt);
\draw[color=black] (0.14,4.01) node {$\{V_3^1,V_3^2\}$};
\draw[color=black] (0.06,1.47) node {$F'$};
\draw [fill=qqqqff] (0.,0.2679491924311226) circle (2.5pt);
\draw[color=black] (0.76,1.13) node {$\{g\}$};
\draw[color=black] (-0.94,1.13) node {$\{f^1,f^2\}$};
\draw[color=black] (0.36,0.13) node {$\{V_4\}$};
\draw[color=black] (0.21,2.67) node {$\{e_1^1\}$};
\draw[color=black] (0.84,3.13) node {$\{e_1^2\}$};
\draw[color=black] (-0.86,2.97) node {$\{e_2^1,e_2^2\}$};
\end{scriptsize}
%\end{axis}
\end{tikzpicture}
\caption{simpliziale Fläche nach einem Cratercut}
\end{figure}

%-------------------------------------------------------
\item Wende einen $Rip Cut$ $R^{c}_{\{e^1_{2},e^2_{2}\}}$ an, um aus dem Knoten $\{V^1_{3},V^2_{3}\}$ die Knoten $\{V^1_{3}\}$ und $\{V^2_{3}\}$ und aus der Kante $\{e^1_{2},e^2_{2}\}$ die Kanten $\{e^1_{2}\}$ und $\{e^2_{2}\}$ zu erhalten, wobei für $\{e_2^1\}$
\[
\{e_2^1\}<F,
\] 
für $\{e_2^2\}$
\[
\{e^2_2\} \nless F',
\]
 für $\{V_3^1\}$
 \[
\{V_3^1\}<\{e_2^1\},\{V_3^1\}<\{e_1^1\},
 \]
 und für $\{V_3^2\}$
 \[
\{V_3^2\}<\{e_2^2\},\{V_3^2\}<\{e_1^2\} 
 \] gilt.\\
 %------------------------------bild-------------
  \begin{figure}[h]
\definecolor{xdxdff}{rgb}{0.49019607843137253,0.49019607843137253,1.}
\definecolor{ffffff}{rgb}{1.,1.,1.}
\definecolor{qqqqff}{rgb}{0.,0.,1.}
\definecolor{ffffqq}{rgb}{1.,1.,0.}
\begin{tikzpicture}[line cap=round,line join=round,>=triangle 45,x=1.cm,y=1.cm]

x=1.cm,y=1.cm,
axis lines=middle,
xmin=-4.0,
xmax=14.0,
ymin=-3.3,
ymax=5.34,
xtick={-9.0,-8.0,...,14.0},
ytick={-5.0,-4.0,...,6.0},]
\clip(-8.,-3.0) rectangle (4.,5.3);
\fill[line width=2.pt,color=ffffqq,fill=ffffqq,fill opacity=\gelb] (4.,-3.) -- (4.,5.) -- (-4.,5.) -- (-4.,-3.) -- cycle;   
\fill[line width=2.pt,color=ffffff,fill=ffffff,fill opacity=1.0] (-2.,1.) -- (2.,1.) -- (0.,4.464101615137755) -- cycle;
\fill[line width=2.pt,color=ffffqq,fill=ffffqq,fill opacity=0.1] (-2.,1.) -- (0.,-2.44) -- (1.9791273890184695,1.012050807568877) -- cycle;
\fill[line width=2.pt,color=ffffqq,fill=ffffqq,fill opacity=0.550000011920929] (-2.,1.) -- (0.,3.48) -- (2.,1.) -- cycle;
\draw [line width=2.pt] (-2.,1.)-- (2.,1.);
\draw [line width=2.pt] (2.,1.)-- (0.,4.464101615137755);
\draw [line width=2.pt] (0.,4.464101615137755)-- (-2.,1.);
\draw [line width=2.pt] (-2.,1.)-- (0.,-2.44);
\draw [line width=2.pt] (0.,-2.44)-- (1.9791273890184695,1.012050807568877);
\draw [line width=2.pt] (1.9791273890184695,1.012050807568877)-- (-2.,1.);
\draw [line width=2.pt] (-2.,1.)-- (0.,3.48);
\draw [line width=2.pt] (0.,3.48)-- (2.,1.);
\draw [line width=2.pt] (2.,1.)-- (-2.,1.);
\begin{scriptsize}
\draw[color=black] (0.12,0.617) node {$\{e_3^1,e_3^2\}$};
\draw [fill=qqqqff] (-2.,1.) circle (2.5pt);
\draw[color=black] (-3.01,1.22) node {$\{V_1^1,V_1^2,V_1^3\}$};
\draw [fill=qqqqff] (2.,1.) circle (2.5pt);
\draw[color=black] (2.59,1.42) node {$\{V_2^1,V_2^2\}$};
%\draw[color=ffffff] (0.48,2.33) node {$Vieleck1$};
%\draw[color=black] (0.07,1.5) node {$e_3$};
\draw[color=black] (1.37,3.12) node {$\{e_1^2\}$};
\draw[color=black] (-1.27,3.12) node {$\{e_2^2\}$};
\draw [fill=qqqqff] (0.,4.464101615137755) circle (2.5pt);
\draw[color=black] (0.42,4.8) node {$\{V_3^2\}$};
\draw [fill=qqqqff] (0.,-2.44) circle (2.5pt);
\draw[color=black] (0.09,-2.84) node {$\{V_4\}$};
\draw[color=black] (0.1,-0.13) node {F'};
\draw[color=black] (-1.74,-0.71) node {$\{f^1,f^2\}$};
\draw[color=black] (1.42,-0.69) node {$\{g\}$};
\draw [fill=xdxdff] (0.,3.48) circle (2.5pt);
\draw[color=black] (0.,2.8) node {$\{V_3^1\}$};
\draw[color=black] (0.06,2.01) node {F};
\draw[color=black] (-0.95,1.76) node {$\{e_2^1\}$};
\draw[color=black] (0.85,1.76) node {$\{e_1^1\}$};
\end{scriptsize}

\end{tikzpicture}
\caption{simpliziale Fläche nach einem Ripcut}
\end{figure}

%----------------------------------------------

 \item Wende einen $Split Cut$ $ S^{c}_{\{e^1_{3},e^2_{3}\}}$ an, um aus der Kante $\{e^1_{3},e^2_{3}\}$ die Kanten $\{e^1_{3}\}$ und $\{e^2_{3}\}$, aus dem Knoten $\{V_1^1,V_1^2,V_1^3\}$ die Knoten $\{V_1^1\}$ und $\{V_1^2,V_1^3\}$ und aus dem Knoten $\{V_2^1,V_2^2\}$ die Knoten $\{V_2^1\}$ und $\{V_2^2\}$, wobei für $\{e_3^1\}$
\[
\{e_3^1\}<F,
\]
für $\{e_3^2\}$
\[
\{e_3^2\} \nless F,\{e_3^2\} < F',
\]
für $\{V_1^1\}$
\[
\{V_1^1\}<\{e_3^1\},\{V_1^1\}<\{e_2^1\},
\]
für $\{V_1^2,V_1^3\}$
\[
\{V_1^2,V_1^3\}<\{e_2^2\},\{V_1^2,V_1^3\}<\{e_3^2\},
\]
für $\{V_2^1\}$
\[
\{V_2^1\}<\{e_1^1\},\{V_2^1\}<\{e_3^1\}
\]
und für $\{V_2^2\}$
\[
\{V_2^2\}<\{e^2_1\},\{V_2^2\}<\{e_3^2\}
\] gilt.

\end{enumerate}
Durch dieses Anwenden der Operatoren auf $X$ erhält man eine simpliziale Fläche $(Y,\prec)\in \mathcal{M}(X)$ mit den zwei Zusammenhangskomponenten $X^{1}$ und $X^{2}$, wobei $X^{1}=\{F,\{V^1_{1}\},\{V^1_{2}\},\{V^1_{3}\},\{e^1_{1}\},\{e^1_{2}\},\{e^1_ {3}\}\}$ mit den Inzidenzen 
\begin{itemize}
 \item $\{e_{i}^1\} < F$ für alle $i \in \{1,2,3\}$,
 \item $\{V_{i}^1\}<\{e_{j}^1\}$ für alle $i \in \{1,2,3\}$ und $j \in \{1,2,3\} \setminus\{i\}$ ,
 \item $\{V_{i}^1\} < F$ für alle $i \in \{1,2,3\}$,
\end{itemize}
das Dreieck und $X^{2}$, die Fläche mit fehlender Fläche $F$, beschreibt.\\
Es entstehen also Randkanten $\{e^1_{i}\},\{e^2_{i}\}$ für $i \in \{1,2,3\}$, wobei die Kanten $\{e^2_{i}\}$ zu der Zusammenhangskomponente $X^2$ gehören. \\
Man bezeichnet die simpliziale Fläche, die nach der Anwendung der Prozedur $P^1$ ensteht, mit \emph{$P_F^1(X):=(S^c_{\{e_3^1,e_3^2\}}\circ R^c_{\{e_2^1,e_2^2\}}\circ C^c_{\{e_1^1,e_1^2\}})(X)$}.
%------------------------bild------------------------
\begin{figure}[h]
\definecolor{qqqqff}{rgb}{0.,0.,1.}
\definecolor{ffffff}{rgb}{1.,1.,1.}
\definecolor{ududff}{rgb}{0.30196078431372547,0.30196078431372547,1.}
\definecolor{ffffqq}{rgb}{1.,1.,0.}
\begin{tikzpicture}[line cap=round,line join=round,>=triangle 45,x=1.cm,y=1.cm]
x=1.0cm,y=1.0cm,
axis lines=middle,
ymajorgrids=true,
xmajorgrids=true,
xmin=-3.5,
xmax=10.0,
ymin=-3.0,
ymax=5.2,
xtick={-9.0,-8.0,...,14.0},
ytick={-5.0,-4.0,...,6.0},]
\clip(-4.,-3.3) rectangle (14.,5.34);
%----------
%\fill[line width=2.pt,color=ffffqq,fill=ffffqq,fill opacity=0.550000011920929] (-4.,-3.) -- (-4.,5.) -- (-3.,5.) -- (-3.,-3.) -- cycle;

%--------------
\fill[line width=2.pt,color=ffffqq,fill=ffffqq,fill opacity=\gelb] (-3.5,-3.) -- (-3.5,5.) -- (3.,5.) -- (3.,-3.) -- cycle;
\fill[line width=2.pt,color=ffffff,fill=ffffff,fill opacity=1.0] (-2.,1.) -- (2.,1.) -- (0.,4.464101615137755) -- cycle;
\fill[line width=2.pt,color=ffffqq,fill=ffffqq,fill opacity=0.20000000298023224] (-2.,1.) -- (0.,-2.44) -- (1.9791273890184695,1.012050807568877) -- cycle;
\fill[line width=2.pt,color=ffffqq,fill=ffffqq,fill opacity=\gelb] (5.,1.) -- (9.,1.) -- (7.,4.464101615137755) -- cycle;
%\draw [line width=2.pt,color=ffffqq] (-3.5,-3.)-- (-3.5,5.);
%\draw [line width=2.pt,color=ffffqq] (-3.,5.)-- (3.,5.);
%\draw [line width=2.pt,color=ffffqq] (3.,5.)-- (3.,-3.);
%\draw [line width=2.pt,color=ffffqq] (3.,-3.)-- (-3.,-3.);
\draw [line width=2.pt] (-2.,1.)-- (2.,1.);
\draw [line width=2.pt] (2.,1.)-- (0.,4.464101615137755);
\draw [line width=2.pt] (0.,4.464101615137755)-- (-2.,1.);
\draw [line width=2.pt] (-2.,1.)-- (0.,-2.44);
\draw [line width=2.pt] (0.,-2.44)-- (1.9791273890184695,1.012050807568877);
\draw [line width=2.pt] (1.9791273890184695,1.012050807568877)-- (-2.,1.);
\draw [line width=2.pt] (5.,1.)-- (9.,1.);
\draw [line width=2.pt] (9.,1.)-- (7.,4.464101615137755);
\draw [line width=2.pt] (7.,4.464101615137755)-- (5.,1.);
\begin{scriptsize}
\draw[color=black] (0,0.7) node {$\{e_3^2\}$};
%\draw[color=black] (7,1.4) node {$e_3^1$};

\draw[color=black] (1.28,2.93) node {$\{e_1^2\}$};
%\draw[color=black] (8.28,2.93) node {$e_1^2$};

\draw[color=black] (-1.28,2.93) node {$\{e_2^2\}$};
\draw[color=black] (5.7,2.93) node {$\{e_2^1\}$};
\draw[color=black] (8.3,2.93) node {$\{e_2^1\}$};

\draw[color=black] (-1.88,-0.6) node {$\{f^1,f^2\}$};
\draw[color=black] (1.68,-0.6) node {$\{g\}$};
\draw [fill=qqqqff] (-2.,1.) circle (2.5pt);
\draw[color=black] (-2.551,1.42) node {$\{V_1^2,V_1^3\}$};
\draw [fill=qqqqff] (2.,1.) circle (2.5pt);
\draw[color=black] (2.29,1.42) node {$\{V_2^2\}$};
\draw[color=ffffff] (0.48,2.33) node {$Vieleck1$};
\draw [fill=qqqqff] (0.,4.464101615137755) circle (2.5pt);
\draw[color=black] (0.43,4.7) node {$\{V_3^2\}$};
\draw [fill=qqqqff] (0.,-2.44) circle (2.5pt);
\draw[color=black] (0.14,-2.75) node {$\{V_4\}$};
\draw[color=black] (0.1,0.03) node {F'};
\draw [fill=qqqqff] (5.,1.) circle (2.5pt);
\draw[color=black] (5.08,0.63) node {$\{V_1^1\}$};
\draw [fill=qqqqff] (9.,1.) circle (2.5pt);
\draw[color=black] (9.1,0.67) node {$\{V_2^1\}$};
\draw[color=black] (7.06,2.33) node {$F$};
\draw[color=black] (7.06,0.70) node {$\{e_3^1\}$};
\draw [fill=qqqqff] (7.,4.464101615137755) circle (2.5pt);
\draw[color=black] (7.14,4.83) node {$\{V_3^1\}$};
\end{scriptsize}

\end{tikzpicture}
\caption{simpliziale Fläche nach einem Splitcut}
\end{figure}

%-------------------------------------------------------

\subsection{Prozedur $P^2$}
 Nun soll das \emph{Loch an der Stelle F} verschoben werden. 
\begin{enumerate}[(i)]
\item Wende einen $Rip Cut$ $R^{c}_{\{f^1,f^2\}}$ an, um aus der Kante $\{f^1,f^2\}$ die Kanten $\{f^1\}$ und $\{f^2\}$ und aus dem Knoten $\{V_1^2,V_1^3\}$ die Knoten $\{V_1^2\}$ und $\{V_1^3\}$ zu erhalten, wobei für $\{f^1\}$
\[
\{f^1\} \nless F,\{f^1\} \nless F',
\]
für $\{f^2\}$
\[
\{f^2\}< F',
\]
für $\{V_1^2\}$
\[
\{V_1^2\}<\{e_3^2\},\{V_1^2\}<\{f^2\}
\]
und für $\{V_1^3\}$
\[
\{V_1^3\}<\{f^1\},\{V_1^3\}<\{e_2^2\}.
\]
gilt.
%----------------bild------------------------
\begin{figure}[h]
\definecolor{ududff}{rgb}{0.30196078431372547,0.30196078431372547,1.}
\definecolor{ffffff}{rgb}{1.,1.,1.}
\definecolor{sqsqsq}{rgb}{0.12549019607843137,0.12549019607843137,0.12549019607843137}
\definecolor{ffffqq}{rgb}{1.,1.,0.}
\definecolor{qqqqff}{rgb}{0.,0.,1.}
\begin{tikzpicture}[line cap=round,line join=round,>=triangle 45,x=1.0cm,y=1.0cm]

x=1.0cm,y=1.0cm,
axis lines=middle,
ymajorgrids=true,
xmajorgrids=true,
xmin=-4.5,
xmax=10.0,
ymin=-5.0,
ymax=5.2,
xtick={-9.0,-8.0,...,14.0},
ytick={-5.0,-4.0,...,6.0},]
\clip(-3.664966779911168,-4.336419420914822) rectangle (10.164271214115164,5.25368189432285);
\fill[line width=2.pt,color=ffffqq,fill=ffffqq,fill opacity=0.10000000149011612] (-2.,0.) -- (0.,-3.481320628255737) -- (2.014912102788271,-0.008609506558991509) -- cycle;
\fill[line width=2.pt,color=ffffqq,fill=ffffqq,fill opacity=0.5] (-3.06633318691027,3.9892123475876073) -- (-3.0412642843067688,-3.8824230699117557) -- (2.6995144118949965,-3.9826986803257602) -- (2.6744455092914956,3.9892123475876073) -- cycle;
\fill[line width=2.pt,color=ffffff,fill=ffffff,fill opacity=1.0] (-2.,0.) -- (2.,0.) -- (0.,3.4641016151377553) -- cycle;
\fill[line width=2.pt,color=ffffff,fill=ffffff,fill opacity=1.0] (0.,-3.481320628255737) -- (-2.,0.) -- (-1.010683173423175,0.028325736234424664) -- cycle;
\fill[line width=2.pt,color=ffffqq,fill=ffffqq,fill opacity=0.44999998807907104] (4.,0.) -- (8.,0.) -- (6.,3.4641016151377553) -- cycle;
\draw [line width=2.pt,color=sqsqsq] (-2.,0.)-- (0.,-3.481320628255737);
\draw [line width=2.pt] (0.,-3.481320628255737)-- (2.014912102788271,-0.008609506558991509);
\draw [line width=2.pt,color=sqsqsq] (2.014912102788271,-0.008609506558991509)-- (-1.,0.);
\draw [line width=2.pt,color=ffffff]  (-3.06633318691027,3.9892123475876073)-- (-3.0412642843067688,-3.8824230699117557);
\draw [line width=2.pt,color=ffffff] (2.6995144118949965,-3.9826986803257602)-- (2.6744455092914956,3.9892123475876073);
\draw [line width=2.pt,color=sqsqsq] (2.,0.)-- (0.,3.4641016151377553);
\draw [line width=2.pt] (0.,3.4641016151377553)-- (-2.,0.);
\draw [line width=2.pt,color=sqsqsq] (0.,-3.481320628255737)-- (-2.,0.);
\draw [line width=2.pt,color=ffffff] (-1.5,0.)-- (-1.010683173423175,0.028325736234424664);
\draw [line width=2.pt,color=sqsqsq] (-1.010683173423175,0.028325736234424664)-- (0.,-3.481320628255737);
\draw [line width=2.pt,color=ffffff] (-2.,0.)-- (-1.010683173423175,0.028325736234424664);
\draw [line width=2.pt,color=sqsqsq] (4.,0.)-- (8.,0.);
\draw [line width=2.pt,color=sqsqsq] (8.,0.)-- (6.,3.4641016151377553);
\draw [line width=2.pt,color=sqsqsq] (6.,3.4641016151377553)-- (4.,0.);
\begin{scriptsize}
\draw [fill=qqqqff] (-2.,0.) circle (2.5pt);
\draw[color=black] (-2.2815938522964825,0.30408366487293736) node {$\{V_1^3\}$};
\draw [fill=qqqqff] (-0.,-3.481320628255737) circle (2.5pt);
\draw[color=black] (0.4555720184676383,-3.6314584334627392) node {$\{V_4\}$};
\draw[color=black] (0.5937265932008994,-0.9368270140003698) node {$F'$};
\draw[color=black] (1.4210003791164376,-1.7139629947089057) node {$\{g\}$};
\draw[color=sqsqsq] (0.3555720184676383,-0.27222548459358455) node {$\{e_3^2\}$};
\draw [fill=qqqqff] (2.014912102788271,-0.108609506558991509) circle (2.5pt);
\draw[color=black] (2.260808616333726,0.3) node {$\{V_2^2\}$};
\draw[color=ffffff] (-3.2295566642470322,0.32915256747643856) node {$\{f^1\}$};
\draw[color=ffffff] (3.21342691526677,0.30408366487293736) node {$h_1$};
\draw[color=ffffff] (0.5937265932008994,1.3695120255217366) node {$Vieleck1$};
\draw[color=sqsqsq] (1.258879343435694,2.209320262739025) node {$\{e_1^2\}$};
\draw[color=black] (-1.199251163777443,2.284526970549529) node {$\{e_2^2\}$};
\draw [fill=qqqqff] (0.,3.4641016151377553) circle (2.5pt);
\draw[color=black] (0.23022750545013249,3.7892123475876073) node {$\{V_3^2\}$};
\draw [fill=ududff] (-1.010683173423175,0.028325736234424664) circle (2.5pt);
\draw[color=black] (-0.7725285986899139,0.3547726909079489) node {$\{V_1^2\}$};
\draw[color=sqsqsq] (-1.0352008551986667,-1.0120337218108733) node {$\{f^2\}$};
\draw[color=ffffff] (-1.4368545176826946,-0.15969103329183398) node {$d$};
\draw[color=sqsqsq] (-1.5363032968546852,-1.5635495790878988) node {$\{f^1\}$};
\draw[color=ffffff] (-1.4368545176826946,-0.15969103329183398) node {$l$};
\draw [fill=qqqqff] (4.,0.) circle (2.5pt);
\draw[color=black] (4.090838506389312,-0.3477078028180926) node {$\{V_1^1\}$};
\draw [fill=qqqqff] (8.,0.) circle (2.5pt);
\draw[color=black] (8.352276338570503,-0.2725010950075892) node {$\{V_2^1\}$};
\draw[color=black] (6.108885165971154,1.4196498307287388) node {$F$};
\draw[color=sqsqsq] (5.96626325083409,-0.22208767938658227) node {$\{e_3^1\}$};
\draw[color=sqsqsq] (7.450347065672471,2.209320262739025) node {$\{e_1^1\}$};
\draw[color=sqsqsq] (4.842905584494346,2.209320262739025) node {$\{e_2^1\}$};
\draw [fill=qqqqff] (6.,3.4641016151377553) circle (2.5pt);
\draw[color=black] (6.321970838100914,3.9892123475876073) node {$\{V_3^1\}$};
\end{scriptsize}

\end{tikzpicture}
\caption{simpliziale Fläche nach einem Ripcut}
\end{figure}
%----------------------------------------------
\item Wende einen $RipMender$ $R^{m}_{\{e^2_{1}\},\{e^2_{3}\}}$ an, um die Kanten $\{e^2_{1}\}$ und $\{e^2_{3}\}$ zu einer Kante $\{e^2_{1},e^2_{3}\}$ und die Knoten $\{V_1^2\}$ und $\{V_3^2\}$ zu dem Knoten $\{V_1^2,V_3^2\}$ zusammenzuführen,wobei für $\{e_3^2,e_1^2\}$
\[
\{e_3^2,e_1^2\}<F'
\]
und für $\{V_1^2,V_3^2\}$
\[
\{V_1^2,V_3^2\}<\{e_2^2\},\{V_1^2,V_3^2\}<\{e^2_{1},e^2_{3}\}
\] gilt.
\end{enumerate}
%\centerline{$\textcolor{red}{Bild4}$}
%-----------------------bild---------------------------
%\begin{figure}[h]

\begin{comment}
\definecolor{ffffff}{rgb}{1.,1.,1.}
\definecolor{qqqqff}{rgb}{0.,0.,1.}
\definecolor{ffffqq}{rgb}{1.,1.,0.}
\begin{tikzpicture}[line cap=round,line join=round,>=triangle 45,x=1.5cm,y=1.5cm]
\begin{axis}[
x=1.0cm,y=1.0cm,
axis lines=middle,
ymajorgrids=true,
xmajorgrids=true,
xmin=-3.583376623376623,
xmax=16.330043290043285,
ymin=-4.489177489177493,
ymax=5.588744588744593,
xtick={-3.0,-2.0,...,16.0},
ytick={-4.0,-3.0,...,5.0},]
\clip(-3.583376623376623,-4.489177489177493) rectangle (16.330043290043285,5.588744588744593);
\fill[line width=2.pt,color=ffffqq,fill=ffffqq,fill opacity=0.5] (-2.,0.) -- (2.,0.) -- (2.,4.) -- (-2.,4.) -- cycle;
\fill[line width=2.pt,color=ffffff,fill=ffffff,fill opacity=1.0] (0.,1.) -- (0.,3.) -- (-1.7320508075688776,2.) -- cycle;
\fill[line width=2.pt,color=ffffqq,fill=ffffqq,fill opacity=0.4000000059604645] (0.,3.) -- (0.,1.) -- (1.7320508075688776,2.) -- cycle;
\draw [line width=2.pt] (0.,1.)-- (0.,3.);
\draw [line width=2.pt] (0.,3.)-- (-1.7320508075688776,2.);
\draw [line width=2.pt] (-1.7320508075688776,2.)-- (0.,1.);
\draw [line width=2.pt] (0.,3.)-- (0.,1.);
\draw [line width=2.pt] (0.,1.)-- (1.7320508075688776,2.);
\draw [line width=2.pt] (1.7320508075688776,2.)-- (0.,3.);
\begin{scriptsize}
\draw[color=ffffqq] (0.3993073593073588,2.151515151515153) node {$Vieleck1$};
\draw [fill=qqqqff] (0.,1.) circle (2.5pt);
\draw[color=qqqqff] (0.12225108225108178,1.3203463203463215) node {$E$};
\draw [fill=qqqqff] (0.,3.) circle (2.5pt);
\draw[color=qqqqff] (0.12225108225108178,3.3116883116883145) node {$F$};
\draw[color=ffffff] (-0.17212121212121256,2.151515151515153) node {$Vieleck2$};
\draw [fill=qqqqff] (-1.7320508075688776,2.) circle (2.5pt);
\draw[color=qqqqff] (-1.6093506493506495,2.3246753246753267) node {$G$};
\draw[color=ffffqq] (0.9880519480519474,2.151515151515153) node {$Vieleck3$};
\draw [fill=qqqqff] (1.7320508075688776,2.) circle (2.5pt);
\draw[color=qqqqff] (1.853852813852813,2.3246753246753267) node {$H$};
\end{scriptsize}
\end{axis}
\end{tikzpicture}
\end{comment}

Für eine nach der Prozedur $P^1$ entstandene simpliziale Fläche $Z=P^1_F(X)$  bezeichnet man die simpliziale Fläche, die nach Anwenden der zweiten Prozedur entsteht mit \emph{$P^2_f(Z)$}$:=(R^m_{\{e_1^2\},\{e_3^2\}}\circ R^c_{\{f^1\},\{f^2\}})(Z)$. Das heißt $P^2_f(P^1_F(X))$ ist die Fläche, die nach Anwenden der beiden Prozeduren $P^1$ und $P^2$ auf die gegebene simpliziale Fläche $X$ entsteht.
%\newpage
%----------------------------------------------------------
%-----------------------bild----------------------------------
\begin{figure}[h]
\definecolor{ffffff}{rgb}{1.,1.,1.}
\definecolor{qqqqff}{rgb}{0.,0.,1.}
\definecolor{ffffqq}{rgb}{1.,1.,0.}
\begin{tikzpicture}[line cap=round,line join=round,>=triangle 45,x=1.4cm,y=1.4cm]
%\begin{axis}[
x=1.0cm,y=1.0cm,
axis lines=middle,
ymajorgrids=true,
xmajorgrids=true,
xmin=-3.583376623376623,
xmax=16.330043290043285,
ymin=-4.489177489177493,
ymax=5.588744588744593,
xtick={-3.0,-2.0,...,16.0},
ytick={-4.0,-3.0,...,5.0},]
\clip(-3.583376623376623,-0.289177489177493) rectangle (16.330043290043285,4.588744588744593);
\fill[line width=2.pt,color=ffffqq,fill=ffffqq,fill opacity=\gelb] (-2.2,0.) -- (2.2,0.) -- (2.2,4.) -- (-2.2,4.) -- cycle;
\fill[line width=2.pt,color=white,fill=ffffff,fill opacity=1.0] (0.,1.) -- (0.,3.) -- (-1.7320508075688776,2.) -- cycle;
\fill[line width=2.pt,color=ffffqq,fill=ffffqq,fill opacity=0.] (0.,3.) -- (0.,1.) -- (1.7320508075688776,2.) -- cycle;
\fill[line width=2.pt,color=ffffqq,fill=ffffqq,fill opacity=\gelb] (3.,1.) -- (5.594112554112552,0.9696969696969713) -- (4.323299471110349,3.231415856986091) -- cycle;
\draw [line width=2.pt] (0.,1.)-- (0.,3.);
\draw [line width=2.pt] (0.,3.)-- (-1.7320508075688776,2.);
\draw [line width=2.pt] (-1.7320508075688776,2.)-- (0.,1.);
\draw [line width=2.pt] (0.,3.)-- (0.,1.);
\draw [line width=2.pt] (0.,1.)-- (1.7320508075688776,2.);
\draw [line width=2.pt] (1.7320508075688776,2.)-- (0.,3.);
\draw [line width=2.pt] (3.,1.)-- (5.594112554112552,0.9696969696969713);
\draw [line width=2.pt] (5.594112554112552,0.9696969696969713)-- (4.323299471110349,3.231415856986091);
\draw [line width=2.pt] (4.323299471110349,3.231415856986091)-- (3.,1.);
\begin{scriptsize}
\draw[color=black] (0.6993073593073588,2.051515151515153) node {$F'$};
\draw [fill=qqqqff] (0.,1.) circle (2.5pt);
\draw[color=black] (0.12225108225108178,0.7203463203463215) node {$\{V_4\}$};
\draw [fill=qqqqff] (0.,3.) circle (2.5pt);
\draw[color=black] (0.10225108225108178,3.2116883116883145) node {$\{V_1^2,V_3^2\}$};
%\draw[color=ffffff] (-0.17212121212121256,2.151515151515153) node {$Vieleck2$};
\draw [fill=qqqqff] (-1.7320508075688776,2.) circle (2.5pt);
\draw[color=black] (-1.8593506493506495,2.2246753246753267) node {$\{V_1^3\}$};
%\draw[color=ffffqq] (0.9880519480519474,2.151515151515153) node {$Vieleck3$};
\draw [fill=qqqqff] (1.7320508075688776,2.) circle (2.5pt);
\draw[color=black] (1.893852813852813,2.2246753246753267) node {$\{V_2^2\}$};
\draw [fill=qqqqff] (3.,1.) circle (2.5pt);
\draw[color=black] (2.717922077922077,1.1246753246753267) node {$\{V_1^1\}$};
\draw [fill=qqqqff] (5.594112554112552,0.9696969696969713) circle (2.5pt);
\draw[color=black] (5.9153246753246725,1.1246753246753267) node {$\{V_2^1\}$};
\draw[color=black] (4.31099567099567,1.8961038961038983) node {$F$};
\draw [fill=qqqqff] (4.323299471110349,3.231415856986091) circle (2.5pt);
\draw[color=black] (4.45125541125541,3.5584415584415625) node {$\{V_3^1\}$};
\draw[color=black] (4.31125541125541,0.745584415584415625) node {$\{e_3^1\}$};
\draw[color=black] (3.31099567099567,2.1961038961038983) node {$\{e_2^1\}$};
\draw[color=black] (5.31099567099567,2.1961038961038983) node {$\{e_1^1\}$};
\draw[color=black] (-0.26099567099567,2.01961038961038983) node {$\{f^2\}$};
\draw[color=black] (-1.01099567099567,1.2961038961038983) node {$\{f^1\}$};
\draw[color=black] (-1.01099567099567,2.6961038961038983) node {$\{e_2^2\}$};
\draw[color=black] (1.01099567099567,1.2961038961038983) node {$\{g\}$};
\draw[color=black] (1.11099567099567,2.6961038961038983) node {$\{e_3^2,e_1^2\}$};
\end{scriptsize}

%\end{axis}
\end{tikzpicture}
\caption{simpliziale Fläche nach Anwenden eines Ripmernders}
\end{figure}

%--------------------------------------------------------------
\subsection{Prozedur $P^3$}
 Zuletzt müssen nun die zwei Zusammenhangskomponenten mithilfe der folgenden Operationen wieder zusammengesetzt werden:
\begin{enumerate}[(i)]
\item Wende einen $SplitMender$ $S^{m}_{(\{V^1_{1}\},\{e^1_{3}\}),(\{V_{4}\},\{f^2\})}$ an, um die Kanten $\{f^2\}$ und $\{e_3^1\}$ zu der Kante $\{e_3^1,f^2\}$ zusammenzuführen und um so ebenfalls aus den Knoten $\{V_1^1\}$ und $\{V_4\}$ den Knoten $\{V_1^1,V_4\}$ und aus den Knoten $\{V_1^2,V_3^2\}$ und $\{V_2^1\}$ den Knoten $\{V_2^1,\{V_1^2,V_3^2\}\}$ hervorzubringen, wobei für $\{e_3^1,f^2\}$
\[
\{e_3^1,f^2\}<F,\{e_3^1,f^2\}<F',
\]
für $\{V_1^1,V_4\}$
\[
\{V_1^1,V_4\}<\{f^1\},\{V_1^1,V_4\}<\{e_2^1\},\{V_1^1,V_4\}<\{e_3^1,f^2\},\{V_1^1,V_4\}<\{g\}
\]
und für $\{V_1^2,V_2^1,V_3^2\}$
\begin{align*}
\{V_2^1,\{V_1^2,V_3^2\}\}<\{e_2^2\},&\{V_2^1,\{V_1^2,V_3^2\}\}<\{e_1^1\},\{V_2^1,\{V_1^2,V_3^2\}\}<\{e_3^1,f^2\},\\
&\{V_2^1,\{V_1^2,V_3^2\}\}<\{e_3^2,e_1^2\}
\end{align*}
%---------------bild----------------------------
\newpage
\begin{figure}[h]
%\begin{comment}
\definecolor{ududff}{rgb}{0.30196078431372547,0.30196078431372547,1.}
\definecolor{ffffff}{rgb}{1.,1.,1.}
\definecolor{qqqqff}{rgb}{0.,0.,1.}
\definecolor{ffffqq}{rgb}{1.,1.,0.}
\begin{tikzpicture}[line cap=round,line join=round,>=triangle 45,x=1.4cm,y=1.4cm]
%\begin{axis}[
x=1.0cm,y=1.0cm,
axis lines=middle,
ymajorgrids=true,
xmajorgrids=true,
xmin=-4.3,
xmax=7.0600000000000005,
ymin=-2.46,
ymax=6.3,
xtick={-4.0,-3.0,...,7.0},
ytick={-2.0,-1.0,...,6.0},]
\clip(-5.3,-0.) rectangle (7.06,4.);
\fill[line width=2.pt,color=ffffqq,fill=ffffqq,fill opacity=\gelb] (-2.,0.) -- (2.1,0.) -- (2.1,4.) -- (-2.,4.) -- cycle;
\fill[line width=2.pt,color=ffffqq,fill=ffffqq,fill opacity=0.1499999940395355] (0.,3.) -- (0.,1.) -- (1.7320508075688776,2.) -- cycle;
\fill[line width=2.pt,color=ffffff,fill=ffffff,fill opacity=1.0] (0.,1.) -- (0.,3.) -- (-1.7320508075688776,2.) -- cycle;
\fill[line width=2.pt,color=ffffqq,fill=ffffqq,fill opacity=\gelb] (0.,3.) -- (-1.,2.) -- (0.,1.) -- cycle;
\draw [line width=2.pt] (0.,3.)-- (0.,1.);
\draw [line width=2.pt] (0.,1.)-- (1.7320508075688776,2.);
\draw [line width=2.pt] (1.7320508075688776,2.)-- (0.,3.);
\draw [line width=2.pt] (0.,1.)-- (0.,3.);
\draw [line width=2.pt] (0.,3.)-- (-1.7320508075688776,2.);
\draw [line width=2.pt] (-1.7320508075688776,2.)-- (0.,1.);
\draw [line width=2.pt] (0.,3.)-- (-1.,2.);
\draw [line width=2.pt] (-1.,2.)-- (0.,1.);
\begin{scriptsize}
\draw[color=black] (-1.04,1.37) node {$\{f^1\}$};
\draw [fill=qqqqff] (0.,3.) circle (2.5pt);
\draw[color=black] (0.14,3.17) node {$\{V_2^1,\{V_1^2,V_3^2\}\}$};
\draw[color=black] (0.1,0.77) node {$\{V_1^1,V_4\}$};
\draw [fill=qqqqff] (0.,1.) circle (2.5pt);
\draw[color=black] (0.64,1.97) node {$F'$};
\draw[color=black] (0.34,2.27) node {$\{e_3^1,f^2\}$};
\draw[color=black] (1.04,2.71) node {$\{e_3^2,e_1^2\}$};
\draw[color=black] (0.94,1.31) node {$\{g\}$};
\draw [fill=qqqqff] (1.7320508075688776,2.) circle (2.5pt);
\draw[color=black] (1.88,2.17) node {$\{V_2^2\}$};
\draw[color=black] (-0.3,2.01) node {$F$};
\draw [fill=qqqqff] (-1.7320508075688776,2.) circle (2.5pt);
\draw[color=black] (-1.8,2.22) node {$\{V_1^3\}$};
\draw [fill=ududff] (-1.,2.) circle (2.5pt);
\draw[color=black] (-1.26,2.0) node {$\{V_3^1\}$};
\draw[color=black] (-1.03,2.66) node {$\{e_2^2\}$};
\draw[color=black] (-0.33,2.37) node {$\{e_1^1\}$};
\draw[color=black] (-0.33,1.62) node {$\{e_2^1\}$};
\end{scriptsize}
%\end{axis}
\end{tikzpicture}
%\end{comment}
\caption{simpliziale Fläche nach Anwenden eines Splitmenders}
\end{figure}
%------------------------------------------------
\item Wende einen $RipMender$ $R^m_{\{e^1_{2}\},\{f^1\}}$ an, um die Kanten $\{e^1_{2}\}$ und $\{f^1\}$ zu der Kante $\{e^1_{2}\},\{f^1\}$ und die Knoten $\{ V_1^3\}$ und $\{V_3^1\}$ zu dem Knoten $\{V_1^3,V_3^1\}$ zusammenzuführen, wobei für $\{e_2^1,f^1\}$
\[
\{e_2^1,f^1\}<F
\]
und für $\{V_1^3,V_3^1\}$
\[
\{V_1^3,V_3^1\}<\{e_1^1\},\{V_1^3,V_3^1\}<\{e_2^2\},\{V_1^3,V_3^1\}<\{e_2^1,f^1\}
\]
gilt.\\
%-------------------------bild-------------------------
\begin{figure}[h]
\definecolor{ududff}{rgb}{0.30196078431372547,0.30196078431372547,1.}
\definecolor{ffffff}{rgb}{1.,1.,1.}
\definecolor{qqqqff}{rgb}{0.,0.,1.}
\definecolor{ffffqq}{rgb}{1.,1.,0.}
\begin{tikzpicture}[line cap=round,line join=round,>=triangle 45,x=1.5cm,y=1.5cm]
%\begin{axis}[
x=1.5cm,y=1.5cm,
axis lines=middle,
ymajorgrids=true,
xmajorgrids=true,
xmin=-4.3,
xmax=7.0600000000000005,
ymin=-2.46,
ymax=6.3,
xtick={-4.0,-3.0,...,7.0},
ytick={-2.0,-1.0,...,6.0},]
\clip(-5.3,-0.46) rectangle (7.06,4.3);
\fill[line width=2.pt,color=ffffqq,fill=ffffqq,fill opacity=\gelb] (-2.5,0.) -- (2.2,0.) -- (2.2,4.) -- (-2.5,4.) -- cycle;
\fill[line width=2.pt,color=ffffqq,fill=ffffqq,fill opacity=0.] (0.,3.) -- (0.,1.) -- (1.7320508075688776,2.) -- cycle;
\fill[line width=2.pt,color=ffffff,fill=ffffff,fill opacity=1.0] (0.,1.) -- (0.,3.) -- (-1.7320508075688776,2.) -- cycle;
\fill[line width=2.pt,color=ffffqq,fill=ffffqq,fill opacity=0.5] (0.,3.) -- (-1.74,2.) -- (0.,1.) -- cycle;
\draw [line width=2.pt] (0.,3.)-- (0.,1.);
\draw [line width=2.pt] (0.,1.)-- (1.7320508075688776,2.);
\draw [line width=2.pt] (1.7320508075688776,2.)-- (0.,3.);
\draw [line width=2.pt] (0.,1.)-- (0.,3.);
\draw [line width=2.pt] (-1.7320508075688776,2.)-- (0.,1.);
\draw [line width=2.pt] (-1.74,2.)-- (0.,1.);
\draw [rotate around={29.886526940424037:(-0.87,2.5)},line width=2.pt,color=ffffff,fill=ffffff,fill opacity=1.0] (-0.87,2.5) ellipse (1.41989009757162764cm and 0.22745816720870826cm);
\draw [rotate around={29.886526940424037:(-0.87,2.5)},line width=2.pt] (-0.87,2.5) ellipse (1.41989009757163032cm and 0.2274581672087103cm);
\begin{scriptsize}
\draw[color=black] (-0.5,1.9) node {$F$};
\draw [fill=qqqqff] (0.,3.) circle (2.5pt);
\draw[color=black] (0.14,3.23) node {$\{V_2^1,\{V_1^2,V_3^2\}\}$};
\draw [fill=qqqqff] (0.,1.) circle (2.5pt);
\draw[color=black] (0.14,0.75) node {$\{V_1^1,V_4\}$};
\draw[color=black] (1.04,2.75) node {$\{e_3^2,e_1^2\}$};
\draw [fill=qqqqff] (1.7320508075688776,2.) circle (2.5pt);
\draw[color=black] (1.88,2.22) node {$\{V_2^2\}$};
\draw[color=black] (0.8,2.) node {$F'$};
\draw[color=black] (0.4,2.32) node {$\{e_3^1,f^2\}$};
\draw [fill=qqqqff] (-1.7320508075688776,2.) circle (2.5pt);
\draw[color=black] (-2.1,2.22) node {$\{V_1^3,V_3^1\}$};
\draw [fill=qqqqff] (-1.74,2.) circle (2.5pt);
\draw[color=black] (-1.0,2.87) node {$\{e_2^2\}$};
\draw[color=black] (-1.0,1.26) node {$\{e_2^1,f^1\}$};
%\draw[color=black] (-1.16,2.25) node {$c$};
\draw[color=black] (-0.65,2.2) node {$\{e^1_1\}$};
\draw[color=black] (0.94,1.31) node {$\{g\}$};
\end{scriptsize}
%\end{axis}
\end{tikzpicture}
\caption{simpliziale Fläche nach Anwenden eines Ripmenders}
\end{figure}
%-------------------------------------------------------

\item Wende einen $Crater Mender$ $R^{m}_{\{e_1^1\},\{e^2_{2}\}}$ an, um die Kanten $\{e_1^1\}$ und $\{e^2_{2}\}$ zu einer Kante $\{e_1^1,e_2^2\}$ zu vereinen, wobei für $\{e_1^1,e_2^2\}$
\[
\{e_1^1,e_2^2\}<F,\{V_1^3,V_3^1\}<\{e_1^1,e_2^2\},\{V_2^1,\{V_1^2,V_3^2\}\}<\{e_1^1,e_2^2\}
\]
gilt.
\end{enumerate}
Für eine durch die Prozedur $P^1$ oder $P^2$ entstandene simpliziale Fläche $Z$ bezeichnet man die simpliziale Fläche, die aus der Prozedur $P^3$ hervorgeht, mit \emph{$P^3_F(Z)$}$:=(S^m_{(\{V_1^1\},\{e_3^1\}),(\{V_4\},\{f^2\})} \circ R^m_{\{e_2^1\},\{f^1\}} \circ C^m_{\{e_1^1\},\{e_2^2\}})(Z)$. \\
Somit ist $P^3_F(P^2_f(P^1_F(X)))$ die simpliziale Fläche, die aus $X$ nach den drei Prozeduren hervorgeht.
%\end{enumerate}
%\centerline{$\textcolor{red}{Bild5}$}
%----------------------------bild------------------
\begin{comment}
\definecolor{uuuuuu}{rgb}{0.26666666666666666,0.26666666666666666,0.26666666666666666}
\definecolor{ududff}{rgb}{0.30196078431372547,0.30196078431372547,1.}
\definecolor{ffffqq}{rgb}{1.,1.,0.}
\begin{tikzpicture}[line cap=round,line join=round,>=triangle 45,x=1.5cm,y=1.5cm]
%\begin{axis}[
x=1.5cm,y=1.5cm,
axis lines=middle,
ymajorgrids=true,
xmajorgrids=true,
xmin=-4.3,
xmax=7.0600000000000005,
ymin=-2.46,
ymax=6.3,
xtick={-4.0,-3.0,...,7.0},
ytick={-2.0,-1.0,...,6.0},]
\clip(-4.3,-0.46) rectangle (3.06,4.3);
\fill[line width=2.pt,color=ffffqq,fill=ffffqq,fill opacity=\gelb] (-2.5,-0.1) -- (2.4,-0.1) -- (2.4,4.1) -- (-2.5,4.1) -- cycle;
%\fill[line width=2.pt,color=ffffqq,fill=ffffqq,fill opacity=\gelb] (-1.,2.) -- (1.,2.) -- (0.,3.7320508075688776) -- cycle;
%\fill[line width=2.pt,color=ffffqq,fill=ffffqq,fill opacity=\gelb] (1.,2.) -- (-1.,2.) -- (0.,0.2679491924311226) -- cycle;
\draw [line width=2.pt] (-1.,2.)-- (1.,2.);
\draw [line width=2.pt] (1.,2.)-- (0.,3.7320508075688776);
\draw [line width=2.pt] (0.,3.7320508075688776)-- (-1.,2.);
\draw [line width=2.pt] (1.,2.)-- (-1.,2.);
\draw [line width=2.pt] (-1.,2.)-- (0.,0.2679491924311226);
\draw [line width=2.pt] (0.,0.2679491924311226)-- (1.,2.);
\begin{scriptsize}
\draw [fill=ududff] (-1.,2.) circle (2.5pt);
\draw[color=black] (-1.6,2.2) node {$\{V_1^1,V_1^2,V_1^3\}$};
\draw [fill=ududff] (1.,2.) circle (2.5pt);
\draw[color=black] (1.54,2.07) node {$\{V_2^1,V_2^2\}$};
\draw[color=black] (0.,2.75) node {$F$};
\draw[color=black] (0.06,1.85) node {$\{e_3^1,e_3^2\}$};
\draw[color=black] (0.87,2.96) node {$\{e_1^1,e_1^2\}$};
\draw[color=black] (-0.87,2.96) node {$\{e_2^1,e_2^2\}$};
\draw [fill=ududff] (0.,3.7320508075688776) circle (2.5pt);
\draw[color=black] (0.,3.91) node {$\{V_3^1,V_3^2\}$};
\draw[color=black] (0.,1.31) node {$F'$};
\draw[color=black] (-1.,1.16) node {$\{f^1,f^2\}$};
\draw[color=black] (0.8,1.16) node {$\{g\}$};
\draw [fill=ududff] (0.,0.2679491924311226) circle (2.5pt);
\draw[color=black] (0.,0.1) node {$\{V_4\}$};
%\draw[color=black] (0.94,1.31) node {$\{g\}$};
\end{scriptsize}
%\end{axis}
\end{tikzpicture}
\end{comment}
%-------------------------------------------
%--------------------bild-------------------------
\begin{figure}[h]
\definecolor{ududff}{rgb}{0.30196078431372547,0.30196078431372547,1.}
\definecolor{ffffff}{rgb}{1.,1.,1.}
\definecolor{qqqqff}{rgb}{0.,0.,1.}
\definecolor{ffffqq}{rgb}{1.,1.,0.}
\begin{tikzpicture}[line cap=round,line join=round,>=triangle 45,x=1.5cm,y=1.5cm]
%\begin{axis}[
x=1.5cm,y=1.5cm,
axis lines=middle,
ymajorgrids=true,
xmajorgrids=true,
xmin=-4.3,
xmax=7.0600000000000005,
ymin=-2.46,
ymax=6.3,
xtick={-4.0,-3.0,...,7.0},
ytick={-2.0,-1.0,...,6.0},]
\clip(-5.3,-0.4) rectangle (7.06,4.3);
\fill[line width=2.pt,color=ffffqq,fill=ffffqq,fill opacity=\gelb] (-2.5,0.) -- (2.2,0.) -- (2.2,4.) -- (-2.5,4.) -- cycle;
\fill[line width=2.pt,color=ffffqq,fill=ffffqq,fill opacity=0.] (0.,3.) -- (0.,1.) -- (1.7320508075688776,2.) -- cycle;
\fill[line width=2.pt,color=ffffff,fill=ffffff,fill opacity=0.] (0.,1.) -- (0.,3.) -- (-1.7320508075688776,2.) -- cycle;
\fill[line width=2.pt,color=ffffqq,fill=ffffqq,fill opacity=0.] (0.,3.) -- (-1.74,2.) -- (0.,1.) -- cycle;
\draw [line width=2.pt] (0.,3.)-- (0.,1.);
\draw [line width=2.pt] (0.,1.)-- (1.7320508075688776,2.);
\draw [line width=2.pt] (1.7320508075688776,2.)-- (0.,3.);
\draw [line width=2.pt] (0.,1.)-- (0.,3.);
\draw [line width=2.pt] (-1.7320508075688776,2.)-- (0.,1.);
\draw [line width=2.pt] (0.,3.)-- (-1.74,2.);
\draw [line width=2.pt] (-1.74,2.)-- (0.,1.);
\begin{scriptsize}
%\draw[color=ffffqq] (-0.84,1.27) node {$Vieleck1$};
%\draw [fill=qqqqff] (0.,3.) circle (2.5pt);
%\draw[color=qqqqff] (0.14,3.37) node {$E$};
%\draw [fill=qqqqff] (0.,1.) circle (2.5pt);
%\draw[color=qqqqff] (0.14,1.37) node {$F$};
%\draw[color=ffffqq] (1.04,3.11) node {$Vieleck2$};
%\draw [fill=qqqqff] (1.7320508075688776,2.) circle (2.5pt);
%\draw[color=qqqqff] (1.88,2.37) node {$G$};
%\draw[color=ffffff] (-0.1,2.17) node {$Vieleck3$};
%\draw [fill=qqqqff] (-1.7320508075688776,2.) circle (2.5pt);
%\draw[color=qqqqff] (-1.6,2.37) node {$H$};
%\draw [fill=ududff] (-1.74,2.) circle (2.5pt);
%\draw[color=ududff] (-1.6,2.37) node {$I$};
%\draw[color=black] (-0.93,3.) node {$f_1$};
%\draw[color=black] (-0.98,1.41) node {$e$};
%\end{comment}
%-----------------------------
%\begin{figure}
\draw[color=black] (-0.6,2.) node {$F$};
\draw [fill=qqqqff] (0.,3.) circle (2.5pt);
\draw[color=black] (0.14,3.23) node {$\{V_2^1,\{V_1^2,V_3^2\}\}$};
\draw [fill=qqqqff] (0.,1.) circle (2.5pt);
\draw[color=black] (0.14,0.75) node {$\{V_1^1,V_4\}$};
\draw[color=black] (1.04,2.75) node {$\{e_3^2,e_1^2\}$};
\draw [fill=qqqqff] (1.7320508075688776,2.) circle (2.5pt);
\draw[color=black] (1.88,2.22) node {$\{V_2^2\}$};
\draw[color=black] (0.8,2.) node {$F'$};
\draw[color=black] (0.4,2.32) node {$\{e_3^1,f^2\}$};
\draw [fill=qqqqff] (-1.7320508075688776,2.) circle (2.5pt);
\draw[color=black] (-2.1,2.22) node {$\{V_1^3,V_3^1\}$};
\draw [fill=qqqqff] (-1.74,2.) circle (2.5pt);
\draw[color=black] (-1.0,2.77) node {$\{e_1^1,e_2^2\}$};
\draw[color=black] (-1.0,1.26) node {$\{e_2^1,f^1\}$};
%\draw[color=black] (-1.16,2.25) node {$c$};
%\draw[color=black] (-0.65,2.2) node {$\{e^1_1\}$};
\draw[color=black] (0.94,1.31) node {$\{g\}$};
\end{scriptsize}
%\end{axis}
\end{tikzpicture}
\caption{simpliziale Fläche nach einem Cratermender}
\end{figure}


%-------------------------------------------------
\begin{comment}
\begin{bemerkung}
\item Es gilt $P^3_F(P^1_F(X))\cong X$
\item Die Prozeduren hängen nicht nur von $F$ ab, sondern auch von der Wahl einer Kante $f$. s gibt für ein nach dem erstem Schritt entferntes Dreieck $F \in X_{2}$ drei Möglichkeiten, um eine von den drei entstandenen Randkanten auszuwählen. Dann gibt es weiterhin zwei Möglichkeiten, eine Kante wie in $(1)$ beschrieben auszuwählen und dann letztlich drei Möglichkeiten, um das im erstem Schritt herausgenommene Dreieck wieder einzufügen. Deshalb ist die nach dem Algorithmus entstandene simpliziale Fläche nicht eindeutig. Weshalb folgende Notation eingeführt wird: \\
Wir nennen die nach Anwenden der drei Prozeduren entstandene simpliziale Fläche $X^{H}_{(F,f)}$, falls nach der ersten Prozedur ein \emph{Loch an der Stelle F} entsteht und in der zweiten Prozedur der Operator \emph{RipCut} auf die Kante $f$ in $X$ bzw. $\{f^1,f^2\}$ in $X(\alpha)$ angewendet wurde.\\ Sind $F$ und $f$ aus dem Kontext klar, so schreibt man nur $X^H$.
\end{bemerkung}
%\centerline{$\textcolor{red}{Bild 6}$}
\begin{bemerkung}
\begin{itemize}
\item Für eine geschlossene simpliziale Fläche $(X,<),F \in X_2$ und $f \in X_1$ ist $X^H_{(F,f)}$ wieder eine geschlossene simpliziale Fläche.
\item Mit den Bezeichnungen wie oben gilt: $(X^H_{(F,f)})^H_{(F,\{e_1^2,e_3^2\})}\cong X$.
\end{itemize}
\end{bemerkung}
Es stellt sich nun die Frage, welche und wie viele simpliziale Flächen mithilfe des obigen Algorithmus konstruiert werden können, wenn man diese mehrfach auf eine simpliziale Fläche anwendet. Dazu werden folgende Definitionen eingeführt.\\\\
\begin{definition}
Für eine geschlossene simpliziale Fläche $(X,<)$ ist 
\[
H(X):=\{(F,f)\in X_{1}\times X_{2} \text{ }\vert \text{ }\vert X_{0}(f) \cap X_{0}(F)\vert = 1 \land X_2(f) \cap X_2(X_1(F))\neq \emptyset\}
\]
die Menge aller Tupel $(F,f)$, auf die der Algorithmus anwendbar ist.
\end{definition}

\begin{definition}
Für eine geschlossene simpliziale Fläche $(X,<)$ wird die m-fache Anwendung des Algorithmus wie folgt definiert:\\
Für $(F_{1},f_{1})\in H(X) $ ist $X^{H}_{(F,f,1)}:=X^{H}_{(F,f)}$. Und für $(F_{i+1},f_{i+1}) \in H(X^{H}_{(F,f,i)})$ ist $X^{H}_{(F,f,i+1)}:=(X^{H}_{(F,f,i)})^{H}_{(F_{i+1},f_{i+1})}$. Wir nennen dann die Sequenz $\sigma=((F_{1},f_{1}),\ldots,(F_{m},f_{m}))$
eine \emph{H-Sequenz} der Länge $m$ von X.\\
Falls jedoch der Bezug zur H-Sequenz $\sigma$ klar ist, so schreibt man 
\[
X^{\sigma}_{i+1}:=X^{H(\sigma)}_{i+1}:=X^{H}_{(F,f,i+1)}:=(X^{H}_{(F,f,i)})^{H}_{(F_{i+1},f_{i+1})}
\]

\end{definition}

\begin{definition}
Für eine geschlossene simpliziale Fläche $(X,<)$ ist 
\[
H_{X}:=\{X^{H(\sigma)}_{m}\vert \sigma \text{ eine H-Sequenz und } m\in \mathbb{N}\}
\]
die Menge aller simplizialen Fläche die mithilfe des obigen Algorithmus ausgehend von $X$ konstruiert werden können.
\end{definition}
\end{comment}
%-------------------------------new shit----
%\newpage
%\section{neuer versuch}
\begin{bemerkung}
\begin{enumerate}
\item Es gilt $P^3_F(P^1_F(X))\cong X$ für ein $F \in X_2 $.
\item Die Prozeduren hängen nicht nur von $F$ ab, sondern auch von der Wahl der Kanten von $F$.
\item Es gibt für ein nach der ersten Prozedur entferntes Dreieck $F \in X_{2}$ drei Möglichkeiten, um eine von den drei entstandenen Randkanten auszuwählen. Dann gibt es weiterhin zwei Möglichkeiten eine Kante, wie im erstem Schritt der Prozedur $P^2$ beschrieben, auszuwählen und dann letztlich drei Möglichkeiten, um das in der Prozedur $P^1$ herausgenommene Dreieck wieder einzufügen. Deshalb ist die nach den drei Prozeduren entstandene simpliziale Fläche nicht eindeutig. Weshalb folgende Notation eingeführt wird: \\
\begin{itemize}
\item Wir nennen die durch Anwenden der drei Prozeduren entstandene simpliziale Fläche $X^{H}_{(F,f)}$, falls zunächst ein \emph{Loch an der Stelle F} entsteht und dann in der Prozedur $P^2$ der Operator \emph{RipCutter} auf die Kante $f$ in $X$ bzw. $\{f^1,f^2\}$ in $X(\alpha)$ angewendet wird. 
%\newpage
\item Für eine geschlossene simpliziale Fläche $(X,<)$ und eine Fläche $F \in X_2$ definiert man \emph{$H_F(X)$} als 
\[
H_F(X):=\{f \in X_1\mid \vert X_{0}(f) \cap X_{0}(F)\vert = 1 \land X_2(f) \cap X_2(X_1(F))\neq \emptyset\} .
\] 
Für den Fall, dass $X$ eine simpliziale Fläche ist, auf die man die Prozedur $P^3$ anwenden kann, so definiert man
\[
H_F(X):=H_F(P^3_F(X)).
\]
%Außerdem ist 
%\[
%H(X):=\bigcup_{F\in X_2} H_F(X).
%\]
Hierbei handelt es sich um die Menge aller Kanten mit denen man die Prozedur $P^2$ auf $X$
durchführen kann, falls man in Prozedur $P^1$ die Fläche $F$ entfernt hat.
 \end{itemize}
 Sind $F$ und $f$ aus dem Kontext klar, so schreibt man nur $X^H$.
 \end{enumerate}
\end{bemerkung}
%\centerline{$\textcolor{red}{Bild 6}$}
\begin{bemerkung}
\begin{itemize}
\item Für eine geschlossene simpliziale Fläche $(X,<),F \in X_2$ und $f \in H_F(X)$ ist $X^H_{(F,f)}$ wieder eine geschlossene simpliziale Fläche.
\item Mit den Bezeichnungen wie oben gilt: $(X^H_{(F,f)})^H_{(F,\{e_1^2,e_3^2\})}\cong X$.


\end{itemize}
%\end{enumerate}
\end{bemerkung}
%\vspace*{}
%---------------------------------

%-----------------------------------
Es stellt sich nun die Frage, welche und wie viele simpliziale Flächen mithilfe der obigen Prozeduren konstruiert werden können, wenn man nach der ersten Prozedur die Prozedur $P^2$ mehrfach auf eine simpliziale Fläche anwendet, bevor dann darauffolgend mithilfe der Prozedur $P^3$ wieder eine geschlossene simpliziale Fläche entsteht.\\
Hierfür führt man einige Notationen ein.

\begin{comment}
\begin{definition}
Sei $(X,<)$ eine geschlossene simpliziale Fläche und $F \in X_2$. \\
Falls $X$ eine geschlossene simpliziale Fläche ist, dann definiert man \emph{$H_F(X)$} als
\[
H_F(X):=\{f \in (P^1_F(X))_1\mid \vert X_{0}(f) \cap X_{0}(F)\vert = 1 \land X_2(f) \cap X_2(X_1(F))\neq \emptyset .
\]
Falls $X$ durch Anwenden einer der ersten beiden P
\end{definition}
\end{comment}
\begin{definition}
Sei $(X,<)$ eine geschlossene simpliziale Fläche, $F \in X_2$ und $n\in \mathbb{N}$. Sei außerdem $(F,f_1,\ldots,f_n)$ ein Tupel mit
\begin{itemize}
%\item $F \in X_2$
\item $f_1 \in H_F(P_F^1(X))$,
%\item $f_2 \in H_F(P^2_{f_1}(P_F^1(X))))$
\item $f_i \in H_F((P^2_{f_{i-1}} \circ \ldots \circ P^2_{f_1}\circ P_F^1 )(X)),$ für $i=2,\ldots ,n$.
\end{itemize}
So ist $X_{(F,f_1,\ldots,f_n)}^H$ definiert durch 
\[
X^{H}_{(F,f_1,\ldots,f_n)}:=(P_F^3\circ P^2_{f_n} \circ \ldots \circ P^2_{f_1}\circ P_F^1)(X)
\]
Das Tupel $(F,f_1,\ldots,f_n)$ nennt man eine \emph{Lochwanderung von $X$} und $X^H_{(F,f_1,\ldots,f_n)}$, die durch die Lochwanderung $(F,f_1,\ldots,f_n)$ entstandene simpliziale Fläche.\\
Für eine simpliziale Fläche $(X,<)$ und eine Lochwanderung $\sigma_1$ von $X$ führt man folgende Konstruktion durch:
\begin{itemize}
\item $X^H_{(\sigma_1)}:=X^H_{\sigma_1}$
\item Für eine Lochwanderung $\sigma_2$ von $X^H_{\sigma_1}$ ist 
\[
X_{(\sigma_1,\sigma_2)}:=(X_{\sigma_1}^H)^H_{\sigma_2}.
\]
\item Sei nun  $X_{(\sigma_1,\ldots, \sigma_i)}^H$ für $i\in \mathbb{N}$ und Lochwanderungen $\sigma_j$ von $X^H_{(\sigma_1,\ldots,\sigma_{j-1})}$ mit $2 \leq j \leq i$ schon konstruiert und $\sigma_{i+1}$ eine Lochwanderung von $X_{(\sigma_1,\ldots, \sigma_i)}^H$, so ist
\[
X_{(\sigma_1,\ldots, \sigma_{i+1})}^H:=(X_{(\sigma_1,\ldots, \sigma_i)}^H)^H_{\sigma_{i+1}}.
\]
\item Für obige Lochwanderungen $\sigma_i$ und $n \in \mathbb{N}$ nennt man $(\sigma_1, \ldots,\sigma_n)$ eine \emph{Lochwanderungssequenz von $X$} und $X^H_{(\sigma_1,\ldots,\sigma_n)}$ die durch die Lochwanderungssequenz $(\sigma_1,\ldots,\sigma_n)$ entstandene simpliziale Fläche.
\end{itemize}
\end{definition}


\begin{definition}
Für eine simpliziale Fläche $(X,<)$ definiert man die Menge aller simplizialen Flächen, die bis auf Isomorphie durch Anwenden einer Lochwanderungssequenz auf $X$ entstehen können, als
\[
\mathcal{H}_X:=\{[X_{\Sigma}^H] \mid \Sigma \text{ ist eine Lochwanderungssequenz von X}\},
\]
wobei
\[
 [Z] := \{Y \mid Y \text{simpliziale Fläche mit } X \cong Z\}
\] die \emph{Isomorphieklasse} einer simplizialen Fläche  $Z$ ist. 

\end{definition}

Für den Beweis der Transitivität der oben definierten Operation Wanderinghole muss noch etwas Vorarbeit geleistet werden.
Zunächst einmal werden die Begriffe einer stark-zusammenhängenden und einer jordan-zusammenhängende Menge eingeführt.
\begin{definition} 

\begin{enumerate}$\textcolor{blue}{(neu)}$
\item Sei $(X,<)$ eine simpliziale Fläche. Ein \emph{Flächenpfad} von $S\in X_{2}$ nach $T \in X_{2}$ in $X$ ist eine Sequenz $(F_1:=S,F_{2},\ldots,F_{k}:=T)$ für ein $k \in \mathbb{N}$ so, dass $F_{i} $ und $F_{i+1}$ für $i=1,\ldots,k-1$ benachbarte Flächen in $X$ sind.
\item Man nennt eine Menge $M\subseteq X_2$  \emph{stark-zusammenhängend}, falls für beliebige $S,T \in M$ ein Flächenpfad $(F_{1}:=S,F_{2},\ldots,F_{k}:=T)$ mit $F_i \in M$ für $1\leq i \leq k$ existiert. 
 \item Die stark-zusammenhängenden Mengen $M\subseteq X_2$ nennt man \emph{Zusammenhangskomponenten von $X$}, falls folgende Bedingung erfüllt ist: Falls ein $M'\supseteq M$ existiert, sodass $M'$ stark-zusammenhängend ist, so gilt schon $M'=M$.
 \item Man nennt die Menge $M \subseteq X_2$ \emph{jordan-Zusammenhängend}, falls $M=X_2$ stark-zusammenhängend ist oder es gilt $M \subsetneq X_2$ und die Mengen $M$ und $X_2\setminus M$ sind stark-zusammenhängend. Falls $X_2$ stark zusammenhängend ist, so nennt man die simpliziale Fläche $(X,<)$ jordan-zusammenhängend.
\end {enumerate}
\end{definition}
An dieser Stelle kann man auch den Begriff einer zusammenhängenden Menge einführen. Da jedoch zusammenhängende Mengen für diesen Untersuchungsaspekt nicht von Relevanz sind, bleibt dies aus.

\begin{bsp}$\textcolor{blue}{(neu)}$
\begin{itemize}
\item Sei $J$ der Janus-Kopf. Dann ist $J_2=\{F_1,F_2\}$. Dann sind die Mengen $\{F_1\}$,$\{F_2\}$ und $\{F_1,F_2\}$ jordan-zusammenhängende Mengen von $J$.
\item Für eine geschlossene simpliziale Fläche $(X,<)$ mit $\chi(X)=2$ bildet $\{F\}$ für ein $F \in X_2$ stets eine jordan-zusammenhängende Menge.
\item Sei $T$ das zuvor definierte Tetraeder. Sei $\emptyset \subsetneq M \subseteq T_2$. So bildet $M$ eine jordan-zusammenhängende Menge.



\end{itemize}
\end{bsp}
Um die Knoten, Kanten und Flächen einer jordan-zusammenhängenden Menge von denen des Komplements unterscheiden zu können, färbt man den Graphen, wie in folgender Definition beschrieben.

\begin{definition}$\textcolor{blue}{(neu)}$
Für eine simpliziale Fläche $(X,<)$ und eine jordan-zusammenhängende Menge $M$definiert man die Abbildung $f_M^X:X \mapsto\{0,1\}$, welche durch
\[
f^X_M(x)=\begin{cases}
1 & \text{falls } x\in M\\
1 & \text{falls } x \in X_i(F) \text{ für ein }F \in M ,i=1,2\\
0 &\text{sonst}\\

\end{cases}
\]
gegeben ist.
Die Abbildung $f_M^X$ nennt man eine \emph{Färbung} von $X$ und $(X,<,f_m^X)$ eine gefärbte simpliziale Fläche.
\end{definition}

\begin{bsp}$\textcolor{blue}{(neu)}$
\begin{itemize}
\item Für eine jordan-zusammenhängende simpliziale Fläche $(X,<)$ ist die Färbung $f^X_{X_2}$ gegeben durch $f_{X_2}^X(x)=1$ für alle $x \in X$.

\item Für den Janus-Kopf $J$ bildet $M_J=\{F_1\}$ eine jordan-zusammenhängende Menge. Damit ist $f^J_{M_J}$ gegeben durch
\[
f^J_{M_J}(x)=\begin{cases}
0 &x=F_2\\
1 & \text{falls } x\in X\setminus \{F_2\}.\\
		\end{cases}
\]
\item Für eine simpliziale Fläche $(X,<)$ mit $\chi(X)=2$ bildet $M=\{F\}$ für ein $F \in X_2$ eine jordan-zusammenhängende Menge. Dann ist $f_M^X(F)=0$ und $f_M^X(x)=1$ für alle $x \in X\setminus \{F\}$.
\item Für das Tetraeder $T$ wie in Abbildung \ref{tetra} definiert, bildet $M_T=\{F_1,F_2\}$ eine jordan-zusammenhängende Menge. Dadurch erhält man die Färbung $f^T_{M_T}$ gegeben durch
\[
f^T_{M_T}(x)=\begin{cases}
		0 & \text{falls } x  \in \{e_6,F_3,F_4\}\\
1 &x \in X\setminus \{e_6,F_3,F_4\}.\\
		\end{cases}
\]
\end{itemize}
\end{bsp}
Und um die Nachbarn einer Fläche in einer simplizialen Fläche besser spezifizieren zu können, führt man folgende Definition ein.
\begin{definition} $\textcolor{blue}{(neu)}$ 
Sei $(X,<)$ eine simpliziale Fläche und $M \subseteq X_2$ eine Menge. Dann ist für $F \in X_2$ die Menge $N_M^X(F)$ definiert durch
\[
N_M^X(F):=\{ F' \in M \mid F'\text{ und F sind adjazent}\} .
\]
Für $S \subseteq X_2$ ist 
\[
N_M(S):=\bigcup_{F\in S}N_M(F).
\]
Falls $M=X_2$ ist, so definiert man auch $N_X^X(F):=N^X_{X_2}(F)$. Ist $X$ aus dem Kontext klar, so schreibt man $N_M^X(F):=N_M(F)$.
\end{definition}
\begin{lemma} \label{lemma1} $\textcolor{blue}{(neu)}$
Sei $(X,<)$  eine simpliziale Fläche mit $\chi (X)=2$, die weder 2- noch 3-Waists enthält und $M \subsetneq X_2$ eine jordan-zusammenhängende Menge. Dann existiert ein $F\in X_2\setminus M$ so, dass $M \cup \{F\}$ jordan-zusammenhängend ist.
\end{lemma}
\begin{proof}
Man betrachte $F \in X_2\setminus M,e_i \in X_1$ und $V_i\in X_0$ mit $e_i<F$ und $V_i<e_j$ für $i\in \{1,2,3\}$ und $j \in \{1,2,3\} \setminus \{ i \}$. Für den Beweis macht man nun Gebrauch von der Färbung $f_M$. 
\begin{itemize}
\item Falls $f_M(x)=1$ für $x\in \{V_1,V_2,V_3,e_1,e_2,e_3\}$ und $f_M(F)=0$ ist, so gilt schon $M=X_2\setminus \{F\}$. Denn angenommen $\vert X_2 \setminus M \vert > 1$, dann existiert ein $F \neq F' \in X_2 \setminus M $ und somit auch ein $F-F'-Weg$ $(F,F_1,\ldots,F')$ in $X_2 \setminus M$. Da aber $f_M(e_1)=f_M(e_2)=f_M(e_3)=1$ ist, gilt $\vert N_M(F)\vert =3$ und daraus kann man schon $X_2 \setminus M$ nicht stark-zusammenhängend schließen, was den gewünschten Widerspruch liefert. Also ist $M=X_2 \setminus \{F\}$ und $M \cup \{F\}=X_2$ ist dann per Definition jordan-zusammenhängend.
\[
bild
\]
\item Falls $F$ eine Fläche mit der Eigenschaft $f_M(x)=1$ für $x \in \{V_1,V_2,V_3,e_1,e_2\}$ und $f_M(F)=f_M(e_3)=0$ ist, so  lässt sich zeigen, dass $M \cup \{F\}$ jordan-zusammenhängend ist. Mein zeigt zunächst den Zusammenhang von $M\cup \{F\}$. Es reicht zu zeigen, dass es $F-T-$Wege in $M\cup \{F\}$ für $T \in M$ beliebig gibt.
Wegen $f(e_1)=1$ existiert ein $F'\in M$ mit $e_1<F'$. Für den Fall $T=F'$ liefert der Weg $(F,F')$ die Behauptung. Betrachte nun $T \neq F,F'$. Da M stark-zusammenhängend ist, existiert ein $F'-T-$Weg $(F',F_1,\ldots,F_n,T)$ in $M$, welcher sich zu einem $F-T-$Weg $(F,F',F_1,\ldots,F_n,T)$ erweitern lässt. Also ist $M \cup \{F\}$ stark-zusammenhängend. Bleibt nur noch nachzuweisen, dass $X_2\setminus (M\cup \{F\})$ stark-zusammenhängend ist. Sei dazu $(F_1,\ldots,F_n)$ ein $F_1-F_n-Weg$ in $X_2 \setminus M $ für $F_1,F_2 \in X_2\setminus M$ mit $F_1,F_n \neq F$. Für $F_i \neq F$ mit $1 \leq i \leq n$ ist $(F_1,\ldots,F_n)$ ebenfalls ein $F_1-F_n$-Weg in $X_2 \setminus (M \cup \{F\})$. Sei also $F_i=F$ für ein $1<i<n$. Dann gilt wegen $\vert N_{X_2\setminus M}(F) \vert=1$ schon $F_{i-1}=F_{i+1}$ und somit kann man den $F_1-F_n$-Weg $(F_1,\ldots,F_{i-1},\ldots,F_n)$ in $X_2\setminus M$ durch Weglassen von $F=F_i$ und $F_{i+1}$ konstruieren. Da $F_j\neq F$ für $j \in \{1\ldots n\}\setminus \{i\}$ gilt, ist $(F_1,\ldots,F_{i-1},\ldots,F_n)$ ein $F_1-F_n-$Weg in $X_2 \setminus (M \cup \{F\})$. Also ist $M$ jordan-zusammenhängend.
\item Falls $F$ eine Fläche mit der Eigenschaft $f_M(x)=0$ für $x \in \{F,e_1,e_2,V_3\}$ und $f_M(y)=1$ für $y \in \{V_1,V_2,e_3\}$ ist, so ist $M \cup \{F\}$ eine jordan-zusammenhängende Menge. Den starken Zusammenhang von $M \cup \{F\}$ kann man mit analoger Argumentation wie im obigen Fall nachweisen. Es bleibt wieder der Zusammenhang der Menge $X_2\setminus (M \cup \{F\})$ zu zeigen.
 Diese Behauptung gilt, denn wegen $f_M(V_3)=0$ existieren $F_1,\ldots,F_n \in X_2\setminus M$ mit $V_3 <F_i$ für $1 \leq i \leq n \in \mathbb{N}$ so, dass das Tupel $(F,F_1,\ldots,F_n)$ einen Schirm bildet.
  Sei also für $n' \in \mathbb{N}$
   $(F_1',\ldots ,F_{n'}')$ ein $F_1'-F_{n'}'-Weg$ in $X_2 \setminus M$, wobei $F_i' \in X_2\setminus M$. 
  Falls $F_i' \neq F$ für $1 \leq i \leq n'$, so ist $(F_1',\ldots,F_n')$ ebenfalls ein Weg in $X_2\setminus (M \cup \{F\})$.
   Sei also $F_i'=F$ für ein $1 < i < n$. Da $f_M(V_3)=f_M(e_1)=f_M(e_2)=0$ ist, folgt $f_M(F_{i-1})=f_M(F_{i+1})=0$. Und aus $F_{i-1}',F_{i+1}' \in X_2(V_3)$ und der Adjazenz von $F'_{i-1}$ und $F'_{i+1}$ zu $F$ kann man $F_{i-1}'=F_1$ und $F_{i+1}'=F_n$ oder $F_{i-1}'=F_n$ und $F_{i+1}'=F_1$ schließen. Für den ersten der beiden auftretenden Fälle bildet $(F_1',\ldots,F_{i-1}'=F_1,F_2, \ldots,F_{i+1}'=F_n\ldots,F_n')$ einen $F_1'-F_n'-Weg$ in $X_2\setminus (M \cup \{F\})$ und für den anderen Fall ist $(F_1',\ldots,F_{i-1}'=F_n,F_{n-1}, \ldots,F_{i+1}'=F_1\ldots,F_n')$ der gesuchte Weg, der $F_1'$ und $F_n'$ verbindet. Obige Konstruktion eines Weges in $X_2 \setminus (M \cup \{F\})$ kann man auch für $F_1'=F$ und $F_n'$ durchführen. 
\end{itemize}
Angenommen keiner der oben skizzierten Fälle tritt ein, das heißt $ \vert\{e <F \mid f_M(e)=1\}\vert \notin \{2,3\}$ und $\vert \{V<F \mid f_M(V)=1\}\vert \neq 2$. Da $X$ geschlossen ist mit $\chi(X)=2$, ist $X$ stark-zusammenhängend und da $M \subset X_2$ ist, existiert ein $F_1\in X_2\setminus M$ und $f_1 \in X_1$ mit $f_1<F_1$ und $f_M(f_1)=1$. Dann ist $M \cup \{F_1\}$ stark-zusammenhängend. Falls $M \cup \{F_1\}$ jordan-zusammenhängend ist, so ist die Behauptung gezeigt. Falls dies nicht der Fall ist, dann zerfällt $X_2 \setminus (M \cup \{F_1\})$  ohne Einschränkung in genau zwei  Zusammenhangskomponenten. Seien also $Z^{(1,1)}$ und $Z^{(1,2)}$ die Zusammenangskomponenten von $X\setminus (M \cup \{F_1\})$. Dann gilt $1\leq \vert Z^{(1,1)}\vert,\vert Z^{(1,2)}\vert < \vert X_2\setminus M\vert$. Dann definiert man $N^1:=Z^{(1,1)}$. Klarerweise ist $N^1$ stark-zusammenhängend in $X_2$. Man konstruiert sich nun eine absteigende Kette von stark-zusammenhängenden Teilmengen von $X_2\setminus M$  wie folgt:
Sei $N^i$ für $i \in \mathbb{N}$ schon mithilfe von $F_i\in X_2\setminus M$ konstruiert. Falls  $\vert N^i \vert>2$ so betrachtet man für ein $F_{i+1}\in N^i$ mit $e_{i+1}<F_{i+1}$ und $f_M(e_{i+1})=1$, wobei $e_{i+1} \in X_1$, die Menge $N^i\setminus \{F_{i+1}\}$. 
Falls $N^i \setminus \{F_{i+1}\}$ stark-zusammenhängend ist, dann ist man fertig, da dann $X_2\setminus (M \cup \{F_{i+1}\})$ ebenfalls stark-zusammenhängend ist.
 Falls jedoch $N^i \setminus \{F\}$ in zwei Zusammenhangskomponenten $Z^{(i+1,1)}$ und $Z^{(i+1,2)}$ zerfällt, so definiert man $N^{i+1}=Z^{(i+1,j)}$  für $j \in \{1,2\}$, falls $F_{i} \notin N_{X_2}(F)$ für  alle $F \in Z^{(i+1,j)}$ gilt. Offensichtlich ist $N^{i+1}$ stark-zusammenhängend und es gilt $\vert N^{i}\vert >\vert N^{i+1} \vert$. Dadurch erhält man die Kette $\vert N^1 \vert>\vert  N^2 \vert> \ldots$. Da $\vert X_2\vert < \infty$ ist, muss nach endlich vielen Schritten der Fall $\vert N^k \vert \in \{1,2\}$ für ein $k \in \mathbb{N}$ eintreten.
 \begin{itemize}
 \item Falls $\vert N^k \vert=1$ ist, so ist $N_{N^{k-1}}(F)=\{F_{k-1}\}$ für $N^k=\{F\}$, somit dann aber auch $N_{X_2}(F)=\{F',F'',F_{k-1}\}$ für $F',F'' \in M$. Also ist $\vert \{e<F \mid f_M(e)=1\}\vert =2 $ und dies ist ein Widerspruch zur oben getroffenen Annahme.
 \item Falls $\vert N^k \vert=2$ ist, so ist $N^K=\{F,F'\}$. Dann gilt $N_{N_k}(F)=\{F'\}$ für $F^*,\tilde{F} \in N^i$. Daraus folgt aber wieder $\vert \{e<F \mid f_M(e)=1\}\vert =2$.
\end{itemize}
\end{proof}
\begin{lemma} \label{lemma2} $\textcolor{blue}{(neu)}$ 
Sei $(X,<)$ eine simpliziale Fläche mit $\chi(X)=2$ und $(S,F_1,F_2,\ldots,F_n,T)$ ein $S-T-Weg$ in $X_2$, wobei $S,T,F_i \in X_2$ für $n \geq 2$ und $1 \leq i \leq n$ ist. Dann existiert eine Lochwanderung $\sigma$ so, dass $(S,F_2, \ldots,F_n,T)$ ein $S-T-Weg$ in $X^H_{\sigma}$ ist.
\end{lemma}
\begin{bemerkung} $\textcolor{blue}{(neu)}$
Induktiv kann also aus dem $S-T-Weg$ $(S,F_1,\ldots,F_n,T)$ durch das Anwenden der Operation Wanderinghole mithilfe einer Lochwanderungssequenz $\Sigma$ eine Fläche $X^H_{\Sigma}$ konstruiert werden, in welcher $S$ und $T$ benachbart sind.
\end{bemerkung}
\begin{proof} $\textcolor{blue}{(neu)}$
Sei $(S,F_1,\ldots,F_n)$ ein $S-T-Weg$, wie  zuvor beschrieben. Dann existieren Kanten $e_1,e_2 \in X_1$ mit $X_2(e_1)= \{S,F_1\}$ und $X_2(e_2)=\{F_1,F_2\}$. Damit ist $e_2 \in H_F(X)$ und Wanderinghole mithilfe des Tupels $(S,e_2)$ anwendbar, denn per Konstruktion gilt $e_1 \notin X_1(S) $ und $\vert X_0(e_1) \cap X_0(S)\vert$ = 1. Dadurch erhalten wir die simpliziale Fläche $X^H_{(S,e_1)}$, in welcher $S$ und $F_2$ benachbart sind.
\[
bild
\]
\end{proof}

\begin{vor} \label{vor2} $\textcolor{blue}{(neu)}$ 
Sei $(X,<)$ eine simpliziale Fläche mit $\chi(X)=2$, die weder 2- noch 3-Waists enthält, $V\in X_0$ ein Knoten und $F_1,\ldots,F_n \in X_2$ Flächen so, dass $(F_1,\ldots,F_n)$ den zu $V$ zugehörigen Schirm bildet. Sei außerdem $1 \leq j<k\leq n$ für $j,k \in \mathbb{N}$. Falls $F_i=F_j$ und $F_{i+1}=F_k$ für ein $1 \leq i < n$ oder $F_1=F_j$ und $F_n=F_k$ gilt, so sind $F_j$ und $F_k$ bereits benachbarte Flächen. Andernfalls bildet $(F_j,F_{j+1},\ldots,F_k)$ einen $F_j-F_k-Weg$ in $X_2$. Für $F_j$ und $F_{j+1}$ existiert eine Kante $f \in X_1$ mit $f<F_{i+1}$ und $f \nless F_j,F_{j+2}$. Damit ist $f \in H_{F_j}(X)$ und $X^H_{(F_j,f)}$ ist eine simpliziale Fläche mit einem Knoten $V'\in X^H_{(F_j.f)}$ so, dass $(F_1,\ldots,F_j,F_{j+2},\ldots,F_n)$ der zugehörige Schirm ist. Induktiv erhält man also durch eine Lochwanderungssequenz $\Sigma$ eine simpliziale Fläche $X_{\Sigma}^H$, in der es einen Knoten $\tilde{V} \in {X_{\Sigma}^H}_0$ gibt, sodass $(F_1,\ldots,F_j,F_k,\ldots,F_n)$ der zu $\tilde{V}$ zugehörige Schirm ist. 
\end{vor}
\begin{satz}$\textcolor{blue}{(neu)}$
Sei $(X,<)$ eine geschlossene simpliziale Fläche mit $\chi (X)=2$, die weder 2- noch 3-Waists enthält. Dann ist die Anwendung von Wanderinghole auf $X$ transitiv, das heißt für alle $(Y,\prec)$ mit $\vert X_2 \vert = \vert Y_2 \vert$ existiert eine Lochwanderungssequenz $\Sigma$ mit $X^H_{\Sigma} \cong Y$.
\end{satz}

\begin{proof}$\textcolor{blue}{(neu)}$
Sei $S:=\{[X^1],\ldots,[X^m]\}$ die Menge aller Isomorphieklassen mit $\vert X^i_2 \vert =n \in \mathbb{N}$ für 
$1 \leq i \leq m \in \mathbb{N}$. Seien zusätzlich $X \in [X^i]$ und $Y \in [X^j]$ für $1 \leq i,j \leq m$ zwei Vertreter der Äquivalenzklassen mit $X_2=Y_2=\{F_1,\ldots,F_n\}$.
 Man definiert sich nun die Menge $M_1:=\{F_1\} \subset X_2=Y_2$ und $Z^1:=X$. Nach \Cref{lemma1} existiert nun ein $F \in Y_2\setminus M_1$, sodass $M_2=M_1 \cup \{F\}$ jordan-zusammenhängend in $Y$ ist. Da $X_2 \setminus M_1$ jordan-zusammenhängend in $X$ ist, existiert ein $F-F_1-Weg$ $(F,F_1',\ldots,F_n',F_1)$ in $X$ und wegen \Cref{lemma2} auch eine Lochwanderungssequenz $\Sigma_2$ so, dass $F$ und $F_1$ benachbarte Flächen in $X^H_{\Sigma_2} $ sind. Setze also nun $Z^2=X^H_{\Sigma_2}$. Seien nun für $1 < i < n$ die Menge $M_i$ und die simpliziale Fläche $Z^i$ schon konstruiert. Nach  \Cref{lemma1} existiert eine Fläche $F \in Y_2 \setminus M_i$ so, dass $M_{i+1}=M_i \cup \{F\}$ jordan-zusammenhängend in $Y$ ist. \\
  Da $M_i$ jordan-zusammenhängend in $W={Z^i}$ ist, existiert eine Lochwanderungssequenz $\Sigma'$, sodass $M_{i+1}$ jordan-zusammenhängend in $W^H_{\Sigma'}$ ist und es für Knoten $V\in Y_0$ mit zugehörigen Schirm $(F_1,\ldots,F_j,F,F_{j+1},\ldots,F_{n_1})$ einen Knoten $V'\in W_0$ mit zugehörigem Schirm $(F_1,\ldots,F_j,F,F'_{j+1},\ldots,F'_{n_2})$ gibt, wobei $F_i \in M$ für $1\leq i\leq j$, $F_k,F_k'\in X_2=Y_2=W_2$ und $n_1,n_2 \in \mathbb{N}$ gilt. Wegen der \Cref{vor2} findet man gegebenfalls eine Lochwanderungsequenz $\tilde{\Sigma}$, sodass $N_{M_{i+1}}^{\tilde{W}}(F)=N_{M_{i+1}}^Y(F)$ für alle $F \in M_{i+1}$ gilt, wobei $\tilde{W}=W^H_{\tilde{\Sigma}}$ ist  und setzt nun $Z^{i+1}=W^H_{\tilde{\Sigma}}$.\\
Somit erhalten wir dann schließlich $M_n=X_2$ und eine simpliziale Fläche $X^H_{\Sigma^*}$.\\ Es bleibt nun zu zeigen, dass $Z=X^H_{\Sigma^*}$ und Y isomorph sind. Nach Konstruktion gilt $N_Z(F)=N_Y(F)$ für alle $F\in Z_2=Y_2$. Damit konstruiert man sich nun folgende Inzidenzgraphen: 
Zu $W\in \{Z,Y\}$ definiert man sich den Graphen  $G^W=(V^W,E^W)$, welcher gegeben ist durch die Knoten $V^W=W_2$ und die Kanten $E^W=\{\{F,F'\} \mid \exists e \in W_1 : e<F,F' \}$. 
Dann gilt $G^Y=G^Z$ und da $\vert N_W(F) \vert =3$ für alle $F\in Y_2$ gilt,  ist $G^Y$ 3-zusammenhängend. Außerdem ist $G^Y=G^Z$ ein planarer Graph. Die Behauptung folgt nun mit dem folgendem Satz.

\end{proof}
\begin{satz}[Satz von Witney]$\textcolor{blue}{(neu)}$
Sei $G=(V,E)$ ein 3-fach zusammenhängender planarer Graph. Dann lässt sich G eindeutig in die Sphäre einbetten.
\end{satz}
\begin{comment}

\begin{definition}
Sei $(X,<)$ eine simpliziale Fläche und $F\in X_2$ eine Fläche in $X$. Dann definieren man die Menge aller benachbarten  Flächen von $F$ in $X$ als 
\[
N_X(F):=\{F' \in X_2 \mid \exists f\in X_1 :f<F,f<F'\}
\]
und $N(X):=\bigcup_{F \in X_2} N(F)$
\end{definition}
\begin{definition}$\textcolor{red}{(new)}$
Sei $(X,<)$ eine simpliziale Fläche und $M\subset X_2$ eine Teilmenge der Flächen von $X$.
 Man identifiziert $M$ mit der simplizialen Fläche $(M^X,\prec)$, die durch die Knotenmenge $M^X_1:=\{V\in X_0 \mid X_2(V)\subseteq M\}$, die Kantenmenge $M^X_1:=\{e\in X_1 \mid X_2(V)\subseteq M\}$ und die Flächenmenge $M^X_2:=M$ gegeben ist.
 Für $x,y\in M^X$ gilt $x\prec y$ in $M^X$, falls $x<y$ in $X$ gilt.
\end{definition}
\begin{definition}
%$\textcolor{red}{!!}$
Sei $(X,<)$ eine simpliziale Fläche. Man nennt eine Menge $\emptyset \subsetneq M \subseteq X_2$ \emph{einfach zusammenhängend}, falls $M=X_2$ gilt oder $M\subset X_2$ und die mit den Mengen $M$ und $X_2\setminus M$ identifizierten simplizialen Flächen zusammenhängend sind.
\end{definition}
\begin{bemerkung}
Für jede simpliziale Fläche $(X,<)$ lassen sich  einfach zusammenhängende Teilmengen finden, denn nach Definition ist $X_2$ einfach zusammenhängend und offensichtlich ist $\{F\}$ für ein $F\in X_2$ einfach zusammenhängend. Falls es zu $F$ eine adjazente Fläche $F'\in X_2$ gibt, so bildet ebenfalls $\{F,F'\}$ eine einfach zusammenhängende Menge von $X$.
\end{bemerkung}
\begin{lemma}% $\textcolor{red}{!!}$
Sei $(X,<)$ eine geschlossene simpliziale Fläche mit $\chi (X)=2$ ohne 2- und 3-Waists. Sei weiterhin $M\subset X_2$ einfach zusammenhängend. Dann existiert eine Fläche $F\in X_2 \setminus M$, sodass $M \cup \{F\}$ wieder einfach zusammenhängend ist.
\end{lemma}
\begin{proof}
$textcolor{red}{nicht zielfuhrend}$
Sei $M\subset X_2$ eine einfach zsammenhängende Menge.
 Für $M=X_2\setminus \{F\}$, wobei $F\in X_2$ ist, ist die Behauptung klar.
  Sei also nun $0<\vert M\vert <\vert X_2\vert -1 $.
   Man betrachte die Mengen $N_1:=\{F \in (X_2 \setminus M)^X_2
    \mid  N_{(X_2\setminus M)^X}(F)=1\}$ 
   und $N_2:=\{F \in (X_2\setminus M)^{X}_2 \mid  N_{(X_2 \setminus M)^X}(F)=2\}$.
    Wegen $\vert M \vert < \vert X_2 \vert -1$ gilt $N_1 \cup N_2 \neq \emptyset$.$\textcolor{red}{begruendung}$ Denn es gilt $\vert X_2\setminus M\vert > 1$.
    Da $X_2 \setminus M$ einfach zusammenhängend ist, existtiert ein $\tilde{F}\in X_2\setminus M$, das du $F$ adjazent ist,
     also $1\leq N_{(X_2 \setminus M)^X}(F)$ Weiterhin existiert ein $F\in X_2 \setminus M $ und ein $F' \in M$ so, dass $F$ und $F'$ adjazent sind,
      daraus folgt $N_{(X_2 \setminus M)^X}(F)
      \leq \vert N_X(F)\vert - \vert\{ F'\}\vert \leq 2$.  
\begin{itemize}
\item Angenommen es gilt $N_1 \neq \emptyset$, dann existiert nach Definition der Menge $N_1$  ein $F \in (X\setminus M)^X$ mit  $N_{(X_2\setminus M)^X}(F)=1$. Da $F$ ursprünglich eine Fläche in $X$ war, gilt $\vert N_X(F)\vert =\vert \{ F_1,F_2,F_3\}\vert=3$ für $F_1,F_2 \in M$ und $F_3 \in X_2\setminus M$, da $X$ geschlossen ist und keine 2 Waists enthält. Dann ist $M \cup \{F\}$ einfach zusammenhängend, denn  offensichtlich ist $X_2\setminus (M\cup \{F\})$ ist klarerweise zusammenhängend und $M\cup\{F\}$ ist zusammenhängend da F adjazent zu $F_1$ ist, das heißt es existiert ein $F-F_1$-Weg und da M zusammenhängend ist existiert ein $F_1-F'$-Weg für ein beliebiges $F'\in M$. Diese beiden setzt man nun zu einem $F-F'$-Weg zusammen.
\item Angenommen es ist $N_2\neq \emptyset$. So existiert ein $F\in X_2\setminus M$ mit $\vert N_{(X_2\setminus M)^X}(F)\vert =\{F_1,F_2,F_3\}=2$ für $F_1\in M$ und $F_2,F_3\in X_2\setminus M$. Man stellt nun die Behauptung $M \cup \{F\}$ ist einfach zusammenhängend und zeigt dies analog zum obigen Fall.
\end{itemize}
\end{proof}
\begin{proof}
Man führt den Beweis dieser Aussage über Induktion. Sei also $\vert X_2 \vert =2k$ für ein $k \in \mathbb{N}$. Für $k = 1$ ist  die Behauptung klar, denn bis auf Isomorphie gibt es nur eine geschlossene simpliziale Fläche mit zwei Flächen, nämlich den Janus-Kopf. Und für diesen gibt es nur die einfach zusammenhängenden Mengen $X_2$ und $\{F\}$ für ein $F\in X_2$. Da nach Voraussetzung $M \subsetneq X_2$ ist, schließt man $M=\{F\}$. Somit folgt die Behauptung. Sei also nun die Aussage für simpliziale Flächen $(Y,\prec)$ mit $\vert Y_2 \vert=2k $ gezeigt, wobei $k \in \mathbb{N}$ fest, aber beliebig ist. Betrachte nun die simpliziale Fläche $(X,<)$ mit $\vert X_2\vert=2(k+1)=2k+2$. Wegen obiger Bemerkung können wir $\vert M \vert \geq 2$ annehmen. Somit existieren $F_1,F_2 \in M$ $e_1,e_2,e_3,e_4,e \in X_1$ mit $e_i<F_1$ und $e_j<F_2$ für $i=1,2 ,j=3,4$ und $e<F_1,F_2$. Nun identifiziert man $e_i=\{e_i^1,e_i^2\}$ für $i=1,2,3,4$ und macht erneut Gebrauch von den Cutter- und Mender-Operatoren.
\begin{itemize}
\item Man führt einen 
\end{itemize}
 
\end{proof}
Für den Beweis machen wir ebenfalls Gebrauch von folgendem Satz (Whitney)$\textcolor{red}{!!}$
\begin{satz}
Die Operation Wanderinghole ist transitiv. Das heißt für beliebige geschlossene simpliziale Flächen $(X,<)$ und $(Y,\prec)$ mit $\vert X_2 \vert=\vert Y_2 \vert$ existiert eine Lochwanderungssequenz $\Sigma$ so, dass $[Y]=[X^H_{\Sigma}]$ ist.
\end{satz}
\begin{proof}$\textcolor{red}{!!}$
Seien $(X,<)$ und $(Y,\prec)$ simpliziale Flächen mit $\vert X \vert=\vert Y \vert$. Man kann zunächst ohne Einschränkung $X_2=Y_2=\{F_1,\ldots,F_n\}$ für ein $n\in \mathbb{N}$ annehmen.
\end{proof}
\end{comment}


%\newpage
%---------------------------------
%\newpage
\subsection{Knotengrade}
Man interessiert sich ebenfalls dafür, wie sich die Grade der Knoten der simplizialen Fläche $X^H_{(F,f)}$ aus den Graden der Knoten von $(X,<)$ ergeben. Hierzu betrachtet man die unten angeführten Tabellen, die die Grade der Knoten nach den einzelnen Schritten der drei Prozeduren zeigen. Hier werden jedoch nur die Knoten angeführt, deren Kanten durch die Operatoren verändert werden, denn für alle anderen Knoten bleibt der Grad nach den drei Prozeduren unverändert. Dazu definiert man den Grad der zu betrachteten Knoten wie folgt:\\
\begin{center}
\begin{tabular}{|c|c|c|c|c|}
\hline
  \textbf{V} & $\{V_1^1,V_1^2,V_1^3\}$ & $\{V_2^1,V_2^2\}$& $\{V_3^1,V_3^2\}$& $\{V_4\}$\\ 
  \hline
   \textbf{deg(V)} & $n_1$ & $n_2$ & $n_3$ & $n_4$ \\  
   \hline
 \end{tabular}
 \end{center}
  Hier gilt $n_1,n_2,n_3,n_4 \in \mathbb{N}$.\\
  
%--------------------------------------
 \textbf{Prozedur $P^1$}
 \begin{itemize}
 \item Nach Anwenden des Cratercutters erhält man folgende Knotengerade.
 \begin{center}
\begin{tabular}{|c|c|c|c|c|}
\hline
  \textbf{V} & $\{V_1^1,V_1^2,V_1^3\}$ & $\{V_2^1,V_2^2\}$& $\{V_3^1,V_3^2\}$& $\{V_4\}$\\ 
  \hline
   \textbf{deg(V)} & $n_1$ & $n_2$ & $n_3$ & $n_4$ \\  
   \hline
 \end{tabular}
 \end{center}

%--------------------------------------
\item %\textbf{Prozedur 1 Schritt 2}
Durch Anwendung des Ripcutters ergibt sich:
\begin{center}
\begin{tabular}{|c|c|c|c|c|c|}
\hline
  \textbf{V} & $\{V_1^1,V_1^2,V_1^3\}$ & $\{V_2^1,V_2^2\}$& $\{V_3^1\}$ & $\{V_3^2\}$& $\{V_4\}$\\ 
  \hline
   \textbf{deg(V)} & $n_1$ & $n_2$ & 1 & $n_3-1$ & $n_4$ \\  
   \hline
 \end{tabular}
 \end{center}
%---------------------------------------
\item %\textbf{Prozedur 1 Schritt 3}
Nach dem Splitcut ergeben sich folgende Grade.
\begin{center}
\begin{tabular}{|c|c|c|c|c|c|c|c|}
\hline
  \textbf{V} & $\{V_1^1\}$ & $\{V_1^2,V_1^3\}$ & $\{V_2^1\}$ & $\{V_2^2\} $ & $\{V_3^1\}$ & $\{V_3^2\}$& $\{V_4\}$\\ 
  \hline
   \textbf{deg(V)} & 1 & $n_1-1$ &1& $n_2-1
   $ & 1 & $n_3-1$ & $n_4$ \\  
   \hline
 \end{tabular}
 \end{center}
 \end{itemize}
%---------------------------------------
\textbf{Prozedur $P^2$}
\begin{itemize}
\item Als Ergebnis des Ripcutters erhält man
\begin{center}
\begin{tabular}{|c|c|c|c|c|c|c|c|c|}
\hline
  \textbf{V} & $\{V_1^1\}$ & $\{V_1^2\}$ & $\{V_1^3\}$ & $\{V_2^1\}$ & $\{V_2^2\} $ & $\{V_3^1\}$ & $\{V_3^2\}$& $\{V_4\}$\\ 
  \hline
   \textbf{deg(V)} & 1 & 1 & $n_1-2$ &1& $n_2-1$ & 1 & $n_3-1$ & $n_4$ \\  
   \hline
 \end{tabular}
 \end{center}
%----------------------------------------
%\textbf{Prozedur 2 Schritt 2}
\item Zusammensetzen der Kanten durch den Ripmender erbringt
\begin{center}
\begin{tabular}{|c|c|c|c|c|c|c|c|}
\hline
  \textbf{V} & $\{V_1^1\}$ & $\{V_1^2,V_3^2\}$ & $\{V_1^3\}$ & $\{V_2^1\}$ & $\{V_2^2\} $ & $\{V_3^1\}$ & $\{V_4\}$\\ 
  \hline
   \textbf{deg(V)} & $1$ & $n_3$ & $n_1-2$ &1& $n_2-1$ & $1$ & $n_4$ \\  
   \hline
 \end{tabular}
 \end{center}
 \end{itemize}
%-----------------------------------------
\textbf{Prozedur $P^3$}
\begin{itemize}
\item Der Splitmender im ersten Schritt der Prozedur $P^3$ führt zu \begin{center}
\begin{tabular}{|c|c|c|c|c|c|c|}
\hline
  \textbf{V} & $\{V_1^1,V_4\}$ & $\{V_2^1,\{V_1^2,V_3^2\}\}$ & $\{V_1^3\}$ & $\{V_2^2\} $  & $\{V_3^1\}$ \\ 
  \hline
   \textbf{deg(V)} & $n_4+1$ & $n_3+1$ & $n_1-2$ & $n_2-1$ & $1$  \\  
   \hline
 \end{tabular}
 \end{center}
%_------------------------------------------
%\textbf{Prozedur 3 Schritt 2}
\item Nach Anwenden der Ripmenders erhält man \begin{center}
\begin{tabular}{|c|c|c|c|c|c|}
\hline
  \textbf{V} & $\{V_1^1,V_4\}$ & $\{V_2^1,\{V_1^2,V_3^2\}\}$ & $\{V_1^3,V_3^1\}$ & $\{V_2^2\} $   \\ 
  \hline
   \textbf{deg(V)} & $n_4+1$ & $n_3+1$ & $n_1-1$ & $n_2-1$  \\  
   \hline
 \end{tabular}
 \end{center}
%-----------------------------------------
%\textbf{Prozedur 3 Schritt 3}
\item Die Anwendung des Cratermenders liefert folgende Knotengrade.
\begin{center}
\begin{tabular}{|c|c|c|c|c|c|c|}
\hline
 \textbf{V} & $\{V_1^1,V_4\}$ & $\{V_2^1,\{V_1^2,V_3^2\}\}$ & $\{V_1^3,V_3^1\}$ & $\{V_2^2\} $   \\ 
  \hline
   \textbf{deg(V)} & $n_4+1$ & $n_3+1$ & $n_1-1$ & $n_2-1$  \\  
   \hline
 \end{tabular}
 \end{center}
 \end{itemize}
%-----------------------------------------
Für durch Lochwanderungen und Lochwanderungssequenzen entstandene simpliziale Flächen betrachtet man die nach dem ersten Prozess veränderten Grade der Knoten und verändert diese, für jede weitere Kante mit welcher der Prozess $P^2$ durchgeführt wird, wie oben vorgeschrieben.\\
Hier sei angemerkt, dass alle Knotengrade größer als zwei sind, da die durch die drei Prozeduren entstandene simpliziale Fläche wieder geschlossen ist, dass heißt es gilt $n_i \geq 2$ für $i=1,2,3,4$.\\

Nun wird die sogenannte Euler-Charakteristik einer simplizialen Fläche genauer betrachtet.
%\newpage

Mit den Bezeichnungen wie in der Beschreibung der obigen drei Prozeduren ist $P_F^1(X)=S^c_{\{e_3^1,e_3^2\}}(R^c_{\{e_2^1,e_2^2\}}(C^c_{\{ e_1^1,e_1^2\} }(X)))$ und somit gilt für die Euler-Charakteristik der simplizialen Fläche $P^1_F(X):$ 
\begin{gather*}
\chi (P^1_F(X))=\chi(S^c_{\{e_3^1,e_3^2\}}(R^c_{\{e_2^1,e_2^2\}}(C^c_{\{ e_1^1,e_1^2\} }(X))))\\
\chi(R^c_{\{e_2^1,e_2^2\}}(C^c_{\{ e_1^1,e_1^2\} }(X))))+1=\chi((C^c_{\{ e_1^1,e_1^2\} }(X)))+1\\
=\chi(X)-1+1=\chi(X).
\end{gather*}
Und durch analoges Vorgehen bei den anderen beiden Prozeduren erhält man dann auch $\chi(P_f^2(P_F^1(X)))=\chi(X)$ und dann schließlich 
\[
\chi(X^H_{(F,f)})=\chi(P^3_F(P_f^2(P_F^1(X))))=\chi(X).
\]
Also verändert sich beim Durchlaufen der drei Prozeduren die Euler-Charakteristik nicht. Wegen $\chi(P^2_f(Y))=\chi(Y)$, für  eine simpliziale Fläche $Y$ auf die man die Prozedur $P^2$ anwendet und $f \in H_f(Y)$ für ein $F\in Y_2$ , kann man, für die durch eine Lochwanderung $\sigma$ entstandene simpliziale Fläche $X^H_{\sigma}$
\[
\chi(X^H_{\sigma})=\chi(X)
\]
und dann schließlich
\[
\chi (X_{(\sigma_1,\ldots, \sigma_n)}^H)=\chi(X)
\]
für eine beliebige Lochwanderungssequenz $(\sigma_1,\ldots, \sigma_n)$ und $n\in \mathbb{N}$, schließen.
%--------------------------------
\begin{bemerkung}
Sei $(X,<)$ eine simpliziale Fläche.
Für $x\in X$, $m \in \mathbb{N}$ schreibt man $x=\{x_1,\ldots,x_m\}$ und dies wird wie folgend beschrieben interpretiert.\\ Man identifiziert $X$ mit der isomorphen simplizialen Fläche $Y \in \mathcal{M}(n \Delta)$, wobei $n\in \mathbb{N}$ die Anzahl der Flächen von $X$ ist. Es gilt $x_1,\ldots,x_m \in n\Delta$ und $\{x_1,\ldots,x_m\}\in Y$. Außerdem gilt $\beta(\{x_1,\ldots,x_m\})=x$ für einen  Isomorphismus $\beta: Y \mapsto X$. \\
Dabei soll hier weniger im Vordergrund stehen, wie man diese simpliziale Fläche konstruiert, sondern diese Notation soll zur Erleichterung der Anwendung der Mender- und Cutteroperatoren auf $X$ dienen.
\end{bemerkung}

\begin{bsp}
Für die simpliziale Fläche $(X,<)$ definiert durch das ordinale Symbol 
\[
\mu((X,<))=(4,5,2;(\{2,3\},\{1,3\},\{1,2\},\{2,4\},\{3,4\}),(\{1,2,3\},\{1,4,5\})
\]
schreibt man $V_1=\{V_1^1,V_1^2\}$, $e_1=\{e_1^1,e_1^2\}$ für $V_1\in X_0,e_1 \in X_1$ und $V_1^1,V_1^2 \in 2 \Delta_0$, $e_1^1,e_1^2 \in 2 \Delta_1$ und meint damit den Knoten und die Kante der simplizialen Fläche $2\Delta(\alpha)$, wobei die Mending Map $\alpha:2\Delta \mapsto X$ gegeben ist durch
\[
\alpha(x)= 
\begin{cases}
e_i & \text{für } x =e_i^j ,i=1,2,3,\,j=1,2\\
V_i &\text{für } x =V_i^j,i=1,2,3,\,j=1,2\\
F_j & \text{ für } x=F_j, j=1,2
\end{cases}
\]
\end{bsp}
%\newpage
\section{Simpliziale Flächen mit 2- und 3-Waists}
\subsection{2-Waists und 3-Waists }
\begin{definition}
Sei $(X,<)$ eine geschlossene simpliziale Fläche und $e_1,\ldots,e_n\in X_1$ mit $ e_i \neq e_j$ für $i \neq j$ und $n \geq 2$.
\begin{enumerate}
\item   Man nennt $(e_1,\ldots,e_n)$ einen \emph{Pfad von Kante $e_1$ zur Kante $e_n$}, falls es Knoten $V_1,\ldots,V_{n-1}\in X_0$ mit 
\[
V_i<e_i \text{ und } V_i<e_{i+1}
\] 
gibt.
Hierbei ist $n$ die Länge des Pfades.
\item Falls es in der obigen Definition einen Knoten $V_n$ mit $V_n<e_n$ und $V_n<e_1$, so nennt man den Pfad $(e_1,\ldots,e_n)$ geschlossen.
\item Einen geschlossenen Pfad der Länge 2 nennt man \emph{2-Waist} und einen geschlossenen Pfad der Länge 3 nennt man \emph{3-Waist}, falls 
\[
\vert \{e_1,e_2,e_3\}\cap X_1(F) \vert< 3\text{ für alle } F \in X_2 \text{ gilt},
\]
wobei $(e_1,e_2,e_3)$ ein geschlossener Pfad ist.
\end{enumerate}
\end{definition}

\begin{bemerkung}
Sei $(X,<)$ eine geschlossene simpliziale Fläche mit $\vert X_2 \vert > 2$, $V \in X_0$ und $\deg(V)=2$. Dann definiert der Knoten $V$ einen 2-Waist. Sei dazu $X_2(V)=\{F_1,F_2\}$ für $F_1,F_2 \in X_2$. Damit haben $F_1$ und $F_2$ genau zwei gemeinsame Kanten und jeweils eine Kante, die nicht zu der anderen Fläche adjazent ist, da $\vert X_2 \vert > 2$. Das heißt, es existieren $e_1,e_2\in X_1$, für welche $e_1<F_1,e_1 \nless F_2, e_2<F_1$ und $e_2  \nless F_2$ gilt. Somit ist durch $(e_1,e_2)$ ein 2-Waist gegeben.\\
Falls $\deg(V)=3$ und $\vert X_2 \vert>3$ ist, so definiert V einen 3-Waist. Dies kann man mit analoger Argumentation zeigen.
\end{bemerkung}
Mit obiger Bemerkung ist es leicht einzusehen, dass eine simpliziale Fläche, die einen Knoten vom Grad 2 enthält, ebenfalls einen 2-Waist enthalten muss. Die Umkehrung gilt jedoch nicht, was durch folgende Konstruktion erläutert wird.\\
 Seien $(T^1,<_1)$ und $(T^2,<_2)$ Tetraeder, wie zuvor definiert, wobei die Knoten bzw. Kanten bzw. Flächen von $T^i$ für $i\in \{1,2\}$ mit $V_1^i,V_2^i,V_3^i,V_4^i$, bzw. $e_1^i,e_2^i,e_3^i,e_4^i,e_5^i,e_6^i$ bzw. $F_1^i,F_2^i,F_3^i,F_4^i$ bezeichnet werden. Man nutzt wieder die Isomorphie von  $T^i$ zu einer simplizialen Fläche $Y\in \mathcal{M}(4 \Delta)$, beschrieben durch $e_3^1 =\{e_3^1,e_3^2\}\in T^1_1 $ und $e_3^2=\{{e_3^1}',{e_3^2}'\}\in T^2_1$ 
   und den zugehörigen Knoten $V_1^1,V_2^1 \in T^1_0$ und $V_1^2,V_2^2 \in T^2_0$  . Mithilfe dieser beiden Mendings konstruiert man sich die simpliziale Fläche $(X,<)$, mit den Knoten $X_0=T^1_0 \cup T^2_0$, Kanten $X_1=T^1_1 \cup T^2_1$, Knoten $X_2=T^1_2 \cup T^2_2$ und $x<y$ genau dann, wenn $x<_1y$ oder $x<_2 y$ ist. Dann führt man folgende Operationen auf X aus:
 \begin{enumerate}
 \item Wende den Cratercut $C^c_{\{e_{3}^1,e_{3}^2\}}$, um die Kanten $\{e_3^1\},\{e_3^2\}$ zu erhalten.
 \item Wende den Cratercut $C^c_{\{{e_{3}^1}',{e_{3}^2}'\}}$, um die Kanten $\{{e_3^1}'\},\{{e_3^2}'\}$ zu erhalten.
\item Wende den Splitmender $S^m_{(\{V_1^1\},\{e_3^1\}),(\{V_1^2\},\{{e_3^1}'\})}$ an, um die Kanten $\{e_3^1\}$ und $\{{e_3^1}'\}$ zu der Kante $\{e_1^1,{e_3^1}'\}$ zusammenzuführen und um die Knoten $\{V_1^1,V_1^2\}$ und $\{V_2^1,V_2^2\}$ zu erhalten.
 \item Wende nun den Cratermender $C^m_{\{e_3^2\},\{{e_3^2}'\}}$ an, um die Kante $\{e_3^2,{e_3^2}'\}$ zu erhalten.
 \end{enumerate}
 Die dadurch entstandene simpliziale Fläche $Z$ ist geschlossen und es gilt
 \[
 \{deg(V)\mid V\in Z_0\}=\{3,6\}.
 \] 
 Damit enthält $Z$ keinen Knoten vom Grad 2, jedoch bildet $(\{e_{3}^1,{e_{3}^1}'\},\{e_{3}^2,{e_{3}^2}'\})$ einen 2-Waist in $Z$. Auf ähnliche Art und Weise kann man zeigen, dass eine simpliziale Fläche einen 3-Waist enthalten kann, ohne einen Knoten vom Grad 3 zu haben.\\
 \subsection{Prozedur $W^2$}
Im Folgendem will man aus einer simplizialen Fläche mit $n$ Flächen, die keinen 2-Waist enthält, eine simpliziale mit $n+2$ Flächen und einem 2-Waist konstruieren. Hierfür macht man Gebrauch von dem zu Beginn definierten Open-Bag.
\begin{bemerkung}
Zur Erinnerung sei hier angemerkt, dass der Open-Bag $(B,<_B)$ durch das ordinale Symbol 
\[
\mu((B,<_B))=(3,4,2;(\{2,3\},\{1,3\},\{1,2\},\{1,3\}),(\{1,2,3\},\{1,3,4\}))
\] definiert wird.
\end{bemerkung}
 
Sei $(B,<_B)$ der Open-Bag und $(X,<)$ eine geschlossene simpliziale Fläche. Dann definiert man zunächst eine simpliziale Fläche $Z$ durch die Knoten $Z_0=X_0 \cup B_0$, die Kanten $Z_1=X_1 \cup B_1$ , die $Z_2=X_2 \cup B_2$ und $x<_Z y$ genau dann, wenn $x<y$ oder $x<_B y$.\\
Für eine Kante $f\in X_1 \subseteq Z_1$ und Knoten $V_1,V_2\in X_0\subseteq Z_0$ mit $V,V'<f$ identifiziert man $Z$ mit einem Mending, welches durch $f=\{f_1,f_2\}$ festgesetzt wird.
 Nun führt man folgende Operationen auf $Z$ durch:
\begin{itemize}
\item Man führt einen Cratercut $C_{\{f_1,f_2\}}^c$ durch, um die Kanten $\{f_1\}$ und $\{f_2\}$ erhalten.
\item Man wendet einen Splitmender $S^m_{(V,{f_1}),(V_1,e_2)}$ an, um die Kanten $\{f_1\}$ und $\{e_2\}$ zusammenzuführen.
\item Man wendet einen Cratermender $C_{\{e_4\},\{f_2\}}^m$ an, um die Kante $\{e_4,f_2\}$ zu erschaffen.
\end{itemize}
Dadurch erhält man eine geschlossene simpliziale Fläche, in welcher man den 2-Waist $(\{e_2,f_1\},\{e_4,f_2\})$ vorfinden kann. Man bezeichnet diese simpliziale Fläche mit \emph{$W^2_f(X)$}, falls durch die obige Prozedur ein 2-Waist an der Stelle $f$ konstruiert wurde.

 Nun stellt man sich die Frage, ob es möglich ist, durch Anwenden einer Lochwanderungssequenz auf $W^2_f(X)$ eine simpliziale Fläche zu konstruieren, die keinen 2-Waist mehr enthält. Hierfür stellt man folgende Vorüberlegung an.
 \begin{vor}\label{vor} $\textcolor{blue}{(neu)}$
 %$\textcolor{red}{safe noch machen}$
 Sei $(X,<)$ eine geschlossene simpliziale Fläche mit $\chi(X)=2$, $F,F',F_1,F_1',F_2,F_2'\in X_2$ Flächen und $V_1,V_2\in X_0$ Knoten in $X$ mit 
$N_{X}(F)=\{F',F_1,F_2\} $, $N_{X}(F')=\{F,F_1',F_2'\},$
 $ V_1 < F_F',F_1,F_1'$ und $V_1 < F_F',F_2,F_2'$.  Sei weiterhin $f\in X_1$ eine Kante mit $X_2(f)=\{F',F_1'\}$, also $f \in H_F(X)$.
 Dann gibt es Kanten $g_x \in X_1$ mit $ X_2(g_x) =\{x,G\}$ für $x \in \{F_1,F_1',F_2,F_2'\}$ und $G\in \{F,F'\}$.
 \[
bild 
 \] 
Man setzt nun $Y=X_{(F,f)}^H$. Wegen $N_Y(F)=\{F',F_1',F_2\}$ und $N_Y(F')=\{F',F_1,F_2'\}$ gibt es zu einer Kante $g^H_x\in Y_1$ mit $x \in \{F_1,F_1',F_2,F_2'\}$ eine Kante $g \in \{g_{F_1},g_{F_1'},g_{F_2},g_{F_2}'\}$ mit $x \in Y_2(g_x^H) \cap X_2(g_x)$. Außerdem gibt es für alle $V_x^H\in Y_1$ mit $V_x^H<g_x^H$ einen Knoten $V_x \in V_x \in X_0$ mit $V_x<g_x$, sodass $ X_2(V_x)\setminus \{F,F'\}=Y_2(V^H_x)\setminus \{F,F'\}$ gilt.
 Damit können wir aufgrund von den Kanten-Inzidenzen in der durch die Anwendung von Wanderinghole entstandenen simplizialen Fläche $Y$ auf Kanten-Inzidenzen in der ursprünglichen Fläche $X$ schließen. Beispielsweise kann man aufgrund der Existenz der Kante $h \in Y_1$ mit $Y_2(h)=\{F,F_1'\}$  auf die Existenz einer Kante $h' \in  X_1$ schließen, die entweder zu $F$ und $F_1'$ oder zu $F'$  und $F_1'$ inzident ist. Dies hängt davon ab, mit welcher Kante die Operation Wanderinghole durchgeführt wurde. 
 
 \[
bild 
 \]
 \end{vor}

\begin{satz}$\textcolor{blue}{(neu)}$
Sei $(X,<)$ eine simpliziale Fläche mit $\chi(X)=2$, die weder 2- noch 3-Waists enthält und $f\in X_1$ eine Kante in $Y$, wobei $Y=W^2_f(X)$ ist. Dann existiert eine simpliziale Fläche $(Z,\prec)\in \mathcal{H}_{Y}$, die weder 2- noch 3-Waists enthält.
\end{satz}
\begin{proof}
Sei $O_2=\{F_1^O,F_2^O\} \subseteq Y_2$ die Flächenmenge des Open-Bags. Es existiert eine Fläche $F_1 \in Y_2$ und eine Kante $f_1 \in Y_2$ so, dass $F_1$ und $F_1^O$ benachbarte Flächen sind und $f_1<F_1$ und $f_1 \nless F_1^O$ gilt. Damit ist $f_1 \in H_{F_1^O}(Y)$, also lässt sich mithilfe von Wanderinghole die simpliziale Fläche $Z=Y^H_{(F_1^O,f_1)}$ konstruieren. Man will nun zeigen, dass $Z$ eine simpliziale Fläche ohne 2- und 3-Waist ist. Angenommen dies ist nicht der Fall und man kann in $Z$ den 2-Waist $W=(g,h)$ mit $g,h\in Z_1$ vorfinden. Um diese Annahme zum Widerspruch zu führen, betrachtet man die Menge $M:=\{g,h\} \cap Z_1(\{F_1,F_1^O,F_2^O\})$. Offensichtlich ist $\vert M \vert \in\{0,1,2\}$.
\begin{enumerate}
\item  Der Fall $\vert M \vert=2$ ist nach Konstruktion nicht möglich, da $\vert Z_0(f) \cap Z_0(g)\vert \leq 1$ für alle $f,g \in Z_1(\{F_1,F_1^O,F_2^O\})$ mit $f \neq g$ gilt. 
\item Falls $\vert M \vert=0$ ist, dann gilt für $g,h \in Z_1$ schon $g,h \in X_1$. Das heißt der 2-Waist $(g,h)$ ist schon in $X$ enthalten. Und dies ist ein Widerspruch dazu, dass man $X$ als eine simplizale Fläche ohne 2-Waist angenommen hat.
\item 
 Falls $\vert M \vert=1$ ist, so sei ohne Einschränkung $g \in M$ und $h \notin M$, damit gilt schon $h \in X_1$. Zunächst einmal ist nach Konstruktion $Z_2(g)\neq \{F_1,F_1^O\}$, denn für $f\in Z_1$ mit $Z_2(f)=\{F_1,F_1^O\}$ und für alle Kanten $\tilde{f} \in Z_1\setminus \{f\}$ gilt $\vert Z_0(f) \cap Z_0(\tilde{f})\vert \leq 1. $ Das heißt mithilfe der \Cref{vor} existiert ein $g_Y \in Y$ so, dass $(g_Y,h)$ einen 2-Waist in $Y$ bildet. Falls $Y_2(g_Y)=\{F_2,F_2^O\}$ ist, dann bildet $(f,h)$ einen 2-Waist in $X$. Andernfalls gilt schon $g_Y \in X_1, und (g_Y,h)$ ist ein 2-Waist in $X$.
\end{enumerate}
Also ist $Z$ eine simpliziale Fläche ohne 2-Waist. Also bleibt nur noch zu zeigen, dass $Z$ keinen 3-Waist enthalten kann. Dies zeigt man wieder über einen Widerspruchsbeweis, indem man sich die Menge $M':=\{g,h,i\} \cap Z_1(\{F_1,F_1^O,F_2^O\})$ zu Hilfe nimmt, wobei hier $W'=(g,h,i)$ der 3-Waist in $Z$ ist. Es gilt $\vert M' \vert \in \{0,1,2,3\}$.
\begin{enumerate}
\item Der Fall $\vert M \vert=3$ ist nach Konstruktion nicht möglich, da falls $\vert Z_0(g) \cap Z_0(h)\cap Z_0(i)\vert =3 $ erfüllt ist,  direkt $g,h,i <F$ für ein $F \in Z_2$ folgt. Für einen 3-Waist $(g,h,i)$ muss aber $Z_1(F) \neq \{g,h,i\} $ für alle $F \in Z_2$ gelten.
\item Falls $\vert M' \vert=0$ ist, dann folgt schon $g,h,i \in X_1$ und somit ist $X$ eine simpliziale Fläche mit einem 3-Waist.
\item  Falls $M'=\{g\}$ ist, dann ist $g$ nach Konstruktion eine Kante mit der Eigenschaft $ Z_2(g) \neq \{F_1,F_1^O\}$. Damit kann man mithilfe der \Cref{vor} auf eine Kante $g_Y\in Y_1$ schließen, sodass $(g_Y,h,i)$ ein 3-Waist in $Y$ ist. Falls $Y_2(g_Y)=\{F_2,F_2^O\}$ ist, dann bildet $(f,h,i)$ einen 3-Waist in $X$. Andernfalls gilt schon $g_Y \in X_1, und (g_y,h,i)$ ist ein 3-Waist in $X$.
\item Angenommen es gilt $  M' =\{g,h\}$. Dann ist $\vert Y_0(g) \cap Y_0(h)\vert =1$ und per Konstruktion gilt $Z(g)\neq Z(h) \neq \{F_1,F_1^O\}$. Damit existieren nach \Cref{vor} die Kanten $g_Y,h_Y \in Y_1$ und der Knoten $V_Y\in Y_0$ mit $V_Y<g_Y,h_Y$. Falls $Y_2(g_Y)=\{F_2,F_2^O\}$ ist, dann bildet $(f,g_Y,h)$ einen 3-Waist in $X$. Ebenso falls $Y_2(h_Y)=\{F_2,F_2^O\}$ ist, dann bildet $(g_Y,f,h)$ einen 3-Waist in $X$. Wenn $Z(g_Y)\neq Z(h_Y) \neq \{F_2,F_2^O\}$ gilt, dann sind $g_Y,h_Y$ bereits Kanten in $X$ und somit auch $(g_Y,h_Y,i)$ ein 3-Waist in $X$.\end{enumerate}
\end{proof}
\subsection{Prozedur $W^3$}
Nun soll aus einer simplizialen Fläche mit $n$ Flächen, eine simpliziale Fläche konstruiert werden, die aus $n+2$ Flächen besteht und einen 3-Waist enthält. Hierfür verwendet man die folgende simpliziale Fläche.

\begin{definition}
Man definiert die simpliziale Fläche \emph{Hut} $(H,<_{H})$ durch die Knoten $H_0=\{V_1,V_2,V_3,V_4\}$, die Kanten $H_1=\{e_1,e_2,e_3,e_4,e_5,e_6\}$ und die Flächen $H_2=\{F_1,F_2,F_3\}$. Die Inzidenz $<$ erhält man durch das ordinale Symbol
\begin{align*}
\mu((H,<_H))=(4,6,3;\{2,3\},\{1,3\},\{2,3\},\{3,4\},\\
\{1,4\},\{2,4\},
 \{1,2,3\},\{2,4,5\},\{3,5,6\}).
\end{align*}
\end{definition}

Für eine geschlossene simpliziale Fläche $(X,<)$ und $F \in X_2$ kontruiert man zunächst die simpliziale Fläche $Y$, die gegeben ist durch $Y_2=P^1_F(X)_2\setminus \{F\}$, $Y_1=P^1_F(X)_1\setminus \{e \in P^1_F(X)_1 \mid e<F\}$ und $Y_0=P^1_F(X)_0\setminus \{V \in P^1_F(X)_0 \mid V<F\}$. Dadurch entstehen die Randkanten $r_1,r_2,r_3 \in Y_1$ mit zugehörigen Knoten $W_1,W_2,W_3\in Y_0$ so, dass 
\[
W_i < r_j \text{ für } j\in \{1,2,3\},i\in\{1,2,3\}\setminus \{j\}
\] gilt.\\
Diese Fläche nutzt man, um die simpliziale Fläche $Z$ zu erhalten, welche durch $Z_0=Y_0 \cup H_0$, $Z_1=Y_1 \cup H_1$, $Z_2=Y_2 \cup H_2$ und $x<_Z y$ genau dann, wenn $x <_Y y$ oder $x <_H y$ definiert ist, wobei mit $<_Y$ die Inzidenz auf $Y$ gemeint ist. Nun führt man folgende Operationen auf $Z$ aus:
\begin{enumerate}
\item Führe einen Splitmender $S^m_{(W_2,r_1),(V_4,e_1)}$ durch, um die Kante $\{e_1,r_1\}$ und Knoten $\{V_4,W_2\}$ und $\{V_1,W_3\}$ zu erhalten.
\item Wende ein Ripmender $R^m_{\{e_4\}\{r_3\}}$ an, um die Kante $\{e_4,r_3\}$ und den Knoten $\{V_3,W_1\}$ zu erhalten. 
\item Wende schließlich einen Cratermender $C^m_{e_6,r_2}$, um die Kante $\{e_6,r2\}$
\end{enumerate}
Dadurch erhält man eine simpliziale Fläche mit $\vert X_2\vert +2$ Flächen, in welcher man den 3-Waist $(\{e_1,r_1\},\{e_4,r_3\},\{e_6,r_2\})$ vorfindet. Diese bezeichnet man mit \emph{$W_F^3(X)$}, falls durch Anwenden der obigen Prozedur die Fläche $F$ entfernt und an dieser Stelle der Hut angesetzt wurde, um den 3-Waist zu kontruieren.\\
Hier stellt sich wiederum die Frage, ob es eine Lochwanderungssequenz gibt, sodass man aus $W_F^3(X)$ durch Anwenden dieser eine simpliziale Fläche kreiert, die keinen 2-Waist enthält. Diese Fragestellung wird im folgendem Satz thematisiert.
\begin{satz} $\textcolor{blue}{(neu)}$
Sei $(X,<)$ eine geschlossene simpliziale Fläche mit $\chi(X)=2$, die weder 2- noch 3-Waists enthält, $F \in X_2$ eine Fläche in $X$ und $Y=W^2_F(X)$. Dann existiert eine Lochwanderungssequenz $\Sigma$ so, dass $Y_{\Sigma}^H$ eine simpliziale Fläche ohne 2- und 3-Waists ist.
\end{satz}
\begin{proof} $\textcolor{blue}{(neu)}$
  Für $\vert X_2 \vert =2$ ist die Behauptung klar, denn dann ist $W^3_F(X)$ zum Tetraeder isomorph. Sei also $\vert X_2 \vert >2$. Sei $T_2=\{F_1^T,F_2^T,F_3^T\} \subseteq Y_2$ die Flächenmenge des Tetraeders. Dann existiert zu $F_1^T$ eine Fläche $F_1 \in Y_2$ und eine Kante $f_1 \in Y_1$, so dass  $F_1^T$ und $F_1$ adjazent sind und $f_1<F_1$ und $f_1 \nless F_1^T$ gilt. Damit ist $f_1 \in H_{F_1^T}(X)$, also kann man mithilfe von Wanderinghole die simpliziale Fläche $Z=Y^H_{(F_1^T,f_1)}$ konstruieren. Man will nun zeigen, dass $Z$ eine simpliziale Fläche ohne 2- und 3-Waists ist. Um zu zeigen, dass $Z$ keinen 2-Waists enthält, nimmt man die Existenz eines 2-Waists $W=(g,h)$ für $g,h\in Z_1$ an und untersucht die Menge $M=\{g,h\} \cap Z_1(\{F_1,F_1^T,F_2^T,F_3^T\})$.
  \begin{enumerate}
  \item Falls $\vert M \vert=0$ ist, so folgt direkt die Existenz eines 2-Waists in der simplizialen Fläche $X$. 
  \item Der Fall $\vert M\vert=2$ ist nach Konstruktion nicht möglich, da $\vert Z_0(g) \cap Z_0(h) \vert \leq 1$ gilt. Für einen 2-Waist $(g,h)$ muss aber $\vert Z_0(g) \cap Z_0(h) \vert=2$ gelten.
  \item Also muss nur noch der Fall $\vert M \vert=1$ betrachtet werden. 
  Sei also $M=\{g\}$. Wegen $\vert M' \vert=1$ muss schon $Z_2(g) \cap \{F_1,F_1^T,F_1^T,F_3^T\}=\{F\}$ für ein $F \in X_2$ gelten.
   Falls $g<F_i^T$ mit $i\in \{2,3\}$ ist dann existiert offensichtlich eine Kante $g'\in Y_2$, sodass $(g' ,h)$ einen 2-Waist in $Y$ bildet. 
   Daraus kann man wieder folgern, dass es eine Kante  $g''\in X_1$ gibt, sodass $(g'',h)$ ein 2-Waist in $X$ ist. Mit analoger Argumentation und der \Cref{vor} erhält man ebenfalls einen 2-Waist in $X$, falls $g<F_1$ oder $g<F_1^T$ ist.  \end{enumerate}
  Damit ist $Z$ eine simpliziale Fläche ohne 3-Waist. Bleibt zu zeigen, dass es in dieser auch keinen 2-Waist geben kann.\\
  Betrachte die Menge $M'=\{f,g,h\} \cap Z_1(\{F_1,F_1^T,F_2^T,F_3^T\})$ für den Fall, dass $(f,g,h)$ einen 3-Waist in $Z$ bildet.
  \begin{enumerate}
  \item Der Fall $\vert M'\vert =3$ ist nach Konstruktion nicht möglich, da aus $\vert Z_0(\{f,g,h\})\vert=3$ folgt, dass $Z_1(F)=\{f,g,h\}$ für ein  $F\in Z_2$ gilt. 
  \item Falls $\vert M'\vert =0 $ ist, so gilt bereits $f,g,h \in X_1$ und somit ist auch $(f,g,h)$ ein 3-Waist in $X$.
  \item Falls $M'=\{f\}$ für ein $f \in Z_1$ ist, dann ist $x\in Z_2(f)$ mit $x\in \{F_1,F_1^T,F_2^T,F_3^T\}$. Für den den Fall, dass $x\in \{F_2^T,F_3^T\}$ gilt, folgt die Existenz einer Kante in $f_X\in X_1$, sodass $(f_X,g,h)$ einen 3-Waist bildet direkt. Falls $x\in \{F_1,F_1^T\}$ gilt, so kann man mithilfe der \Cref{vor} die Existenz einen 3-Waists in $Y$ folgern und dann schließlich die existenz eines 3-Waists in $X$. 
  \item Angenommen es gilt $M'=\{f,g\}$. Dann folgt direkt schon $h \in X_1$ und es existiert ein Knoten $V \in Z_0$, sodass $V\in Z_0(f)\cap Z_0(g)$ ist.
   Und außerdem gilt per Kontruktion $\vert Z_(f)\cap \{F_1,F_1^T,F_2^T,F_3^T\}\vert=\vert Z_(g)\cap \{F_1,F_1^T,F_2^T,F_3^T\}\vert=1$.
    Mithilfe der \Cref{vor} folgt dann, dass es einen Knoten $V'\in Y_0$ und Kanten $g',h' \in Y_1$ mit $V'<g',h'$ gibt so, dass $(f',g',h)$ einen 2-Waist in $Y$ mit $\vert Y_2(f') \cap \{F_1^T,F_2^T,F_3^T\}\vert =1$ und $Y_2(g') \cap \{F_1^T,F_2^T,F_3^T\}\vert=1$, bildet.
     Aufgrund dessen findet man aber dann $f'',g''\in X_1$, sodass $(f'',g'',h)$ einen 3-Waist in $X$ bildet.
  \end{enumerate}
  \end{proof}

Man betrachtet nun die Euler-Charakteristik der oben konstruierten simplizialen Flächen. Seien dazu wieder $(X,<)$ eine geschlossene simpliziale Fläche, $f\in X_1$ eine Kante und $F\in X_2$ eine Fläche.
\begin{itemize}
\item Durch die Prozedur $W^2$ erhält man zunächst die simpliziale Fläche $Z$, welche die disjunkte Vereininung von X und dem Open-Bag $B$ darstellt. Deshalb ist $\vert Z_i\vert= \vert X_i \vert+\vert B_i\vert$ für $i=0,1,2$ und damit auch $\chi(Z)=\chi(X)+\chi(B)=\chi(X)+1.$ Daraufhin geht $W^2_f(X)$ durch Anwenden von einem Cratercutter, einem Splitmender und einem Cratermender aus $Z$ hervor. Damit gilt mit den obigen Bezeichnungen 
\begin{align*}
\chi(W^2_f(X))&=\chi(C_{\{e_4\},\{f_2\}}^m(S^m_{(W_2,r_1),(V_4,e_1)}(C_{\{f_1,f_2\}}^c(Z))))\\
&=\chi(S^m_{(W_2,r_1),(V_4,e_1)}(C_{\{f_1,f_2\}}^c(Z)))+1\\
&=\chi(C_{\{f_1,f_2\}}^c(Z))\\
&=\chi(Z)-1\\
&\chi(X)
\end{align*}

\item Beim Anwenden der Prozedur $W^3$ erhält man auch zunächst eine simpliziale Fläche $Z$ als disjunkte Vereinung der geschlossenen simplizialen Fläche $X$ und dem oben definierten Hut $H$, das heißt $\vert Z_i\vert= \vert X_i \vert+\vert H_i\vert$ für $i=0,1,2$ und damit auch $\chi(Z)=\chi(X)+\chi(H)=\chi(X)+1.$ Durch entfernten der Fläche $F$ von Z erhält man die simpliziale Fläche $Y$ mit $\vert Y_0\vert= \vert Z_0 \vert-3,\vert Y_1\vert= \vert Z_1 \vert-3$ und $\vert Y_2\vert= \vert Z_2 \vert-1$. Dadurch kann man 
\begin{align*}
\chi(Y)&=\vert Y_0 \vert-\vert Y_1 \vert+\vert Y_2 \vert\\
&=\vert Z_0 \vert-3-(\vert Z_1 \vert-3)+\vert Z_2 \vert-1\\
&=\chi(Z)-1\\
&=\chi(X)
\end{align*}
schließen. Schließlich erhält man $W^3_F(X)$ durch Anwenden von den Operatoren Splitmender, Ripmender und Cratermender auf $Y$. Mit den obigen Bezeichnungen ist dann 
\begin{align*}
\chi(W_F^3)&=\chi(C^m_{\{e_6\},\{r_2\}}(R^m_{\{e_4\},\{r_3\}}(S^m_{(\{W_2\},\{r_1\}),(\{V_4\},\{e_1\})}(Y))))\\
&=\chi(R^m_{\{e_4\},\{r_3\}}(S^m_{(\{W_2\},\{r_1\}),(\{V_4\},\{e_1\})}(Y)))+1\\
&=\chi(S^m_{(\{W_2\},\{r_1\}),(\{V_4\},\{e_1\})}(Y))+1\\
&=\chi(Y)\\
&=\chi(X)
\end{align*}
\end{itemize}
Das heißt durch das Anwenden der Prozeduren $W^2$ und $W^3$ wird die Euler-Charakteristik einer simplizialen Fläche nicht verändert.
\newpage
\section{Weitere Konstruktionen}
\subsection{Vereinigung Simplizialer Flächen} 
Bisher wurde die Erweiterung einer simplizialen Fläche um den Open-Bag bzw. Tetraeder mit einer fehlenden Fläche skizziert. Dies will man nun verallgemeinern und führt zu diesem Zweck die Vereinigung von simplizialen Flächen ein.
\begin{definition} Seien $(X^1,<_1), \ldots,(X^n,<_n)$ simpliziale Flächen für ein $n \in \mathbb{N}$. Dann definiert man die simpliziale Fläche $Z=\biguplus\limits_{i=1}^{n} X^i$ durch
die Knoten 
\[
Z_0= Z=\bigcup\limits_{i=1}^{n} (X^i)_0,
\]
die Kanten 
\[
Z_1=\bigcup\limits_{i=1}^{n} (X^i)_1
\]
und die Flächen 
\[
Z_2=\bigcup\limits_{i=1}^{n} (X^i)_2.
\]
Weiterhin sei $<_Z$ die Inzidenz der simplizialen Fläche $Z$. Dann gilt $x<_{Z} y$ für $x,y \in Z$ genau dann, wenn $x<y$ in $X^i$ gilt, wobei $i \in \{1,\ldots,n\}$ ist. Man nennt diese simpliziale Fläche $Z$, die \emph{aus $X^1,\ldots,X^n$ zusammengesetzte simpliziale Fläche}.
\end{definition}
\begin{bemerkung}
 Seien $(X^1,<_1), \ldots,(X^n,<_n)$ simpliziale Flächen für ein $n \in \mathbb{N}$. 
%\begin{enumerate}
 Für eine wie in obiger Definition aus $X^1,\ldots,X_n$ zusammengesetzte simpliziale Fläche $Z$ gelten folgende Aussagen
\begin{itemize}
\item Für $i=0,1,2$ ist $\vert Z_i\vert= \sum_{j=1}^n \vert (X^j)_i\vert$. 
\item  Es gilt die Gleichheit $\chi(Z)=\sum_{j=1}^n \chi(X^j)$, denn \begin{align*}
&\chi(Z)\\
=&\vert Z_0 \vert-\vert Z_1 \vert +\vert Z_2 \vert \\
=&\sum_{j=1}^n \vert (X^j)_0\vert-\sum_{j=1}^n \vert (X^j)_1\vert+\sum_{j=1}^n \vert (X^j)_2\vert\\
=&\sum_{j=1}^n \vert (X^j)_0\vert-\vert (X^j)_1\vert+\vert (X^j)_2\vert\\
=&\sum_{j=1}^n \chi(X^j).
\end{align*}
\item Falls $X^1,\ldots,X^n$ geschlossene simpliziale Flächen sind, so ist auch Z geschlossen.
\item $Z$ ist nicht zusammenhängend, denn es gilt 
\begin{align*}
&\vert \{ U \mid \text{U ist Zusammenhangskomponente von Z } \} \vert \\
=\sum_{i=1}^n &\vert \{U' \mid U' \text{ ist Zusammenhangskomponente von $X^i$ } \}\vert \geq  n
\end{align*}
\item Falls $X^1= \ldots =X^n$  gilt, so schreibt man auch $n X^1 :=\biguplus\limits_{i=1}^{n} X^i$.
\end{itemize}
%\end{enumerate}
\end{bemerkung}

Nun sollen die Konstruktionen, die in den Prozeduren $W^2$ und $W^3$ beschrieben wurden, verallgemeinert werden. Hierbei beschränkt man sich aber zunächst auf den Fall, dass man zwei geschlossene simpliziale Flächen durch einen 2-Waist bzw. 3-Waist zu einer geschlossenen simplizialen Fläche verbinden will. Denn der allgemeinere Fall folgt dann induktiv.
\subsection{Verallgemeinerung der Prozedur $P^3$}
Seien deshalb $(X,<)$ und $(Y,\prec)$ geschlossene simpliziale Flächen mit $ \vert X_2\vert , \vert Y_2\vert \geq 4$, $F_X \in X_2$ und $F_Y \in Y_2$. Dann bildet man die simplizialen Flächen $X'=P^1_{F_X}(X)\setminus D_X$ und $Y'=P^1_{F_X}(X)\setminus D_{Y}$, wobei die Mengen $D_X$ und $D_Y$ definiert sind durch
$D_X:=\{F_X\} \cup \{x \mid x <F_X\}$ und
$D_Y:=\{F_Y\} \cup \{x \mid x<F_Y\}$.
Dadurch erhält man die aus $X'$ und $Y'$ zusammengesetzte simpliziale Fläche $Z=X' \uplus Y'$ mit den Eckknoten $V_i^{X},V_i^{Y}$ und den Randkanten $f_i^{X},f_i^{Y}$ mit den zugehörigen Inzidenzen 
\[
V^x_i < f^x_j \text{ für } j\in \{1,2,3\},i\in\{1,2,3\}\setminus \{j\}, x\in \{X,Y\}.
\]
Nun führt man folgende Operationen auf $Z$ aus:
\begin{enumerate}
\item Führe einen Splitmender $S^m_{(V_1^X,f_3^X),(V_1^Y,f_3^Y)}$ durch, um die Kante $\{f_3^X,f_3^Y\}$ und Knoten $\{V_1^X,V_1^Y\}$ und $\{V_2^X,V_2^Y\}$ zu erhalten.
\item Wende ein Ripmender $R^m_{f_2^X, f_2^Y}$ an, um die Kante $\{f_2^X ,f_2^Y\}$ und den Knoten $\{V_3^X,V_3^Y\}$ zu erhalten. 
\item Wende schließlich einen Cratermender $C^m_{f_1^X,f_1^Y}$, um die Kante $\{f_1^X,f_1^Y\}$ zu erhalten.
\end{enumerate}
Hierdurch entsteht eine geschlossene simpliziale Fläche, in welcher man den 3-Waist $(\{f_1^X,f_1^Y\},\{f_2^X,f_2^Y\},\{f_3^X,f_3^Y\})$ vorfindet. Man bezeichnet die entstandene simpliziale Fläche mit $W^3_{F_X,F_Y}(X,Y)$. Falls $Y=H$ ist, wobei $H$ den oben definierten Hut beschreibt, so ist $W^3_{F_X,F_Y}(X,Y)=W^3_{F_X}(X)=W^3_{F_X,F_Y}(Y,X)$.
\begin{folgerung} $\textcolor{blue}{(neu)}$
Für simpliziale Flächen $(X,<)$ und $(Y, \prec)$ mit $\chi (X)= \chi (Y) =2$, die beide weder 2- noch 3-Waists enthalten, und $F\in X_2 ,F' \in Y_2$ gibt es eine Lochwanderungssequenz $ \Sigma$ so, dass $W^H_{\Sigma}$ eine simpliziale Fläche ist, in der keine 2-Waists und 3-Waists existieren, wobei hier $W=W^3_{F,F'}(X,Y)$ ist. 
\end{folgerung} 
 
\begin{proof}
$\textcolor{blue}{(neu)}$ 
Falls $X=J$ ist, wobei $J$ der Janus-Kopf ist, dann ist die Behauptung klar, denn dann gilt $W_{F,F'}^3(X,Y) \cong Y$ für $F\in X_2$ und $F' \in Y_2$. Somit ist $W_{F,F'}^3(X,Y)$ eine simpliziale Fläche ohne 2-Waists und 3-Waists.
Seien also nun weder $X$ noch $Y$ zum Janus-Kopf isomorph. Dann ist $\vert X \vert ,\vert Y \vert>2$ und es existiert eine zu $F$ adjazente  Fläche $F^X\in X_2$. Da $X$ eine simpliziale Fläche ohne 2-Waists ist, gilt $\vert N_X(F^X)\vert =3$ und daher gibt es ein $F\neq F_2^X \in N_X(F_1^X)\subseteq X_2$ und eine Kante $f^X \in X_1$  so, dass $f \in H_X(F_1^X)$ ist. Dann ist also $f  \in H_Z(F_1^X)$, wobei hier $Z=W_{F,F'}^3(X,Y)$ ist. Man konstruiere nun die simpliziale Fläche $Z^H_{(F_X^1,f^X)}$ und zeigt mit ähnlichen Argumenten wie in obigen Beweisen die Aussage.
\end{proof} 
$\textcolor{blue}{(neu)}$
Bei Verwendung der gleichen Bezeichnungen wie in der obigen Konstruktion ergibt sich für die Euler-Charakteristik folgendes:
\begin{align*}
&\chi(C^m_{f_1^X,f_1^Y}(R^m_{F_2^X F_2^Y}(S^m_{(V_1^X,f_3^X),(V_1^Y,f_3^Y)}(Z)))\\
=&\chi(R^m_{F_2^X F_2^Y}(S^m_{(V_1^X,f_3^X),(V_1^Y,f_3^Y)}(Z)))+1\\
=&\chi(S^m_{(V_1^X,f_3^X),(V_1^Y,f_3^Y)}(Z)))+1\\
=&\chi(Z)-1+1\\
=&\chi(X')+\chi(Y')\\
=&\chi(P^1_{F_X})+\chi(P^1_{F_Y})-2\\
=&\chi(X)+ \chi(Y)-2
\end{align*}

  \subsection{Verallgemeinerung der Prozedur $P^2$}
Seien nun wieder $(X,<)$ und $(Y,\prec)$ geschlossene simpliziale Flächen, $f_X\in X_1$ eine Kante in $X$ mit zugehörigen Knoten $V^X,W^X \in X_0$ und $f_Y\in Y$ eine Kante in $Y$ mit zugehörigen Knoten $V^X,V^Y\in Y_0$. Zunächst führe man $X$ und $Y$ zu der simplizialen $Z=X \uplus Y$ zusammen und nutzt an dieser Stelle wieder die Isomorphie von $Z$ zu einer simplizialen Fläche, die das Anwenden der Mender- und Cutteroperatoren erleichtert, nämlich die Fläche festgesetzt durch $f^X=\{f_1^X,f^X_2\}$ und $f^Y=\{f_1^Y,f^Y_2\}$. Nun führt man folgende Operationen auf $Z$ aus:
\begin{enumerate}
 \item Wende einen Cratercut $C^c_{\{f_1^X,f_2^X\}}$ an, um die Kanten $\{f_1^X\},\{f_2^X\}$ zu erhalten.
 \item Wende einen Cratercut $C^c_{\{f_1^Y,f_2^Y\}}$ an, um die Kanten $\{f_1^Y\},\{{f_2^Y}\}$ zu erhalten.
\item Wende einen Splitmender $S^m_{(\{V^X\},\{f_1^X\}),(\{V^Y\},\{f_1^Y\})}$ an, um die Kanten $\{f_1^X\}$ und $\{{f_1^X}\}$ zu der Kante $\{f_1^X,f_1^Y\}$ zusammenzuführen und um die Knoten $\{V^X,V^Y\}$ und $\{W^X,W^Y\}$ zu erhalten.
 \item Wende nun den Cratermender $C^m_{\{f_2^X\},\{{f_2^Y}\}}$ an, um die Kante $\{f_2^X,f_2^Y\}$ zu erhalten.
 \end{enumerate}
 
 \begin{folgerung} $\textcolor{blue}{(neu)}$
 Für simpliziale Flächen $(X,<)$ und $(Y,\prec)$ simpliziale Flächen mit $\chi(X)=\chi(Y)=2,$ die beide weder 2- noch Waists enthalten und $f\in X_1,f'\in Y_1$ gibt es eine Lochwanderungssequenz $\Sigma$ so, dass $Y^H_{\Sigma}$ eine simpliziale Fläche ist, die weder 2- noch 3-Waists enthält, wobei $Y=W^2_{f,f'}(X,Y)$ ist.
 \end{folgerung}
Ein Beweis wird hier jedoch nicht angeführt. Die Beweisstruktur ist jedoch ähnlich, wie die der Beweise der vorherigen Sätze und Folgerungen. Man konstruiert durch Anwenden von Wanderinghole auf die simpliziale Fläche $W^2_{f,f'}(X,Y)$ eine simpliziale Fläche von der man die Behauptung zeigen will. 
$ \textcolor{blue}{(neu)}$\\
Beim Betrachten der Euler-Charakteristik der simplizialen Fläche $W_2(X,Y)$, wobei hier die Bezeichnungen wie in der obigen Konstruktion übernommen werden, erkennt man 
\begin{align*}
&\chi(W^2_f(X,Y))\\
=&\chi(C^m_{\{f_2^X\}\{{f_2^Y}\}}(S^m_{(\{V^X\},\{f_1^X\}),(\{V^Y\},\{f_1^Y\})}(C^c_{\{f_1^Y,f_2^Y\}}(C^c_{\{f_1^X,f_2^X\}}(Z)))))\\
=&\chi(S^m_{(\{V^X\},\{f_1^X\}),(\{V^Y\},\{f_1^Y\})}(C^c_{\{f_1^Y,f_2^Y\}}(C^c_{\{f_1^X,f_2^X\}}(Z))))+1\\
=&\chi(C^c_{\{f_1^Y,f_2^Y\}}(C^c_{\{f_1^X,f_2^X\}}(Z)))+1-1\\
=&\chi(C^c_{\{f_1^X,f_2^X\}}(Z))-1\\
=&\chi(Z)-2=\chi(X)+\chi(Y)-2.
\end{align*}
\newpage
\section*{Anhang}

 %\docite{*}\bibliography{literatur}
%\bibliographystyle{plain}
\end{document}

