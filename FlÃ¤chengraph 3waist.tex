\documentclass[12pt,titlepage,twoside,cleardoublepage]{article}
\usepackage[ngerman]{babel}
\usepackage[utf8]{inputenc}
\usepackage[a4paper,lmargin={4cm},rmargin={2cm},
tmargin={2.5cm},bmargin = {2.5cm}]{geometry}
\usepackage{amsmath}
\usepackage{amssymb}
\usepackage{pdfpages} 
%\usepackage[pdftex,article]{geometry}
\usepackage{amsthm}
%\usepackage{ngerman,amsthm}
\usepackage{lineno} 
\usepackage{lineno, blindtext} 
\usepackage{cleveref}
\usepackage{enumerate}
\usepackage{float}
\usepackage{thmtools}
\usepackage{tabularx}
\linespread{1.25}
\usepackage{color}
\usepackage{verbatim}
\newcommand{\gelb}{0.550000011920929}
\usepackage{pgf,tikz,pgfplots}
\pgfplotsset{compat=1.15}
\usepackage{mathrsfs}
\usepackage{mathrsfs}
\usetikzlibrary{arrows}
%\numberwithin{equation}{chapter}
%\usepackage{scrheadings}
\pagestyle{headings}
\usepackage{titlesec}     
\usepackage{tikz}           % für Kontrolle der Abschnittüberschriften
\begin{comment}
\makeatother
\theoremstyle{nummermitklammern}
\theorembodyfont{\rmfamily}
\theoremsymbol{\ensuremath{\diamond}}
\newtheorem{temp}{}[section]
\newtheorem{vor}[temp]{Vorüberlegung}
\newtheorem{lemma}[temp]{Lemma}
\newtheorem{folgerung}[temp]{Folgerung}
\newtheorem{bsp}[temp]{Beispiel}
\newtheorem{herleitung}[temp]{Herleitung}
\newtheorem{definition}[temp]{Definition}
\newtheorem{bemerkung}[temp]{Bemerkung}
\newtheorem{satz}[temp]{Satz}
\newtheorem{beweisidee}[temp]{Beweisidee}
\theoremsymbol{\ensuremath{\square}}
\end{comment}
%\begin{comment}
\newtheorem{zahl}{}[section]
%\setcounter{zahl}{1}
%\newtheorem{section}{section}[section]
\newtheorem{definition}[zahl]{Definition}
\newtheorem{vor}[zahl]{Vorüberlegung}
\newtheorem{lemma}[zahl]{Lemma}
\newtheorem{folgerung}[zahl]{Folgerung}
\newtheorem{bsp}[zahl]{Beispiel}
\newtheorem{herleitung}[zahl]{Herleitung}
\newtheorem{bemerkung}[zahl]{Bemerkung}
\newtheorem{satz}[zahl]{Satz}
\newtheorem{beweisidee}[zahl]{Beweisidee}
\numberwithin{equation}{section}


%-----------------------------------------------

%\end{comment}
 %Nummerierung mit Kapitelnummern
%-------------------------
%\newcommand{\secnumbering}[1]{% 
 % \setcounter{chapter}{0}% 
  %\setcounter{section}{0}% 
  %\renewcommand{\thechapter}{\csname #1\endcsname{chapter}.}% nach Duden gehört 
                                  % der Punkt hier hin bei gemischten Zählungen 
%  \renewcommand{\thesection}{\thechapter\csname #1\endcsname{section}}% 
%}
%------------------------------
\begin{document}

\subsection{Flächengraphen von Sphären mit 3-Waists}
Seien $X$ mit $F\in X_2$ und $Y$ mit $F'\in Y_2$ zwei vertex-treue Sphären. Weiterhin sei die Abbildung $\phi:X_2(F)\to Y_2(F')$ bijektiv. An dieser Stelle wollen wir beschreiben, wie sich der Flächengraph der Sphäre $X\#_{\phi}Y$ aus den Flächengraphen der Sphären $X$ und $Y$ ergibt. Seien hierzu $F_1,F_2,F_3$ bzw. $F_1',F_2',F_3'$ die Nachbar-Flächen von $F$ in $X$ bzw. von $F'$ in $Y.$ Durch das Zusammensetzen der Sphären an den Flächen $F$ und $F'$ erhalten wir die konstruierte Sphäre $Z=X \#_\phi Y$ und einen 3-Waist $W=(e_1,e_2,e_3)$ in $Z,$ sodass ohne Einschränkung 
\[
Z_2(e_i)=\{F_i,F_i'\}
\] für $i=1,2,3$ gilt. Also sind $F_i$ und $F_i'$ benachbarte Flächen in $Z$.
\begin{figure}[H]
\begin{center}
\includegraphics[viewport=4cm 23.5cm 5cm 27.5cm]{3waist}
\end{center}
\caption{Ausschnitt der Sphäre $Z$}
\end{figure}   
 Weiterhin sei nun $G_X=(V_X,E_X)$ bzw. $G_Y=(V_Y,E_Y)$ der zu $X$ bzw. $Y$ gehörige Flächengraph. Den Flächengraph $G_Z$ erhalten wir durch folgende Konstruktion:
Sei $G=(V_X\cup V_Y,E_X \cup E_Y)$ die Vereinigung der Graphen $G_X$ und $G_Y.$
\begin{figure}[H]
\begin{center}
\includegraphics[viewport=14cm 18.5cm 5cm 23cm]{Image_fg10}
\end{center}
\caption{Ausschnitt der Vereinigung der Graphen $G_X$ und $G_Y$}
\end{figure}
Als erstes müssen die zu den Flächen $F$ und $F'$ gehörigen Knoten in $G$ und damit auch die inzidenten Kanten dem Graphen entnommen werden. 
\begin{figure}[H]
\begin{center}
\includegraphics[viewport=14cm 18.5cm 5cm 23cm]{Image_fg11}
\end{center}
\caption{Ausschnitt eines aus dem Graphen $G$ konstruiertem Graphen}
\end{figure}
Da in $Z$ die Flächen $F_i$ und $F_i'$ benachbart sind, erhalten wir den zugehörigen Flächengraphen, indem wir nun die Kanten $\{F_1,F_1'\},\{F_1,F_1'\},\{F_1,F_1'\}$ dem zuvor konstruierten Graphen hinzufügen. 
\begin{figure}[H]
\begin{center}
\includegraphics[viewport=14cm 18.5cm 5cm 22.5cm]{Image_fg12}
\end{center}
\caption{Ausschnitt des Graphen $G_Z$}
\end{figure}
\end{document}