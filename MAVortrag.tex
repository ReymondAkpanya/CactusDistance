\documentclass{beamer} 
\usetheme{Warsaw}             % Falls Ihnen das Layout nicht gefällt, können Sie hier
                              % auch andere Themes wählen. Ein Verzeichnis der möglichen 
                              % Themes finden Sie im Kapitel 15 des beameruserguide.

\usepackage[utf8]{inputenc}
\usepackage[ngerman]{babel}
\usepackage{amsmath}
\usepackage{amsfonts}
\usepackage{amssymb}
\usepackage{float}
\usepackage{graphicx}
\usepackage{pdfpages}


\usepackage{verbatim}
\newcommand{\gelb}{0.550000011920929}
\usepackage{pgf,tikz,pgfplots}
\pgfplotsset{compat=1.15}
\usetikzlibrary{arrows}
\AtBeginSection[]{\frame<beamer>{\frametitle{Übersicht} \tableofcontents[current]}}

\newcommand{\defin}[1]{\textit{\color{blue}#1}}

% ========== Abkürzungen ==========
\newcommand{\N}{\mathbb{N}}
\newcommand{\Z}{\mathbb{Z}}
\newcommand{\Q}{\mathbb{Q}}
\newcommand{\R}{\mathbb{R}}
\newcommand{\C}{\mathbb{C}}
\DeclareMathOperator{\Pot}{Pot}

\DeclareMathOperator{\Aut}{Aut}
\author{Reymond Akpanya}
\title{Klassifikation der Sphären ohne Zweier-Taillen}
\date{14.05.2021 }

\begin{document}
\frame{\maketitle}
\frame{\tableofcontents[currentsection]}
\begin{frame}
\begin{figure}[H]
\begin{center}
\includegraphics[viewport=1cm 23.6cm 5.5cm 26.7cm]{DSC_0050}
\end{center}
%\caption{Tetraeder}
\end{figure}
\end{frame}
\begin{frame}{Gliederung}
\tableofcontents
\end{frame}
\section{Kantendrehung}
\begin{frame}
\textbf{5 Kantendrehungen}
\begin{itemize}
\item 5.1 Transitivität der Kantendrehung
\item 5.2 Kantendrehung am Schirmzeiger
\end{itemize}
\end{frame}
\begin{frame}
\textbf{5 Kantendrehungen}
\begin{itemize}
\item \textcolor{red}{5.1 Transitivität der Kantendrehung}
\item 5.2 Kantendrehung am Schirmzeiger
\end{itemize}
\end{frame}
%\subsection{Transitivität der Kantendrehung}
\begin{frame}
\begin{block}{Definition 3.7}
Sei $X$ eine vertex-treue Sphäre. Eine Kante $e\in X_1$ heißt  \emph{drehbar}, falls es keine Kante $e'\in X_1\setminus \{e\}$ mit $X_0(e')=X_0(X_2(e))-X_0(e)$ gibt.
\end{block}
\end{frame}
\begin{frame}
\textbf{Tetraeder:}
\begin{columns}

    \column{0.5\textwidth}
\begin{figure}[H]
\begin{center}
\includegraphics[viewport=1cm 23.6cm 5.5cm 26.7cm]{ET_Example1}
\end{center}
%\caption{Tetraeder}
\end{figure}

    \column{0.5\textwidth}

\end{columns}
\end{frame}
\begin{frame}
\textbf{Tetraeder:}
\begin{columns}

    \column{0.5\textwidth}
\begin{figure}[H]
\begin{center}
\includegraphics[viewport=1cm 23.6cm 5.5cm 26.7cm]{etvortrag1}
\end{center}
%\caption{Tetraeder}
\end{figure}

    \column{0.5\textwidth}
  \pause  $X_0(X_2(e_2))-X_0(e_2)=\{2,4\}$\\%\pause
    
\end{columns}
\end{frame}
\begin{frame}
\textbf{Tetraeder:}
\begin{columns}

    \column{0.5\textwidth}
\begin{figure}[H]
\begin{center}
\includegraphics[viewport=1cm 23.6cm 5.5cm 26.7cm]{etvortrag2}
\end{center}
%\caption{Tetraeder}
\end{figure}

    \column{0.5\textwidth}
    $X_0(X_2(e_2))-X_0(e_2)=\{2,4\}$\\%\pause
     $=X_0(e_5)$
\end{columns}
\end{frame}
\begin{frame}
\textbf{Oktaeder:}
\begin{figure}[H]
\begin{center}
\includegraphics[viewport=17cm 17cm 5cm 20.7cm]{Image_Octahedron}
\end{center}
%\caption{Oktaeder}
\end{figure}
\end{frame}
\begin{frame}
\begin{block}{Definition 3.8}
Sei $X$ eine vertex-treue Sphäre, $\xi$ der zugehörige Flächen-Träger und $e$ eine drehbare Kante in $X$. Dann definieren wir die durch die Kantendrehung $e$ entstandene Sphäre $X^e$ durch den Flächen-Träger $\xi \Delta \Pot_3(X_0(X_2(e))).$
\end{block}
\end{frame}
\begin{frame}
\textbf{Kantendrehung in einer Skizze:}
\begin{figure}[H]
\begin{center}
\includegraphics[viewport=5cm 22.5cm 4cm 27cm]{Image_Kantendrehung}
\end{center}
%\caption{Kantendrehung}
\end{figure}
\end{frame}

\begin{frame}
\textbf{Doppel-6-Gon:}
\begin{figure}[H]
\begin{center}
\includegraphics[scale=0.85,viewport=12cm 11cm 18cm 18cm,scale=0.85]{Double6gon}
\end{center}
%\caption{Doppel-6-Gon}
\end{figure}
\end{frame}
\begin{comment}
\begin{frame}
\textbf{Ausschnitt einer Sphäre}:
\begin{figure}[H]
\begin{center}
\includegraphics[viewport=2cm 21.5cm 5cm 26.5cm]{facgraKan}
\end{center}
%\caption{Ausschnitt der Sphäre $X$}
\end{figure}
\end{frame}

\begin{frame}
\textbf{Ausschnitt einer Sphäre}:
\begin{figure}[H]
\begin{center}
\includegraphics[viewport=2cm 21.5cm 5cm 26.5cm]{fgVortrag}
\end{center}
\end{figure}
\end{frame}
\begin{frame}{Beispiel}

\begin{columns}
    \column{0.5\textwidth}
   %\center{simpliziale Fläche $Y$}\\
   \includegraphics[scale=0.5,viewport=0cm 15.5cm 5cm 26.5cm]{facgraKan}\\
 %  \includegraphics[scale=0.5,viewport=0cm 21.5cm 5cm 28cm]{fgVortrag}
    \column{0.5\textwidth}

\includegraphics[scale=0.6,viewport=8cm 12cm 19cm 22.5cm]{Image_fg8}\\
%\includegraphics[scale=0.6,viewport=8cm 18cm 19cm 22.5cm]{Image_fg9}
\end{columns}
\end{frame}

\begin{frame}{Beispiel}

\begin{columns}
    \column{0.5\textwidth}
   %\center{simpliziale Fläche $Y$}\\
   \includegraphics[scale=0.5,viewport=0cm 15.5cm 5cm 26.5cm]{fgvortrag1}\\
 %  \includegraphics[scale=0.5,viewport=0cm 21.5cm 5cm 28cm]{fgVortrag}
    \column{0.5\textwidth}

\includegraphics[scale=0.6,viewport=8cm 12cm 19cm 22.5cm]{fg8vortrag}\\
%\includegraphics[scale=0.6,viewport=8cm 18cm 19cm 22.5cm]{Image_fg9}
\end{columns}
\end{frame}


\begin{frame}{Beispiel}

\begin{columns}
    \column{0.5\textwidth}
   %\center{simpliziale Fläche $Y$}\\
   \includegraphics[scale=0.5,viewport=0cm 21.5cm 5cm 26.5cm]{facgraKan}\\
   \includegraphics[scale=0.5,viewport=0cm 21.5cm 5cm 28cm]{fgVortrag}
    \column{0.5\textwidth}

\includegraphics[scale=0.6,viewport=8cm 17cm 19cm 22.5cm]{Image_fg8}\\
\includegraphics[scale=0.6,viewport=8cm 18cm 19cm 22.5cm]{Image_fg9}
\end{columns}
\end{frame}

\end{comment}
\begin{frame}
\begin{block}{Definition 5.6}
Sei $X$ eine Sphäre und $e_1,\ldots,e_n$ Kanten in $X$. Wir nennen $E=(e_1,\ldots,e_n)$ eine \emph{drehbare Kantensequenz}, falls Folgendes erfüllt ist: 
\begin{itemize}
\item Die Kante $e_1$ ist eine drehbare Kante in $X^{()}:=X.$ 
\item Für alle $1\leq i < n$ ist die Kante $e_{i+1}$ eine drehbare Kante in 
\[
X^{(e_1,\ldots,e_i)}:=(X^{(e_1,\ldots,e_{i-1})})^{e_i}.
\] 
\end{itemize}
Wir nennen $X^E$ die durch die Kantensequenz $E$ entstandene Sphäre. 
\end{block}
\end{frame}
\begin{frame}
\textbf{Frage:} Gibt es für zwei beliebige Sphären $X,Y$ eine drehbare Kantensequenz $E$ in $X,$ sodass $X^E\cong Y$ ist?
\end{frame}
\begin{frame}
\begin{block}{Satz 5.9}
Seien $X$ und $Y$ vertex-treue Sphären mit $\vert X_2\vert=\vert Y_2\vert=2n$.
Dann existiert eine drehbare Kantensequenz $E=(e_1,\ldots,e_l)$ in $X,$ sodass  $X^E$ zu $Y$ isomorph ist. 
\end{block}
\end{frame}

\begin{frame}
\textbf{Beweisidee:} Wir zeigen induktiv, dass jede Sphäre durch eine drehbare Kantensequenz in einen Doppel-$n$-Gon umgeformt werden kann. \pause (Induktion über die Anzahl der Flächen)
\end{frame}
\begin{frame}
\begin{enumerate}
\item Wie wird aus einer Sphäre $X$ mit $2n$ Flächen eine Sphäre $Y$ mit $2n-2$ Flächen konstruiert?
\end{enumerate}
\end{frame}
\begin{frame}
\begin{itemize}
\item Wir können die Existenz einer Ecke vom Grad 3 in $X$ annehmen\pause
\item die Sphäre $Y=T_V(X)$ hat dann $2n-2$ Flächen
\end{itemize}
\end{frame}
\begin{frame}
\textbf{Tetraeder-Erweiterung und Tetraeder-Entfernung:}
\begin{figure}[H]
\begin{center}
\includegraphics[scale=0.8,viewport=18cm 22.5cm -3cm 27cm]{TetraedererwVortrag}
\end{center}
%\caption{Tetraeder-Erweiterung und Tetraeder-Entfernung}
\end{figure}
\end{frame}
\begin{frame}
\begin{itemize}
\item Wir können die Existenz einer Ecke vom Grad 3 in $X$ annehmen\pause
\item die Sphäre $Y=T_V(X)$ hat dann $2n-2$ Flächen\pause
\item es existiert eine drehbare Kantensequenz $E'$ in $Y$ mit $Y^{E'}\cong (n-1)^2$  
\end{itemize}
\end{frame}
\begin{frame}
\begin{enumerate}
\item Wie wird aus einer Sphäre $X$ mit $2n$ Flächen eine Sphäre $Y$ mit $2n-2$ Flächen konstruiert?
\item Wie wird eine drehbare Kantensequenz auf $Y$ zu einer drehbaren Kantensequenz auf $X$ übersetzt? 
\end{enumerate}
\end{frame}
%%%%%%%%%%%%%%%%%%%%%%55
\begin{frame}
\textbf{Verträglichkeit der Tetraeder-Erweiterung und Kantendrehung:}
\begin{figure}[H]
\includegraphics[scale=.65,viewport=0cm 17cm 16cm 26.3cm]{comutevortrag1}
%\caption{Verträglichkeit der Tetraeder-Erweiterung und Kantendrehung}
\end{figure}
\end{frame}
\begin{frame}
\textbf{Verträglichkeit der Tetraeder-Erweiterung und Kantendrehung:}
\begin{figure}[H]
\includegraphics[scale=.65,viewport=0cm 17cm 16cm 26.3cm]{comutevortrag2}
%\caption{Verträglichkeit der Tetraeder-Erweiterung und Kantendrehung}
\end{figure}
\end{frame}
\begin{frame}
\textbf{Verträglichkeit der Tetraeder-Erweiterung und Kantendrehung:}
\begin{figure}[H]
\includegraphics[scale=.65,viewport=0cm 17cm 16cm 26.3cm]{comutevortrag3}
%\caption{Verträglichkeit der Tetraeder-Erweiterung und Kantendrehung}
\end{figure}
\end{frame}
\begin{frame}
\textbf{Verträglichkeit der Tetraeder-Erweiterung und Kantendrehung:}
\begin{figure}[H]
\includegraphics[scale=.65,viewport=0cm 17cm 16cm 26.3cm]{comutevortrag4}
%\caption{Verträglichkeit der Tetraeder-Erweiterung und Kantendrehung}
\end{figure}
\end{frame}
\begin{frame}
\textbf{Verträglichkeit der Tetraeder-Erweiterung und Kantendrehung:}
\begin{figure}[H]
\includegraphics[scale=.65,viewport=0cm 17cm 16cm 26.3cm]{comutevortrag5}
%\caption{Verträglichkeit der Tetraeder-Erweiterung und Kantendrehung}
\end{figure}
\end{frame}
\begin{frame}
\textbf{Verträglichkeit der Tetraeder-Erweiterung und Kantendrehung:}
\begin{figure}[H]
\includegraphics[scale=.65,viewport=0cm 17cm 16cm 26.3cm]{comutevortrag6}
%\caption{Verträglichkeit der Tetraeder-Erweiterung und Kantendrehung}
\end{figure}
\end{frame}
\begin{frame}
\textbf{Verträglichkeit der Tetraeder-Erweiterung und Kantendrehung:}
\begin{figure}[H]
\includegraphics[scale=.65,viewport=0cm 17cm 16cm 26.3cm]{comutevortrag7}
%\caption{Verträglichkeit der Tetraeder-Erweiterung und Kantendrehung}
\end{figure}
\end{frame}


%%%%%%%%%%%%%%%%%%%%%%%%%%%%%%%5
\begin{frame}
\textbf{Verträglichkeit der Tetraeder-Erweiterung und Kantendrehung:}
\begin{figure}[H]
\includegraphics[scale=.65,viewport=0cm 17cm 16cm 26.3cm]{comutevortrag9}
%\caption{Verträglichkeit der Tetraeder-Erweiterung und Kantendrehung}
\end{figure}
\end{frame}
\begin{frame}
\textbf{Verträglichkeit der Tetraeder-Erweiterung und Kantendrehung:}
\begin{figure}[H]
\includegraphics[scale=.65,viewport=0cm 17cm 16cm 26.3cm]{comutevortrag81}
%\caption{Verträglichkeit der Tetraeder-Erweiterung und Kantendrehung}
\end{figure}
\end{frame}
\begin{frame}
\textbf{Verträglichkeit der Tetraeder-Erweiterung und Kantendrehung:}
\begin{figure}[H]
\includegraphics[scale=.65,viewport=0cm 17cm 16cm 26.3cm]{comutevortrag82}
%\caption{Verträglichkeit der Tetraeder-Erweiterung und Kantendrehung}
\end{figure}
\end{frame}
\begin{frame}
\textbf{Verträglichkeit der Tetraeder-Erweiterung und Kantendrehung:}
\begin{figure}[H]
\includegraphics[scale=.65,viewport=0cm 17cm 16cm 26.3cm]{comutevortrag8}
%\caption{Verträglichkeit der Tetraeder-Erweiterung und Kantendrehung}
\end{figure}
\end{frame}
\begin{frame}
\begin{itemize}
\item Wegen Lemma 5.6 kann also die Existenz einer Ecke vom Grad 3 in $X$ angenommen werden
\item die Sphäre $Y=T_V(X)$ hat dann $2n-2$ Flächen und die Fläche $F\in Y_2$ ersetzt den Tetraeder an der Stelle $V$
\item es existiert eine drehbare Kantensequenz $E'$ in $Y$ mit $Y^{E'}\cong (n-1)^2$  
\item es existiert eine drehbare Kantensequenz $E$ mit 
\[
X^{E}\cong T^F(Y^{E'})=T^F((n-1)^2)
\] 
\end{itemize}
\end{frame}
\begin{frame}
\begin{block}{Definition 7.20}
Sei $n\in \mathbb{N}$ gerade. Dann definieren wir den \emph{Sphären-Graphen} $\mathcal{G}^3_n$ durch die Knotenmenge $\mathcal{S}^3_n.$ Zwei Knoten $V,V'$ sind in diesem Graphen durch eine ungerichtete Kante verbunden, falls die zu $V$ zugehörige Sphäre durch genau eine Kantendrehung aus der zu $V'$ zugehörigen Sphäre hervorgeht. 
\end{block}
\end{frame}
\begin{frame}
\textbf{Oktaeder:}
\begin{figure}
\begin{center}
\includegraphics[viewport=17cm 17cm 5cm 20.7cm]{Image_Octahedron}
\end{center}
%\caption{Oktaeder}
\end{figure}
\end{frame}
\begin{frame}
\textbf{Sphäre mit 8 Flächen:}
\begin{figure}[H]
\begin{center}
\includegraphics[viewport=19cm 19cm 0cm 23cm]{Image_MultiTetraeder1}
\end{center}
%\caption{Multi-Tetraeder mit 8 Flächen}
\end{figure}
\end{frame}

\begin{frame}

\textbf{Der Sphären-Graph $\mathcal{G}_8^3$:}
\begin{figure}[H]
\begin{center}
\includegraphics[viewport=0cm 26.5cm 5cm 27.5cm]{spgr}
\end{center}
%\caption{der Graph $\mathcal{G}_8^3$}
\end{figure}
\end{frame}
\section{Multi-Tetraeder}
\begin{frame}
\begin{figure}[H]
\begin{center}
\includegraphics[scale=0.14,viewport=0cm 30.5cm 60cm 32.5cm]{bild10}
\end{center}
%\caption{der Graph $\mathcal{G}_8^3$}
\end{figure}
\end{frame}
\begin{frame}
\begin{figure}[H]
\begin{center}
\includegraphics[scale=0.15,viewport=0cm 20.5cm 66cm 20.5cm]{bild11}
\end{center}
%\caption{der Graph $\mathcal{G}_8^3$}
\end{figure}
\end{frame}
\begin{frame}
\begin{figure}[H]
\begin{center}
\includegraphics[scale=0.15,viewport=0cm 20.5cm 66cm 33.5cm]{bild13}
\end{center}
%\caption{der Graph $\mathcal{G}_8^3$}
\end{figure}
\end{frame}
\begin{frame}
\textbf{6 Multi-Tetraeder}
\begin{itemize}
\item 6.1 Konstruktion und Klassifikation
\item 6.2 Kaktus-Distanz
\item 6.3 Tetraeder-Zerlegung
\item 6.4 Wilde Färbungen auf Multi-Tetraedern
\item 6.5 Flächengraphen von Multi-Tetraedern
\item 6.6 Das Tetraeder-Polynom
\end{itemize}
\end{frame}
\begin{frame}
\textbf{6 Multi-Tetraeder}
\begin{itemize}
\item \textcolor{red}{6.1 Konstruktion und Klassifikation}
\item \textcolor{red}{6.2 Kaktus-Distanz}
\item 6.3 Tetraeder-Zerlegung
\item \textcolor{red}{6.4 Wilde Färbungen auf Multi-Tetraedern}
\item 6.5 Flächengraphen von Multi-Tetraedern
\item 6.6 Das Tetraeder-Polynom
\end{itemize}
\end{frame}
%\subsection{Konstruktion und Klassifikation}
\begin{frame}
\begin{block}{Definition 6.1}
\begin{enumerate}
\item Sei $X$ eine geschlossene simpliziale Fläche mit $\vert X_0\vert \geq 6.$
 Die simpliziale Fläche, die durch Entfernen aller Tetraeder entsteht, bezeichnen wir mit $X^{(1)}$. Für $i>1$ definieren wir analog 
\[
X^{(i)}:=(X^{(i-1)})^{(1)}.
\]
\item Wir nennen $X$ einen \emph{Multi-Tetraeder vom Grad} $k$, falls $X^{(k-1)}$ ein Tetraeder oder ein Doppel-Tetraeder ist.
\end{enumerate}
\end{block}
\end{frame}
\begin{frame}
\textbf{Tetraeder:}
\begin{figure}[H]
\begin{center}
\includegraphics[viewport=19cm 19cm 0cm 23cm]{MultiVortrag2}
\end{center}
%\caption{Multi-Tetraeder mit 8 Flächen}
\end{figure}
\end{frame}
\begin{frame}
\textbf{Doppel-Tetraeder:}
\begin{figure}[H]
\begin{center}
\includegraphics[viewport=22cm 12.5cm 5cm 17cm]{Image_DoubleTetraeder}
\end{center}
%\caption{Doppel-Tetraeder}
\end{figure}
\end{frame}
\begin{frame}
\textbf{Sphäre mit 8 Flächen:}
\begin{figure}[H]
\begin{center}
\includegraphics[viewport=19cm 19cm 0cm 23cm]{Image_MultiTetraeder1}
\end{center}
%\caption{Multi-Tetraeder mit 8 Flächen}
\end{figure}
\end{frame}
\begin{frame}
\textbf{Tetraeder:}
\begin{figure}[H]
\begin{center}
\includegraphics[viewport=19cm 19cm 0cm 23cm]{MultiVortrag}
\end{center}
%\caption{Multi-Tetraeder mit 8 Flächen}
\end{figure}
\end{frame}
\begin{frame}
\textbf{simpliziale Parallelepiped $P$:}
\begin{figure}[H]
\begin{center}
\includegraphics[scale=0.8,viewport=28cm 14cm 0cm 19cm]{Image_Parallelepiped}
\end{center}
%\caption{Parallelepiped}
\end{figure}
\end{frame}
\begin{frame}
\textbf{simpliziale Fläche $P^{(1)}$:}
\begin{figure}[H]
\begin{center}
\includegraphics[scale=0.8,viewport=28cm 14cm 0cm 19cm]{OctaVortrag}
\end{center}
%\caption{Parallelepiped}
\end{figure}
\end{frame}

%\end{center}
\begin{frame}
\textbf{Frage:} Wieviele Multi-Tetraeder mit einer gegebenen Anzahl von Flächen gibt es?
\end{frame}
\begin{frame}
\textbf{Anzahl der Multi-Tetraeder:}
\begin{center}
\begin{tabular}[h]{|c|c|c|c|c|c|c|}
\hline
\textbf{ 4} &  \textbf{6}& \textbf{8} &\textbf{ 10} &\textbf{ 12} & \textbf{14}&\textbf{16}\\
\hline
 1& 1& 1& 3& 7& 24& 93\\
\hline
\end{tabular}
\end{center}
\textcolor{white}{..}\\
\begin{center}
\begin{tabular}[h]{|c|c|c|c|c|c|}
\hline
\textbf{18}&\textbf{20}&\textbf{22}&\textbf{24}&\textbf{26}&\textbf{28}\\
\hline
 434& 2110& 11003& 58598& 321726& 1614848
 \\
 \hline
\end{tabular}
\end{center}
\end{frame}


\begin{frame}
\begin{block}{Satz 6.17}
Sei $X$ ein Multi-Tetraeder und $\Aut(X)$ die Automorphismengruppe von $X$. Dann ist $\Aut(X)$ auflösbar.
Weiterhin seien $l,l'$ minimal mit der Eigenschaft, dass
\[
\vert \Aut(X)^l\vert=\vert \Aut(X^{(1)})^{l'}\vert =1
\] ist. Dann gilt $l\leq l'$.
\end{block}
\end{frame}
\begin{frame}
\textbf{Beweisskizze:}
\begin{itemize}
\item Sei $X$ ein Multi-Tetraeder mit $\vert X_2\vert >6$ und $G$ die zugehörige Automorphismengruppe\pause
\item $V_1,\ldots,V_k$ die Ecken vom Grad 3 in $X$\pause 
\item Definiere $M_t:=\{V_t\} \cup X_1(V_t) \cup X_2(V_t)$\pause
\item dann gilt $\phi(M_t)=M_{\pi(t)}$ für alle $\phi \in G$\pause
\item in $Y=X^{(1)}$ werden die Tetraeder an den Ecken $V_1,\ldots,V_k$ durch Fläche $F_1,\ldots,F_k$ ersetzt
\end{itemize}
\end{frame}
\begin{frame}
\begin{itemize}
\item betrachte die Abbildung $\psi:G\to \Aut(Y),$ die durch  
\[
\psi(\phi)(x)=
\begin{cases}
F_t,&\text{falls } x\in M_t, \\
x,& \text{sonst},\\
\end{cases}
\]\pause
\item $\psi(G)$ ist eine Untergruppe von $Aut(Y)$\pause
\item $\psi(G)$ ist als Untergruppe einer auflösbaren Untergruppe ebenfalls auflösbar
\end{itemize}
\end{frame}

\begin{frame}
\begin{figure}[H]
\begin{center}
\includegraphics[scale=0.14,viewport=0cm 20.5cm 63cm 40.5cm]{bild9}
\end{center}
%\caption{der Graph $\mathcal{G}_8^3$}
\end{figure}
\end{frame}
\begin{frame}
\begin{figure}[H]
\begin{center}
\includegraphics[scale=0.14,viewport=0cm 20.5cm 85cm 30.5cm]{bild3}
\end{center}
%\caption{der Graph $\mathcal{G}_8^3$}
\end{figure}
\end{frame}
\begin{frame}
\begin{figure}[H]
\begin{center}
\includegraphics[scale=0.2,viewport=0cm 5.5cm 48cm 35.5cm]{bild4}
\end{center}
%\caption{der Graph $\mathcal{G}_8^3$}
\end{figure}
\end{frame}
\begin{frame}
\begin{figure}[H]
\begin{center}
\includegraphics[scale=0.1,viewport=0cm 5.5cm 80cm 73.5cm]{bild7}
\end{center}
%\caption{der Graph $\mathcal{G}_8^3$}
\end{figure}
\end{frame}
\begin{frame}
\begin{figure}[H]
\begin{center}
\includegraphics[scale=0.15,viewport=0cm 5.5cm 55cm 45cm]{bild6}
\end{center}
%\caption{der Graph $\mathcal{G}_8^3$}
\end{figure}
\end{frame}
\begin{frame}
\textbf{Verallgemeierung:}
\begin{block}{Satz 7.16}
Sei $Y$ eine Multi-Sphäre und $X$ eine Sphäre, sodass die Blöcke von $Y$ zu $X$ isomorph sind. Falls $\Aut(X)$ auflösbar ist, dann folgt die Auflösbarkeit von $\Aut(Y).$ 
\end{block}
\end{frame}
\begin{frame}
\begin{center}
\huge{Kaktus-Distanz}
\end{center}
\end{frame}
\begin{frame}
\begin{block}{Definition 6.18}
Sei $X$ eine vertex-treue Sphäre. Die minimale Anzahl an Kantendrehungen, die es braucht, um aus $X$ einen Multi-Tetraeder zu  konstruieren, nennen wir die \emph{Kaktus-Distanz} $\zeta(X)$ von $X$.
\end{block}
\end{frame}
\begin{frame}
\begin{block}{Lemma 6.20}
Für alle Sphären ist die Kaktus-Distanz endlich.
\end{block}
\pause
\begin{itemize}
\item folgt direkt aus der Transitivität der Kantendrehung
\end{itemize}
\end{frame}
\begin{frame}
\begin{center}
$\fbox{
\parbox{10cm}{
\textcolor{red}{gap$>$} \textcolor{blue}{IsCactus(O);}\\
false\\
\textcolor{red}{gap$>$}\textcolor{blue}{ O;}\\
simplicial surface (6 vertices, 12 edges, and 8 faces)\\
\textcolor{red}{gap$>$} \textcolor{blue}{EdgeTurn(O,1);}\\
simplicial surface (6 vertices, 12 edges, and 8 faces)\\
\textcolor{red}{gap$>$} \textcolor{blue}{IsCactus(last);}\\
true
}}$
\end{center}
\end{frame}
\begin{frame}
\begin{block}{Lemma 6.21}
Sei $X$ eine vertex-treue Sphäre ohne 3-Taillen, die  sich durch eine Kantendrehung in einen Multi-Tetraeder umformen lässt. Dann ist $X$  zum Doppel-$n$-Gon isomorph.[6]
\end{block}
\end{frame}
\begin{frame}
Falls vertex-treue Sphären mit 3-Taillen zugelassen werden, dann ist die Aussage falsch. \pause
\begin{figure}[H]
\begin{center}
\includegraphics[scale=0.8,viewport=28cm 14cm 0cm 19cm]{Image_Parallelepiped}
\end{center}
%\caption{Parallelepiped}
\end{figure}
\end{frame}
\begin{frame}
Falls vertex-treue Sphären mit 3-Taillen zugelassen werden, dann ist die Aussage falsch. 
\begin{figure}[H]
\begin{center}
\includegraphics[scale=0.8,viewport=28cm 14cm 0cm 19cm]{ppvortrag4}
\end{center}
%\caption{Parallelepiped}
\end{figure}
\end{frame}
\begin{frame}
Falls vertex-treue Sphären mit 3-Taillen zugelassen werden, dann ist die Aussage falsch. 
\begin{figure}[H]
\begin{center}
\includegraphics[scale=0.8,viewport=28cm 14cm 0cm 19cm]{ppvortrag2}
\end{center}
%\caption{Parallelepiped}
\end{figure}
\end{frame}
\begin{frame}
Falls vertex-treue Sphären mit 3-Taillen zugelassen werden, dann ist die Aussage falsch. 
\begin{figure}[H]
\begin{center}
\includegraphics[scale=0.8,viewport=28cm 14cm 0cm 19cm]{ppvortrag1}
\end{center}
%\caption{Parallelepiped}
\end{figure}
\end{frame}
\begin{frame}
Falls vertex-treue Sphären mit 3-Taillen zugelassen werden, dann ist die Aussage falsch. 
\begin{figure}[H]
\begin{center}
\includegraphics[scale=0.8,viewport=28cm 14cm 0cm 19cm]{ppvortrag3}
\end{center}
%\caption{Parallelepiped}
\end{figure}
\end{frame}
\begin{frame}
Falls vertex-treue Sphären mit 3-Taillen zugelassen werden, dann ist die Aussage falsch. 
\begin{figure}[H]
\begin{center}
\includegraphics[scale=0.8,viewport=28cm 14cm 0cm 19cm]{ppvortrag6}
\end{center}
%\caption{Parallelepiped}
\end{figure}
\end{frame}
\begin{frame}
Falls vertex-treue Sphären mit 3-Taillen zugelassen werden, dann ist die Aussage falsch. 
\begin{figure}[H]
\begin{center}
\includegraphics[scale=0.8,viewport=28cm 14cm 0cm 19cm]{ppvortrag5}
\end{center}
%\caption{Parallelepiped}
\end{figure}
\end{frame}

\begin{frame}
\begin{block}{Satz 6.24}
Sei $X$ eine vertex-treue Sphäre, die $\zeta(X)=1$ erfüllt. Dann existiert ein $n\in \mathbb{N}, $ sodass $X$ zum Doppel-$n$-Gon isomorph ist oder durch eine endliche Anzahl von Tetraeder-Erweiterungen aus diesem hervorgeht.

\end{block}
\end{frame}
\begin{frame}
\textbf{Beweisskizze}\\
Fallunterscheidung:
\begin{itemize}
\item $X$ ist eine vertex-treue Sphäre ohne 3-Taillen\pause
\item $X$ ist eine vertex-treue Sphäre mit mindestens zwei  ohne Ecken vom Grad 3\pause 
\item $X$ ist eine vertex-treue Sphäre mit einer Ecke vom Grad 3
\end{itemize}
\end{frame}
\begin{frame}
\begin{center}
\begin{huge}
\center{Algorithmus 1}
\end{huge}
\end{center}
\end{frame}
\begin{frame}
Wir benötigen:
\begin{itemize}
\item Zwei Ecken $V_1$ und $V_2$ einer Sphäre $X$ sind nicht benachbart und erfüllen
\[
X_0(X_1(V_1))\cap X_0(X_1(V_2))\neq \emptyset.
\] 
\end{itemize}
  \begin{figure}[H]
\begin{center}
\includegraphics[scale=0.6,viewport=-1cm 22.5cm 7cm 26.5cm]{kantenvortrag}
\end{center}
%\caption{Ausschnitt der Sphäre $X$}
\end{figure}
\pause
Wir betrachten als Beispiel, die Sphäre $(6)\overline{6}(6).$
\end{frame}
\begin{frame}
\textbf{Doppel-6-Gon:}
\begin{figure}[H]
\begin{center}
\includegraphics[scale=0.85,viewport=12cm 11cm 18cm 18cm,scale=0.85]{Double6gon}
\end{center}
%\caption{Doppel-6-Gon}
\end{figure}
\end{frame}
\begin{frame}
\textbf{Doppel-6-Gon:}
\begin{figure}[H]
\begin{center}
\includegraphics[scale=0.85,viewport=12cm 11cm 18cm 18cm,scale=0.85]{d6gvortrag}
\end{center}
%\caption{Doppel-6-Gon}
\end{figure}
\end{frame}
\begin{frame}
\textbf{6-Streifen:}
\begin{figure}[H]
\begin{center}
\includegraphics[viewport=1cm 25.5cm 8cm 26.5cm]{nstreifen}
\end{center}
%\caption{ $n$-Streifen für $n=6$}
\end{figure}
\end{frame}
\begin{frame}
\begin{figure}[H]
\begin{center}
\includegraphics[viewport=8cm 21.7cm 5cm 26.5cm]{n2lnV}
\end{center}
%\caption{Eine Skizze der Sphäre $(6)\overline{6}(6)$; der einsetzte 6-Strip ist durch die grüne Farbe hervorgehoben}
\end{figure}
\end{frame}

\begin{frame}
\begin{figure}[H]
\begin{center}
\includegraphics[viewport=8cm 21.7cm 5cm 26.5cm]{n2lnVortrag}
\end{center}
%\caption{Eine Skizze der Sphäre $(6)\overline{6}(6)$; der einsetzte 6-Strip ist durch die grüne Farbe hervorgehoben}
\end{figure}
\end{frame}
\begin{frame}
\begin{figure}[H]
\begin{center}
\includegraphics[viewport=8cm 21.7cm 5cm 26.5cm]{n2lnVortrag1}
\end{center}
%\caption{Eine Skizze der Sphäre $(6)\overline{6}(6)$; der einsetzte 6-Strip ist durch die grüne Farbe hervorgehoben}
\end{figure}
\end{frame}
\begin{frame}
\begin{figure}[H]
\begin{center}
\includegraphics[viewport=8cm 21.7cm 5cm 26.5cm]{n2lnVortrag2}
\end{center}
%\caption{Eine Skizze der Sphäre $(6)\overline{6}(6)$; der einsetzte 6-Strip ist durch die grüne Farbe hervorgehoben}
\end{figure}
\end{frame}
\begin{frame}
\begin{figure}[H]
\begin{center}
\includegraphics[viewport=8cm 21.7cm 5cm 26.5cm]{n2lnVortrag3}
\end{center}
%\caption{Eine Skizze der Sphäre $(6)\overline{6}(6)$; der einsetzte 6-Strip ist durch die grüne Farbe hervorgehoben}
\end{figure}
\end{frame}
\begin{frame}
\begin{figure}[H]
\begin{center}
\includegraphics[viewport=8cm 21.7cm 5cm 26.5cm]{n2lnVortrag4}
\end{center}
%\caption{Eine Skizze der Sphäre $(6)\overline{6}(6)$; der einsetzte 6-Strip ist durch die grüne Farbe hervorgehoben}
\end{figure}
\end{frame}
\begin{frame}
\begin{figure}[H]
\begin{center}
\includegraphics[viewport=8cm 21.7cm 5cm 26.5cm]{n2lnVortrag5}
\end{center}
%\caption{Eine Skizze der Sphäre $(6)\overline{6}(6)$; der einsetzte 6-Strip ist durch die grüne Farbe hervorgehoben}
\end{figure}
\end{frame}
\begin{frame}
\begin{figure}[H]
\begin{center}
\includegraphics[viewport=8cm 21.7cm 5cm 26.5cm]{n2lnVortrag6}
\end{center}
%\caption{Eine Skizze der Sphäre $(6)\overline{6}(6)$; der einsetzte 6-Strip ist durch die grüne Farbe hervorgehoben}
\end{figure}
\end{frame}
\begin{frame}
\begin{figure}[H]
\begin{center}
\includegraphics[viewport=8cm 21.7cm 5cm 26.5cm]{n2lnVortrag7}
\end{center}
%\caption{Eine Skizze der Sphäre $(6)\overline{6}(6)$; der einsetzte 6-Strip ist durch die grüne Farbe hervorgehoben}
\end{figure}
\end{frame}
\begin{frame}
\begin{figure}[H]
\begin{center}
\includegraphics[viewport=8cm 21.7cm 5cm 26.5cm]{n2lnVortrag8}
\end{center}
%\caption{Eine Skizze der Sphäre $(6)\overline{6}(6)$; der einsetzte 6-Strip ist durch die grüne Farbe hervorgehoben}
\end{figure}
\end{frame}
\begin{frame}
\begin{figure}[H]
\begin{center}
\includegraphics[viewport=8cm 21.7cm 5cm 26.5cm]{n2lnVortrag9}
\end{center}
%\caption{Eine Skizze der Sphäre $(6)\overline{6}(6)$; der einsetzte 6-Strip ist durch die grüne Farbe hervorgehoben}
\end{figure}
\end{frame}
\begin{frame}
\begin{figure}[H]
\begin{center}
\includegraphics[viewport=8cm 21.7cm 5cm 26.5cm]{n2lnVortrag10}
\end{center}
%\caption{Eine Skizze der Sphäre $(6)\overline{6}(6)$; der einsetzte 6-Strip ist durch die grüne Farbe hervorgehoben}
\end{figure}
\end{frame}
\begin{frame}
\begin{block}{Satz 6.27}
Sei $X$ eine vertex-treue Sphäre, die kein Multi-Tetraeder ist. Dann ist 
\[
\zeta(X)\leq m+1,
\] wobei $m:=\underset{{(V,V')\in D}}{\min}\left\{\left|\frac{\vert X_2 \vert}{2}-\deg_X(V)\right| +\left| \frac{\vert X_2 \vert}{2}-\deg_X(V')\right|\right\}$
 ist.
\end{block}
\end{frame}
\begin{frame}
\begin{block}{Lemma 6.34}
Seien $n\in \mathbb{N}$ und $0\leq l \leq n.$ Dann ist 
\[
\zeta((n)\overline{2l}(n))\leq 2l+1.
\]
\end{block}
\end{frame}
\begin{frame}
\begin{center}
\begin{tabular}{|c|c|c|}
\hline

$\textbf{X}$&$\textbf{$\zeta$(X)}$&\textbf{Algorithmus 1}\\
\hline
$T$&0&0\\
\hline
$(4)^2$&1&1\\
\hline
$(5)^2$&1&1\\
\hline
$(6)^2$&1&1\\
\hline
$(5)\overline{2}(5)$&2&3\\
\hline
$(6)\overline{2}(6)$&2&3\\
\hline
$(7)^2$&1&1\\
\hline
$(6)\overline{3}(5)$&2&4\\
\hline
$(5)\overline{4}(5)$&3&5\\
\hline
\end{tabular}
\end{center}

\end{frame}
\begin{frame}
\begin{center}
\begin{tabular}{|c|c|c|}
\hline

$\textbf{X}$&$\textbf{$\zeta$(X)}$&\textbf{Algorithmus 1}\\
\hline
$T$&0&0\\
\hline
$(4)^2$&1&1\\
\hline
$(5)^2$&1&1\\
\hline
$(6)^2$&1&1\\
\hline
\colorbox{yellow}{$(5)\overline{2}(5)$}&\colorbox{yellow}{2}&\colorbox{yellow}{3}\\
\hline
$(6)\overline{2}(6)$&2&3\\
\hline
$(7)^2$&1&1\\
\hline
$(6)\overline{3}(5)$&2&4\\
\hline
$(5)\overline{4}(5)$&3&5\\
\hline
\end{tabular}
\end{center}
\end{frame}
\begin{frame}
\begin{block}{Satz 6.38}
Sei $n \geq 4.$ Dann ist $\zeta((n)\overline{2}(n))=2.$
\end{block}
\end{frame}
\begin{frame}
\begin{block}{Folgerung 6.39}
Sei $n \geq 4.$ Dann ist $\zeta((n+1)\overline{3}(n))=2.$
\end{block}
\end{frame}
\begin{frame}
\begin{center}
\huge{Flächengraphen von Multi-Tetraedern}
\end{center}
\end{frame}
\begin{frame}
\textbf{Ausschnitt eines Flächengraphen}
\begin{figure}[H]
\begin{center}
\includegraphics[viewport=3cm 19.1cm 14cm 23.5cm]{Image_fg1}
\end{center}
%\caption{Ausschnitt eines Flächengraphen eines Multi-Tetraeders}
\end{figure}
\end{frame}
\begin{frame}
\textbf{Ausschnitt eines Flächengraphen}
\begin{figure}[H]
\begin{center}
\includegraphics[viewport=3cm 19.1cm 14cm 23.5cm]{Image_fg2}
\end{center}
\end{figure}
\end{frame}
\begin{frame}
\textbf{Ausschnitt eines Flächengraphen}
\begin{figure}[H]
\begin{center}
\includegraphics[viewport=3cm 19.1cm 14cm 23.5cm]{Image_fg3}
\end{center}
\end{figure}
\end{frame}
\begin{frame}
\textbf{Flächengraph des Tetraeders:}
\begin{figure}[H]
\begin{center}
\includegraphics[scale=0.8,viewport=10cm 19.5cm 5cm 22.5cm]{mttry1}

\end{center}
%\caption{Kantendrehung}
\end{figure}
\end{frame}
\begin{frame}
\textbf{Flächengraph des Doppel-Tetraeders:}
\begin{figure}[H]
\begin{center}
\includegraphics[viewport=10cm 19.5cm 5cm 23.cm]{Mttry4}
\end{center}
\end{figure}
\end{frame}
\begin{frame}
\textbf{Flächengraph eines Multi-Tetraeders mit 8 Flächen}
\begin{figure}[H]
\begin{center}
\includegraphics[scale=0.5,viewport=16cm 14.5cm 5cm 24cm]{bsp9}
\end{center}

\end{figure}
\end{frame}
\begin{frame}

\textbf{Flächengraph eines Multi-Tetraeders mit 10 Flächen}
\begin{figure}[H]
\begin{center}
\includegraphics[scale=0.6,viewport=15.5cm 14.cm 5cm 23cm]{bsp10}
\end{center}
\end{figure}
\end{frame}
\begin{frame}
\textbf{Flächengraph eines Multi-Tetraeders mit 12 Flächen}
\begin{figure}[H]
\begin{center}
\includegraphics[scale=0.6,viewport=16cm 14cm 5cm 23cm]{bsp11}
\end{center}
\end{figure}
\end{frame}
\begin{frame}
\begin{center}
\huge{Wilde Färbungen}
\end{center}
\end{frame}
\begin{figure}[H]
\begin{center}
\includegraphics[viewport=0cm 23.7cm 6cm 28cm,scale=1.3]{colouredTriangle}
\end{center}
%\caption{Ausschnitt der wild gefärbten Sphäre $Y$}
\end{figure}
\begin{frame}
\begin{block}{Satz 6.44}
Sei $X$ ein Multi-Tetraeder. Dann besitzt $X$ bis auf Permutation der Farben $a,b,c$ genau eine wilde Färbung.
\end{block}
\end{frame}
\begin{frame}
Die gleichschenkligen Färbungen der Multi-Tetraeder sind jedoch nicht eindeutig.
\end{frame}
\begin{frame}
\textbf{Doppel-Tetraeder:}
\begin{columns}

    \column{0.5\textwidth}
\begin{figure}[H]
\begin{center}
\includegraphics[viewport=24cm 12.5cm 5cm 17cm]{coloureddbvortrag1}
\end{center}
%\caption{Doppel-Tetraeder}
\end{figure}  
  \column{0.5\textwidth}
\begin{figure}[H]
\begin{center}
\includegraphics[viewport=24cm 12.5cm 5cm 17cm]{coloureddbvortrag2}
\end{center}
%\caption{Doppel-Tetraeder}
\end{figure}  
\end{columns}
\end{frame}

\begin{figure}[H]
\begin{center}
\includegraphics[scale=1.3,viewport=1.5cm 24.cm 13cm 28.4cm]{deffaerbung}
\end{center}
%\caption{Typen einer Kante einer wild gefärbten Sphäre}
\end{figure}
\begin{frame}
\begin{block}{Lemma 6.45}
Sei $X$ ein Multi-Tetraeder. Falls $X$ eine zahme Färbung besitzt, muss diese eine $rrr$-Struktur sein.
\end{block}
\end{frame}

\begin{frame}
\textbf{Ausschnitt eines Multi-Tetraeders:}
\begin{figure}[H]
\begin{center}
\includegraphics[scale=0.8,viewport=0cm 19.5cm 10cm 26.5cm]{notcol3gon}
\end{center}
%\caption{Ausschnitt der Sphäre $X$}
\end{figure}
\end{frame}


\begin{frame}
\textbf{Ausschnitt eines Multi-Tetraeders:}
\begin{figure}[H]
\begin{center}
\includegraphics[scale=0.8,viewport=0cm 19.5cm 10cm 26.5cm]{col3gonvortrag1}
\end{center}
%\caption{Ausschnitt der Sphäre $X$}
\end{figure}
\end{frame}
\begin{frame}
\textbf{Ausschnitt eines Multi-Tetraeders:}
\begin{figure}[H]
\begin{center}
\includegraphics[scale=0.8,viewport=0cm 19.5cm 10cm 26.5cm]{col3gonvortrag2}
\end{center}
%\caption{Ausschnitt der Sphäre $X$}
\end{figure}
\end{frame}
\begin{frame}
\textbf{Ausschnitt eines Multi-Tetraeders:}
\begin{figure}[H]
\begin{center}
\includegraphics[scale=0.8,viewport=0cm 19.5cm 10cm 26.5cm]{col3gonvortrag3}
\end{center}
%\caption{Ausschnitt der Sphäre $X$}
\end{figure}
\end{frame}
\begin{frame}
\textbf{Ausschnitt eines Multi-Tetraeders:}
\begin{figure}[H]
\begin{center}
\includegraphics[scale=0.8,viewport=0cm 19.5cm 10cm 26.5cm]{col3gonvortrag4}
\end{center}
%\caption{Ausschnitt der Sphäre $X$}
\end{figure}
\end{frame}
\begin{frame}
\textbf{Ausschnitt eines Multi-Tetraeders:}
\begin{figure}[H]
\begin{center}
\includegraphics[scale=0.8,viewport=0cm 19.5cm 10cm 26.5cm]{col3gonvortrag5}
\end{center}
%\caption{Ausschnitt der Sphäre $X$}
\end{figure}
\end{frame}

\begin{frame}

\textbf{Ausschnitt eines Multi-Tetraeders:}
\begin{figure}[H]
\begin{center}
\includegraphics[scale=0.8,viewport=0cm 19.5cm 10cm 26.5cm]{col3gon2}
\end{center}
%\caption{Ausschnitt der Sphäre $X$}
\end{figure}
\end{frame}
\begin{frame}
\begin{block}{Lemma 7.18}
Seien $X$ und $Y$ vertex-treue Sphären und $\omega_X$ beziehungsweise $\omega_Y$ eine zahme Färbung auf $X$ beziehungsweise $Y.$ Weiterhin seien $F\in X_2$ beziehungsweise $F'\in Y_2$ eine Fläche und $\phi:X_0(F)\to Y_0(F')$ eine bijektive Abbildung. 
Falls $X$ und $Y$ $mmm$-Strukturen besitzen, dann existiert eine $mmm$-Struktur auf $X\#_{\phi}Y$.
\end{block}
\end{frame}
\end{document}
