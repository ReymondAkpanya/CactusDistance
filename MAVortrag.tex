\documentclass{beamer} 
\usetheme{Warsaw}             % Falls Ihnen das Layout nicht gefällt, können Sie hier
                              % auch andere Themes wählen. Ein Verzeichnis der möglichen 
                              % Themes finden Sie im Kapitel 15 des beameruserguide.

\usepackage[utf8]{inputenc}
\usepackage[ngerman]{babel}
\usepackage{amsmath}
\usepackage{amsfonts}
\usepackage{amssymb}
\usepackage{float}
\usepackage{graphicx}
\usepackage{pdfpages}

\usepackage{verbatim}
\newcommand{\gelb}{0.550000011920929}
\usepackage{pgf,tikz,pgfplots}
\pgfplotsset{compat=1.15}
\usetikzlibrary{arrows}
\AtBeginSection[]{\frame<beamer>{\frametitle{Übersicht} \tableofcontents[current]}}

\newcommand{\defin}[1]{\textit{\color{blue}#1}}

% ========== Abkürzungen ==========
\newcommand{\N}{\mathbb{N}}
\newcommand{\Z}{\mathbb{Z}}
\newcommand{\Q}{\mathbb{Q}}
\newcommand{\R}{\mathbb{R}}
\newcommand{\C}{\mathbb{C}}
\DeclareMathOperator{\Pot}{Pot}

\DeclareMathOperator{\Aut}{Aut}
\author{Reymond Akpanya}
\title{Klassifikation der Sphären ohne Zweier Taillen}
\date{14.05.2021 }

\begin{document}
\frame{\maketitle}
\frame{\tableofcontents[currentsection]}
\begin{frame}{Gliederung}
\tableofcontents
\end{frame}
\section{Kantendrehung}
\subsection{Transitivität der Kantendrehung}
\begin{frame}{definition}
Sei $X$ eine vertex-treue Sphäre. Eine Kante $e\in X_1$ heißt  \emph{drehbar}, falls es keine Kante $e'\in X_1\setminus \{e\}$ mit $X_0(e')=X_0(X_2(e))-X_0(e)$ gibt.
\end{frame}
\begin{frame}
\begin{figure}[H]
\begin{center}
\includegraphics[viewport=17cm 17cm 5cm 20.7cm]{Image_Octahedron}
\end{center}
\caption{Oktaeder}
\end{figure}
\end{frame}
\begin{frame}
\begin{figure}[H]
\begin{center}
\includegraphics[viewport=1cm 23.6cm 5.5cm 26.7cm]{ET_Example1}
\end{center}
\caption{Tetraeder}
\end{figure}
\end{frame}
\begin{frame}{definition}
Sei $X$ eine vertex-treue Sphäre, $\xi$ der zugehörige Flächen-Träger und $e$ eine drehbare Kante in $X$. Dann definieren wir die durch die Kantendrehung $e$ entstandene Sphäre $X^e$ durch den Flächen-Träger $\xi \Delta \Pot_3(X_0(X_2(e))).$
\end{frame}
\begin{figure}[H]
\begin{center}
\includegraphics[viewport=5cm 22.5cm 4cm 27cm]{Image_Kantendrehung}
\end{center}
\caption{Kantendrehung}
\end{figure}

\begin{figure}[H]
\begin{center}
\includegraphics[viewport=12cm 11cm 18cm 18cm,scale=0.85]{Double6gon}
\end{center}
\caption{Doppel-6-Gon}
\end{figure}
\begin{frame}

\begin{block}{Definition}
Sei $X$ eine Sphäre in $X$ und $e_1,\ldots,e_n$ Kanten in $X$. Wir nennen $E=(e_1,\ldots,e_n)$ eine \emph{drehbare Kantensequenz}, falls Folgendes erfüllt ist: 
\begin{itemize}
\item Die Kante $e_1$ ist eine drehbare Kante in $X^{()}:=X.$ 
\item Für alle $1\leq i < n$ ist die Kante $e_{i+1}$ eine drehbare Kante in 
\[
X^{(e_1,\ldots,e_i)}:=(X^{(e_1,\ldots,e_{i-1})})^{e_i}.
\] 
\end{itemize}
Wir nennen $X^E$ die durch die Kantensequenz $E$ entstandene Sphäre. 
\end{block}
\end{frame}
\begin{frame}
\begin{block}{Lemma 5.6}
Sei $X$ eine vertex-treue Sphäre und $V$ eine beliebige Ecke in $X,$ die $\deg(X)\geq 4$ erfüllt. Dann gibt es eine drehbare Kante in $X_1(V).$ 
\end{block}
\end{frame}
\begin{frame}
\begin{block}{Satz 5.7}
Sei $X$ eine vertex-treue Sphäre mit $2n$ Flächen, wobei $n\geq 3$ ist. Dann existiert eine drehbare Kantensequenz $E,$ sodass $X^E$ zum Doppel-$n$-Gon isomorph ist. 
\end{block}
\end{frame}
\begin{frame}
\begin{figure}[H]
\begin{center}
\includegraphics[viewport=18cm 22.5cm -3cm 27cm]{Image_Tetraedererweiterung}
\end{center}
\caption{Tetraeder-Erweiterung und Tetraeder-Entfernung}
\end{figure}
\end{frame}
\begin{frame}
\begin{figure}[H]
\includegraphics[scale=.95,viewport=0cm 17cm 19cm 27cm]{comute}
\caption{Verträglichkeit der Tetraeder-Erweiterung und Kantendrehung}
\end{figure}
\end{frame}
\begin{frame}
\begin{block}{satz}
Seien $X$ und $Y$ vertex-treue Sphären mit $\vert X_2\vert=\vert Y_2\vert=2n$.
Dann existiert eine drehbare Kantensequenz $E=(e_1,\ldots,e_l)$ in $X,$ sodass  $X^E$ zu $Y$ isomorph ist. 
\end{block}
\end{frame}

\section{Multi-Tetraeder}
\subsection{Konstruktion und Klassifikation}
\begin{frame}
\begin{block}{Definition 6.1}
\begin{enumerate}
\item Sei $X$ eine geschlossene simpliziale Fläche mit $\vert X_0\vert > 6.$
 Die simpliziale Fläche, die durch Entfernen aller Tetraeder entsteht, bezeichnen wir mit $X^{(1)}$. Für $i>1$ definieren wir analog 
\[
X^{(i)}:=(X^{(i-1)})^{(1)}.
\]
\item Wir nennen $X$ einen \emph{Multi-Tetraeder vom Grad} $k$, falls $X^{(k-1)}$ ein Tetraeder oder ein Doppel-Tetraeder ist. Hierfür bezeichne $a_0$ die Anzahl der Ecken vom Grad 3 in $X_0$ oder anders gesagt die Anzahl der Tetraeder, die von $X$ entfernt wurden und analog bezeichne $a_i$ die Anzahl der Ecken vom Grad 3 in $X^{(i)}$. Falls $X^{(i)}$ ein Tetraeder ist, so ist $a_i$ als 1 definiert. Das damit konstruierte Tupel $(a_0,a_1,\ldots,a_k)$ nennen wir den \emph{Typ} von $X$ und $T:=\sum_{i=0}^{k} a_i$ die \emph{Tetraeder-Zahl} von $X$.\\
\end{enumerate}
\end{block}
\end{frame}
\begin{frame}
\begin{figure}[H]
\begin{center}
\includegraphics[viewport=22cm 12.5cm 5cm 17cm]{Image_DoubleTetraeder}
\end{center}
\caption{Doppel-Tetraeder}
\end{figure}
\end{frame}
\begin{frame}
\begin{figure}[H]
\begin{center}
\includegraphics[viewport=24cm 14cm 0cm 19cm]{Image_Parallelepiped}
\end{center}
\caption{Parallelepiped}
\end{figure}
\end{frame}
\begin{figure}[H]
\begin{center}
\includegraphics[viewport=19cm 19cm 0cm 23cm]{Image_MultiTetraeder1}
\end{center}
\caption{Multi-Tetraeder mit 8 Flächen}
\end{figure}
\begin{frame}
\begin{block}{Definition 6.9}
Seien $X$ und $Y$ Multi-Tetraeder. Falls $X=Y^{(1)}$ ist, dann nennen wir $Y$ einen \emph{Enkel} von $X.$ 
\end{block}
\end{frame}
\begin{frame}
\begin{block}{Lemma 6.21}
Sei $X$ ein Multi-Tetraeder und $\phi \in \Aut(X).$ Für eine Überdeckung $M\in U_X^l$ mit  $0\leq l\leq \vert X_2\vert -a_k$ ist $\{\phi(m)\mid m\in M\}$ wieder eine $a_k+l$-elementige Überdeckung.
\end{block}
\end{frame}
\begin{frame}
\begin{block}{Bemerkung 6.15}
Sei $X$ ein Multi-Tetraeder vom Typ $(a_0,\ldots ,a_k)$ und $G$ seine Automorphismengruppe.
Durch die obige Erkenntnis kann nun die in 6.10 eingeführte Gruppenoperation leicht abgeändert werden, um so die Anzahl der Enkel eines Multi-Tetraeders zu bestimmen. Für $0\leq l\leq \vert X_2 \vert -a_k$ ist die Abbildung
\[
\theta_l: G\times U_X^l \to U_X^l, (\phi, M)\mapsto \phi(M):=\{\phi(F)\mid F\in M\}
\] 
wohldefiniert und es lässt sich leicht nachprüfen, dass diese eine Gruppenoperation bildet.
\end{block}
\end{frame}
\begin{frame}
\begin{block}{Satz 6.17}
Sei $X$ ein Multi-Tetraeder und $\Aut(X)$ die Automorphismengruppe von $X$. Dann ist $\Aut(X)$ auflösbar.
Weiterhin seien $l,l'$ minimal mit der Eigenschaft, dass
\[
\vert \Aut(X)^l\vert=\vert \Aut(X^{(1)})^{l'}\vert =1
\] ist. Dann gilt $l\leq l'$.
\end{block}
\end{frame}
\subsection{Kaktus-Distanz}
\begin{frame}
\begin{block}{Definition 6.18}
Sei $X$ eine vertex-treue Sphäre. Die minimale Anzahl an Kantendrehungen, die es braucht, um aus $X$ einen Multi-Tetraeder zu  konstruieren, nennen wir die \emph{Kaktus-Distanz} $\zeta(X)$ von $X$.
\end{block}
\end{frame}
\subsection{Wilde Färbungen auf Multi-Tetraedern}
\section{2-taillenfreie Sphären}
\subsection{Multi-Sphären}
\begin{frame}
\begin{block}{Satz 7.16}
Sei $Y$ eine Multi-Sphäre und $X$ eine Sphäre, sodass die Blöcke von $Y$ zu $X$ isomorph sind. Falls $\Aut(X)$ auflösbar ist, dann folgt die Auslösbarkeit von $\Aut(Y).$ 

\end{block}
\end{frame}
\subsection{Flächengraph von Sphären mit 3-Taillen}
\subsection{Der Sphären Graph}
\begin{frame}
\begin{block}{Definition 7.19}
Sei $n\in \mathbb{N}$ gerade. Die Menge der Isomorphieklassen der Sphären ohne 2- oder 3-Taillen bezeichnen wir mit $\mathcal{S}_n.$ Weiterhin bezeichnen wir die Menge der Isomorphieklassen der Sphären ohne 2-Taillen mit $\mathcal{S}_n^3.$
\end{block}
\end{frame}
\begin{frame}
Es gilt $\mathcal{S}_2=\mathcal{S}_2^3=\emptyset.$ 
 Für $n=4$ ist $\mathcal{S}_n^3=\mathcal{S}_n=\{T\},$ wobei $T$ ein Tetraeder ist.
 Falls $n\geq 6$ ist, dann herrscht eine echte Teilmengenbeziehung zwischen $\mathcal{S}_n$ und $\mathcal{S}_n^3.$ Beispielsweise ist 
 \[
 \mathcal{S}_6=\emptyset \subset \{DT\}=\mathcal{S}_6^3.
 \]
\end{frame}
\begin{frame}
\begin{block}{Definition 7.23}
Sei $n\in \mathbb{N}$ gerade. Dann definiert das Tripel $\mathcal{G}_n=(V,E,c)$ den \emph{taillenfreien Sphären-Graph}. Die Knotenmenge erhalten wir durch $V=\mathcal{S}_n.$ Zwei Knoten $V_X,V_Y$ mit zugehörigen vertex-treuen Sphären $X,Y$ sind durch eine Kante verbunden, falls $X^e\cong Y$ für eine drehbare Kante $e\in X_1$ ist oder es eine drehbare Kantensequenz $E=(e_1,\ldots,e_m)$ in $X$ gibt, sodass $X^E\cong Y$ ist und für alle $1\leq i < m $ die Sphäre $X^{(e_1,\ldots,e_i)}$ eine 3-Taille besitzt. Die Gewichtsfunktion $c:E\to\mathbb{Z}^+$  erhalten wir folgendermaßen: Sei $\{V_X,V_Y\}=e\in E.$ Das Gewicht $c(e)$ ist die minimale Länge einer Kantensequenz, die $X$ isomorph in $Y$ umwandelt und obige Bedingung erfüllt.
\end{block}
\end{frame}
 \begin{figure}[H]
\begin{center}
\includegraphics[scale=0.7,viewport=1cm 15.5cm 24cm 22cm]{spheregraph}
\end{center}
\caption{der taillenfreie Sphären-Graph $\mathcal{G}_{14}$}
\end{figure}

\end{document}
