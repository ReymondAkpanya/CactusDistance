\documentclass{beamer} 
\usetheme{Warsaw}             % Falls Ihnen das Layout nicht gefällt, können Sie hier
                              % auch andere Themes wählen. Ein Verzeichnis der möglichen 
                              % Themes finden Sie im Kapitel 15 des beameruserguide.

\usepackage[utf8]{inputenc}
\usepackage[ngerman]{babel}
\usepackage{amsmath}
\usepackage{amsfonts}
\usepackage{amssymb}
\usepackage{float}
\usepackage{graphicx}
\usepackage{pdfpages}
\include{comment}
\usepackage{comment}
\newcommand{\gelb}{0.550000011920929}
\usepackage{pgf,tikz,pgfplots}
\pgfplotsset{compat=1.15}
\usetikzlibrary{arrows}
\AtBeginSection[]{\frame<beamer>{\frametitle{Übersicht} \tableofcontents[current]}}

\newcommand{\defin}[1]{\textit{\color{blue}#1}}

% ========== Abkürzungen ==========
\newcommand{\N}{\mathbb{N}}
\newcommand{\Z}{\mathbb{Z}}
\newcommand{\Q}{\mathbb{Q}}
\newcommand{\R}{\mathbb{R}}
\newcommand{\C}{\mathbb{C}}

\author{Reymond Akpanya}
\title{Manipulation diskreter simplizialer Flächen}
\date{19.10.2018 \\[.5\baselineskip] Vortrag zur Bachelorarbeit}

\begin{document}
\frame{\maketitle}
%\frame{\tableofcontents[currentsection]}

\begin{frame}{Beispiel}
\begin{figure}[H]
\definecolor{sqsqsq}{rgb}{0.12549019607843137,0.12549019607843137,0.12549019607843137}
\definecolor{ttqqqq}{rgb}{0.2,0.,0.}
\definecolor{ffffqq}{rgb}{1.,1.,0.}
\definecolor{qqqqff}{rgb}{0.,0.,1.}
\begin{tikzpicture}[line cap=round,line join=round,>=triangle 45,x=.90cm,y=.90cm]

x=1.0cm,y=1.0cm,
axis lines=middle,
ymajorgrids=true,
xmajorgrids=true,
xmin=-5.056290110700678,
xmax=5.380866801866215,
ymin=-0.9227448489396118,
ymax=4.359364127681193,
xtick={-5.0,-4.5,...,5.0},
ytick={-0.5,0.0,...,4.0},]
\clip(-3.056290110700678,-0.4) rectangle (5.380866801866215,3.859364127681193);
\fill[line width=2.pt,color=ffffqq,fill=ffffqq,fill opacity=0.5] (-2.,0.) -- (2.,0.) -- (0.,3.4641016151377553) -- cycle;
\fill[line width=2.pt,color=ffffqq,fill=ffffqq,fill opacity=0.5] (0.,3.4641016151377553) -- (2.,0.) -- (4.,3.464101615137754) -- cycle;
\draw [line width=2.pt] (-2.,0.)-- (2.,0.);
\draw [line width=2.pt,color=black] (2.,0.)-- (0.,3.4641016151377553);
\draw [line width=2.pt] (0.,3.4641016151377553)-- (-2.,0.);
\draw [line width=2.pt,color=black] (0.,3.4641016151377553)-- (2.,0.);
\draw [line width=2.pt] (2.,0.)-- (4.,3.464101615137754);
\draw [line width=2.pt,color=black] (4.,3.464101615137754)-- (0.,3.4641016151377553);
\begin{scriptsize}
\draw [fill=qqqqff] (-2.,0.) circle (2.5pt);
\visible<2->{
\draw[color=black] (-2.347666183295526,0.0934108260899206) node {$V_1$};
\draw[color=black] (4.087566923569883,3.75145261535057) node {$V_1$};
\draw[color=black] (2.390727102068595,0.0934108260899206) node {$V_2$};
\draw[color=sqsqsq] (0.04884098783747624,1.2554444197699985) node {$F_1$};
\draw[color=black] (0.057916776457099625,-0.22407545041275472) node {$e_3$};
\draw[color=black] (1.174238776670776,1.9089012003828816) node {$e_1$};
\draw[color=black] (-1.1673146871920574,1.9089012003828816) node {$e_2$};
\draw[color=black] (0.08514414231596978,3.75145261535057) node {$V_3$};
\draw[color=sqsqsq] (2.0455144841546207,2.417145363081791) node {$F_2$};
\draw[color=black] (3.4162912160860375,1.9089012003828816) node {$e_3$};
\draw[color=black] (2.054590272774244,3.6603620787860503) node {$e_4$};}
\draw [fill=qqqqff] (4.,3.464101615137754) circle (2.5pt);
\draw [fill=qqqqff] (2.,0.) circle (2.5pt);
\draw [fill=qqqqff] (0.,3.4641016151377553) circle (2.5pt);

\end{scriptsize}
\end{tikzpicture}
%\caption{Open-Bag}
\end{figure}
\visible<3->{\begin{align*}
  X_{0}=\{\,V_{1},&V_{2},V_{3}\,\},X_{1}=\{\,e_{1},e_{2},e_{3},e_{4} \,\}, X_{2}=\{\,F_{1},F_{2}\,\} \text{ und } x<y \Leftrightarrow\\
 (x,y)\in\{&\,(e_{1},F_{1}),(e_{1},F_{2}),(e_{2},F_{1}),(e_{3},F_{1}),(e_{3},F_{2}),(e_{4},F_{2}),(V_{1},e_{2}),\\&(V_{1},e_{3}),(V_{1},e_{4}),
  (V_{1},F_{1}),(V_{1},F_{2}),(V_{2},e_{1}),(V_{2},e_{3})
 (V_{2},F_{1}),\\& (V_{2},F_{2}), (V_{3},e_{1}),(V_{3},e_{2}),(V_{3},e_{4}),(V_{3},F_{1}),(V_3,F_2) \,\}.
 \end{align*}}
 
\end{frame}
\begin{frame}{Gliederung}
\tableofcontents
\end{frame}
\section{Wanderinghole}

\begin{frame}
\begin{center}
\begin{figure}[H]
\definecolor{uuuuuu}{rgb}{0.26666666666666666,0.26666666666666666,0.26666666666666666}
\definecolor{ududff}{rgb}{0.30196078431372547,0.30196078431372547,1.}
\definecolor{ffffqq}{rgb}{1.,1.,0.}
\definecolor{qqqqff}{rgb}{0.,0.,1.}
\begin{tikzpicture}[line cap=round,line join=round,>=triangle 45,x=1.5cm,y=1.5cm]
%\begin{axis}[
x=1.5cm,y=1.5cm,
axis lines=middle,
ymajorgrids=true,
xmajorgrids=true,
xmin=-4.3,
xmax=7.0600000000000005,
ymin=-2.46,
ymax=6.3,
xtick={-4.0,-3.0,...,7.0},
ytick={-2.0,-1.0,...,6.0},]
\clip(-3.8,-0.46) rectangle (8.06,4.3);
\fill[line width=2.pt,color=ffffqq,fill=ffffqq,fill opacity=\gelb] (-2.2,-0.1) -- (2.,-0.1) -- (2.,4.1) -- (-2.2,4.1) -- cycle;
%\fill[line width=2.pt,color=ffffqq,fill=ffffqq,fill opacity=\gelb] (-1.,2.) -- (1.,2.) -- (0.,3.7320508075688776) -- cycle;
%\fill[line width=2.pt,color=ffffqq,fill=ffffqq,fill opacity=\gelb] (1.,2.) -- (-1.,2.) -- (0.,0.2679491924311226) -- cycle;
\draw [line width=2.pt] (-1.,2.)-- (1.,2.);
\draw [line width=2.pt] (1.,2.)-- (0.,3.7320508075688776);
\draw [line width=2.pt] (0.,3.7320508075688776)-- (-1.,2.);
\draw [line width=2.pt] (1.,2.)-- (-1.,2.);
\draw [line width=2.pt] (-1.,2.)-- (0.,0.2679491924311226);
\draw [line width=2.pt] (0.,0.2679491924311226)-- (1.,2.);
\begin{scriptsize}
\draw [fill=qqqqff] (-1.,2.) circle (2.5pt);
%\draw[color=black] (-1.6,2.2) node {$\{V_1^1,V_1^2,V_1^3\}$};
\draw [fill=qqqqff] (1.,2.) circle (2.5pt);
%\draw[color=black] (1.54,2.07) node {$\{V_2^1,V_2^2\}$};
\draw[color=black] (0.,2.75) node {$F$};
%\draw[color=black] (0.06,1.8) node {$\{e_3^1,e_3^2\}$};
%\draw[color=black] (0.87,2.96) node {$\{e_1^1,e_1^2\}$};
%\draw[color=black] (-0.87,2.96) node {$\{e_2^1,e_2^2\}$};
\draw [fill=qqqqff] (0.,3.7320508075688776) circle (2.5pt);
%\draw[color=black] (0.,3.91) node {$\{V_3^1,V_3^2\}$};
\draw[color=black] (0.,1.31) node {$F'$};
%\draw[color=black] (-1.,1.16) node {$\{f^1,f^2\}$};
%\draw[color=black] (0.8,1.16) node {$\{g\}$};
\draw [fill=qqqqff] (0.,0.2679491924311226) circle (2.5pt);
%\draw[color=black] (0.,0.05) node {$\{V_4\}$};
\end{scriptsize}
%\end{axis}
\end{tikzpicture}
\caption{Ausschnitt eines Mendings einer simplizialen Fläche}\label{abb16}
\end{figure}
\end{center}
\end{frame}

\subsection{Prozedur $P^1$}
%\subsection{Prozedur P^1}
\begin{frame}
\center{\huge{\bf{Prozedur $P^1$}}}
\end{frame}
\begin{frame}
Anwenden des $Cratercuts$:\\
\begin{figure}[H]
\definecolor{ffffff}{rgb}{1.,1.,1.}
\definecolor{qqqqff}{rgb}{0.,0.,1.}
\definecolor{ududff}{rgb}{0.30196078431372547,0.30196078431372547,1.}
\definecolor{ffffqq}{rgb}{1.,1.,0.}
\begin{tikzpicture}[line cap=round,line join=round,>=triangle 45,x=1.4cm,y=1.38cm]
%\begin{axis}[
x=1.0cm,y=1.0cm,
axis lines=middle,
ymajorgrids=true,
xmajorgrids=true,
xmin=-4.3,
xmax=18.7,
ymin=-5.34,
ymax=6.3,
xtick={-4.0,-3.0,...,18.0},
ytick={-5.0,-4.0,...,6.0},]
\clip(-4.1,-0.) rectangle (3.7,4.3);
\fill[line width=2.pt,color=ffffqq,fill=ffffqq,fill opacity=\gelb] (-2.2,0.) -- (2.,0.) -- (2.,4.2) -- (-2.2,4.2) -- cycle;
%\fill[line width=2.pt,color=ffffqq,fill=ffffqq,fill opacity=\gelb] (-1.,2.) -- (1.,2.) -- (0.,3.7320508075688776) -- cycle;
%\fill[line width=2.pt,color=ffffqq,fill=ffffqq,fill opacity=\gelb] (1.,2.) -- (-1.,2.) -- (0.,0.2679491924311226) -- cycle;
\draw [line width=2.pt] (0.,3.7320508075688776)-- (-1.,2.);
\draw [line width=2.pt] (1.,2.)-- (-1.,2.);
\draw [line width=2.pt] (-1.,2.)-- (0.,0.2679491924311226);
\draw [line width=2.pt] (0.,0.2679491924311226)-- (1.,2.);
\draw [rotate around={-60.:(0.5,2.8660254037844513)},line width=2.pt,color=ffffff,fill=ffffff,fill opacity=1.0] (0.5,2.8660254037844513) ellipse (1.4633824013732526cm and 0.1641459454658895cm);
\draw [rotate around={-60.:(0.5,2.866025403784439)},line width=2.pt] (0.5,2.866025403784439) ellipse (1.463382401373216cm and 0.16414594546590086cm);
\begin{scriptsize}
%\draw[color=black] (0.48,2.17) node {$Vieleck1$};
\draw [fill=qqqqff] (-1.,2.) circle (2.5pt);
%\draw[color=black] (-1.56,2.27) node {$\{V_1^1,V_1^2,V_1^3\}$};
\draw [fill=qqqqff] (1.,2.) circle (2.5pt);
%\draw[color=black] (1.49,2.12) node {$\{V_2^1,V_2^2\}$};
\draw[color=black] (.,2.7) node {$F$};
%\draw[color=black] (-0.,1.8) node {$\{e_3^1,e_3^2\}$};
\draw [fill=qqqqff] (0.,3.7320508075688776) circle (2.5pt);
%\draw[color=black] (0.14,4.01) node {$\{V_3^1,V_3^2\}$};
\draw[color=black] (0.06,1.47) node {$F'$};
\draw [fill=qqqqff] (0.,0.2679491924311226) circle (2.5pt);
%\draw[color=black] (0.76,1.13) node {$\{g\}$};
%\draw[color=black] (-0.94,1.13) node {$\{f^1,f^2\}$};
%\draw[color=black] (0.36,0.13) node {$\{V_4\}$};
%\draw[color=black] (0.21,2.67) node {$\{e_1^1\}$};
%\draw[color=black] (0.84,3.13) node {$\{e_1^2\}$};
%\draw[color=black] (-0.86,2.97) node {$\{e_2^1,e_2^2\}$};
\end{scriptsize}
%\end{axis}
\end{tikzpicture}
%\caption{simpliziale Fläche nach einem Cratercut}
\end{figure}


\end{frame}
\begin{frame}
Anwenden des $Ripcuts$:\\
 \begin{figure}[H]
\definecolor{xdxdff}{rgb}{0.49019607843137253,0.49019607843137253,1.}
\definecolor{ffffff}{rgb}{1.,1.,1.}
\definecolor{qqqqff}{rgb}{0.,0.,1.}
\definecolor{ffffqq}{rgb}{1.,1.,0.}
\begin{tikzpicture}[line cap=round,line join=round,>=triangle 45,x=.78cm,y=.78cm]

x=1.cm,y=1.cm,
axis lines=middle,
xmin=-4.0,
xmax=14.0,
ymin=-3.3,
ymax=5.34,
xtick={-9.0,-8.0,...,14.0},
ytick={-5.0,-4.0,...,6.0},]
\clip(-5.1,-3.0) rectangle (4.,5.3);
\fill[line width=2.pt,color=ffffqq,fill=ffffqq,fill opacity=\gelb] (4.,-3.) -- (4.,5.) -- (-4.,5.) -- (-4.,-3.) -- cycle;   
\fill[line width=2.pt,color=ffffff,fill=ffffff,fill opacity=1.0] (-2.,1.) -- (2.,1.) -- (0.,4.464101615137755) -- cycle;
\fill[line width=2.pt,color=ffffqq,fill=ffffqq,fill opacity=0.1] (-2.,1.) -- (0.,-2.44) -- (1.9791273890184695,1.012050807568877) -- cycle;
\fill[line width=2.pt,color=ffffqq,fill=ffffqq,fill opacity=0.550000011920929] (-2.,1.) -- (0.,3.48) -- (2.,1.) -- cycle;
\draw [line width=2.pt] (-2.,1.)-- (2.,1.);
\draw [line width=2.pt] (2.,1.)-- (0.,4.464101615137755);
\draw [line width=2.pt] (0.,4.464101615137755)-- (-2.,1.);
\draw [line width=2.pt] (-2.,1.)-- (0.,-2.44);
\draw [line width=2.pt] (0.,-2.44)-- (1.9791273890184695,1.012050807568877);
\draw [line width=2.pt] (1.9791273890184695,1.012050807568877)-- (-2.,1.);
\draw [line width=2.pt] (-2.,1.)-- (0.,3.48);
\draw [line width=2.pt] (0.,3.48)-- (2.,1.);
\draw [line width=2.pt] (2.,1.)-- (-2.,1.);
\begin{scriptsize}
%\draw[color=black] (0.12,0.617) node {$\{e_3^1,e_3^2\}$};
\draw [fill=qqqqff] (-2.,1.) circle (2.5pt);
%\draw[color=black] (-3.01,1.22) node {$\{V_1^1,V_1^2,V_1^3\}$};
\draw [fill=qqqqff] (2.,1.) circle (2.5pt);
%\draw[color=black] (2.59,1.42) node {$\{V_2^1,V_2^2\}$};
%\draw[color=ffffff] (0.48,2.33) node {$Vieleck1$};
%\draw[color=black] (0.07,1.5) node {$e_3$};
%\draw[color=black] (1.37,3.12) node {$\{e_1^2\}$};
%\draw[color=black] (-1.27,3.12) node {$\{e_2^2\}$};
\draw [fill=qqqqff] (0.,4.464101615137755) circle (2.5pt);
%\draw[color=black] (0.42,4.8) node {$\{V_3^2\}$};
\draw [fill=qqqqff] (0.,-2.44) circle (2.5pt);
%\draw[color=black] (0.09,-2.84) node {$\{V_4\}$};
\draw[color=black] (0.1,-0.13) node {F'};
%\draw[color=black] (-1.74,-0.71) node {$\{f^1,f^2\}$};
%\draw[color=black] (1.42,-0.69) node {$\{g\}$};
\draw [fill=qqqqff] (0.,3.48) circle (2.5pt);
%\draw[color=black] (0.,2.8) node {$\{V_3^1\}$};
\draw[color=black] (0.06,2.01) node {F};
%\draw[color=black] (-0.95,1.76) node {$\{e_2^1\}$};
%\draw[color=black] (0.85,1.76) node {$\{e_1^1\}$};
\end{scriptsize}

\end{tikzpicture}
\end{figure}

\end{frame}
\begin{frame}
 Anwenden des $Splitcuts$:\\
 \begin{figure}[H]
\definecolor{qqqqff}{rgb}{0.,0.,1.}
\definecolor{ffffff}{rgb}{1.,1.,1.}
\definecolor{ududff}{rgb}{0.30196078431372547,0.30196078431372547,1.}
\definecolor{ffffqq}{rgb}{1.,1.,0.}
\begin{tikzpicture}[line cap=round,line join=round,>=triangle 45,x=.7cm,y=.7cm]
x=.50cm,y=.50cm,
axis lines=middle,
ymajorgrids=true,
xmajorgrids=true,
xmin=-3.5,
xmax=10.0,
ymin=-3.0,
ymax=5.2,
xtick={-9.0,-8.0,...,14.0},
ytick={-5.0,-4.0,...,6.0},]
\clip(-5.,-3.3) rectangle (14.,5.34);
%----------
%\fill[line width=2.pt,color=ffffqq,fill=ffffqq,fill opacity=0.550000011920929] (-4.,-3.) -- (-4.,5.) -- (-3.,5.) -- (-3.,-3.) -- cycle;

%--------------
\fill[line width=2.pt,color=ffffqq,fill=ffffqq,fill opacity=\gelb] (-3.5,-3.) -- (-3.5,5.) -- (3.,5.) -- (3.,-3.) -- cycle;
\fill[line width=2.pt,color=ffffff,fill=ffffff,fill opacity=1.0] (-2.,1.) -- (2.,1.) -- (0.,4.464101615137755) -- cycle;
\fill[line width=2.pt,color=ffffqq,fill=ffffqq,fill opacity=0.20000000298023224] (-2.,1.) -- (0.,-2.44) -- (1.9791273890184695,1.012050807568877) -- cycle;
\fill[line width=2.pt,color=ffffqq,fill=ffffqq,fill opacity=\gelb] (5.,1.) -- (9.,1.) -- (7.,4.464101615137755) -- cycle;
%\draw [line width=2.pt,color=ffffqq] (-3.5,-3.)-- (-3.5,5.);
%\draw [line width=2.pt,color=ffffqq] (-3.,5.)-- (3.,5.);
%\draw [line width=2.pt,color=ffffqq] (3.,5.)-- (3.,-3.);
%\draw [line width=2.pt,color=ffffqq] (3.,-3.)-- (-3.,-3.);
\draw [line width=2.pt] (-2.,1.)-- (2.,1.);
\draw [line width=2.pt] (2.,1.)-- (0.,4.464101615137755);
\draw [line width=2.pt] (0.,4.464101615137755)-- (-2.,1.);
\draw [line width=2.pt] (-2.,1.)-- (0.,-2.44);
\draw [line width=2.pt] (0.,-2.44)-- (1.9791273890184695,1.012050807568877);
\draw [line width=2.pt] (1.9791273890184695,1.012050807568877)-- (-2.,1.);
\draw [line width=2.pt] (5.,1.)-- (9.,1.);
\draw [line width=2.pt] (9.,1.)-- (7.,4.464101615137755);
\draw [line width=2.pt] (7.,4.464101615137755)-- (5.,1.);
\begin{scriptsize}
%\draw[color=black] (0,0.7) node {$\{e_3^2\}$};
%\draw[color=black] (7,1.4) node {$e_3^1$};

%\draw[color=black] (1.28,2.93) node {$\{e_1^2\}$};
%\draw[color=black] (8.28,2.93) node {$e_1^2$};

%\draw[color=black] (-1.28,2.93) node {$\{e_2^2\}$};
%\draw[color=black] (5.7,2.93) node {$\{e_2^1\}$};
%\draw[color=black] (8.3,2.93) node {$\{e_1^1\}$};

%\draw[color=black] (-1.88,-0.6) node {$\{f^1,f^2\}$};
%\draw[color=black] (1.68,-0.6) node {$\{g\}$};
\draw [fill=qqqqff] (-2.,1.) circle (2.5pt);
%\draw[color=black] (-2.551,1.42) node {$\{V_1^2,V_1^3\}$};
\draw [fill=qqqqff] (2.,1.) circle (2.5pt);
%\draw[color=black] (2.29,1.42) node {$\{V_2^2\}$};
%\draw[color=ffffff] (0.48,2.33) node {$Vieleck1$};
\draw [fill=qqqqff] (0.,4.464101615137755) circle (2.5pt);
%\draw[color=black] (0.43,4.7) node {$\{V_3^2\}$};
\draw [fill=qqqqff] (0.,-2.44) circle (2.5pt);
%\draw[color=black] (0.14,-2.75) node {$\{V_4\}$};
\draw[color=black] (0.1,0.03) node {F'};
\draw [fill=qqqqff] (5.,1.) circle (2.5pt);
%\draw[color=black] (5.08,0.63) node {$\{V_1^1\}$};
\draw [fill=qqqqff] (9.,1.) circle (2.5pt);
%\draw[color=black] (9.1,0.67) node {$\{V_2^1\}$};
\draw[color=black] (7.06,2.33) node {$F$};
%\draw[color=black] (7.06,0.70) node {$\{e_3^1\}$};
\draw [fill=qqqqff] (7.,4.464101615137755) circle (2.5pt);
%\draw[color=black] (7.14,4.83) node {$\{V_3^1\}$};
\end{scriptsize}

\end{tikzpicture}
%\caption{simpliziale Fläche nach einem Splitcut}
\end{figure}
\end{frame}

\subsection{Prozedur $P^2$}
\begin{frame}
\center{\huge{\bf{Prozedur $P^2$}}}
\end{frame}
\begin{frame}
Anwenden des $Ripcuts$ \\
\begin{figure}
\definecolor{ududff}{rgb}{0.30196078431372547,0.30196078431372547,1.}
\definecolor{ffffff}{rgb}{1.,1.,1.}
\definecolor{sqsqsq}{rgb}{0.12549019607843137,0.12549019607843137,0.12549019607843137}
\definecolor{ffffqq}{rgb}{1.,1.,0.}
\definecolor{qqqqff}{rgb}{0.,0.,1.}
\begin{tikzpicture}[line cap=round,line join=round,>=triangle 45,x=.750cm,y=.750cm]

x=1.0cm,y=1.0cm,
axis lines=middle,
ymajorgrids=true,
xmajorgrids=true,
xmin=-4.5,
xmax=10.0,
ymin=-5.0,
ymax=5.2,
xtick={-9.0,-8.0,...,14.0},
ytick={-5.0,-4.0,...,6.0},]
\clip(-4.34966779911168,-10.336419420914822) rectangle (10.164271214115164,4.25368189432285);
%\fill[line width=2.pt,color=ffffqq,fill=ffffqq,fill opacity=\gelb] (-2.,0.) -- (0.,-3.481320628255737) -- (2.014912102788271,-0.008609506558991509) -- cycle;
\fill[line width=2.pt,color=ffffqq,fill=ffffqq,fill opacity=\gelb] (-3.06633318691027,3.9892123475876073) -- (-3.0412642843067688,-3.8824230699117557) -- (2.6995144118949965,-3.9826986803257602) -- (2.6744455092914956,3.9892123475876073) -- cycle;
\fill[line width=2.pt,color=ffffff,fill=ffffff,fill opacity=1.0] (-2.,0.) -- (2.,0.) -- (0.,3.4641016151377553) -- cycle;
\fill[line width=2.pt,color=ffffff,fill=ffffff,fill opacity=1.0] (0.,-3.481320628255737) -- (-2.,0.) -- (-1.010683173423175,0.028325736234424664) -- cycle;
\fill[line width=2.pt,color=ffffqq,fill=ffffqq,fill opacity=0.44999998807907104] (4.,0.) -- (8.,0.) -- (6.,3.4641016151377553) -- cycle;
\draw [line width=2.pt,color=sqsqsq] (-2.,0.)-- (0.,-3.481320628255737);
\draw [line width=2.pt] (0.,-3.481320628255737)-- (2.014912102788271,-0.008609506558991509);
\draw [line width=2.pt,color=sqsqsq] (2.014912102788271,-0.008609506558991509)-- (-1.,0.);
\draw [line width=2.pt,color=ffffff]  (-3.06633318691027,3.9892123475876073)-- (-3.0412642843067688,-3.8824230699117557);
\draw [line width=2.pt,color=ffffff] (2.6995144118949965,-3.9826986803257602)-- (2.6744455092914956,3.9892123475876073);
\draw [line width=2.pt,color=sqsqsq] (2.,0.)-- (0.,3.4641016151377553);
\draw [line width=2.pt] (0.,3.4641016151377553)-- (-2.,0.);
\draw [line width=2.pt,color=sqsqsq] (0.,-3.481320628255737)-- (-2.,0.);
\draw [line width=2.pt,color=ffffff] (-1.5,0.)-- (-1.010683173423175,0.028325736234424664);
\draw [line width=2.pt,color=sqsqsq] (-1.010683173423175,0.028325736234424664)-- (0.,-3.481320628255737);
\draw [line width=2.pt,color=ffffff] (-2.,0.)-- (-1.010683173423175,0.028325736234424664);
\draw [line width=2.pt,color=sqsqsq] (4.,0.)-- (8.,0.);
\draw [line width=2.pt,color=sqsqsq] (8.,0.)-- (6.,3.4641016151377553);
\draw [line width=2.pt,color=sqsqsq] (6.,3.4641016151377553)-- (4.,0.);
\begin{scriptsize}
\draw [fill=qqqqff] (-2.,0.) circle (2.5pt);
%\draw[color=black] (-2.2815938522964825,0.30408366487293736) node {$\{V_1^3\}$};
\draw [fill=qqqqff] (-0.,-3.481320628255737) circle (2.5pt);
%\draw[color=black] (0.4555720184676383,-3.6314584334627392) node {$\{V_4\}$};
\draw[color=black] (0.5937265932008994,-0.9368270140003698) node {$F'$};
%\draw[color=black] (1.4210003791164376,-1.7139629947089057) node {$\{g\}$};
%\draw[color=sqsqsq] (0.3555720184676383,-0.27222548459358455) node {$\{e_3^2\}$};
\draw [fill=qqqqff] (2.014912102788271,-0.108609506558991509) circle (2.5pt);
%\draw[color=black] (2.260808616333726,0.3) node {$\{V_2^2\}$};
%\draw[color=ffffff] (-3.2295566642470322,0.32915256747643856) node {$\{f^1\}$};
%\draw[color=ffffff] (3.21342691526677,0.30408366487293736) node {$h_1$};
%\draw[color=ffffff] (0.5937265932008994,1.3695120255217366) node {$Vieleck1$};
%\draw[color=sqsqsq] (1.258879343435694,2.209320262739025) node {$\{e_1^2\}$};
%\draw[color=black] (-1.199251163777443,2.284526970549529) node {$\{e_2^2\}$};
\draw [fill=qqqqff] (0.,3.4641016151377553) circle (2.5pt);
%\draw[color=black] (0.23022750545013249,3.7892123475876073) node {$\{V_3^2\}$};
\draw [fill=ududff] (-1.010683173423175,0.028325736234424664) circle (2.5pt);
%\draw[color=black] (-0.7725285986899139,0.3547726909079489) node {$\{V_1^2\}$};
%\draw[color=sqsqsq] (-1.0352008551986667,-1.0120337218108733) node {$\{f^2\}$};
\draw[color=ffffff] (-1.4368545176826946,-0.15969103329183398) node {$d$};
%\draw[color=sqsqsq] (-1.5363032968546852,-1.5635495790878988) node {$\{f^1\}$};
%\draw[color=ffffff] (-1.4368545176826946,-0.15969103329183398) node {$l$};
\draw [fill=qqqqff] (4.,0.) circle (2.5pt);
%\draw[color=black] (4.090838506389312,-0.3477078028180926) node {$\{V_1^1\}$};
\draw [fill=qqqqff] (8.,0.) circle (2.5pt);
%\draw[color=black] (8.352276338570503,-0.2725010950075892) node {$\{V_2^1\}$};
\draw[color=black] (6.108885165971154,1.4196498307287388) node {$F$};
%\draw[color=sqsqsq] (5.96626325083409,-0.22208767938658227) node {$\{e_3^1\}$};
%\draw[color=sqsqsq] (7.450347065672471,2.209320262739025) node {$\{e_1^1\}$};
%\draw[color=sqsqsq] (4.842905584494346,2.209320262739025) node {$\{e_2^1\}$};
\draw [fill=qqqqff] (6.,3.4641016151377553) circle (2.5pt);
%\draw[color=black] (6.321970838100914,3.9892123475876073) node {$\{V_3^1\}$};
\end{scriptsize}

\end{tikzpicture}
\caption{simpliziale Fläche nach einem Ripcut}
\end{figure}
\end{frame}
\begin{frame}
Anwenden des $Ripmenders$
\begin{figure}[H]
\definecolor{ffffff}{rgb}{1.,1.,1.}
\definecolor{qqqqff}{rgb}{0.,0.,1.}
\definecolor{ffffqq}{rgb}{1.,1.,0.}
\begin{tikzpicture}[line cap=round,line join=round,>=triangle 45,x=1.2cm,y=1.2cm]
%\begin{axis}[
x=1.0cm,y=1.0cm,
axis lines=middle,
ymajorgrids=true,
xmajorgrids=true,
xmin=-3.583376623376623,
xmax=16.330043290043285,
ymin=-4.489177489177493,
ymax=5.588744588744593,
xtick={-3.0,-2.0,...,16.0},
ytick={-4.0,-3.0,...,5.0},]
\clip(-2.8383376623376623,-0.289177489177493) rectangle (16.330043290043285,4.0588744588744593);
\fill[line width=2.pt,color=ffffqq,fill=ffffqq,fill opacity=\gelb] (-2.2,0.) -- (2.2,0.) -- (2.2,4.) -- (-2.2,4.) -- cycle;
\fill[line width=2.pt,color=white,fill=ffffff,fill opacity=1.0] (0.,1.) -- (0.,3.) -- (-1.7320508075688776,2.) -- cycle;
\fill[line width=2.pt,color=ffffqq,fill=ffffqq,fill opacity=0.] (0.,3.) -- (0.,1.) -- (1.7320508075688776,2.) -- cycle;
\fill[line width=2.pt,color=ffffqq,fill=ffffqq,fill opacity=\gelb] (3.,1.) -- (5.594112554112552,0.9696969696969713) -- (4.323299471110349,3.231415856986091) -- cycle;
\draw [line width=2.pt] (0.,1.)-- (0.,3.);
\draw [line width=2.pt] (0.,3.)-- (-1.7320508075688776,2.);
\draw [line width=2.pt] (-1.7320508075688776,2.)-- (0.,1.);
\draw [line width=2.pt] (0.,3.)-- (0.,1.);
\draw [line width=2.pt] (0.,1.)-- (1.7320508075688776,2.);
\draw [line width=2.pt] (1.7320508075688776,2.)-- (0.,3.);
\draw [line width=2.pt] (3.,1.)-- (5.594112554112552,0.9696969696969713);
\draw [line width=2.pt] (5.594112554112552,0.9696969696969713)-- (4.323299471110349,3.231415856986091);
\draw [line width=2.pt] (4.323299471110349,3.231415856986091)-- (3.,1.);
\begin{scriptsize}
\draw[color=black] (0.6993073593073588,2.051515151515153) node {$F'$};
\draw [fill=qqqqff] (0.,1.) circle (2.5pt);
%\draw[color=black] (0.12225108225108178,0.7203463203463215) node {$\{V_4\}$};
\draw [fill=qqqqff] (0.,3.) circle (2.5pt);
%\draw[color=black] (0.10225108225108178,3.2116883116883145) node {$\{V_1^2,V_3^2\}$};
%\draw[color=ffffff] (-0.17212121212121256,2.151515151515153) node {$Vieleck2$};
\draw [fill=qqqqff] (-1.7320508075688776,2.) circle (2.5pt);
%\draw[color=black] (-1.8593506493506495,2.2246753246753267) node {$\{V_1^3\}$};
%\draw[color=ffffqq] (0.9880519480519474,2.151515151515153) node {$Vieleck3$};
\draw [fill=qqqqff] (1.7320508075688776,2.) circle (2.5pt);
%\draw[color=black] (1.893852813852813,2.2246753246753267) node {$\{V_2^2\}$};
\draw [fill=qqqqff] (3.,1.) circle (2.5pt);
%\draw[color=black] (2.717922077922077,1.1246753246753267) node {$\{V_1^1\}$};
\draw [fill=qqqqff] (5.594112554112552,0.9696969696969713) circle (2.5pt);
%\draw[color=black] (5.9153246753246725,1.1246753246753267) node {$\{V_2^1\}$};
\draw[color=black] (4.31099567099567,1.8961038961038983) node {$F$};
\draw [fill=qqqqff] (4.323299471110349,3.231415856986091) circle (2.5pt);
%\draw[color=black] (4.45125541125541,3.5584415584415625) node {$\{V_3^1\}$};
%\draw[color=black] (4.31125541125541,0.745584415584415625) node {$\{e_3^1\}$};
%\draw[color=black] (3.31099567099567,2.1961038961038983) node {$\{e_2^1\}$};
%\draw[color=black] (5.31099567099567,2.1961038961038983) node {$\{e_1^1\}$};
%\draw[color=black] (-0.26099567099567,2.01961038961038983) node {$\{f^2\}$};
%\draw[color=black] (-1.01099567099567,1.2961038961038983) node {$\{f^1\}$};
%\draw[color=black] (-1.01099567099567,2.6961038961038983) node {$\{e_2^2\}$};
%\draw[color=black] (1.01099567099567,1.2961038961038983) node {$\{g\}$};
%\draw[color=black] (1.11099567099567,2.6961038961038983) node {$\{e_1^2,e_3^2\}$};
\end{scriptsize}

%\end{axis}
\end{tikzpicture}
%\caption{simpliziale Fläche nach Anwendung eines Ripmernders}
\end{figure}

\end{frame}

\subsection{Prozedur $P^3$}
\begin{frame}
\center{\huge{\bf{Prozedur $P^3$}}}
\end{frame}
\begin{frame}
Anwenden des $Splitmenders$:\\
\begin{figure}[H]
%\begin{comment}
\definecolor{ududff}{rgb}{0.30196078431372547,0.30196078431372547,1.}
\definecolor{ffffff}{rgb}{1.,1.,1.}
\definecolor{qqqqff}{rgb}{0.,0.,1.}
\definecolor{ffffqq}{rgb}{1.,1.,0.}
\begin{tikzpicture}[line cap=round,line join=round,>=triangle 45,x=1.5cm,y=1.5cm]
%\begin{axis}[
x=1.0cm,y=1.0cm,
axis lines=middle,
ymajorgrids=true,
xmajorgrids=true,
xmin=-4.3,
xmax=7.0600000000000005,
ymin=-2.46,
ymax=6.3,
xtick={-4.0,-3.0,...,7.0},
ytick={-2.0,-1.0,...,6.0},]
\clip(-3.8,-0.) rectangle (7.06,4.);
\fill[line width=2.pt,color=ffffqq,fill=ffffqq,fill opacity=\gelb] (-2.,0.) -- (2.1,0.) -- (2.1,4.) -- (-2.,4.) -- cycle;
\fill[line width=2.pt,color=ffffqq,fill=ffffqq,fill opacity=0.1499999940395355] (0.,3.) -- (0.,1.) -- (1.7320508075688776,2.) -- cycle;
\fill[line width=2.pt,color=ffffff,fill=ffffff,fill opacity=1.0] (0.,1.) -- (0.,3.) -- (-1.7320508075688776,2.) -- cycle;
\fill[line width=2.pt,color=ffffqq,fill=ffffqq,fill opacity=\gelb] (0.,3.) -- (-1.,2.) -- (0.,1.) -- cycle;
\draw [line width=2.pt] (0.,3.)-- (0.,1.);
\draw [line width=2.pt] (0.,1.)-- (1.7320508075688776,2.);
\draw [line width=2.pt] (1.7320508075688776,2.)-- (0.,3.);
\draw [line width=2.pt] (0.,1.)-- (0.,3.);
\draw [line width=2.pt] (0.,3.)-- (-1.7320508075688776,2.);
\draw [line width=2.pt] (-1.7320508075688776,2.)-- (0.,1.);
\draw [line width=2.pt] (0.,3.)-- (-1.,2.);
\draw [line width=2.pt] (-1.,2.)-- (0.,1.);
\begin{scriptsize}
%\draw[color=black] (-1.04,1.37) node {$\{f^1\}$};
\draw [fill=qqqqff] (0.,3.) circle (2.5pt);
%\draw[color=black] (0.14,3.17) node {$\{V_2^1,\{V_1^2,V_3^2\}\}$};
%\draw[color=black] (0.1,0.77) node {$\{V_1^1,V_4\}$};
\draw [fill=qqqqff] (0.,1.) circle (2.5pt);
\draw[color=black] (0.64,1.97) node {$F'$};
%\draw[color=black] (0.34,2.27) node {$\{e_3^1,f^2\}$};
%\draw[color=black] (1.04,2.71) node {$\{e_1^2,e_3^2\}$};
%\draw[color=black] (0.94,1.31) node {$\{g\}$};
\draw [fill=qqqqff] (1.7320508075688776,2.) circle (2.5pt);
%\draw[color=black] (1.88,2.17) node {$\{V_2^2\}$};
\draw[color=black] (-0.3,2.01) node {$F$};
\draw [fill=qqqqff] (-1.7320508075688776,2.) circle (2.5pt);
%\draw[color=black] (-1.8,2.22) node {$\{V_1^3\}$};
\draw [fill=ududff] (-1.,2.) circle (2.5pt);
%\draw[color=black] (-1.26,2.0) node {$\{V_3^1\}$};
%\draw[color=black] (-1.03,2.66) node {$\{e_2^2\}$};
%\draw[color=black] (-0.33,2.37) node {$\{e_1^1\}$};
%\draw[color=black] (-0.33,1.62) node {$\{e_2^1\}$};
\end{scriptsize}
%\end{axis}
\end{tikzpicture}
%\end{comment}
%\caption{simpliziale Fläche nach Anwendung eines Splitmenders}
\end{figure}
\end{frame}

\begin{frame}
Anwenden des $Ripmenders$  \\
\begin{figure}[H]
\definecolor{ududff}{rgb}{0.30196078431372547,0.30196078431372547,1.}
\definecolor{ffffff}{rgb}{1.,1.,1.}
\definecolor{qqqqff}{rgb}{0.,0.,1.}
\definecolor{ffffqq}{rgb}{1.,1.,0.}
\begin{tikzpicture}[line cap=round,line join=round,>=triangle 45,x=1.4cm,y=1.4cm]
%\begin{axis}[
x=1.5cm,y=1.5cm,
axis lines=middle,
ymajorgrids=true,
xmajorgrids=true,
xmin=-4.3,
xmax=7.0600000000000005,
ymin=-2.46,
ymax=6.3,
xtick={-4.0,-3.0,...,7.0},
ytick={-2.0,-1.0,...,6.0},]
\clip(-4.3,-0.46) rectangle (7.06,4.3);
\fill[line width=2.pt,color=ffffqq,fill=ffffqq,fill opacity=\gelb] (-2.5,0.) -- (2.2,0.) -- (2.2,4.) -- (-2.5,4.) -- cycle;
\fill[line width=2.pt,color=ffffqq,fill=ffffqq,fill opacity=0.] (0.,3.) -- (0.,1.) -- (1.7320508075688776,2.) -- cycle;
\fill[line width=2.pt,color=ffffff,fill=ffffff,fill opacity=1.0] (0.,1.) -- (0.,3.) -- (-1.7320508075688776,2.) -- cycle;
\fill[line width=2.pt,color=ffffqq,fill=ffffqq,fill opacity=0.5] (0.,3.) -- (-1.74,2.) -- (0.,1.) -- cycle;
\draw [line width=2.pt] (0.,3.)-- (0.,1.);
\draw [line width=2.pt] (0.,1.)-- (1.7320508075688776,2.);
\draw [line width=2.pt] (1.7320508075688776,2.)-- (0.,3.);
\draw [line width=2.pt] (0.,1.)-- (0.,3.);
\draw [line width=2.pt] (-1.7320508075688776,2.)-- (0.,1.);
\draw [line width=2.pt] (-1.74,2.)-- (0.,1.);
\draw [rotate around={29.886526940424037:(-0.87,2.5)},line width=2.pt,color=ffffff,fill=ffffff,fill opacity=1.0] (-0.87,2.5) ellipse (1.41989009757162764cm and 0.22745816720870826cm);
\draw [rotate around={29.886526940424037:(-0.87,2.5)},line width=2.pt] (-0.87,2.5) ellipse (1.41989009757163032cm and 0.2274581672087103cm);
\begin{scriptsize}
\draw[color=black] (-0.5,1.9) node {$F$};
\draw [fill=qqqqff] (0.,3.) circle (2.5pt);
%\draw[color=black] (0.14,3.23) node {$\{V_2^1,\{V_1^2,V_3^2\}\}$};
\draw [fill=qqqqff] (0.,1.) circle (2.5pt);
%\draw[color=black] (0.14,0.75) node {$\{V_1^1,V_4\}$};
%\draw[color=black] (1.04,2.75) node {$\{e_1^2,e_3^2\}$};
\draw [fill=qqqqff] (1.7320508075688776,2.) circle (2.5pt);
%\draw[color=black] (1.88,2.22) node {$\{V_2^2\}$};
\draw[color=black] (0.8,2.) node {$F'$};
%\draw[color=black] (0.4,2.32) node {$\{e_3^1,f^2\}$};
\draw [fill=qqqqff] (-1.7320508075688776,2.) circle (2.5pt);
%\draw[color=black] (-2.1,2.22) node {$\{V_1^3,V_3^1\}$};
\draw [fill=qqqqff] (-1.74,2.) circle (2.5pt);
%\draw[color=black] (-1.0,2.87) node {$\{e_2^2\}$};
%\draw[color=black] (-1.0,1.26) node {$\{e_2^1,f^1\}$};
%\draw[color=black] (-1.16,2.25) node {$c$};
%\draw[color=black] (-0.65,2.2) node {$\{e^1_1\}$};
%\draw[color=black] (0.94,1.31) node {$\{g\}$};
\end{scriptsize}
%\end{axis}
\end{tikzpicture}
\caption{simpliziale Fläche nach Anwendung eines Ripmenders}
\end{figure}

\end{frame}
\begin{frame}
 Anwenden des $Cratermenders$ :\\
 \begin{figure}[H]
\definecolor{ududff}{rgb}{0.30196078431372547,0.30196078431372547,1.}
\definecolor{ffffff}{rgb}{1.,1.,1.}
\definecolor{qqqqff}{rgb}{0.,0.,1.}
\definecolor{ffffqq}{rgb}{1.,1.,0.}
\begin{tikzpicture}[line cap=round,line join=round,>=triangle 45,x=1.4cm,y=1.4cm]
%\begin{axis}[
x=1.3cm,y=1.3cm,
axis lines=middle,
ymajorgrids=true,
xmajorgrids=true,
xmin=-4.3,
xmax=7.0600000000000005,
ymin=-2.46,
ymax=6.3,
xtick={-4.0,-3.0,...,7.0},
ytick={-2.0,-1.0,...,5.0},]
\clip(-4.3,-0.) rectangle (7.06,4.);
\fill[line width=2.pt,color=ffffqq,fill=ffffqq,fill opacity=\gelb] (-2.5,0.) -- (2.2,0.) -- (2.2,4.) -- (-2.5,4.) -- cycle;
\fill[line width=2.pt,color=ffffqq,fill=ffffqq,fill opacity=0.] (0.,3.) -- (0.,1.) -- (1.7320508075688776,2.) -- cycle;
\fill[line width=2.pt,color=ffffff,fill=ffffff,fill opacity=0.] (0.,1.) -- (0.,3.) -- (-1.7320508075688776,2.) -- cycle;
\fill[line width=2.pt,color=ffffqq,fill=ffffqq,fill opacity=0.] (0.,3.) -- (-1.74,2.) -- (0.,1.) -- cycle;
\draw [line width=2.pt] (0.,3.)-- (0.,1.);
\draw [line width=2.pt] (0.,1.)-- (1.7320508075688776,2.);
\draw [line width=2.pt] (1.7320508075688776,2.)-- (0.,3.);
\draw [line width=2.pt] (0.,1.)-- (0.,3.);
\draw [line width=2.pt] (-1.7320508075688776,2.)-- (0.,1.);
\draw [line width=2.pt] (0.,3.)-- (-1.74,2.);
\draw [line width=2.pt] (-1.74,2.)-- (0.,1.);
\begin{scriptsize}
%\draw[color=ffffqq] (-0.84,1.27) node {$Vieleck1$};
%\draw [fill=qqqqff] (0.,3.) circle (2.5pt);
%\draw[color=qqqqff] (0.14,3.37) node {$E$};
%\draw [fill=qqqqff] (0.,1.) circle (2.5pt);
%\draw[color=qqqqff] (0.14,1.37) node {$F$};
%\draw[color=ffffqq] (1.04,3.11) node {$Vieleck2$};
%\draw [fill=qqqqff] (1.7320508075688776,2.) circle (2.5pt);
%\draw[color=qqqqff] (1.88,2.37) node {$G$};
%\draw[color=ffffff] (-0.1,2.17) node {$Vieleck3$};
%\draw [fill=qqqqff] (-1.7320508075688776,2.) circle (2.5pt);
%\draw[color=qqqqff] (-1.6,2.37) node {$H$};
%\draw [fill=ududff] (-1.74,2.) circle (2.5pt);
%\draw[color=ududff] (-1.6,2.37) node {$I$};
%\draw[color=black] (-0.93,3.) node {$f_1$};
%\draw[color=black] (-0.98,1.41) node {$e$};
%\end{comment}
%-----------------------------
%\begin{figure}
\draw[color=black] (-0.6,2.) node {$F$};
\draw [fill=qqqqff] (0.,3.) circle (2.5pt);
%\draw[color=black] (0.14,3.23) node {$\{V_2^1,\{V_1^2,V_3^2\}\}$};
\draw [fill=qqqqff] (0.,1.) circle (2.5pt);
%\draw[color=black] (0.14,0.75) node {$\{V_1^1,V_4\}$};
%\draw[color=black] (1.04,2.75) node {$\{e_1^2,e_3^2\}$};
\draw [fill=qqqqff] (1.7320508075688776,2.) circle (2.5pt);
%\draw[color=black] (1.88,2.22) node {$\{V_2^2\}$};
\draw[color=black] (0.8,2.) node {$F'$};
%\draw[color=black] (0.4,2.32) node {$\{e_3^1,f^2\}$};
\draw [fill=qqqqff] (-1.7320508075688776,2.) circle (2.5pt);
%\draw[color=black] (-2.1,2.22) node {$\{V_1^3,V_3^1\}$};
\draw [fill=qqqqff] (-1.74,2.) circle (2.5pt);
%\draw[color=black] (-1.0,2.77) node {$\{e_1^1,e_2^2\}$};
%\draw[color=black] (-1.0,1.26) node {$\{e_2^1,f^1\}$};
%\draw[color=black] (-1.16,2.25) node {$c$};
%\draw[color=black] (-0.65,2.2) node {$\{e^1_1\}$};
%\draw[color=black] (0.94,1.31) node {$\{g\}$};
\end{scriptsize}
%\end{axis}
\end{tikzpicture}
%\tiny{\caption{simpliziale Fläche nach einem Cratermender}}
\end{figure}
 \end{frame}
 
 


 \begin{frame}
 \begin{itemize}
 \item entstandene simpliziale Fläche bezeichnet man mit $X^H_{(F,f)}$ \pause
 \item $\sigma =(F,f_1,\ldots,f_n)$ nennt man ein eine Lochwanderung \pause
 \item entstandene simpliziale Fläche bezeichnet man mit  $X^H_{\sigma}$ \pause
 \item für Lochwanderungen $\sigma_1,\ldots, \sigma_n$ ist $\Sigma =(\sigma_1, \ldots, \sigma_n)$ eine Lochwanderungssequerenz \pause
 \item entstandene simpliziale Fläche bezeichnet man mit $X^H_{\Sigma}$
 \end{itemize}
 
\end{frame}

\section{Transitivität der Operation Wanderinghole}
%\subsection{Beweisidee}
\begin{frame}{Vorüberlegung}
 Seien $(X,<)$ und $(Y,\prec)$ geschlossene simpliziale Flächen mit $\chi(X)=\chi(Y)$ und $\vert X_2 \vert =\vert Y_2 \vert $.\\\pause
 \textbf{Ziel}: Durch Anwenden der Operation Wanderinghole aus $X$ eine simpliziale Fläche konstruieren, die dieselben Nachbarschaften wie $Y$ aufweist.
\end{frame}
\begin{frame}
simpliziale Fläche $X$:\\
\includegraphics[scale=0.8, viewport=-1.5cm 20cm 8cm 28cm]{surfacey}
\end{frame}
\begin{frame}
simpliziale Fläche $Y$:\\
\includegraphics[scale=0.8, viewport=-1.5cm 20cm 8cm 27cm]{surfacex}
\end{frame}
\begin{frame}{{Vorüberlegung}}
\begin{columns}
    \column{0.5\textwidth}
   \center{simpliziale Fläche $Y$}\\
    \includegraphics[scale=0.6, viewport=.5cm 18cm 9cm 27.5cm]{surfacex}\\
 
    \column{0.5\textwidth}
    \center{simpliziale Fläche $X$}\\
    \includegraphics[scale=0.6, viewport=1.5cm 18cm 9cm 27.5cm]{surfacey}
    
\end{columns}
\end{frame}
\begin{frame}{{Vorüberlegung}}
\begin{columns}
    \column{0.5\textwidth}
   \center{simpliziale Fläche $Y$}\\
    \includegraphics[scale=0.6, viewport=.5cm 18cm 9cm 27.5cm]{surfacex}
 
    \column{0.5\textwidth}
    \center{simpliziale Fläche nach Anwenden von WH auf $X$}\\
    \includegraphics[scale=0.6, viewport=1.5cm 18cm 9cm 27.5cm]{s1}
\end{columns}
\end{frame}
\begin{frame}{{Vorüberlegung}}
\begin{columns}
    \column{0.5\textwidth}
   \center{simpliziale Fläche $Y$}\\
    \includegraphics[scale=0.6, viewport=0.5cm 18cm 9cm 27.5cm]{surfacex}
 
    \column{0.5\textwidth}
    \center{simpliziale Fläche nach Anwenden von WH auf $X$}\\
    \includegraphics[scale=0.6, viewport=1.5cm 18cm 9cm 27.5cm]{s2}
\end{columns}
\end{frame}
\begin{frame}{{Vorüberlegung}}
\begin{columns}
    \column{0.5\textwidth}
   \center{simpliziale Fläche $Y$}\\
    \includegraphics[scale=0.6, viewport=.5cm 18cm 9cm 27.5cm]{surfacex}
 
    \column{0.5\textwidth}
    \center{simpliziale Fläche nach Anwenden von WH auf $X$}\\
    \includegraphics[scale=0.6, viewport=1.5cm 18cm 9cm 27.5cm]{s3}
\end{columns}
\end{frame}
%------------------------------------------

\begin{frame}{Vorüberlegung}
\begin{itemize}
\item Ist es immer möglich in $X$ die Nachbarschaften aus der simplizialen Fläche $Y$ nachzuahmen?\pause
\item Können durch das Anwenden von Lochwanderungssequenzen auf $X$ schon erfolgreich konstruierte Nachbarschaften zerstört werden?\pause
\item Ist die konstruierte simpliziale Fläche, deren Nachbarschaften mit denen der simplizialen Fläche $Y$ übereinstimmen, isomorph zu $Y$? 
\end{itemize}
\end{frame}



\begin{frame}
\begin{block}{Definition 4.10}
\begin{itemize}
\item Sei $(X,<)$ eine simpliziale Fläche. Ein \emph{Flächenpfad} von $S\in X_{2}$ nach $T \in X_{2}$ in $X$ ist eine Sequenz $(F_1:=S,F_{2},\ldots,F_{k}:=T)$ für ein $k \in \mathbb{N}$ so, dass $F_{i} $ und $F_{i+1}$ für $i=1,\ldots,k-1$ benachbarte Flächen in $X$ sind.
\end{itemize}
\end{block}
\end{frame}
\begin{frame}
\begin{block}{Definition 4.10}
\begin{itemize}
\item Man nennt eine Menge $M\subseteq X_2$  \emph{stark-zusammenhängend}, falls für beliebige $S,T \in M$ ein Flächenpfad $(F_{1}:=S,F_{2},\ldots,F_{k}:=T)$ mit $F_i \in M$ für $1\leq i \leq k$ existiert. 
\end{itemize}
\end{block}
\end{frame}

\begin{frame}
\begin{block}{Defintion 4.10}
\begin{itemize}
\item Sei $(X,<)$ eine simpliziale Fläche. Man nennt die Menge $M \subseteq X_2$ \emph{Jordan-zusammenhängend} in $X$, falls $M=X_2$ stark-zusammenhängend ist oder  $M \subsetneq X_2$ gilt und die Mengen $M$ und $X_2\setminus M$ stark-zusammenhängend sind. Falls $X_2$ stark zusammenhängend ist, so nennt man die simpliziale Fläche $(X,<)$ Jordan-zusammenhängend.
\end{itemize}
\end{block}
\end{frame}
\begin{frame}{Tetraeder}
\begin{figure}[H]
 \definecolor{ffffqq}{rgb}{1.,1.,0.}
\definecolor{qqqqff}{rgb}{0.,0.,1.}
\definecolor{qqffqq}{rgb}{0.,1.,0.}
\definecolor{yqyqyq}{rgb}{0.5019607843137255,0.5019607843137255,0.5019607843137255}
\begin{tikzpicture}[line cap=round,line join=round,>=triangle 45,x=.8cm,y=.8cm]
%\begin{axis}[
x=1.0cm,y=1.0cm,
axis lines=middle,
ymajorgrids=true,
xmajorgrids=true,
xmin=-8.620000000000001,
xmax=14.38,
ymin=-3.72,
ymax=4.32,
xtick={-8.0,-7.0,...,14.0},
ytick={-5.0,-4.0,...,5.0},]
\clip(-7.12,-3.82) rectangle (18.38,4.032);
\fill[line width=2.pt,color=ffffqq,fill=ffffqq,fill opacity=\gelb] (-2.,0.) -- (2.,0.) -- (0.,3.4641016151377553) -- cycle;
\fill[line width=2.pt,color=ffffqq,fill=ffffqq,fill opacity=\gelb] (2.,0.) -- (-2.,0.) -- (0.,-3.4641016151377553) -- cycle;
\fill[line width=2.pt,color=ffffqq,fill=ffffqq,fill opacity=\gelb] (0.,3.4641016151377553) -- (2.,0.) -- (4.,3.464101615137754) -- cycle;
\fill[line width=2.pt,color=ffffqq,fill=ffffqq,fill opacity=\gelb] (-2.,0.) -- (0.,3.4641016151377553) -- (-4.,3.464101615137757) -- cycle;
\draw [line width=2.pt] (-2.,0.)-- (2.,0.);
\draw [line width=2.pt] (2.,0.)-- (0.,3.4641016151377553);
\draw [line width=2.pt] (0.,3.4641016151377553)-- (-2.,0.);
\draw [line width=2.pt] (2.,0.)-- (-2.,0.);
\draw [line width=2.pt] (-2.,0.)-- (0.,-3.4641016151377553);
\draw [line width=2.pt] (0.,-3.4641016151377553)-- (2.,0.);
\draw [line width=2.pt] (0.,3.4641016151377553)-- (2.,0.);
\draw [line width=2.pt] (2.,0.)-- (4.,3.464101615137754);
\draw [line width=2.pt] (4.,3.464101615137754)-- (0.,3.4641016151377553);
\draw [line width=2.pt] (-2.,0.)-- (0.,3.4641016151377553);%%%%
\draw [line width=2.pt] (0.,3.4641016151377553)-- (-4.,3.464101615137757);
\draw [line width=2.pt] (-4.,3.464101615137757)-- (-2.,0.);
\begin{scriptsize}
\draw [fill=blue] (-2.,0.) circle (2.5pt);
\draw[color=black] (-2.32,-0.09) node {$V_1$};
\draw [fill=blue] (2.,0.) circle (2.5pt);
\draw[color=black] (2.34,0.) node {$V_2$};
\draw[color=black] (0.,1.33) node {$F_1$};
\draw[color=black] (0.06,-0.25) node {$e_3$};
\draw[color=black] (1.32,2.07) node {$e_1$};
\draw[color=black] (-1.22,2.07) node {$e_2$};
\draw [fill=blue] (0.,3.4641016151377553) circle (2.5pt);
\draw[color=black] (0.14,3.83) node {$V_3$};
\draw[color=black] (0.,-1.33) node {$F_3$};
\draw[color=black] (-1.27,-1.71) node {$e_5$};
\draw[color=black] (1.37,-1.71) node {$e_6$};
\draw [fill=blue] (0.,-3.4641016151377553) circle (2.5pt);
\draw[color=black] (0.49,-3.29) node {$V_4$};
\draw [fill=blue] (-2.,0.) circle (2.5pt);
%\draw[color=qqqqff] (-1.86,0.37) node {$E$};
\draw[color=black] (2.,2.19) node {$F_4$};
\draw[color=black] (3.37,1.75) node {$e_6$};
\draw[color=black] (2.11,3.87) node {$e_4$};
\draw [fill=blue] (4.,3.464101615137754) circle (2.5pt);
\draw[color=black] (4.14,3.88) node {$V_4$};
\draw[color=black] (-2.,2.19) node {$F_2$};
\draw[color=black] (-1.94,3.87) node {$e_4$};
\draw[color=black] (-3.3,1.75) node {$e_5$};
\draw [fill=blue] (-4.,3.464101615137757) circle (2.5pt);
\draw[color=black] (-3.86,3.88) node {$V_4$};
\end{scriptsize}
%\end{axis}
\end{tikzpicture}

 %\caption{Tetraeder }
 %\label{Tetraeder}
 \end{figure}
 \end{frame}
\begin{frame}
\begin{block}{Lemma 4.22}
Seien $(X,<)$  eine geschlossene Jordan-zusammenhängende simpliziale Fläche und $M \subsetneq X_2$ eine Jordan-zusammenhängende Menge. Dann existiert ein $F\in X_2\setminus M$ so, dass die Menge $M \cup \{F\}$ Jordan-zusammenhängend ist.
\end{block}
\end{frame}
\begin{frame}
\begin{figure}[H]
\definecolor{ffffff}{rgb}{1.,1.,1.}
\definecolor{qqffqq}{rgb}{0.,1.,0.}
\definecolor{ffffqq}{rgb}{1.,1.,0.}
\definecolor{yqyqyq}{rgb}{0.5019607843137255,0.5019607843137255,0.5019607843137255}
\begin{tikzpicture}[line cap=round,line join=round,>=triangle 45,x=1.2cm,y=1.2cm]
%\begin{axis}[
x=1.2cm,y=1.2cm,
axis lines=middle,
ymajorgrids=true,
xmajorgrids=true,
xmin=-10.24,
xmax=12.76,
ymin=-4.62,
ymax=7.0200000000000005,
xtick={-8.0,-7.0,...,12.0},
ytick={-4.0,-3.0,...,7.0},]
\clip(-4.504,-1.02) rectangle (12.76,4.02);
\fill[line width=2.pt,color=ffffqq,fill=ffffqq,fill opacity=0.5] (-3.,-1.) -- (3.,-1.) -- (3.,0.) -- (-3.,0.) -- cycle;
\fill[line width=2.pt,color=yqyqyq,fill=yqyqyq,fill opacity=0.5] (-3.,0.) -- (-3.,4.) -- (3.,4.) -- (3.,0.) -- cycle;
\fill[line width=2.pt,color=ffffff,fill=ffffff,fill opacity=1.0] (-2.,0.) -- (2.,0.) -- (0.,3.4641016151377553) -- cycle;
\fill[line width=2.pt,color=ffffqq,fill=ffffqq,fill opacity=\gelb] (-2.,0.) -- (2.,0.) -- (0.,3.4641016151377553) -- cycle;
\draw [line width=2.pt,color=ffffff] (2.,0.)-- (0.,3.4641016151377553);
\draw [line width=2.pt,color=ffffff] (0.,3.4641016151377553)-- (-2.,0.);
\draw [line width=2.pt] (-2.,0.)-- (2.,0.);
\draw [line width=2.pt,color=yqyqyq] (2.,0.)-- (0.,3.4641016151377553);
\draw [line width=2.pt,color=yqyqyq] (0.,3.4641016151377553)-- (-2.,0.);
\begin{scriptsize}
%\draw[color=ffffff] (0.48,1.33) node {$Vieleck1$};
\draw [fill=yqyqyq] (0.,3.4641016151377553) circle (2.5pt);
%\draw[color=black] (0.14,3.83) node {$I$};
\draw [fill=yqyqyq] (-2.,0.) circle (2.5pt);
\draw[color=black] (-2.16,0.37) node {$V_1$};
\draw [fill=yqyqyq] (2.,0.) circle (2.0pt);
\draw[color=black] (2.14,0.33) node {$V_2$};
\draw[color=black] (0.,1.33) node {$F$};
\draw[color=black] (0.1,-0.2) node {$e_3$};
\draw[color=black] (1.32,2.07) node {$e_1$};
\draw[color=black] (-1.22,2.07) node {$e_2$};
\draw [fill=yqyqyq] (0.,3.4641016151377553) circle (2.5pt);
\draw[color=black] (0.14,3.83) node {$V_3$};
\end{scriptsize}
%\end{axis}
\end{tikzpicture}
%\caption{Ausschnitt gefärbte simpliziale Fläche}
\end{figure}
\end{frame}
\begin{frame}
\begin{block}{Lemma 4.24}
Seien $(X,<)$ eine geschlossene Jordan-zusammenhängende simpliziale Fläche und $(S,F_1,F_2,\ldots,F_n,F_{n+1}:=T)$ für $n\geq 1$ ein $S$-$T$-Weg in $X$ ohne Flächenwiederholung, wobei $F_i \in X_2$ für $1 \leq i \leq n$ ist und $S,T\in X_2$ zwei nicht benachbarte Flächen sind. 
Dann existiert eine Lochwanderungssequenz $\Sigma$ so, dass $(S,F_2, \ldots,F_n,T)$ ein $S$-$T$-Weg ohne Flächenwiederholung in $X^H_{\Sigma}$ ist.
\end{block}
\end{frame}
\begin{frame}
\begin{figure}[H]
%\begin{center}
\definecolor{ffffff}{rgb}{1.,1.,1.}
\definecolor{qqqqff}{rgb}{0.,0.,1.}
\definecolor{ududff}{rgb}{0.30196078431372547,0.30196078431372547,1.}
\definecolor{ffffqq}{rgb}{1.,1.,0.}
\begin{tikzpicture}[line cap=round,line join=round,>=triangle 45,x=0.9cm,y=0.9cm]
%\begin{axis}[
x=1.0cm,y=1.0cm,
axis lines=middle,
ymajorgrids=true,
xmajorgrids=true,
xmin=-2.296265478762931,
xmax=13.597363682451801,
ymin=-0.7463544498652792,
ymax=7.297203960419052,
xtick={-2.0,-1.0,...,13.0},
ytick={-0.0,1.0,...,7.0},]
\clip(-.096265478762931,-0.07463544498652792) rectangle (11.597363682451801,4.297203960419052);
\fill[line width=2.pt,color=ffffqq,fill=ffffqq,fill opacity=\gelb] (0.,0.) -- (4.,0.) -- (4.,4.) -- (0.,4.) -- cycle;
\fill[line width=2.pt,color=ffffqq,fill=ffffqq,fill opacity=\gelb] (7.,0.) -- (11.,0.) -- (11.,4.) -- (7.,4.) -- cycle;
%\fill[line width=2.pt,color=ffffqq,fill=ffffqq,fill opacity=\gelb] (0.62,1.54) -- (1.94,1.54) -- (1.28,2.6831535329954592) -- cycle;
%\fill[line width=2.pt,color=ffffqq,fill=ffffqq,fill opacity=\gelb] (1.94,1.54) -- (3.26,1.54) -- (2.6,2.6831535329954592) -- cycle;
%\fill[line width=2.pt,color=ffffqq,fill=ffffqq,fill opacity=\gelb] (1.28,2.6831535329954592) -- (1.94,1.54) -- (2.6,2.683153532995459) -- cycle;
%\fill[line width=2.pt,color=ffffqq,fill=ffffqq,fill opacity=\gelb] (7.66,1.44) -- (8.98,1.42) -- (8.33732050807569,2.57315353299546) -- cycle;
\fill[line width=2.pt,color=ffffff,fill=ffffff,fill opacity=1.0] (8.98,1.42) -- (10.3,1.44) -- (9.62267949192431,2.573153532995459) -- cycle;
\fill[line width=2.pt,color=ffffqq] (9.62267949192431,2.573153532995459) -- (8.33732050807569,2.57315353299546) -- (8.98,1.46) -- cycle;
\fill[line width=2.pt,color=ffffff,fill=ffffff,fill opacity=1.0] (8.33732050807569,2.57315353299546) -- (9.62267949192431,2.573153532995459) -- (10.3,1.44) -- (8.98,1.42) -- cycle;
\fill[line width=2.pt,color=ffffqq,fill=ffffqq,fill opacity=\gelb] (8.33732050807569,2.57315353299546) -- (10.3,1.44) -- (8.98,1.42) -- cycle;
\fill[line width=2.pt,color=ffffqq,fill=ffffqq,fill opacity=\gelb] (8.33732050807569,2.57315353299546) -- (9.62267949192431,2.573153532995459) -- (10.3,1.44) -- cycle;
\draw [line width=2.pt] (0.62,1.54)-- (1.94,1.54);
\draw [line width=2.pt] (1.94,1.54)-- (1.28,2.6831535329954592);
\draw [line width=2.pt] (1.28,2.6831535329954592)-- (0.62,1.54);
\draw [line width=2.pt] (1.94,1.54)-- (3.26,1.54);
\draw [line width=2.pt] (3.26,1.54)-- (2.6,2.6831535329954592);
\draw [line width=2.pt] (2.6,2.6831535329954592)-- (1.94,1.54);
\draw [line width=2.pt] (1.28,2.6831535329954592)-- (1.94,1.54);
\draw [line width=2.pt] (1.94,1.54)-- (2.6,2.683153532995459);
\draw [line width=2.pt] (2.6,2.683153532995459)-- (1.28,2.6831535329954592);
\draw [line width=2.pt] (5.,2.)-- (6.,2.);
\draw [line width=2.pt] (6.,2.)-- (5.86,2.28);
\draw [line width=2.pt] (5.86,2.28)-- (6.,2.);
\draw [line width=2.pt] (6.,2.)-- (5.86,1.72);
\draw [line width=2.pt] (7.66,1.44)-- (8.98,1.42);
\draw [line width=2.pt] (8.98,1.42)-- (8.33732050807569,2.57315353299546);
\draw [line width=2.pt] (8.33732050807569,2.57315353299546)-- (7.66,1.44);
\draw [line width=2.pt] (8.98,1.42)-- (10.3,1.44);
\draw [line width=2.pt,color=ffffff] (10.3,1.44)-- (9.62267949192431,2.573153532995459);
\draw [line width=2.pt] (9.62267949192431,2.573153532995459)-- (8.33732050807569,2.57315353299546);
\draw [line width=2.pt] (8.33732050807569,2.57315353299546)-- (8.98,1.46);
\draw [line width=2.pt] (8.33732050807569,2.57315353299546)-- (9.62267949192431,2.573153532995459);
\draw [line width=2.pt] (9.62267949192431,2.573153532995459)-- (10.3,1.44);
\draw [line width=2.pt] (10.3,1.44)-- (8.98,1.42);
\draw [line width=2.pt] (8.98,1.42)-- (8.33732050807569,2.57315353299546);
\draw [line width=2.pt] (8.33732050807569,2.57315353299546)-- (10.3,1.44);
\draw [line width=2.pt] (10.3,1.44)-- (8.98,1.42);
\draw [line width=2.pt] (8.98,1.42)-- (8.33732050807569,2.57315353299546);
\draw [line width=2.pt] (8.33732050807569,2.57315353299546)-- (9.62267949192431,2.573153532995459);
\draw [line width=2.pt] (9.62267949192431,2.573153532995459)-- (10.3,1.44);
\draw [line width=2.pt] (10.3,1.44)-- (8.33732050807569,2.57315353299546);
\begin{scriptsize}
\draw[color=black] (1.919797251537696,2.221409072701729) node {$F_1$};
\draw[color=black] (9.326814629568887,1.7121409072701729) node {$S$};
\draw [fill=ududff] (0.62,1.54) circle (2.5pt);
%\draw[color=ududff] (0.716613788319514,1.7897159423807258) node {$I$};
\draw [fill=ududff] (1.94,1.54) circle (2.5pt);
%\draw[color=ududff] (2.0433863096035263,1.7897159423807258) node {$J$};
\draw[color=black] (1.28043849728318963,1.904831165782173) node {$F_2$};
\draw [fill=qqqqff] (1.28,2.6831535329954592) circle (2.5pt);
%\draw[color=qqqqff] (1.3800000489615203,2.936821351407529) node {$K$};
\draw [fill=ududff] (3.26,1.54) circle (2.5pt);
%\draw[color=ududff] (3.3563382837908304,1.7897159423807258) node {$L$};
\draw[color=black] (2.620547586323667,1.904831165782173) node {$S$};
\draw[color=black] (2.55547586323667,2.17604831165782173) node {$e_1$};
\draw[color=black] (1.387586323667,2.19604831165782173) node {$e_2$};
\draw [fill=qqqqff] (2.6,2.6831535329954592) circle (2.5pt);
%\draw[color=qqqqff] (2.692952023148824,2.936821351407529) node {$M$};
%\draw[color=black] (2.8311574941159092,2.9506418985042377) node {$Vieleck5$};
\draw [fill=qqqqff] (2.6,2.683153532995459) circle (2.5pt);
%\draw[color=qqqqff] (2.692952023148824,2.936821351407529) node {$N$};
%\draw[color=black] (5.5399847250707674,1.9002803191543935) node {$d$};
\draw[color=black] (5.3161909344422648,2.3701789204424815) node {$(S,e_2)$};
%\draw[color=black] (5.795664846359874,2.1974220817336256) node {$f_1$};
%\draw[color=black] (5.795664846359874,2.114498799153375) node {$g_1$};
\draw [fill=ududff] (7.66,1.44) circle (2.5pt);
%\draw[color=ududff] (7.7512722605441216,1.6929721127037665) node {$S$};
\draw [fill=ududff] (8.98,1.42) circle (2.5pt);
%\draw[color=ududff] (9.078044781828133,1.679151565607058) node {$T$};
\draw[color=black] (8.349607821830172,1.941741960444519) node {$F_2$};
\draw [fill=qqqqff] (8.33732050807569,2.57315353299546) circle (2.5pt);
%\draw[color=qqqqff] (8.428479068282837,2.826256974633861) node {$U$};
\draw [fill=ududff] (10.3,1.44) circle (2.5pt);
%\draw[color=ududff] (10.390996756015438,1.6929721127037665) node {$V$};
%\draw[color=black] (9.962559796017477,1.941741960444519) node {$Vieleck7$};
\draw [fill=qqqqff] (9.62267949192431,2.573153532995459) circle (2.5pt);
%\draw[color=qqqqff] (9.713789948276723,2.826256974633861) node {$W$};
\draw[color=black] (9.39917353537547,2.328717279152356) node {$F_1$};
\draw [fill=qqqqff] (8.98,1.46) circle (2.5pt);
%\draw[color=qqqqff] (9.078044781828133,1.7206132068971833) node {$Z$};
%\draw[color=black] (9.008942046344592,2.4807432972161494) node {$u$};
%\draw[color=black] (9.810533777953681,2.010844695928061) node {$w$};
%\draw[color=black] (9.672328306986598,1.7758953952840173) node {$v$};
%\draw[color=black] (8.919108490215987,2.266524817217168) node {$t_1$};
%\draw[color=black] (9.278442714730406,1.97629332818629) node {$t_2$};
%\draw[color=black] (9.70687967472837,1.8104467630257883) node {$u_1$};
%\draw[color=black] (8.919108490215987,2.266524817217168) node {$v_1$};
%\draw[color=black] (9.043493414086363,2.5152946649579206) node {$v_2$};
%\draw[color=black] (9.845085145695453,2.0453960636698323) node {$u_2$};
%\draw[color=black] (9.499571468277743,2.3632686468941273) node {$w_1$};
\end{scriptsize}
%\end{axis}
\end{tikzpicture}
%\caption{Anwendung von Wanderinghole}
%\end{center}
\end{figure}
\end{frame}
\begin{frame}
\begin{block}{Lemma 4.27}
 Seien $(X,<)$ eine geschlossene Jordan-zusammenhängende simpliziale Fläche mit $\chi(X)=2$, $M \subsetneq X_2$ eine Jordan-zusammenhängende Teilmenge der Flächen und $F \in X_2\setminus M$ eine Fläche in $X$. Dann existiert eine Lochwanderungssequenz $\Sigma$ so, dass $M \cup \{F\}$ Jordan-zusammenhängend in $X^H_{\Sigma}$ ist.

\end{block}

\end{frame}
%\subsection{Beweis}
\begin{frame}
\begin{block}{Satz 4.29}
Sei $(X,<)$ eine geschlossene Jordan-zusammenhängende simpliziale Fläche. Dann ist die Anwendung von Wanderinghole auf $X$ transitiv, das heißt, für alle geschlossenen Jordan-zusammenhängenden simplizialen Flächen $(Y,\prec)$, die keine Knoten vom Grad 2 enthalten, wobei $\vert X_2 \vert = \vert Y_2 \vert$ und $\chi(X)=\chi(Y)=2$ ist, existiert eine Lochwanderungssequenz $\Sigma$ mit $X^H_{\Sigma} \cong Y$.
\end{block}
\end{frame}
\begin{frame}{Tetraeder}
\begin{figure}[H]
 \definecolor{ffffqq}{rgb}{1.,1.,0.}
\definecolor{qqqqff}{rgb}{0.,0.,1.}
\definecolor{qqffqq}{rgb}{0.,1.,0.}
\definecolor{yqyqyq}{rgb}{0.5019607843137255,0.5019607843137255,0.5019607843137255}
\begin{tikzpicture}[line cap=round,line join=round,>=triangle 45,x=.8cm,y=.8cm]
%\begin{axis}[
x=1.0cm,y=1.0cm,
axis lines=middle,
ymajorgrids=true,
xmajorgrids=true,
xmin=-8.620000000000001,
xmax=14.38,
ymin=-3.72,
ymax=4.32,
xtick={-8.0,-7.0,...,14.0},
ytick={-5.0,-4.0,...,5.0},]
\clip(-7.12,-3.82) rectangle (18.38,4.032);
\fill[line width=2.pt,color=ffffqq,fill=ffffqq,fill opacity=\gelb] (-2.,0.) -- (2.,0.) -- (0.,3.4641016151377553) -- cycle;
\fill[line width=2.pt,color=ffffqq,fill=ffffqq,fill opacity=\gelb] (2.,0.) -- (-2.,0.) -- (0.,-3.4641016151377553) -- cycle;
\fill[line width=2.pt,color=ffffqq,fill=ffffqq,fill opacity=\gelb] (0.,3.4641016151377553) -- (2.,0.) -- (4.,3.464101615137754) -- cycle;
\fill[line width=2.pt,color=ffffqq,fill=ffffqq,fill opacity=\gelb] (-2.,0.) -- (0.,3.4641016151377553) -- (-4.,3.464101615137757) -- cycle;
\draw [line width=2.pt] (-2.,0.)-- (2.,0.);
\draw [line width=2.pt] (2.,0.)-- (0.,3.4641016151377553);
\draw [line width=2.pt] (0.,3.4641016151377553)-- (-2.,0.);
\draw [line width=2.pt] (2.,0.)-- (-2.,0.);
\draw [line width=2.pt] (-2.,0.)-- (0.,-3.4641016151377553);
\draw [line width=2.pt] (0.,-3.4641016151377553)-- (2.,0.);
\draw [line width=2.pt] (0.,3.4641016151377553)-- (2.,0.);
\draw [line width=2.pt] (2.,0.)-- (4.,3.464101615137754);
\draw [line width=2.pt] (4.,3.464101615137754)-- (0.,3.4641016151377553);
\draw [line width=2.pt] (-2.,0.)-- (0.,3.4641016151377553);%%%%
\draw [line width=2.pt] (0.,3.4641016151377553)-- (-4.,3.464101615137757);
\draw [line width=2.pt] (-4.,3.464101615137757)-- (-2.,0.);
\begin{scriptsize}
\draw [fill=blue] (-2.,0.) circle (2.5pt);
\draw[color=black] (-2.32,-0.09) node {$V_1$};
\draw [fill=blue] (2.,0.) circle (2.5pt);
\draw[color=black] (2.34,0.) node {$V_2$};
\draw[color=black] (0.,1.33) node {$F_1$};
\draw[color=black] (0.06,-0.25) node {$e_3$};
\draw[color=black] (1.32,2.07) node {$e_1$};
\draw[color=black] (-1.22,2.07) node {$e_2$};
\draw [fill=blue] (0.,3.4641016151377553) circle (2.5pt);
\draw[color=black] (0.14,3.83) node {$V_3$};
\draw[color=black] (0.,-1.33) node {$F_3$};
\draw[color=black] (-1.27,-1.71) node {$e_5$};
\draw[color=black] (1.37,-1.71) node {$e_6$};
\draw [fill=blue] (0.,-3.4641016151377553) circle (2.5pt);
\draw[color=black] (0.49,-3.29) node {$V_4$};
\draw [fill=blue] (-2.,0.) circle (2.5pt);
%\draw[color=qqqqff] (-1.86,0.37) node {$E$};
\draw[color=black] (2.,2.19) node {$F_4$};
\draw[color=black] (3.37,1.75) node {$e_6$};
\draw[color=black] (2.11,3.87) node {$e_4$};
\draw [fill=blue] (4.,3.464101615137754) circle (2.5pt);
\draw[color=black] (4.14,3.88) node {$V_4$};
\draw[color=black] (-2.,2.19) node {$F_2$};
\draw[color=black] (-1.94,3.87) node {$e_4$};
\draw[color=black] (-3.3,1.75) node {$e_5$};
\draw [fill=blue] (-4.,3.464101615137757) circle (2.5pt);
\draw[color=black] (-3.86,3.88) node {$V_4$};
\end{scriptsize}
%\end{axis}
\end{tikzpicture}

 %\caption{Tetraeder }
 %\label{Tetraeder}
 \end{figure}
 \end{frame}
\begin{frame}
\begin{block}{Satz von Whitney}
Sei $G=(V,E)$ ein 3-fach zusammenhängender planarer Graph. Dann lässt sich G eindeutig in die Sphäre einbetten.
\end{block}
\end{frame}
\section{Fazit}
%--------------------------------------------------
%----------------------------------------------


\end{document}
