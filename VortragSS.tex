\documentclass{beamer} 
\usetheme{Warsaw}             % Falls Ihnen das Layout nicht gefällt, können Sie hier
                              % auch andere Themes wählen. Ein Verzeichnis der möglichen 
                              % Themes finden Sie im Kapitel 15 des beameruserguide.

\usepackage[utf8]{inputenc}
\usepackage[ngerman]{babel}
\usepackage{amsmath}
\usepackage{amsfonts}
\usepackage{amssymb}

\AtBeginSection[]{\frame<beamer>{\frametitle{Ãœbersicht} \tableofcontents[current]}}

\newcommand{\defin}[1]{\textit{\color{blue}#1}}

% ========== Abkürzungen ==========
\newcommand{\N}{\mathbb{N}}
\newcommand{\Z}{\mathbb{Z}}
\newcommand{\Q}{\mathbb{Q}}
\newcommand{\R}{\mathbb{R}}
\newcommand{\C}{\mathbb{C}}

\author{Vorname Nachname}
\title{Vortragstitel}
\date{Datum \\[.5\baselineskip] Vortrag zum Seminar zur Funktionentheorie}

\begin{document}
\frame{\maketitle}

\section{Erster Abschnitt}

\subsection{Erster Unterabschnitt}

\frame{\frametitle{Definitionen: Stetigkeit}
\begin{block}{(1.1) Definition}
Eine Funktion $f: D \to \R$ mit $D \subset \R$ heißt \defin{stetig} in einem Punkt $x_0 \in D$, wenn zu jedem $\varepsilon > 0$ ein $\delta > 0$ existiert, so dass für alle $x \in D$ mit $|x - x_0| < \delta$ stets $|f(x) - f(x_0)| < \varepsilon$ gilt.
\end{block}

\begin{block}{(1.2) Definition}<2->
Eine Funktion $f: D \to \R$ mit $D \subset \R$ heißt \defin{stetig} (auf ihrem Definitionsbereich), wenn $f$ in jedem $x_0 \in D$ stetig ist.
\end{block}


\frame{\frametitle{Stetigkeit unter arithmetischen Verknüpfungen}
\begin{alertblock}{(1.3) Lemma}
Seien $f, g: D \to \R$ stetig, $D \subset \R$. Dann sind auch die folgenden Funktionen stetig:
\begin{itemize}
\item<2-> $f + g$
\item<3-> $f - g$
\item<4-> $f \cdot g$
\end{itemize}
\end{alertblock}

\begin{exampleblock}{(1.4) Beispiel}<5->
Die Funktion $f: \R \to \R$, $x \mapsto 2 \cdot x + 3$ ist stetig, da die Identität und die konstanten Funktionen stetig sind.
\end{exampleblock}
}

\subsection{Zweiter Unterabschnitt}

\frame{\frametitle{Definition: gleichmäßige Stetigkeit}
\begin{block}{(1.6) Definition}
Eine Funktion $f: D \to \R$ mit $D \subset \R$ heißt \defin{gleichmäßig stetig}, wenn zu jedem $\varepsilon > 0$ ein $\delta > 0$ existiert, so dass für alle $x, y \in D$ mit $|x - y| < \delta$ stets $|f(x) - f(y)| < \varepsilon$ gilt.
\end{block}

\vspace*{3ex}
\uncover<2->{Es macht keinen Sinn, von einer gleichmäßigen Stetigkeit in einem Punkt zu sprechen.}
}

\section{Zweiter Abschnitt}

\end{document}
