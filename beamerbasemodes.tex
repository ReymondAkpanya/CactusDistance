% Copyright 2003--2007 by Till Tantau
% Copyright 2010 by Vedran Mileti\'c
% Copyright 2012,2013,2015 by Vedran Mileti\'c, Joseph Wright
% Copyright 2016
% Copyright 2017,2018 by Louis Stuart, Joseph Wright
%
% This file may be distributed and/or modified
%
% 1. under the LaTeX Project Public License and/or
% 2. under the GNU Public License.
%
% See the file doc/licenses/LICENSE for more details.

% Force e-TeX and provide \patchcmd
\RequirePackage{etoolbox}
% Booleans for options available for both beamer and beamerarticle
\newif\ifbeamer@blocks    \beamer@blockstrue
\newif\ifbeamer@ams       \beamer@amstrue
\newif\ifbeamer@amssymb   \beamer@amssymbtrue
\newif\ifbeamer@countsect \beamer@countsectfalse % A no-op but for clarity
\newif\ifbeamer@keywords  \beamer@keywordstrue

% Scratch space
\newbox\beamer@tempbox
\newcount\beamer@tempcount

% The pauses counter is public and global, and required by both
% beamer and beamerarticle
\newcounter{beamerpauses}

\RequirePackage{beamerbasedecode}

%
% Presentation/article stuff
%
% Copyright notice: Part of the following code is taken from the
% package comment.sty by Victor Eijkhout
%

%
% Comment stuff. This will also be needed, if we switch to another
% class, so define it here already.
%
\newif\ifbeamer@inpresentation
\beamer@inpresentationtrue

\def\beamer@startcomment{%
  \begingroup
    \let\do\@makeother\dospecials
    \catcode`\^^L=12 % and whatever other special cases
    \endlinechar`\^^M %
    \catcode`\^^M=12 %
    \beamer@processline}

\begingroup
  \catcode`\^^M=12\relax\endlinechar=-1\relax%
  \long\gdef\beamer@processline#1^^M{%
    \def\beamer@test{#1}%
    \ifx\beamer@test\beamer@stopmodestar%
      \let\next=\beamer@modeoutsideframe%
    \else%
      \ifx\beamer@test\beamer@stopmode%
        \let\next=\mode%
      \else%
        \ifx\beamer@test\beamer@stoparticle%
          \let\next=\article%
        \else%
          \ifx\beamer@test\beamer@stoppresentation%
            \let\next=\presentation%
          \else%
            \ifx\beamer@test\beamer@stopcommon%
              \let\next=\common%
            \else%
              \ifx\beamer@test\beamer@stopdocument%
                \let\next=\beamer@enddocument%
              \else
                \ifx\beamer@test\beamer@begindocument%
                  \let\next=\beamer@startdocument%
                \else
                  \let\next=\beamer@processline%
                \fi%
              \fi%
            \fi%
          \fi%
        \fi%
      \fi%
    \fi%
    \next}
\endgroup

\def\beamer@enddocument{\beamer@closer\end{document}}
\def\beamer@startdocument{%
  \beamer@closer
  \begin{document}%
  \gdef\beamer@closer{}%
  \mode<all>}
\def\beamer@closer{}
\begingroup
  \escapechar=-1\relax
  \xdef\beamer@stopmode{\string\\mode}
  \xdef\beamer@stopmodestar{\string\\mode*}
  \xdef\beamer@stoparticle{\string\\article}
  \xdef\beamer@stoppresentation{\string\\presentation}
  \xdef\beamer@stopcommon{\string\\common}
  \xdef\beamer@stopdocument{\string\\end\string\{document\string\}}
  \xdef\beamer@begindocument{\string\\begin\string\{document\string\}}
\endgroup

\newbox\beamer@commentbox
\def\beamer@startcommentinframe{%
  \begingroup
    \xdef\beamer@closer{\egroup\beamer@closer}%
    \setbox\beamer@commentbox=\vbox\bgroup\leavevmode}

\def\beamer@outsidemode{\afterassignment\beamer@treat\let\beamer@nexttoken=}

% Deals with the various tokens that need to be 'active' even when skipping
% material outside frames
\def\beamer@treat{%
  \ifx\beamer@nexttoken\frame\let\next=\beamer@stopoutsidemode\fi
  \ifx\beamer@nexttoken\lecture\let\next=\beamer@stopoutsidemode\fi
  \ifx\beamer@nexttoken\note\let\next=\beamer@stopoutsidemode\fi
  \ifx\beamer@nexttoken\appendix\let\next=\beamer@stopoutsidemode\fi
  \ifx\beamer@nexttoken\againframe\let\next=\beamer@stopoutsidemode\fi
  \ifx\beamer@nexttoken\section\let\next=\beamer@stopoutsidemode\fi
  \ifx\beamer@nexttoken\subsection\let\next=\beamer@stopoutsidemode\fi
  \ifx\beamer@nexttoken\subsubsection\let\next=\beamer@stopoutsidemode\fi
  \ifx\beamer@nexttoken\part\let\next=\beamer@stopoutsidemode\fi
  \ifx\beamer@nexttoken\article\let\next=\beamer@stopoutsidemode\fi
  \ifx\beamer@nexttoken\mode\let\next=\beamer@stopoutsidemode\fi
  \ifx\beamer@nexttoken\common\let\next=\beamer@stopoutsidemode\fi
  \ifx\beamer@nexttoken\presentation\let\next=\beamer@stopoutsidemode\fi
  \ifx\beamer@nexttoken\begin\let\next=\beamer@checkbeginframe\fi
  \ifx\beamer@nexttoken\end\let\next=\beamer@checkenddoc\fi
  \next}

\def\beamer@stopoutsidemode{\beamer@nexttoken}
\def\beamer@checkenddoc#1{%
  \def\beamer@temp{#1}%
  \ifx\beamer@temp\beamer@enddoc
    \let\next=\beamer@enddocument
  \else
    \let\next=\beamer@outsidemode
  \fi
  \next}
\def\beamer@enddoc{document}

\def\beamer@checkbeginframe#1{%
  \def\beamer@temp{#1}%
  \ifx\beamer@temp\beamer@frametext
    \let\next=\beamer@beginframeenv
  \else
    \let\next=\beamer@outsidemode
  \fi
  \next}
\def\beamer@beginframeenv{\begin{frame}}

\def\beamer@modeoutsideframe{%
  \beamer@closer
  \gdef\beamer@closer{}%
  \gdef\beamer@mode{\beamer@modeoutsideframe}%
  \ifbeamer@inpresentation
    \let\next=\beamer@outsidemode
  \else
    \let\next=\relax
  \fi%
  \next}

% Obsolete, do not use!
\newrobustcmd*\presentation{\mode<presentation>}
\newrobustcmd*\article{\mode<article>}
\newrobustcmd*\common{\mode<all>}

%
% Mode command
%
\newrobustcmd*\mode{\@ifstar\beamer@modeoutsideframe\beamer@@@mode}
\def\beamer@@@mode<#1>{%
  \beamer@closer
  \@ifnextchar\bgroup
    {\beamer@modeinline<#1>}%
    {\beamer@switchmode<#1>}}
\long\def\beamer@modeinline<#1>#2{%
  \gdef\beamer@closer{}%
  \gdef\beamer@doifnotinframe{\@gobble}%
  \def\beamer@doifinframe{\@firstofone}%
  \begingroup
    \beamer@saveanother
    \beamer@slideinframe=1\relax%
    \beamer@masterdecode{#1}%
    \beamer@restoreanother
  \endgroup
  \beamer@donow{#2}%
  \beamer@mode\par}
\def\beamer@switchmode<#1>{%
  \gdef\beamer@mode{\beamer@switchmode<#1>}%
  \gdef\beamer@doifnotinframe{%
    \let\next=\beamer@startcomment
    \gdef\beamer@closer{\endgroup}%
  }%
  \def\beamer@doifinframe{%
    \let\next=\relax
    \gdef\beamer@closer{}%
  }%
  \begingroup
    \beamer@saveanother
    \beamer@slideinframe=1\relax%
    \beamer@masterdecode{#1}%
    \beamer@restoreanother
  \endgroup
  \beamer@donow
  \next}

\def\beamer@saveanother{\let\beamer@savedif=\ifbeamer@anotherslide}
\def\beamer@restoreanother{\global\let\ifbeamer@anotherslide=\beamer@savedif}

\mode
<all>

\newcount\beamer@modecount
\def\beamer@pushmode#1{%
  \expandafter\gdef\csname beamer@savedmode@\the\beamer@modecount\endcsname{#1}%
  \global\advance\beamer@modecount by1\relax}
\def\beamer@popmode{
  \global\advance\beamer@modecount by-1\relax%
  \csname beamer@savedmode@\the\beamer@modecount\endcsname}

\def\beamer@savemode{\expandafter\beamer@pushmode\expandafter{\beamer@mode}}
\def\beamer@resumemode{\beamer@popmode}

%
% Stuff needed in both article and presentation version
%
\newcommand*\jobnamebeamerversion{}

\newrobustcmd*\includeslide[2][]{%
  \ifx\jobnamebeamerversion\@empty
    \ClassError{beamer}{Invoke macro "setjobnamebeamerversion" first}\@ehc
  \else
    \edef\beamer@args
      {[{#1,page=\csname beamer@slide#2\endcsname}]{\jobnamebeamerversion}}%
    \expandafter\pgfimage\beamer@args
  \fi}

\newrobustcmd*\setjobnamebeamerversion[1]{%
  \gdef\jobnamebeamerversion{#1}%
  \begingroup
    \makeatletter
    \@input{\jobnamebeamerversion.snm}%
  \endgroup}
