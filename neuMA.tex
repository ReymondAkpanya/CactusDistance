\documentclass[12pt,titlepage,twoside,cleardoublepage]{article}
\usepackage[ngerman]{babel}
\usepackage[utf8]{inputenc}
\usepackage[a4paper,lmargin={4cm},rmargin={2cm},
tmargin={2.5cm},bmargin = {2.5cm}]{geometry}
\usepackage{amsmath}
\usepackage{amssymb}
\usepackage{pdfpages} 
%\usepackage[pdftex,article]{geometry}
\usepackage{amsthm}
%\usepackage{ngerman,amsthm}
\usepackage{lineno} 
\usepackage{lineno, blindtext} 
\usepackage{cleveref}
\usepackage{enumerate}
\usepackage{float}
\usepackage{thmtools}
\usepackage{tabularx}
\linespread{1.25}
\usepackage{color}
\usepackage{verbatim}
\newcommand{\gelb}{0.550000011920929}
\usepackage{pgf,tikz,pgfplots}
\pgfplotsset{compat=1.15}
\usepackage{mathrsfs}
\usepackage{mathrsfs}
\usetikzlibrary{arrows}
%\numberwithin{equation}{chapter}
%\usepackage{scrheadings}
\pagestyle{headings}
\usepackage{titlesec}     
\usepackage{tikz}           % für Kontrolle der Abschnittüberschriften
\begin{comment}
\makeatother
\theoremstyle{nummermitklammern}
\theorembodyfont{\rmfamily}
\theoremsymbol{\ensuremath{\diamond}}
\newtheorem{temp}{}[section]
\newtheorem{vor}[temp]{Vorüberlegung}
\newtheorem{lemma}[temp]{Lemma}
\newtheorem{folgerung}[temp]{Folgerung}
\newtheorem{bsp}[temp]{Beispiel}
\newtheorem{herleitung}[temp]{Herleitung}
\newtheorem{definition}[temp]{Definition}
\newtheorem{bemerkung}[temp]{Bemerkung}
\newtheorem{satz}[temp]{Satz}
\newtheorem{beweisidee}[temp]{Beweisidee}
\theoremsymbol{\ensuremath{\square}}
\end{comment}
%\begin{comment}
\newtheorem{zahl}{}[section]
%\setcounter{zahl}{1}
%\newtheorem{section}{section}[section]
\newtheorem{definition}[zahl]{Definition}
\newtheorem{vor}[zahl]{Vorüberlegung}
\newtheorem{lemma}[zahl]{Lemma}
\newtheorem{folgerung}[zahl]{Folgerung}
\newtheorem{bsp}[zahl]{Beispiel}
\newtheorem{herleitung}[zahl]{Herleitung}
\newtheorem{bemerkung}[zahl]{Bemerkung}
\newtheorem{satz}[zahl]{Satz}
\newtheorem{beweisidee}[zahl]{Beweisidee}
\numberwithin{equation}{section}
\DeclareMathOperator{\Aut}{Aut}
\DeclareMathOperator{\Pot}{Pot}
%\DeclareMathOperator{\deg}{deg}


%-----------------------------------------------

%\end{comment}
 %Nummerierung mit Kapitelnummern
%-------------------------
%\newcommand{\secnumbering}[1]{% 
 % \setcounter{chapter}{0}% 
  %\setcounter{section}{0}% 
  %\renewcommand{\thechapter}{\csname #1\endcsname{chapter}.}% nach Duden gehört 
                                  % der Punkt hier hin bei gemischten Zählungen 
%  \renewcommand{\thesection}{\thechapter\csname #1\endcsname{section}}% 
%}
%------------------------------
\begin{document}
\begin{titlepage}
    \begin{center}
      \large
      \textsc{Rheinisch-Westf\"alische Technische Hochschule Aachen}\\
      Lehrstuhl B für Mathematik \\
      Univ.-Prof. Dr.  Alice Niemeyer\\
      \vspace{3 cm}
      \huge  Klassifikation der Sphären ohne Zweier Taillen\\
      \vspace{1 cm}
      \large Masterarbeit\\
      %\large Bachelor's Thesis\\
      \vspace{2 cm}
       \vspace{1 cm}
      \Large Reymond Oluwaseun Akpanya\\
      \large Matrikelnummer: 357115\\
      %\vspace{2 cm}
      %\large Vorgelegt am: 28.09.2018
      \vspace{3.5 cm}
%      last build:
 %    \today \\[4em]
\begin{flalign*}
&\text{Vorgelegt am:}&\text{...}&\\
&\text{Gutachter:}&\text{Prof. Dr. Alice Niemeyer}&\\
&\text{Zweitgutachter:}&\text{Prof. Dr. Wilhelm Plesken}&\\[1em]
\end{flalign*}
    \end{center}
% \begin{flalign*}
 %&\text{ } &\text{ 25.09.2018 }
 %\end{flalign*}
\end{titlepage}
%---------------------------
%\farb
%\section*{Inhaltsverzeichnis}
\newpage 
\thispagestyle{empty}
\quad 
\newpage
\thispagestyle{empty}

\tableofcontents
%\addcontentsline{toc}{section}{Einleitung}
\newpage
\setcounter{page}{1}
%\cleardoubelepage
\section{Einleitung}
\subsection{Herangehensweise}
\subsection{Aufbau dieser Arbeit}
\subsection{Vorkenntnisse und Notationen}
\section{Grundlagen} \label{Grundlagen}
Zu Beginn dieser Arbeit werden, um ein Verständnis von simplizialen Flächen zu entwickeln, grundlegende Definitionen und einführende Beispiele präsentiert. Diese basieren auf dem Skript "\emph{Simplicial Surfaces of Congruent Triangles}". Da dieses Kapitel nur als Einführung in die Thematik dienen soll, werden die benötigten Resultate ohne Beweise angeführt. Aufgrund der Tatsache, dass in diesem Teil der Arbeit keine neuen Resultate präsentiert werden, sondern lediglich bekannte Erkenntnisse reproduziert werden, kann dieses Kapitel bei bereits vorhandener Vertrautheit mit simplizialen Flächen übersprungen werden. Für ein tieferes Verständnis der angeführten Resultate sei auf das oben genannte Skript verwiesen.
\begin{figure}[H]
\begin{center}
\includegraphics[viewport=3cm 22.8cm 14cm 23cm]{new}
\end{center}
%\caption{Kantendrehung}
\end{figure}
\subsection{Grundlegende Konzepte}
\begin{definition}  \label{def1} 
Seien $X_0,X_1,X_2$ nichtleere Mengen so, dass $X=X_{0} \biguplus X_{1} \biguplus X_{2}$ eine abzählbare Menge und $<$ eine transitive Relation auf   
\[
(X_{0}\times X_{1}) \cup (X_{1}\times X_{2})\cup (X_{0}\times X_{2})
\]
 bildet.
 Wir nennen $X_{0}$ \emph{die Menge der Ecken}, $X_{1}$ \emph{die Menge der Kanten}, $X_{2}$ \emph{die Menge der Flächen} und $<$ die \emph{Inzidenz} der \emph{simplizialen Fläche} $(X,<)$, falls folgende Axiome erfüllt sind:
 \begin{enumerate}
\item Für jede Kante $e \in X_{1}$ existieren genau zwei Ecken $V_1,V_2 \in X_{0}$ mit $V_1,V_2 < e$. 
\begin{figure}[H]
\begin{center}
\includegraphics[viewport=0cm 25.5cm 4cm 27cm]{Image_Def11}
\end{center}
%\caption{Kantendrehung}
\end{figure}
\item Für jede Fläche $F\in X_2$ gibt es genau drei Kanten $e_1,e_2,e_3 \in X_{1}$ mit der Eigenschaft $e_1,e_2,e_3 < F$.
\begin{figure}[H]
\begin{center}
\includegraphics[viewport=0cm 25.5cm 4cm 26.5cm]{Image_Def12}
\end{center}
%\caption{Kantendrehung}
\end{figure} 
\item Für jede Kante $e \in X_{1}$ gibt es entweder genau zwei Flächen $F_{1},F_{2} \in X_{2}$ mit $e <F_{1},F_2$ oder
genau eine Fläche $F \in X_{2}$ mit $e < F$. Im ersten Fall sind $F_{1}$ und $F_{2}$ \emph{$(e)$-Nachbarn} und $e$ ist eine \emph{innere Kante}, im zweitem Fall ist $e$ eine \emph{Randkante}. 
 \item Für jede Ecke $V \in X_{0}$ existieren endlich viele Flächen $F\in X_{2}$ mit $V < F$.
  Für ein $n\in \mathbb{N}$ können diese $F_{i}\in X_2$ so in einem Tupel $(F_{1},\ldots,F_{n})$ angeordnet werden, dass $e_i<F_{i}$ und $e_i<F_{i+1}$ für $i=1,\ldots,n-1$ ist, wobei $e_i\in X_1$ eine Kante  in $X$ ist, für die ebenfalls $V<e_i$ gilt. 
  Das Tupel $(F_1,\ldots,F_n)$ wird auch \emph{Schirm} genannt. Falls es zudem eine Kante $e\in X_1$ mit $e<F_{1},F_{n}$ gibt, so ist $V$ eine \emph{innere Ecke}. Ist $V$ keine innere Ecke, so ist sie eine \emph{Randecke}.
\begin{figure}[H]
\begin{center}
\includegraphics[viewport=0cm 24cm 12cm 27cm]{Image_Def14}
\end{center}
%\caption{Kantendrehung}
\end{figure} 
 \item Seien $V \in X_0$ eine Ecke in $X$ und $(F_1,\ldots,F_n)$ der zu $V$ gehörige Schirm, wobei die $F_i$ für $i=1,\ldots ,n$ und $n\in \mathbb{N}$ Flächen in $X$ sind. Dann ist n der \emph{Grad der Ecke} $V$. Für den Grad einer Ecke $V$ in $X$ schreiben wir $\deg_X(V)$. Falls $X$ aus dem Kontext heraus klar ist, so schreiben wir nur $\deg(V)$.

\end{enumerate}
\end{definition}

\begin{bemerkung}
\begin{itemize}
%\item Die Menge aller inneren Knoten einer Kante $e \in X_1$ bezeichnet man mit $X_0^0(e).$
\item Für eine gegebene Ecke $V \in X_0$ einer simplizialen Fläche $X$ gibt es eine endliche Anzahl von Schirmen. Diese sind jedoch alle äquivalent, da sie durch zyklische Permutationen umgeordnet werden können.
\item Zur Vereinfachung identifizieren wir $(X,<)$ mit der Menge $X$. 
\item Die Definition einer simplizialen Fläche $(X,<)$ lässt abzählbar unendliche Mengen $X_i$ für $i=0,1,2$ zu, jedoch sind für diese Arbeit nur endliche simpliziale Flächen von Interesse. Im Folgenden sei also ohne Einschränkung $\vert X_0\vert,\vert X_1\vert,\vert X_2\vert < \infty$.
\end{itemize}
\end{bemerkung}
 
 \begin{bsp}\label{bspsimp}
 \begin{enumerate}
\item 
 Bis auf Isomorphie gibt es genau eine simpliziale Fläche bestehend aus einer Fläche. Diese wird durch die Mengen $
D_{0}=\{\,V_{1},V_{2},V_{3}\,\},$  $D_{1}=\{\,e_{1},e_{2},e_{3}\,\}, D_{2}=\{\,F_{1}\,\}$  und der Relation $x<y$ genau dann, wenn
\begin{align*}
 (x,y)\in \{\,&(e_{1},F_{1}),(e_{2},F_{1}),(e_{3},F_{1}),(V_{1},e_{2}),(V_{1},e_{3}),(V_{1},F_{1}),\\ &(V_{2},e_{1}), (V_{2},e_{3}),(V_{2},F_{1}),
 (V_{3},e_{1}),(V_{3},e_{2}),(V_{3},F_{1})\,\} 
\end{align*} 
beschrieben. Diese simpliziale Fläche wird \emph{Dreieck} genannt. 
%--------------------------Bild-------------------------
\begin{figure}[H]
\begin{center}
\includegraphics[viewport=1cm 25.cm 4cm 26.7cm]{oneface}
\end{center}
\caption{Dreieck}
\end{figure}
%-------------------------------------------------------
 \item
 Für $n \in \mathbb{N}$ definieren wir das \emph{n-fache Dreieck} $n \Delta$ durch die Mengen 
 \begin{align*}
  &n\Delta_0=\{ \,V_{j}^{k}\,\vert\, j=1,2,3,\,k=1,\ldots,n\,\},\\
   &n\Delta_1=\{\,e_{j}^{k}\,\vert\, j=1,2,3,\,k=1,\ldots,n\,\},\\
   &n\Delta_2=\{F_{1},\ldots,F_{n}\} .
   \end{align*}
   Für $x,y\in n\Delta$ gilt $x<y$ genau dann, wenn
   \begin{align*}
 (x,y)\in \{\,&(e_{1}^k,F_{k}),(e_{2}^k,F_{k}),(e_{3}^k,F_{k}),(V_{1}^k,e_{2}^k),(V_{1}^k,e_{3}^k),(V_{1}^k,F_{k}), (V_{2}^k,e_{1}^k),\\ &(V_{2}^k,e_{3}^k),(V_{2}^k,F_{k}),(V_{3}^k,e_{1}^k),(V_{3}^k,e_{2}^k),(V_{3}^k,F_{k})\mid k=1,\ldots,n\} 
\end{align*}
 \begin{figure}[H]
\begin{center}
\includegraphics[scale=1,viewport=0cm 24.5cm 16cm 26.5cm]{ndelta}
\end{center}
\caption{$n\Delta$ im Fall $n=3$}
\end{figure}
 \item 
 Der \emph{Janus-Kopf} $J$ ist eine geschlossene simpliziale Fläche, die aus zwei Flächen besteht. Sie besitzt 3 innere Ecken und 3 innere Kanten und wird durch die Mengen
 \begin{align*}
 &J_{0}=\{\,V_{1},V_{2},V_{3}\,\} ,\\
 &J_{1}=\{\,e_{1},e_{2},e_{3}\,\},\\
 &J_{2}=\{\, F_{1},F_{2}\,\}
\end{align*}
definiert. Für $x,y\in J$ gilt $x<y$ genau dann, wenn
\begin{align*} 
 (x,y)\in\{&\,(e_{1},F_{1}),(e_{1},F_{2}),(e_{2},F_{1}),(e_{2},F_{2}),(e_{3},F_{1}),(e_{3},F_{2}),(V_{1},e_{2}),(V_{1},e_{3}),\\ &(V_{1},F_{1}),
  (V_{1},F_{2}),(V_{2},e_{1}),(V_{2},e_{3}),(V_{2},F_{1})
 (V_{2},F_{2}), (V_{3},e_{1}), (V_{3},e_{2}),\\&(V_{3},F_{1}),(V_{3},F_{2}) \,\}.
 \end{align*}

 %----bild----------------------------
\begin{figure}[H]
\begin{center}
\includegraphics[viewport=1cm 25cm 5.5cm 26.5cm]{JanusHead}
\end{center}
\caption{Janus-Kopf}
\end{figure}
 %------------------------------------
 \item 
 Die \emph{offene Tasche} ist eine simpliziale Fläche, die aus dem \emph{Janus-Kopf} durch Verdopplung der Kante $e_{2},$ hervorgeht. Das heißt, sie wird durch die Mengen
%\begin{figure}[h]
 \begin{align*}
&  OB_{0}=\{\,V_{1},V_{2},V_{3}\,\},\\
 & OB_{1}=\{\,e_{1},e_{2},e_{3},e_{4} \,\},\\
  &OB_{2}=\{\,F_{1},F_{2}\,\}
  \end{align*}
  beschrieben. Für  $x<y\in OB$ gilt $x<y$ genau dann, wenn \begin{align*}
 (x,y)\in\{&\,(e_{1},F_{1}),(e_{1},F_{2}),(e_{2},F_{1}),(e_{3},F_{1}),(e_{3},F_{2}),(e_{4},F_{2}),(V_{1},e_{2}),(V_{1},e_{3}),\\ &(V_{1},e_{4}),
  (V_{1},F_{1}),(V_{1},F_{2}),(V_{2},e_{1}),(V_{2},e_{3})
 (V_{2},F_{1}), (V_{2},F_{2}), (V_{3},e_{1}),\\&(V_{3},e_{2}),(V_{3},e_{4}),(V_{3},F_{1}),(V_3,F_2) \,\}.
 \end{align*}
 \end{enumerate}
%--------------------------------------------
\begin{figure}[H]
\begin{center}
\includegraphics[viewport=1cm 25cm 5.5cm 26.5cm]{OB}
\end{center}
\caption{offene Tasche}
\end{figure}
\end{bsp}
%\newpage
\begin{bemerkung}
Die Anzahl der Flächen einer geschlossenen simplizialen Fläche ist durch $2$ teilbar, da
\[
\vert X_{2} \vert = \frac{2\vert X_{1}\vert}{3}
\]
ist.
Die Anzahl der Kanten ist insbesondere durch 3 teilbar. 
\end{bemerkung}
Um simpliziale Flächen vollständig und vor allem einfacher beschreiben zu können, wird eine weitere Notation eingeführt. Diese hängt von der Nummerierung der Ecken, Kanten und Flächen ab, davon abgesehen ist sie jedoch eindeutig.
\begin{definition}
 Sei $(X,<)$ eine simpliziale Fläche, deren Ecken $V_{1},\ldots,V_{n}$, Kanten $e_{1},\ldots,e_{k}$ und Flächen $F_{1},\ldots,F_{m}$ ausgehend von ihrer Nummerierung linear geordnet sind. Das \emph{Symbol} von $(X,<)$ ist definiert durch 
\[
\mu((X,<)):=(n,k,m;(X_{0}(e_{1}),\ldots,X_{0}(e_{k})),(X_{1}(F_{1}),\ldots,X_{1}(F_{m}))).
\]
Im Symbol können die Ecken $V_{i}$ durch $i$, die Kanten $e_{j}$ durch $j$ und die Flächen $F_{l}$ durch $l$ ersetzt werden und das resultierende Symbol wird dann das \emph{ordinale Symbol} $\omega((X,<))$ von $(X,<)$ genannt. Selbiges gilt von nun an für die folgenden Abbildungen. Beispielsweise sind die Ecken $V_i$ in Abbildungen durch $i$ repräsentiert. 
\end{definition}
\begin{bsp}
Beispielsweise kann der Tetraeder $(T,<)$ durch das Symbol 
\begin{align*}
\omega ((T,<)):=(4,6,4;&(\{1,2\}, \{1,3\},\{1,4\},\{2,3\},\{2,4\},\{2,4\},\{3,4\})\\
;&(\{4,5,6\},\{2,3,6\},\{1,3,5\},\{1,2,4\}))
\end{align*}
beschrieben werden.
\end{bsp}
\begin{figure}[H]
\begin{center}
\includegraphics[viewport=1.5cm 24cm 5cm 26cm]{ET_Example1}
\end{center}
\caption{Tetraeder}
\end{figure}

Diese Notation wird im Kapitel \ref{manipulation} behilflich sein, einen vereinfachten Zugang zu der Manipulation simplizialer Flächen zu finden. Für gewisse simpliziale Flächen reichen jedoch die Inzidenzen zwischen den Ecken und Flächen vollkommen aus, um diese bis auf Isomorphie eindeutig festzulegen. Bevor diese jedoch betrachtet werden, wird an dieser Stelle der Begriff der \emph{Euler-Charakteristik} eingeführt.
\begin{definition}
Für eine simpliziale Fläche $X$ definieren wir die \emph{Euler-Charakteristik} $\chi (X)$ als 
\[
\chi(X):=\vert X_0\vert-\vert X_1\vert+\vert X_2\vert.
\]
\end{definition}
\begin{bsp}
\begin{enumerate}
Beim Betrachten der im obigem Beispiel eingeführten simplizialen Flächen ergeben sich folgende Euler-Charakteristiken:
\item $\chi(D):=\vert D_0\vert-\vert D_1\vert+\vert D_2\vert=3-3+1=1$
\item $\chi(n\Delta):=\vert n\Delta_0\vert-\vert n\Delta_1\vert+\vert n\Delta_2\vert=3n-3n+n=n$
\item $\chi(J):=\vert J_0\vert-\vert J_1\vert+\vert J_2\vert=3-3+2=2$
\item $\chi(OB):=\vert OB_0\vert-\vert OB_1\vert+\vert OB_2\vert=3-6+4=1$
\end{enumerate}
\end{bsp}


%\begin{vor}
%Man kann eine genauere Aussage über die Anzahl der Kanten und %Flächen einer geschlossenen simplizialen Fläche treffen, indem man sich folgendes überlegt:
%\begin{itemize}
%\item
%abzahlargument 
%\end{itemize}

%\end{vor}
%\begin{lemma}
%Sei $X$ eine geschlossene simpliziale Fläche und $\epsilon$ der Flächen-Kantenparameter von $X$. Dann gilt 
%\[
%\epsilon=ggt(\vert X_1 \vert,\vert X_2\vert).
%\] 
%\end{lemma}
%\begin{proof}
%Um die obige Aussage zu nachzuprüfen, beweist man zunächst $\epsilon\vert ggt(\vert X_1 \vert,\vert X_2\vert)$ und dann $\epsilon\vert ggt(\vert X_1 \vert,\vert X_2\vert)$. 
% Der Flächen-Kanten-Parameter $\epsilon$ teilt $\vert X_2 \vert$ und er teilt $\vert X_2 \vert$ nach obiger Bemerkung. Daraus kann man $\epsilon \vert ggt(\vert X_1\vert ,\vert X_2\vert)$ schließen. \newline
% Man nehme nun $ ggt(\vert X_1\vert ,\vert X_2\vert) \nmid \epsilon$ an. Dann existiert ein $a \in \mathbb{N}\setminus \{1\}$, sodass $\epsilon=a* ggt(\vert X_1\vert ,\vert X_2\vert)$ gilt. Durch Nachrechnen erhält man 
% \[kgV(\vert X_1\vert,\vert X_2\vert)=2*3*\epsilon
% \]
%  und 
% \[
% kgV(\vert X_1\vert,\vert X_2\vert)=\frac{\vert X_1\vert \vert X_2\vert}{ggt(\vert X_1\vert,\vert X_2\vert)}=\frac{2*3*\epsilon^2 *a}{a*ggt(\vert X_1\vert,\vert X_2\vert)}=2*3*a*\epsilon.
% \]
%  Also ist $2*3*\epsilon=2*3*a*\epsilon$, was $a=1$ impliziert und den gewünschten Widerspruch erzeugt.

%\end{proof}
%Insbesondere ist $\epsilon=\frac{\vert X_1 \vert}{3}.$




 %------------------------------------
%\newpage
Um den Umgang mit simplizialen Flächen weiter zu vereinfachen, führen wir eine Notation ein, die die Beschreibung der Inzidenzen einzelner Elemente einer simplizialen Fläche erlaubt.
\begin{definition} 
Sei $(X,<)$ eine simpliziale Fläche und $i,j \in \{\,0,1,2\,\}$ mit $i \neq j$. Dann definieren wir für ein $x \in X_{i}$ die Menge 
\[
X_{j}(x):=\{\,y \in X_{j}\,|\,x < y\,\} \text{, falls $i < j$  }
\]
bzw. 
\[
X_{j}(x):=\{\,y \in X_{i}\,|\,y < x\}, \text{ falls $j<i$}.
\]
Für $S \subseteq X_{i}$ ist 
\[
X_j(S):= \bigcup_{x\in S}X_{j}(x).
\]
\end{definition}
%\newpage
\begin{bemerkung}
Für eine simpliziale Fläche $(X,<)$ können die Axiome aus \Cref{def1} mit obiger Definition neu umformuliert werden:
\begin{itemize}
\item $\vert X_{0}(e)\vert=2$ für alle $e \in X_{1}$,
\item $\vert X_{0}(F)\vert=3$ für alle $F \in X_{2}$,
\item $\vert X_{1}(F)\vert=3$ für alle $F \in X_{2}$,
\item $1\leq  \vert X_{2}(e)\vert \leq 2$ für alle $e \in X_{1}$.

\end{itemize}
\end{bemerkung}
Wie der Titel dieser Arbeit bereits verrät, werden wir \emph{Sphären} ohne sogenannte \emph{2-Taillen} untersuchen. Diese Definitionen werden an dieser Stelle eingeführt.

\begin{definition}  Eine \emph{geschlossene} simpliziale Fläche ist eine simpliziale Fläche, deren Kanten alle innere Kanten sind. Eine geschlossene simpliziale Fläche der Euler-Charakteristik  2 nennen wir eine \emph{Sphäre}.
\end{definition} 

\begin{definition}
Sei $X$ eine simpliziale Fläche.

 Falls es verschiedene Kanten $e_1,e_2$ in $X$ mit $X_0(e_1)=X_0(e_2)$ gibt, dann nennen wir $(e_1,e_2)$ eine 2-Taille.
 \begin{figure}[H]
\begin{center}
\includegraphics[viewport=0cm 25.5cm 5cm 26.5cm]{Image_2Waist}
\end{center}
%\caption{Kantendrehung}
\end{figure} 
 Falls es paarweise verschiedene Kanten $e_1,e_2,e_3$ in $X$ gibt, die die Gleichheit $\vert X_0(e_1)\cup X_0(e_2)\cup X_0(e_3) \vert=3$ und $X_2(e_i)\cap X_2(e_j)=\emptyset$ für $i \neq j\in\{1,2,3\}  $ erfüllen, dann nennen wir $(e_1,e_2,e_3)$ eine 3-Taille. 
\end{definition}
Mit der offenen Tasche ist bereits ein Beispiel für eine simpliziale Fläche mit einer 2-Taille bekannt. Beispiele für simpliziale Flächen mit 3-Taillen folgen mit der Einführung der sogenannten \emph{Multi-Tetraeder} in Kapitel \ref{cactus}. 
\begin{definition} \label{2waistk}
Sei $X$ eine Sphäre mit einer 3-Taille $(e_1,e_2,e_3)$. Dann lässt sich $X_2$ in  Mengen $M_1,M_2$ aufteilen, die Folgendes erfüllen:
\begin{itemize}
\item $M_1,M_2$ sind nichtleer und disjunkt
\item $M_1\cup M_2=X_2$
\item $\vert M_i \cap X_2(e_j)\vert =1$ für $i=1,2$ und $j=1,2,3.$
\item $M_1,M_2$ sind maximal bezüglich Inklusion mit der Eigenschaft, dass es für jedes $F$ in $M_1$ bzw. $M_2$ eine benachbarte Fläche $F'$ in $M_1$ bzw. $M_2$ gibt.
\end{itemize}  
 $M_1,M_2$ sind dann die \emph{3-Taillen Komponenten}. 
\end{definition}
Analog werden auch \emph{2-Taillen Komponenten} definiert.
\begin{definition}
Sei $X$ eine Sphäre mit einer 2-Taille $(e_1,e_2)$. Dann lässt sich $X_2$ in  Mengen $M_1,M_2$ aufteilen, die Folgendes erfüllen:
\begin{itemize}
\item $M_1,M_2$ sind nichtleer und disjunkt
\item $M_1\cup M_2=X_2$
\item $\vert M_i \cap X_2(e_j)\vert =1$ für $i=1,2$ und $j=1,2.$
\item $M_1,M_2$ sind maximal bezüglich Inklusion mit der Eigenschaft, dass es für jedes $F$ in $M_1$ bzw. $M_2$ eine benachbarte Fläche $F'$ in $M_1$ bzw. $M_2$ gibt.
\end{itemize}  
 $M_1,M_2$ sind dann die \emph{2-Taillen Komponenten}. 
\end{definition}
Sei $X$ beispielsweise die Sphäre, die durch das ordinale Symbol
\begin{align*}
(4,6,2;&(\{1,2,\},\{1,3,\},\{2,3\},\{2,3\},\{2,4\},\{3,4\});\\&(\{1,3,4\},\{2,3,4\},\{3,4,5\},\{3,4,6\}))
\end{align*}
beschrieben wird. Dann bildet $(e_3,e_4)$ eine 2-Taille in $X$.
\begin{comment}
%beispiel angefangen aber nicht zugeende geführt OB^2
Die offene Tasche aus dem Beispiel \ref{bspsimp} ist wie schon erwähnt eine Sphäre mit einer 2-Taille. Für diese bilden $M_1=\{\}$ und $M_2=\{\}$ die 2-Taillen Komponenten.
\end{comment}
\subsection{Homomorphismen}
An dieser Stelle wollen wir den Begriff eines \emph{Homomorphismus} zwischen simplizialen Flächen einführen.
\begin{definition} Seien $(X,<)$ und $(Y,\prec)$ simpliziale Flächen.
\begin{enumerate}
 \item Eine bijektive Abbildung $\alpha: X \to Y$ wird ein \emph{Isomorphismus} genannt, falls $A<B$ in $(X,<)$ genau dann gilt, wenn $\alpha(A) \prec \alpha(B)$ in $(Y,\prec)$  ist. In diesem Fall schreiben wir $X \cong Y$.
\item Eine surjektive Abbildung $\alpha: X \to Y$ heißt \emph{Überdeckung}, falls aus $A<B$ in $(X,<)$ folgt, dass $\alpha(A) \prec \alpha(B)$ in $(Y,\prec)$ gilt. 
\end{enumerate}
\end{definition}
\begin{bemerkung}
\begin{itemize}
\item

Für $i=0,1,2$ induziert eine Überdeckung $\alpha:X\to Y$ surjektive Abbildungen $X_{i} \to Y_{i}$.
\item 
Für $i=0,1,2$ induziert ein Isomorphismus $\beta:X \to Y$  bijektive Abbildungen $X_{i} \to Y_{i}$.
\item
Für eine simpliziale Fläche $X$ und eine Fläche $F$ in $X$ sei $X(F)$ die simpliziale Fläche, die mit 
\begin{align*}
&X(F)_0:=X_0(F)\\
&X(F)_1:= X_1(F)\\
&X(F)_2:= \{F\} 
\end{align*}
 identifiziert wird.
Dann gilt für simpliziale Flächen $X$ und $Y$ mit Flächen $F \in X_2$ und $F' \in Y_2$, dass $X(F)$ und $Y(F')$ isomorph sind. Für einen Isomorphismus $\alpha : X(F)\mapsto Y(F')$ gibt es genau 6 Möglichkeiten.
\item
 Zwei isomorphe simpliziale Flächen $(X,<)$ und $(Y, \prec)$ haben dieselbe Euler-Charakteristik, denn eine bijektive Abbildung $\alpha:X \to Y$ impliziert, wie oben schon erwähnt, für $i=0,1,2$ bijektive Abbildungen  $X_i \to Y_i$. Damit gilt dann $\vert X_i \vert =\vert Y_i \vert $, woraus
 \[
\chi(X) =\vert X_0 \vert - \vert X_1\vert +\vert X_2 \vert = \vert Y_0 \vert - \vert Y_1\vert +\vert Y_2 \vert =\chi(Y)
 \]
 folgt. \\
 Die Umkehrung gilt im Allgemeinen nicht.
\end{itemize}
\end{bemerkung}
\begin{definition}
Sei $X$ eine simpliziale Fläche. Ein Isomorphismus $\phi$ von
$X$ nach $X$ wird \emph{Automorphismus} genannt. Die Menge aller Automorphismen von $X$ nach $X$ mit der Verkettung von Abbildungen als Verknüpfung ist die Automorphismengruppe von $X$ und wir bezeichnen diese mit $\Aut(X)$. 
\end{definition}
\begin{bsp}
Die Automorphismengruppe des Dreiecks $D$ besteht aus drei 
Spiegelungen und 3 Drehungen. Also ist
\[
\Aut(D)\cong D_3.
\]
\end{bsp}
\begin{bemerkung}
Sei $X$ eine Sphäre und $F$ eine Fläche in $X.$ Für $i=0,1$ sei 
\[
f:X_i(F)\mapsto X_i(F)  
\] 
eine bijektive Abbildung. Dann gibt es höchstens einen Automorphismus $\phi \in \Aut(X)$ mit 
\[
\phi(x)=f(x) 
\]
für $x\in X_i(F).$
\end{bemerkung}


\subsection{Vertex-treue Sphären}
Von besonderem Interesse sind jene simplizialen Flächen, für die die Angabe der Inzidenzen zwischen den Ecken und Flächen ausreichend ist, um die simpliziale Fläche eindeutig festzulegen. An dieser Stelle werden einführende Beispiele und Definitionen präsentiert, damit diese Klasse von simplizialen Flächen im weiteren Verlauf dieser Arbeit genauer untersucht werden kann.
\begin{definition}
Eine simpliziale Fläche $X$ heißt $\emph{vertex-treu}$, falls die Abbildung
\[
X_1 \cup X_2 \to \Pot(X_0),S \mapsto X_0(S)
\]
 injektiv ist. 

Die Kanten und Flächen einer vertex-treuen simplizialen Fläche $X$ können  mit ihren Bildern unter obiger Abbildung identifiziert werden. In diesem Fall gilt also $X_1 \subseteq \Pot_2(X_0)$ bzw. $X_2\subseteq \Pot_3(X_0)$. 
\end{definition}
\begin{bsp}
\begin{itemize}
\item Eine simpliziale Fläche mit einer 2-Taille ist nicht vertex-treu.
\item Der Janus-Kopf ist nicht vertex-treu, da die beiden Flächen der Sphäre  zu den selben drei Ecken inzident sind.
\end{itemize}
\end{bsp}
\begin{definition}
Sei $P$ eine endliche Menge mit mindestens 3 Elementen. Eine nicht-leere Menge $\xi \subseteq \Pot_3(P)$ für die $P=\bigcup_{F\in \xi} F$ gilt, heißt ein \emph{Flächen-Träger} auf $P$, falls für alle $V\in P$ die Menge aller $F_i \in \xi$ mit $V \in F_i$ in einem Zykel  $(F_1,\ldots ,F_n)$ geschrieben werden kann, sodass $\vert F_i \cap F_{i+1}\vert=2 $ für $i=1,\ldots n-1$ gilt und aus $\vert F_i \cap F_{i+1}\vert=2$ entweder $\vert i-j\vert =1$ oder $\{i,j\}=\{1,n\}$ folgt.  
\end{definition}
\begin{lemma}
Sei $P$ eine endliche Menge und $\xi \subseteq \Pot_3(P)$ ein Flächen-Träger. Dann definiert $\mathcal{S}(\xi)$ mit den Mengen 
\[
\mathcal{S}(\xi)_i:=\{A\subseteq F\mid F\in \xi,\vert A\vert=i+1\}\text{ für }i=0,1,2 
\]eine vertex-treue simpliziale Fläche, wobei die Inzidenz $<$ durch Mengeninklusion gegeben ist. Wir nennen $\mathcal{S},$ die durch $\xi$ \emph{getragene} simpliziale Fläche.
\end{lemma}
\begin{proof}
Der Beweis kann dem Skript oben genannten Skript entnommen werden.
\begin{comment}
Man muss nachweisen, dass $S(\xi)$ die Axiome in \Cref{def1} erfüllt.
\begin{itemize}
\item Das es zu jeder Kante genau zwei Ecken gibt, die inzident zu dieser sind, ist klar, denn zu einer zweielementigen gibt es genau zwei einelementigen Teilmengen.
\item Da es zu einer dreielementigen Menge genau drei einelementigen Teilmengen gibt, gibt es in $\mathcal{S}$ zu jeder Fläche genau drei Knoten. 
\item Eine Kante ist durch die Mengeninklusion inzident zu mindestens einen Fläche und durch obige Definition eines Flächen-Trägers erhält man, dass eine Kante zu höchstens zwei Flächen inzident ist. 
\item Die Anordnung der Flächen einer Ecke in einem Schirm wird in Definition 4.3 verlangt und ist somit klarerweise erfüllt.
\end{itemize}
\end{comment}
\end{proof}
\begin{bsp} \label{bspO}
\begin{enumerate}
Der Flächen-Träger 
\[
\zeta=\{\{1,2,3\},\{1,3,4\},\{1,2,4\},\{2,3,4\}\}
\]
bildet eine getragene simpliziale Fläche, die zu einem Tetraeder isomorph ist.
\item Der Schmetterling lässt sich durch 
\[
\zeta =\{\{1,2,3\},\{2,3,4\}\}
\] darstellen.
\begin{figure}[H]
\begin{center}
\includegraphics[viewport=1cm 25.5cm 4cm 27cm]{Butterfly}
\end{center}
\caption{Schmetterling}
\end{figure}
\item
Für $n\in \mathbb{N}$ definieren wir den \emph{$n$-Streifen} durch den Flächen-Träger
\[
\zeta=\{\{l,l+1,l+2\}\mid 1\leq l \leq n\}.
\]
\begin{figure}[H]
\begin{center}
\includegraphics[viewport=1cm 25.5cm 8cm 26.5cm]{nstreifen}
\end{center}
\caption{der $n$-Streifen für $n=6$}
\end{figure}
\item 
Für $n\geq 3$ definieren wir den Doppel-$n$-Gon $(n)^2$ durch den Flächen-Träger  
\[
\zeta=\{\{1,i,i+1\},\{i,i+1,n+2\}\mid i=2\ldots n-1\}\cup \{\{1,2,n+1\},\{2,n+1,n+2\}\}.
\]  
\item
Für $n=4$ ist beispielsweise der Oktaeder $O$ durch den  Flächen-Träger
\begin{align*}
\zeta=\{&\{1,2,3\},\{1,3,4\},\{1,4,5\},\{1,2,5\},\\
&\{6,2,3\},\{6,3,4\},\{6,4,5\},\{6,2,5\}\}
\end{align*} 
gegeben.
\begin{figure}[H]
\begin{center}
\includegraphics[viewport=17cm 17cm 5cm 20.7cm]{Image_Octahedron}
\end{center}
\caption{Oktaeder}
\end{figure} 
\end{enumerate}
\end{bsp}

%\begin{definition}
 % Seien $(X^1,<_1)\ldots (X^n,<_n)$ für $n\in \mathbb{N}$ und simpliziale Flächen, die $X^j\cap X^i=\emptyset $ fuer $i\neq j$ erfüllen. Dann bildet die Menge $Y=\bigcup_{i=1}^{n}X^i$ mit den Identitaeten
%\begin{align*}
%  Y_0&=\bigcup_{i=1}^{n}X_0^i\\
%  Y_1&=\bigcup_{i=1}^{n}X_1^i\\
%  Y_2&=\bigcup_{i=1}^{n}X_2^i\\
%\end{align*}   
%und der Relation 
%\[
%x<y \text{ in } Y \Leftrightarrow x <_i y \text{ in } X^i \text{ fuer } 1\leq i\leq n 
%\]
%die simpliziale Fläche $(Y,<)$.
%  \end{definition}
\subsection{Wilde Färbungen}
In diesem Abschnitt der Arbeit werden Färbungen auf simplizialen Flächen betrachtet. Genauer führen wir hier eine Färbung der Kanten ein. Wir beschränken uns jedoch auf simpliziale Flächen vom sphärischen Typ und geben deshalb die Resultate in diesen vereinfachten Fall reduziert an. Für eine allgemeinere Definition sei an dieser Stelle wieder auf das zugrundeliegende Skript verwiesen.
\begin{definition}
Sei $X$ eine Sphäre. Eine \emph{wilde Färbung} auf $X$ ist eine Abbildung $\omega:X_1\to \{a,b,c\}$ so dass für ein beliebiges $F\in X_2$ die Einschränkung von $\omega$ auf $X_1(F)$ bijektiv ist. Wir nennen $\omega(e)$ die \emph{Farbe} der Kante $e$ und das Tupel $(X,\omega)$ die \emph{wild gefärbte Sphäre}. 
\end{definition}
Wir können die Kanten unter einer Färbung in zwei Typen unterteilen. Dies wird in der folgenden Definition skizziert.
\begin{definition}
Sei $(X,\omega)$ eine wild gefärbte Sphäre und $e\in X_1$ eine innere Kante mit $X_2(e)=\{F,F'\}.$ Die Kante $e$ ist vom Typ $m,$ falls der Isomorphismus $X(F)\to X(F'),$ der $e$ fixiert und die Färbung der Kanten berücksichtigt, ebenfalls die beiden zu $e$ inzidenten Ecken fixiert. Andernfalls ist $e$ eine Kante vom Typ $r.$
\end{definition}
\begin{figure}[H]
\begin{center}
\includegraphics[viewport=1cm 24.cm 10cm 27cm]{deffaerbung}
\end{center}
\caption{die verschiedenen Typen einer Kante}
\end{figure}
\begin{definition}
Sei $\omega$ eine wilde Färbung auf der Sphäre $X.$ Für $x\in \{a,b,c\}$ nennen wir $\omega$ eine \emph{$x$-zahme} Färbung, falls  alle Kanten in $\omega^{-1}(\{x\})$ vom selben Typ sind. Wir nennen $\omega$ eine \emph{zahme Färbung}, falls $\omega$ für alle $x\in \{a,b,c\}$ eine $x-$zahme Färbung ist. 
\end{definition}
In dem Fall, dass auf $X$ eine zahme Färbung $\omega$ existiert, können wir der Sphäre ein Tupel $(t_a,t_b,t_c)$ zuordnen, wobei $t_x$ für $x\in \{a,b,c\}$ der Typ der Kanten in $\omega^{-1}(\{x\})$ ist. Wir sagen dann, dass $X$ eine $(t_a,t_b,t_c)-$Struktur besitzt.\\
Der Oktaeder beispielsweise kann mit vier verschiedenen Strukturen ausgestattet werden:
\begin{center}
$\fbox{
\parbox{14cm}{
\textcolor{red}{gap$>$}\textcolor{blue}{Octahedron();}\newline
simplicial surface (6 vertices, 12 edges, and 8 faces)\newline
\textcolor{red}{gap$>$} \textcolor{blue}{AllTameColouredSurfaces(last);}\newline
[ tame coloured surface (MMM with 6 vertices, 12 edges and 8 faces)
    , \newline
     tame coloured surface (MRM with 6 vertices, 12 edges and 8 faces)
    ,\newline
     tame coloured surface (MMR with 6 vertices, 12 edges and 8 faces)
    ,\newline
  tame coloured surface (RMM with 6 vertices, 12 edges and 8 faces)]
}}
$
\end{center}
\section{Manipulation simplizialer Flächen}\label{manipulation}
Dieses Kapitel beschreibt das Manipulieren simplizialer Flächen, um neue simpliziale Flächen zu konstruieren. Im Einzelnen werden nun die Operationen
\begin{itemize}
\item Schmetterlings-Entfernung,
\item Kantendrehung,
 \item Schmetterlings-Erweiterung,
 \item und Tetraeder-Erweiterung
\end{itemize}
 eingeführt.
Diese Operationen können ebenfalls dem zugrundeliegendem Skript entnommen werden und sind somit nur als eine Wiederholung zu verstehen. Für die Definition der Konstruktionen nehmen wir an, dass $X$ eine Sphäre mit $\vert X_2 \vert \geq 4$ ist.

%Zur Einfachhalt halber nimmt man an, dass eine simpliziale Fläche $X$ mit k Ecken, m Kanten und n Flächen durch die Menge 
%\[
%X_0\cup X_1\cup X_2=\{V_1\ldots V_k\}\cup\{e_1\ldots e_m\}\cup\{F_1\ldots F_n\}
%\] 
%dargestellt wird.
  \subsection{Tetraeder-Erweiterung}
 Um das Erweitern durch Tetraeder zu beschreiben, wird die zusätzliche Annahme getroffen, dass die betrachteten Sphären vertex-treu sind. Es ist möglich, eine Definition der Erweiterungen auch in dem allgemeinen Fall anzugeben, aber hierauf wird an dieser Stelle verzichtet.
 Hilfreich hierfür ist die Definition von getragenen Flächen. Wir betrachten zunächst folgendes Lemma.
\begin{lemma}
Seien $U,P,Q$ endliche Mengen für die $U=P\cup Q$ und $\vert P\cap Q \vert \geq 3$ gilt. Außerdem sind $U\setminus P\neq \emptyset$ und $U\setminus Q \neq \emptyset$. Weiter seien $\xi \subseteq \Pot_3(P)$ ein Flächen-Träger auf P und $\zeta \subseteq \Pot_3(Q)$ ein Flächen-Träger auf $Q$ so, dass $\zeta \cap \xi$ ein Flächen-Träger auf $P \cap Q$ ist. Dann ist die symmetrische Differenz $\pi :=\xi \Delta \zeta$ ein Flächen-Träger einer simplizialen Fläche, falls es eine Ecke gibt, die in all den Mengen $\mathcal{S}(\xi),\mathcal{S}(\zeta),\mathcal{S}(\xi\cap\zeta)$ liegt. 
\end{lemma}

\begin{definition}
Sei $X$ beschrieben durch den Flächen-Träger $\xi$ und $F$ eine Fläche in $X$ mit $X_0(F)=\{V_1,V_2,V_3\}.$ Sei nun $P\notin X_0$, dann definieren wir einen Tetraeder durch den Flächen-Träger
\[
\xi_T=\{\{V_1,V_2,V_3\},\{P,V_1,V_2\},\{P,V_1,V_3\},\{P,V_2,V_3\}\}
\] 
und damit die Sphäre $Y$, die durch $\xi \Delta \xi_T$ getragen wird. Wir sagen, dass die Sphäre $Y$ durch das Anhängen eines Tetraeders an $X$ an der Fläche $F$ entstanden ist und bezeichnen sie mit $T^F(X)$.
\end{definition}

\begin{definition}
Sei $X$ eine Sphäre, die nicht zu einem Tetraeder isomorph ist.
Weiterhin sei $\xi$ der Flächen-Träger von $X$ und $P\in X_0$ eine Ecke vom Grad 3. Dann gibt es Ecken $V_1,V_2,V_3$ in $X,$ sodass
\[
X_0(X_2(P))=\{P,V_1,V_2,V_3\}
\] 
ist. Da $X$ vertex-treu ist, identifizieren wir die Flächen in $X_2$  mit den inzidenten Ecken. Durch den Flächen-Träger 
\[
\xi_T=\{\{V_1,V_2,V_3\},\{P,V_1,V_2\},\{P,V_1,V_3\}\{P,V_2,V_3\}\}
\]
 wird dann ein Tetraeder definiert. 
  Dadurch entsteht die Sphäre $Y$, die durch $\xi \Delta \xi_T$ getragen wird. Wir sagen, die Sphäre $Y$ ist durch Entfernen des Tetraeders an der Stelle $P$ entstanden und bezeichnet sie mit $T_P(X)$.
\begin{figure}[H]
\begin{center}
\includegraphics[viewport=18cm 22.5cm -3cm 27cm]{Image_Tetraedererweiterung}
\end{center}
\caption{Tetraeder-Erweiterung und Tetraeder-Entfernung}
\end{figure}
\begin{bemerkung}
An dieser Stelle wurde bewusst eine Notation für die durch eine Tetraeder-Erweiterung entstandene Sphäre gewählt, die von der Notation im Skript abweicht. Die hier verwendete Notation wurde aus Gründen der besseren Lesbarkeit benutzt. Für Beispiele dieser Operation verweisen wir auf die Multi-Tetraeder in Kapitel $\ref{cactus}$.
\end{bemerkung}
\end{definition}

  \subsection{Schmetterlings-Entfernung}
 \begin{definition}
 Sei $X$ eine Sphäre mit paarweise verschiedenen Ecken $V_1\ldots ,V_4$, Kanten $e_1,\ldots,e_5$ und Flächen $F_1,F_2$,
 % und $V_1,V_2,V_3,V_4$ Ecken, $e_1,e_2,e_3,e_4,e_5$  Kanten und  $F_1,F_2,$ Flächen 
 die folgende Relationen erfüllt: 
 \begin{itemize}
 \item $X_2(e_1)=\{F_1,F_2\}$
 %\item $X_0(X_2(e_1))=\{V_1,V_2,V_3,V_4\}$
% \item $X_1(X_2(e_1))=\{e_1,e_2,e_3,e_4,e_5\}$
  \item $(X_1(F_1),X_1(F_2))=(\{e_1,e_2,e_5\},\{e_1,e_3,e_4\})$
 \item   $(X_0(e_1),X_0(e_2),X_0(e_3),X_0(e_4),X_0(e_5)) =$ \\$ (\{V_2,V_4\},\{V_1,V_2\},\{V_2,V_3\},\{V_3,V_4\},\{V_1,V_4\})$
\end{itemize}  

\begin{figure}[H]
\begin{center}
\includegraphics[viewport=-1cm 23cm 5cm 26.5cm]{butt}
\end{center}
\caption{Ausschnitt einer simplizialen Fläche}
\end{figure}

Wir erhalten die Sphäre ${}^{e_1}\beta(X)$, die durch das Symbol $\omega({}^{e_1}\beta(X))$ beschrieben wird, durch das Anwenden der folgenden Schritte beim ordinalen Symbol $\mu((X,<))$: 
\begin{itemize}
\item die Anzahl der Ecken wird um 1, die Flächenanzahl um 2 und die Kantenanzahl um 3 verringert.
\item der Eintrag $X_0(e_1)$ an der Stelle $e_1$, der Eintrag $X_0(e_2)$ an der Stelle $e_2$ und Eintrag $X_0(e_3)$ an der Stelle $e_2$ werden gelöscht. 
\item der Eintrag $X_1(F_1)$ an der Stelle $F_1$ und der Eintrag $X_1(F_2)$ an der Stelle $F_2$ werden gelöscht. 
\
\item an jeder Stelle ${F}'$ in der $e_2$ in $X_1({F}')$ vorkommt, wird $e_2$ durch $e_5$ ersetzt
\item an jeder Stelle ${F}'$ in der $e_3$ in $X_1({F}')$ vorkommt, wird $e_3$ durch $e_4$ ersetzt,
\item an jeder Stelle ${e}'$ an der $V_2$ in $X_0({e}')$ vorkommt, wird $V_2$ durch $V_4$ ersetzt.
\end{itemize}
\begin{figure}[H]
\begin{center}
\includegraphics[viewport=-1cm 23cm 16cm 26.5cm]{butt2}
\end{center}
\caption{Schmetterlings-Entfernung}
\end{figure}
Zur Durchführung der Schmetterlings-Entfernung reicht jedoch die Angabe der Kante $e_1$, da die Flächen durch $X_2(e_1)$, die Kanten durch $X_1(X_2(e_1))$ und die Ecken $X_0(X_2(e_1))$ eindeutig festgelegt sind. Es gibt 2 Möglichkeiten die obigen Kanten und Flächen zu wählen, doch beide Wahlen liefern isomorphe simpliziale Flächen.
 \end{definition}
\begin{comment} 
 Dies soll an dieser Stelle mit dem Oktaeder $O$ und der Kante 1 durchgeführt werden. Außerdem gebraucht, wird die Wahl 
\begin{align*}
&(V_1,V_2,V_3,V_4,e_1,e_2,e_3,e_4,e_5,F_1,F_2)=\\&(3,1,5,2,1,2,4,6,5,1,3)
\end{align*}
\end{comment} 
 Wählen wir als Sphäre den Oktaeder, der durch das Symbol
 \begin{align*}
 \omega((O,<))=&(6,12,8;(\{1,2\},\{1,3\},\{1,4\},\{1,5\},\{2,3\},\{2,5\},\\
 &\{2,6\},\{3,4\},\{3,6\},\{4,5\},\{4,6\},\{5,6\});\\
 &(\{1,2,5\},\{6,7,12\},\{1,4,6\},\{5,7,9\},\\&\{3,4,10\},\{8,9,11\},\{2,3,8\},\{10,11,12\}))
 \end{align*}
 beschrieben wird, dann erhalten wir bei Durchführung der Schmetterlings-Entfernung an der Kante $1$ bis auf Isomorphie den Doppel-3-Gon. 
 \begin{comment}
 Zunächst werden die Anzahlen der Ecken, Kanten und Flächen angepasst.
 \begin{align*}
 &(5,9,6;(\{1,2\},\{1,3\},\{1,4\},\{1,5\},\{2,3\},\{2,5\},\\
 &\{2,6\},\{3,4\}\{3,6\},\{4,5\},\{4,6\},\{5,6\});\\
 &(\{1,2,5\},\{6,7,12\},\{1,4,6\},\{5,7,9\},\{3,4,10\},\{8,9,11\},\{2,3,8\},\{10,11,12\}))
 \end{align*}
 Nun werden die Einträge $X_0(1),X_0(2),X_0(4)$ gelöscht 
 \begin{align*}
 &(5,9,6;(\{1,4\},\{2,3\},\{2,5\},\\
 &\{2,6\},\{3,4\}\{3,6\},\{4,5\},\{4,6\},\{5,6\});\\
 &(\{1,2,5\},\{6,7,12\},\{1,4,6\},\{5,7,9\},\{3,4,10\},\{8,9,11\},\{2,3,8\},\{10,11,12\}))
 \end{align*}
 Löschen der Einträge $X_1(1),X_1(3)$ führt zu:
\begin{align*}
 &(5,9,6;(\{1,4\},\{2,3\},\{2,5\},\\
 &\{2,6\},\{3,4\}\{3,6\},\{4,5\},\{4,6\},\{5,6\});\\
 &(\{6,7,12\}\{5,7,9\},\{3,4,10\},\{8,9,11\},\{2,3,8\},\{10,11,12\}))
 \end{align*} 
 Nun ersetzte in der Beschreibung für die Kanten-Flächen-Inzidenz jedes Vorkommen der Kante 2 bzw. 4 durch die Kante 5 bzw. 6.
 \begin{align*}
 &(5,9,6;(\{1,4\},\{2,3\},\{2,5\},\\
 &\{2,6\},\{3,4\}\{3,6\},\{4,5\},\{4,6\},\{5,6\});\\
 &(\{6,7,12\},\{5,7,9\},\{3,6,10\},\{8,9,11\},\{3,8,10\},\{10,11,12\}))
 \end{align*}
 \end{comment}
 Dieser wird durch das folgende ordinale Symbol beschrieben.
 \begin{align*}
 &(5,9,6;(\{2,4\},\{2,3\},\{2,5\},\\
 &\{2,6\},\{3,4\}\{3,6\},\{4,5\},\{4,6\},\{5,6\});\\
 &(\{6,7,12\},\{5,7,9\},\{3,6,10\},\{8,9,11\},\{3,8,10\},\{10,11,12\})),
 \end{align*}
 Wir werden die Sphäre von nun an als \emph{Doppel-Tetraeder} bezeichnen.
 \begin{figure}[H]
\begin{center}
\includegraphics[viewport=22cm 12.5cm 5cm 17cm]{Image_DoubleTetraeder}
\end{center}
\caption{Doppel-Tetraeder}
\end{figure}
 \subsection{Schmetterlings-Erweiterung}
 \begin{definition}
 Sei $X$ eine Sphäre mit paarweise verschiedenen Ecken $V,V_1,V_2$, Kanten  $e_1,\ldots, e_5\in X_1$ und Flächen $F,F_1,F_2\in X_2$, die $ X_0(e_1)\cap X_0(e_2) =\{V\}$ erfüllen. 
 
 Falls $X_2(e_1)\cap X_2(e_2)=\{F\}$ für ein $F\in X_2$ ist, können folgende Relationen festgesetzt werden: 
\begin{itemize}
\item $(X_0(e_1),X_0(e_2))=(\{V,V_1\},\{V,V_2\})$
\item $(X_2(e_1),X_2(e_2))=(\{F,F_1\},\{F,F_2\})$ für Flächen $F_1,F_2$.
%\item $(X_1(F_1),X_1(F_2),X_1(F))=(\{e_1,e_1,e_2\},\{e_2,e_3,e_4\},\{e_1,e_2,e_5\})$ 
\end{itemize}
\begin{figure}[H]
\begin{center}
\includegraphics[viewport=4cm 24.5cm 5cm 27.5cm]{butt3}
\end{center}
\caption{Ausschnitt einer simplizialen Fläche}
\end{figure}
Die Sphäre $\beta(X)_{e_1,e_2}$ wird dann durch
\[
\beta(X)_{e_1,e_2}= T^F(X)
\]
definiert.
\begin{comment}
Man erhält das ordinale Symbol der simplizialen Fläche $\beta(x)_{e_1,e_2}$ nun, indem eine neue Ecke $V'$, neue Kanten $e,e_1',e_2'$ und  Flächen $F',F''$ eingeführt werden und beim ordinale Symbol $\mu((X,<))$
 \begin{itemize}
 %%\item die Einträge $X_0(e_1)$ und $X_0(e_2)$ löscht,
%% \item an der Stelle $F_1$ beim Eintrag $X_0(F_1)$ die Kante $e_1$ durch $e_1'$ ersetzt,
 %%\item an der Stelle $F_2$ beim Eintrag $X_0(F_2)$ die Kante $e_2$ durch $e_2'$ ersetzt,
% \item an der Stelle $F$ den Eintrag $X_1(F)$ durch $\{e_1',e_2',e_5\}$ ersetzt,
\item die Eckenanzahl wird um 1, die Flächenanzahl um $2$ und die Kantenanzahl um $3$ erhöht, 
\item an der Stelle $F$ in $X_1(F)$ die Kante $e_1$ durch $e_1'$ und die Kante $e_2$ durch $e_2'$ ersetzt,
 \item an jeder Stelle $\overline{e}$, in der $V$ in $X_0(\overline{e})$ vorkommt, $V$ durch $V'$ ersetzt,  
\item  bei der Beschreibung  für die Ecken-Kanten-Inzidenz $\{V',V''\}$ für die Kante $e$, $\{V,V_1\}$ für die Kante $e_1'$ und  $\{V,V_2\}$ für die Kante $e_2'$ hinzufügt,
\item und bei der Beschreibung für die Kanten-Flächen Inzidenz  man $\{e,e_1,e_1'\}$ für die Fläche $F'$ und $\{e,e_2,e_2'\}$ für die Fläche $F''$ hinzugefügt wird.
 \end{itemize}
 \end{comment}
Falls aber $X_2(e_1)\cap X_2(e_2)=\emptyset$ ist, so gibt es zusätzliche Flächen $F_1'F_2'$, sodass sich folgende Relationen ergeben: 
\begin{itemize}
\item $(X_0(e_1),X_0(e_2))=(\{V,V_1\},\{V,V_2\}),$
\item $((F_1,F_2,\ldots F_2',F_1'\ldots))$ ist der zu V zugehörige Schirm,
\item $(X_2(e_1),X_2(e_2))=(\{F_1,F_2\},\{F_1',F_2'\}).$
%\item $(X_1(F_1^1),X_1(F_1^2),X_1(F_2^1),X_1(F_2^2))=(\{e_1,e_1,e_2\},\{e_2,e_3,e_4\},\{e_1,e_2,e_5\})$ 
\end{itemize}

\begin{figure}[H]
\begin{center}
\includegraphics[viewport=2cm 23.5cm 5cm 26.5cm]{butt4}
\end{center}
\caption{Ausschnitt einer Sphäre}
\end{figure}

Die Sphäre $\beta(X)_{e_1,e_2}$, welche durch das ordinale Symbol $\mu(\beta(X)_{e_1,e_2})$ beschrieben wird, erhalten wir, indem eine neue Ecke $V'$, Kanten $e,e_1',e_2'$ und Flächen $F',F''$ einführt werden und beim ordinale Symbol $\mu((X,<))$

 \begin{itemize}
 \item die Eckanzahl um 1, die Flächenanzahl um $2$ und die Kantenanzahl um $3$ erhöht wird 
 %\item die Einträge $X_0(e_1)$ und $X_0(e_2)$ löscht,
 \item an der Stelle $F_1^1$ beim Eintrag $X_0(F_1^1)$ die Kante $e_1$ durch $e_1'$ ersetzt wird,
 \item an der Stelle $F_2^1$ beim Eintrag $X_0(F_2^1)$ die Kante $e_2$ durch $e_2'$ ersetzt wird,
 %\item an der Stelle $F$ den Eintrag $X_1(F)$ durch $\{e_1^1,e_2^1,e_5\}$ ersetzt,
 \item an jeder Stelle $i$ aus $M$ in der $V$ in $X_0(i)$ vorkommt,$V$ durch $V'$ ersetzt wird,  
\item  bei der Beschreibung  für die Ecken-Kanten-Inzidenz die Menge $\{V,V'\}$ für die Kante $e$, die Menge $\{V,V_1\}$ für die Kante $e_1'$ und $\{V,V_2\}$ für die Kante $e_2'$ hinzufügt wird,
%\item und bei der Beschreibung für die Kanten-Flächen Inzidenz fügt man $\textcolor{red}{X}_1(F')=\{e,e_1^1,e_1^2\}$ und $\textcolor{red}{X}_1(F'')=\{e,e_2^1,e_2^2\}$ hinzu
\item und bei der Beschreibung für die Kanten-Flächen Inzidenz  die Menge $\{e,e_1,e_1'\}$ für die Fläche $F'$ und die Menge $\{e,e_2,e_2'\}$ für die Fläche $F''$ hinzufügt wird.
 \end{itemize}
  Klarerweise sind Schmetterlings-Entfernung und Schmetterlings-Erweiterung invers zueinander.
 \end{definition}
\subsection{Kantendrehung}
Im Vergleich zu den zuvor vorgestellten Prozeduren ist die Kantendrehung eine Manipulation der Sphäre, die die Anzahl der Ecken, Kanten und Flächen unberührt lässt. In Kapitel \ref{edget} werden wir die Eigenschaften dieser Operation genauer betrachten. Zunächst geben wir hier aber nur die Definition und ein Beispiel an.
\begin{definition}
Sei $X$ eine vertex-treue Sphäre. Eine Kante $e\in X_1$ heißt  \emph{drehbar}, falls es keine Kante $e'\in X_1\setminus \{e\}$ mit $X_0(e')=X_0(X_2(e))-X_0(e)$ gibt.
\end{definition}

\begin{definition}
Sei $X$ eine vertex-treue Sphäre, $\xi$ der zugehörige Flächen-Träger und $e$ eine drehbare Kante in $X$. Dann definieren wir die durch die Kantendrehung $e$ entstandene Sphäre $X^e$ durch den Flächen-Träger $\xi \Delta \Pot_3(X_0(X_2(e))).$
\end{definition}
\begin{figure}[H]
\begin{center}
\includegraphics[viewport=5cm 22.5cm 5cm 26.5cm]{Image_Kantendrehung}
\end{center}
\caption{Kantendrehung}
\end{figure}
 Wir führen nun die Kantendrehung am Beispiel des $(6)^2$  durch, welcher durch den Flächen-Träger
\begin{align*}
\xi=\{&\{1,2,3\},\{1,3,4\},\{1,4,5\},\{1,5,6\},\{1,6,7\},\{1,2,7\},\\ 
&\{8,2,3\},\{8,3,4\},\{8,4,5\},\{8,5,6\},\{8,6,7\},\{8,2,7\}\}
\end{align*}
dargestellt wird. 
\begin{figure}[H]
\begin{center}
\includegraphics[viewport=12cm 11cm 18cm 18cm,scale=0.85]{Double6gon}
\end{center}
\caption{Doppel-6-Gon}
\end{figure}
Durch näheres Betrachten ist zu erkennen, dass alle Kanten des Doppel-6-Gons drehbar sind. Bis auf Isomorphie gibt es jedoch nur zwei Kanten in dem Doppel-6-Gon, nämlich 
\begin{itemize}
\item Kanten, die zu zwei Ecken vom Grad 4 inzident sind und
\item Kanten, die zu einer Ecke vom Grad 6 und zu einer Ecke vom Grad 4 inzident sind.
\end{itemize} 
Die Kante $e$, die zu den Ecken 3 und 4 inzident ist, gehört zu den ersteren Kanten und durch Drehen dieser Kante wird der Flächen-Träger 
\begin{align*}
\xi=\{&\{1,2,3\},\{1,3,8\},\{1,4,5\},\{1,5,6\},\{1,6,7\},\{1,2,7\},\\ 
&\{8,2,3\},\{8,1,4\},\{8,4,5\},\{8,5,6\},\{8,6,7\},\{8,2,7\}\}
\end{align*}
mit der zugehörigen Sphäre ${((6)^2)}^e$ konstruiert.
\begin{figure}[H]
\begin{center}
\includegraphics[viewport=0cm 20.5cm 9cm 27cm,scale=0.9]{Edgeturn2}
\end{center}
\caption{Kantendrehung am $(6)^2$ }
\end{figure}
 Durch Drehen der Kante $e',$ die zu den Ecken 1 und 2 inzident ist, wird der Flächen-Träger 
\begin{align*}
\xi=\{&\{1,3,7\},\{1,3,8\},\{1,4,5\},\{1,5,6\},\{1,6,7\},\{2,3,7\},\\ 
&\{8,2,3\},\{8,1,4\},\{8,4,5\},\{8,5,6\},\{8,6,7\},\{8,2,7\}\}
\end{align*}
erzeugt und somit die Sphäre ${((6)^2)}^{e'}$ konstruiert.
\begin{figure}[H]
\begin{center}
\includegraphics[viewport=0cm 19cm 9cm 27cm,scale=0.85]{Edgeturn}
\end{center}
\caption{Kantendrehung am $(6)^2$ }
\end{figure}

Beachte, die obigen Sphären sind nicht isomorph und enthalten beide nicht drehbare Kanten.

\section{Der Flächengraph}
\textbf{benötigte Vorkenntnisse}\\
$\fbox{
\parbox{14cm}{\begin{itemize} 
\item elementare Eigenschaften simplizialer Fl\"achen
\item Manipulation simplizialer Fl\"achen
\end{itemize}
}}$\\\\
In diesem Abschnitt der Arbeit wird der Flächengraph einer simplizialen Fläche behandelt. Genauer gesagt wird thematisiert, wie viel Informationen die Kanten-Flächen-Inzidenzen in einer simplizialen Fläche zur Beschreibung dieser liefern können. Deshalb wird im ersten Teil des Kapitels eine Einleitung in die Thematik der Flächengraphen gegeben. Das Hauptresultat des ersten Abschnittes ist jedoch das Aufstellen einer Normalform der Flächen-Inzidenz Matrix einer Sphäre mit einer 2- oder 3-Taille. Im zweiten Teil widmen wir uns dann den in Kapitel \ref{manipulation} vorgestellten Prozeduren zum Manipulieren einer Sphäre. Genauer untersuchen wir, wie sich eine Manipulation der Sphäre auf den zugehörigen Flächengraph auswirkt.  
Wir nehmen zur Vereinfachung an, dass die Flächenmenge einer simplizialen Fläche mit $n$ Flächen  durch $\{1,\ldots,n\}$ gegeben ist.
\subsection{Grundlagen}
\begin{definition}
Sei $X$ eine Sphäre. Wir definieren den Flächengraph $G_X=(V,E)$ von $X$ durch die Knotenmenge $V=X_2$ und die Kantenmenge $E=X_1.$ Zwei Knoten $F,F'$ des Graphen sind adjazent, falls es eine Kante $e\in X_1$ gibt, die $X_2(e)=\{F,F'\}$ erfüllt. 
\end{definition}
\begin{bemerkung}
Da wir in diesem Abschnitt nur geschlossene simpliziale Flächen betrachten, sind die zugehörigen Graphen einfach. 
\end{bemerkung}
\begin{bsp}
Der Flächengraph des Tetraeders bildet einen vollständigen Graphen mit vier Knoten.
\begin{figure}[H]
\begin{center}
\includegraphics[viewport=1.5cm 20.5cm 20cm 23cm]{Image_FaceGraphTetraeder}
\end{center}
\caption{Flächengraph des Tetraeders}
\end{figure}
\end{bsp}
\begin{bemerkung}
Abgesehen vom Tetraeder ist, hat jede Sphäre einen  Flächengraph, der $3-$färbbar ist. Die Flächengraphen von Sphären, die ausschließlich gerade Eckengrade besitzen sind sogar 2-färbbar.
\end{bemerkung}
\begin{definition}
Sei $X$ eine Sphäre. Dann definieren wir die Matrix 
$F_X\in \{0,1\}^{n \times n}$ durch
\[
F_{i,j}=
\Biggl\{
\begin{tabular}[l]{lcr}
1,&\textcolor{black}{falls $\{i,j\}\in X_2(X_1)$} \\
0,& sonst\\
\end{tabular}
\] und nennen $F_X$ die Flächen-Inzidenz Matrix. 
\end{definition}
\begin{bsp}
Für einen Tetraeder erhalten wir die Flächen-Inzidenz-Matrix  durch  
\[
F_T=
\left( \begin{array}{rrrrrrrr}
0 & 1 & 1 & 1 \\ 
1 & 0 & 1 & 1 \\
1 & 1 & 0 & 1 \\
1 & 1 & 1 & 0  
\end{array}
\right)
\]

\end{bsp}

\begin{bemerkung}
\begin{itemize}
\item In jeder Spalte und Zeile der Flächen-Inzidenz-Matrix einer vertex-treuen simplizialen Fläche befinden sich genau 3 Einsen. 
\item Die Flächen-Inzidenz-Matrix einer simplizialen Fläche  ist symmetrisch.
\item $\lambda =3$ ist ein Eigenwert der Flächen-Inzidenz-Matrix einer vertex-treuen Sphäre. 
\end{itemize}
\end{bemerkung}
\begin{lemma}
Seien $X$ und $Y$ zwei isomorphe Sphären. Dann existiert eine Permutationsmatrix $P\in \{0,1\}^{n \times n}$ so, dass 
\[
F_X=PF_YP^{-1}
\] 
\end{lemma}
\begin{proof}
Dieser Zusammenhang ist dem Skript \emph{Simplicial Surfaces of Congruent Triangles} zu entnehmen.
\end{proof}
\begin{comment}
\begin{proof}
Sei $\alpha:X \to Y $ ein Isomorphismus von $X$ nach $Y$. Dieser induziert eine bijektive Abbildung $\beta :X_2\to Y_2$, wobei $X_2=Y_2=\{1,\ldots,n\}$ gilt. Mithilfe der Abbildung $\beta,$ kann die Permutationsmatrix $P\in \{0,1\}^n$ mit
\[
P_{i,j}=
\Biggl\{
\begin{tabular}[l]{lcr}
1,&\textcolor{black}{$\beta(i)=j$} \\
0,& sonst\\
\end{tabular}
\]
konstruiert werden.
Dies liefert die obige Behauptung, denn es gilt
\[
(PM_YP^{-1})_{ij}=(M_Y)_{\beta(i),\beta(j)}=(M_X)_{i,j}
\] .
\end{proof}
\end{comment}
\begin{bemerkung}
Die Umkehrung ist jedoch nur richtig, wenn wir uns auf den Fall $\chi(X)=2$ beschränken.  Dies wird hier jedoch nicht ausgeführt. 
\end{bemerkung}
In Kapitel \ref{Grundlagen} haben wir bereits Sphären mit 2-Taillen kennengelernt. Die Flächen-Inzidenz Matrix solcher Sphären lässt sich als eine Blockmatrix schreiben, was im Folgenden skizziert wird.
\begin{definition}
Für $k\leq l$ und $k\leq m$ ist die Matrix $I^{l,m}_k\in \{0,1\}^{l \times m}$ definiert als
\[
\left( 
\begin{array}{cccc} 
  I_k & 0_{k,m-k} \\
  0_{l-k,k} & 0_{l-k,m-k}\\
\end{array} 
\right).
\]
\end{definition}
Mithilfe dieser Matrix können wir die oben erwähnte Normalform für die Flächen-Inzidenz Matrix von Sphären mit 2-Taillen aufstellen.
\begin{satz}\label{mat2w}
Sei $X$ eine Sphäre mit einer 2-Taille. Dann gibt es eine Permutationsmatrix $P\in \{0,1\}^{n\times n}$ so, dass die Flächen-Inzidenz Matrix $F_X$ sich in die Gestalt 
\[
PF_XP^{-1}=
\left[ 
\begin{array}{c|c} 
  A & I^{k,n-k}_2 \\ 
  \hline 
  I^{n-k,k}_2 & B 
\end{array} 
\right]
\] 
bringen lässt, wobei $k$ die Anzahl der Flächen in der 2-Taillen Komponente $M_1$ ist und $A,B$ symmetrische Matrizen sind.
\end{satz}
\begin{proof}
Sei $W=(e_1,e_2)$ für Kanten $e_1,e_2\in X_1$ eine 2-Taille in $X.$ Dann gibt es die zu $W$ zugehörigen 2-Taillen Komponenten $M_1,M_2\subseteq X_2$ mit $M_1=\{i_1,\ldots,i_k\}$ und $M_2=\{j_1,\ldots,j_{n-k}\}$. Zudem seien ohne Einschränkung die Flächen $i_1\in M_1$ und $j_1\in M_2$ inzident zu $e_1$ und die Flächen $i_2\in M_1$ und $j_2\in M_2$ durch $e_2$ verbunden. Dann werden die folgenden Zeilen und Spalten der Matrix $F_X$ vertauscht:
\begin{itemize}
\item Die $i_1-te$ Zeile wird mit der ersten Zeile und die  $i_1-$te Spalte mit der ersten Spalte vertauscht.
\item Die $i_2-te$ Zeile wird mit der zweiten Zeile und die $i_2-$te Spalte mit der zweiten Spalte vertauscht.
\item Die $j_1-te$ Zeile wird mit der $k+1$-ten Zeile und die $j_1-$te Spalte mit der $k+1$-ten Spalte vertauscht.
\item Die $j_2-te$ Zeile wird mit der $k+2$-ten Zeile und die $j_2-$te Spalte mit der $k+1$-ten Spalte vertauscht.
\item Falls $i\in M_2$ für $3\leq i \leq k$ ist, dann existiert ein $k+3\leq j\leq n$ mit $j \in M_1.$ 
Wir tauschen dann die $i$-te Zeile mit der $j$-ten Zeile und die $i$-te Spalte mit der $j$-ten Spalte.
\end{itemize} 
So erhalten wir eine Flächen-Inzidenz Matrix, in der die ersten $k$ Zeilen bzw. Spalten zu Flächen in $M_1$ und die restlichen $n-k$ Zeilen bzw. Spalten zu Flächen in $M_2$  gehören. Bei genauerer Betrachtung ist die gewünschte Gestalt  bei der durch Vertauschen der Zeilen und Spalten entstandenen Matrix zu erkennen. Da diese Gestalt ausschließlich durch simultanes Vertauschen der Zeilen bzw. Spalten der Matrix $F_X$ erzielt wurde, existiert also eine Permutationsmatrix $P\in\{0,1\}^{n\times n}$, sodass die Multiplikation von links und die Multiplikation des Inversen von rechts  die skizzierte Form hervorbringt.
\end{proof}

\begin{satz}
Sei $X$ eine Sphäre mit einer 3-Taille. Dann gibt es eine Permutationsmatrix $P\in \{0,1\}^{n \times n}$ so, dass $F_X$ sich auf die Gestalt 
\[
PF_XP=
\left[ 
\begin{array}{c|c} 
  A & I^{l,n-l}_3 \\ 
  \hline 
  I^{n-l,l}_3 & B 
\end{array} 
\right]
\] 
bringen lässt, wobei $A\in \{0,1\}^{l}$ und $B\in \{0,1\}^{n-l}$ symmetrische Matrizen sind und $k$ die Anzahl der Flächen inder 3-Taillen Komponente $M_1$ ist.
\end{satz}
\begin{proof}
Diese Aussage wird analog zum Beweis von \Cref{mat2w} geführt.
\end{proof}

\subsection{Manipulation des Flächengraphen }
Wie bereits erwähnt studieren wir hier den Zusammenhang der Flächengraphen von Sphären, die durch eine Manipulation der Ecken, Kanten und Flächen auseinander hervorgehen. Es sei angemerkt, dass die ausführliche Definition der in Kapitel \ref{manipulation} vorgestellten Prozeduren hier eher nebensächlich ist. Deswegen werden die Voraussetzungen zum Anwenden dieser eher oberflächlich aufgestellt. Diese können jedoch bei Bedarf in Kapitel \ref{manipulation} nachgelesen werden. Zur Vereinfachung nehmen wir hier zur Annahme $X_2  =\{1,\ldots,n\}$ zusätzlich an, dass $X$ eine Sphäre mit zugehörigem Flächengraph $G_X$ und zugehöriger Flächen-Inzidenz Matrix $F_X$ ist.
\begin{enumerate}
\item Schmetterlings-Entfernung
\begin{itemize}
\item Sei $e$ eine Kante in $X,$ sodass die Schmetterlings-Entfernung durchführbar ist. Dann ist $X_2(e)=\{i',j'\}$ für geeignete $i',j'.$ Es gibt dann genau zwei Flächen $i_1,i_2\in X_2,$ die zu $i'$ und genau zwei Flächen $j_1,j_2\in X_2,$ die zu $j'$ adjazent sind. 
\begin{figure}[H]
\begin{center}
\includegraphics[viewport=2cm 21.5cm 5cm 26.5cm]{Image_butfac}
\end{center}
\caption{Ausschnitt der Sphäre $X$}
\end{figure}
Zudem sei $F\in \{0,1\}^{n\times n}$ die Matrix, die durch
 \[
F_{i,j}=
\Biggl{\{\begin{tabular}[l]{lcr}
1,&\textcolor{black}{$(i,j)\in \{(i_1,i_2),(i_2,i_1)\}$} \\
1,&\textcolor{black}{$(i,j)\in \{(j_1,j_2),(j_2,j_1)\}$} \\
$(F_{X})_{i,j}$,& sonst\\
\end{tabular}}
\] definiert ist. Dann geht die Flächen-Inzidenz Matrix der Sphäre ${}^e\beta(X)$ durch Streichen der Zeilen $i',j'$ und der Spalten $i',j'$ aus $F$ hervor.\\
Beim Betrachten des zugehörigen Flächengraphen $G_X$ und den zu den obigen Flächen zugehörigen Knoten in $G_X$ ergibt sich folgender Zusammenhang:
\begin{figure}[H]
\begin{center}
\includegraphics[viewport=2cm 18.cm 19cm 22.5cm]{Image_fg4}
\end{center}
\caption{Ausschnitt des Flächengraphen der Sphäre $X$}
\end{figure}
Zunächst werden die zu $i'$ und $j'$ zugehörigen Knoten und damit auch die inzidenten Kanten aus dem Graphen entfernt. 
\begin{figure}[H]
\begin{center}
\includegraphics[viewport=2cm 18cm 19cm 22.5cm]{Image_fg5}
\end{center}
\caption{Ausschnitt eines aus dem Flächengraph der Sphäre $X$ konstruierten Graph}
\end{figure}
Daraufhin werden die Kanten $\{i_1,i_2\}$ und $\{j_1,j_2\}$ hinzugefügt, um so den Flächengraph der Sphäre ${}^e\beta(X)$ zu konstruieren. 
\begin{figure}[H]
\begin{center}
\includegraphics[viewport=2cm 18cm 19cm 22.5cm]{Image_fg6}
\end{center}
\caption{Ausschnitt des Flächengraphen der Sphäre ${}^e\beta(X)$}
\end{figure}
\end{itemize}
\item Schmetterlings-Erweiterung
\begin{itemize}
\item Seien $e_1,e_2$ Kanten in $X,$ für die eine Schmetterlings-Erweiterung durchführbar ist. Dann existieren geeignete Flächen, sodass $X_2(e_1)=\{i_1,i_2\}$ und $X_2(e_2)=\{j_1,j_2\}$ gilt. 
\begin{figure}[H]
\begin{center}
\includegraphics[viewport=2cm 23.5cm 4cm 26.5cm]{Image_butint}
\end{center}
\caption{Ausschnitt der Sphäre $X$}
\end{figure}
Zudem sei $F\in \{0,1\}^{n\times n} $ die Matrix, die durch   
\[
F_{i,j}=
\Biggl{\{\begin{tabular}[l]{lcr}
0,&\textcolor{black}{$(i,j)\in \{(i_1,i_2),(i_2,i_1)\}$} \\
0,&\textcolor{black}{$(i,j)\in \{(j_1,j_2),(j_2,j_1)\}$} \\
$(F_{X})_{i,j}$,& sonst\\
\end{tabular}}
\]
definiert ist. Mithilfe der Vektoren $v\in \{0,1\}^n $ mit 
\[
v_{i}=
\Biggl{\{\begin{tabular}[l]{lcr}
1,& $i=i_1,i_2$\\
0,& sonst\\
\end{tabular}}
\]
und $w\in \{0,1\}^n $ mit 
\[
w_{i}=
\Biggl{\{\begin{tabular}[l]{lcr}
1,& $i=j_1,j_2$\\
0,& sonst\\ 
\end{tabular}}
\]
erhalten wir die Flächen-Inzidenz Matrix der Sphäre $\beta_{e_1,e_2}(X)
,$ indem wir die Blockmatrix 
\[
\left[ 
\begin{array}{c|cc} 
  F & v& w \\ 
  \hline 
  v^{tr}& 0 &1 \\
  w^{tr} &1 &0  \\
\end{array} 
\right]
\]
zusammensetzen. Betrachten wir nun die Auswirkungen dieser Manipulation auf den Flächengraph. 
\begin{figure}[H]
\begin{center}
\includegraphics[viewport=2cm 18cm 19cm 22.5cm]{Image_fg6}
\end{center}
\caption{Ausschnitt des Flächengraphen von $X$}
\end{figure}
Im ersten Schritt werden $\{i_1,i_2\}$ und $\{j_1,j_2\}$ aus der Menge der Kanten entfernt.
\begin{figure}[H]
\begin{center}
\includegraphics[viewport=2cm 18cm 19cm 22.5cm]{Image_fg5}
\end{center}
\caption{Ausschnitt eines aus dem Flächengraph der Sphäre $X$ konstruierten Graph}
\end{figure}
Dann werden neue Knoten $i'$ und $j'$ eingeführt. Außerdem müssen die Kanten 
\[
\{i',i_1\},\{i',i_2\},\{j',j_1\},\{j',j_2\},\{i',j'\}
\]
ergänzt werden, um so den Flächengraph der Sphäre $\beta_{e_1,e_2}(X)$ zu erhalten. 
\begin{figure}[H]
\begin{center}
\includegraphics[viewport=2cm 18cm 19cm 22.5cm]{Image_fg4}
\end{center}
\caption{Ausschnitt des Flächengraphen der Sphäre $\beta_{e_1,e_2}(X)$}
\end{figure}
In den Flächengraphen der Sphären ist also der Zusammenhang zwischen der Schmetterlings-Erweiterung und Schmetterlings-Entfernung ebenfalls erkennbar. 
\end{itemize} 
\item Kantendrehung
\begin{itemize} 

\item Sei $e \in X_1$ eine drehbare Kante und $X_2(e)=\{i_1,i_2\}.$ Dann existieren geeignete Flächen $j_1,j_2,j_3,j_4,$ die 

\begin{align*}
&X_2(X_1(i_1))-\{i_1,i_2\}=\{j_1,j_3\},\\
&X_2(X_1(i_2))-\{i_1,i_2\}=\{j_2,j_4\},\\
\end{align*}
und 
\begin{align*}
&X_0(e)\cap X_0(j_1) \cap X_0(j_2)\neq \emptyset\\
 &X_0(e)\cap X_0(j_3) \cap X_0(j_4)\neq \emptyset\\
\end{align*}
erfüllen.
\begin{figure}[H]
\begin{center}
\includegraphics[viewport=2cm 21.5cm 5cm 26.5cm]{facgraKan}
\end{center}
\caption{Ausschnitt der Sphäre $X$}
\end{figure}
 Die Flächen-Inzidenz-Matrix $F_{X^e}\in \{0,1\}^{n\times n}$ erhalten wir bis auf Äquivalenz durch
\[
{F_{X^e}}_{i,j}=
\Biggl{\{\begin{tabular}[l]{lcr}
1,&\textcolor{black}{$(i,j)\in \{(i_1,j_2),(j_2,i_1)\}$} \\
0,&\textcolor{black}{$(i,j)\in \{(i_1,j_3),(j_3,i_1)\}$} \\
1,&\textcolor{black}{$(i,j)\in \{(i_2,j_3),(j_3,i_2)\}$} \\
0,&\textcolor{black}{$(i,j)\in \{(i_2,j_2),(j_2,i_2)\}$} \\
$(F_{X})_{i,j}$,& sonst\\
\end{tabular}}
\]
Betrachten wir nun wieder den zugehörigen Flächengraph $G_X.$ 
\begin{figure}[H]
\begin{center}
\includegraphics[viewport=2cm 18cm 19cm 22.5cm]{Image_fg8}
\end{center}
\caption{Ausschnitt des Flächengraphen der Sphäre $X$}
\end{figure}
Beim Drehen der Kanten $e$ werden die Kanten $\{i_1,j_3\}$ und $\{i_2,j_2\}$ aus dem Graphen entfernt und die Kanten $\{i_2,j_2\}$ und $\{i_2,j_2\}$ zu der Kantenmenge hinzugefügt. Auf diese Art und Weise erhalten wir den Flächengraph der Sphäre $X^e.$ 
\begin{figure}[H]
\begin{center}
\includegraphics[viewport=2cm 18cm 19cm 22.5cm]{Image_fg9}
\end{center}
\caption{Ausschnitt des Flächengraphen der Sphäre $X^e$}
\end{figure}
\end{itemize}
\item Tetraeder-Erweiterung
\begin{itemize}  
\item Seien $i$ eine Fläche in $X$ und die Nachbar-Flächen von $i$ durch geeignete $j,k,l$ gegeben.
\begin{figure}[H]
\begin{center}
\includegraphics[viewport=2cm 23.cm 4cm 27cm]{Image_tetfac}
\end{center}
\caption{Ausschnitt der Sphäre $X$}
\end{figure}
Weiterhin sei $F$ die Matrix, die durch Streichen der $i$-ten Zeile und Spalte aus $F_X$ hervorgeht. Dann ergibt sich die Flächen-Inzidenz Matrix von $T^i(X)$ durch die Blockmatrix
\[
\left[ 
\begin{array}{c|ccc} 
  F & e_j& e_k &e_l \\ 
  \hline 
  {e_j}^{tr} & 0 & 1 & 1  \\
  {e_k}^{tr} & 1 & 0 & 1 \\
  {e_l}^{tr} & 1 & 1 & 0 \\
\end{array} 
\right]
\]
wobei $e_j,e_k,e_l$ die zugehörigen Einheitsvektoren in $n-1$ Eintragen sind. 
\item Sei $V\in X_0$ eine Ecke vom Grad 3 in $X$ mit $X_2(V)=\{i_1,i_2,i_3\}.$ Dann ist $X_2(X_1(X_2(V)))-X_2(V)=\{j,k,l\}$ für geeignete $j,k,l$. Diese bilden die Nachbarn der Flächen $i_1,i_2,i_3.$ Genauer nehmen wir an, dass $i_1$ zu $j$, $i_2$ zu $k$ und $i_3$ zu $l$ benachbart ist.
\begin{figure}[H]
\begin{center}
\includegraphics[viewport=2cm 23.cm 4cm 27cm]{Image_tetfac2}
\end{center}
\caption{Ausschnitt der Sphäre $X$}
\end{figure}  
Sei $F\in \{0,1\}^{n-3\times n-3}$ die Matrix, die durch Streichen der $i_1-ten,i_2-ten$ und $i_3-ten$ Zeilen und Spalten  aus $F_X$ entsteht und $v\in \{0,1\}^{n-3}$ der Vektor mit
\[
v_{i}=
\Biggl{\{\begin{tabular}[l]{lcr}
1,&\textcolor{black}{$i\in \{j,k,l\}$} \\
0& sonst\\
\end{tabular}}
\] Dann ist $F_{T_V(X)}$ durch 
\[
\left[ 
\begin{array}{c|ccc} 
  F & v \\ 
  \hline 
  {v}^{tr} & 0\\
\end{array} 
\right]
\]
gegeben.
Die Auswirkungen auf den Flächengraph werden wir zu einem späteren Zeitpunkt noch genauer formulieren. Dies wird im Kontext der Multi-Tetraeder geschehen.
\end{itemize}
\end{enumerate}



 \section{Kantendrehungen}\label{edget}
Dieses Kapitel soll als Wiederholung der Resultate der Bachelorarbeit "Manipulation diskreter simplizialer Flächen"  dienen und zugleich einen anderen Zugang zu der Thematik der Kantendrehungen liefern. Dort wurde der Zugang durch die sogenannten Mender- und Cutteroperatoren ermöglicht, wohingegen hier versucht wird, die symmetrische Differenz zur Durchführung der Kantendrehungen auszunutzen. Deshalb werden hier zunächst die einführenden Definitionen umformuliert und die daraus entstehenden Resultate ohne Beweis zusammengefasst, um so die Transitivität der Kantendrehungen auf der Menge der Isomorphieklassen Sphären ohne 2-Taillen unter strikteren Einschränkungen zu beweisen. 
 \subsection{Transitivität der Kantendrehung}
\begin{comment} 
 \begin{definition}
 Sei $(X,<)$ eine geschlossene simpliziale Fläche mit paarweise verschiedenen Ecken $V_1\ldots V_4$, Kanten $e_1,\ldots,e_5$ und Flächen $F_1,F_2$ in $X$,
  die folgendes erfüllen:
\begin{itemize}
\item $X_2(e_1))=\{F_1,F_2\}$
 \item $X_0(X_2(e_1))=\{V_1,V_2,V_3,V_4\}$
\item $deg(V)\neq 2$ für  alle $V\in X_0(e_1)=\{V_2, V_4\}$
\item $(X_0(e_2),X_0(e_3),X_0(e_4),X_0(e_5))=(\{V_1,V_2\},\{V_2,V_3\},\{V_3,V_4\},\{V_1,V_4\})$
\item $(X_1(F_1),X_1(F_2))=(\{e_1,e_2,e_5\},\{e_1,e_3,e_4\})$
\end{itemize} 
\begin{figure}[H]
\begin{center}
\includegraphics[viewport=13cm 19cm 5cm 22cm]{Image_ButterflyDeletion}
\end{center}
\caption{Dreieck}
\end{figure}
  Dann definiert man die durch die Kantendrehung $e_1$ entstandene simpliziale Fläche $X^{e_1}$ durch das ordinale Symbol $\mu (X^{e_1})$, welches entsteht, wenn man beim ordinalen Symbol $\mu((X,<))$
 \begin{itemize}
 \item an der Stelle $e_1$ den Eintrag $X_0(e_1)$ durch $\{V_1,V_3\}$ ersetzt,
 \item an der Stelle $F_1$ den Eintrag $X_1(F_1)$ durch $\{e_1,e_2,e_3\}$ ersetzt, 
 \item und an der Stelle $F_2$ den Eintrag $X_1(F_2)$ durch den Eintrag $\{e_1,e_4,e_5\}$ ersetzt.
 \begin{figure}[H]
\begin{center}
\includegraphics[viewport=18cm 12cm 5cm 17cm]{Image_Edgeturn}
\end{center}
\caption{Dreieck}
\end{figure}
 \end{itemize}
  Für $X^{e_1}$ gilt dann 
 \[
X_i =X^{e_1}_i \text{ für i=0,1,2}
 \]
 und 
 \[
\chi (X)=\chi(X^{e_1}). 
 \]
 \end{definition}
\end{comment}
Für die weitere Untersuchung der Kantendrehung, muss zunächst das iterative Anwenden der Kantendrehung definiert werden.

\begin{definition}
Sei $X$ eine Sphäre in $X$ und $e_1,\ldots,e_n$ Kanten in $X$. Wir nennen $E=(e_1,\ldots,e_n)$ eine \emph{drehbare Kantensequenz}, falls Folgendes erfüllt ist: 
\begin{itemize}
\item Die Kante $e_1$ ist eine drehbare Kante in $X.$ 
\item Für alle $1< i < n$ ist die Kante $e_{i+1}$ eine drehbare Kante in 
\[
X^{(e_1,\ldots,e_i)}:=(X^{(e_1,\ldots,e_{i-1})})^{e_i}.
\] 
\end{itemize}
Wir nennen $X^E$ die durch die Kantensequenz $E$ entstandene Sphäre. 
\end{definition}

\begin{bemerkung}
Kantendrehungen sind nicht kommutativ. Das heißt im Allgemeinen gilt für drehbare Kanten $e_1,e_2$ in $X$ zwischen $X^{(e_1,e_2)}$ und $X^{(e_2,e_1)}$ keine Isomorphie. 

\end{bemerkung}

Wie oben schon erwähnt wurden die Kantendrehungen in der Bachelorarbeit "Manipulation diskreter simplizialer Flächen"  \,
 allgemeiner formuliert. Es wurde zugelassen, dass Kantendrehungen auch an nicht vertex-treuen Sphären durchgeführt werden konnten. Für die Formulierung des Hauptresultates der Bachelorarbeit bezeichnen wir diese Kantensequenzen mit \emph{allgemeinen Kantensequenzen}.
\begin{satz}
Sei $X$ und $Y$ Sphären mit $\vert X_2\vert=\vert Y_2\vert.$ Dann existiert eine allgemeine Kantensequenz $E$ in $X$ so, dass 
\[
X^E \cong Y
\]
ist. 
\end{satz}
Einen Beweis dieser Aussage haben wir bereits in der Bachelorarbeit gesehen. An dieser Stelle wird nun ein weiterer Beweis vorgestellt, der erlaubt die Voraussetzungen des Satzes schärfer zu formulieren. Doch hierfür wird zunächst noch etwas Vorarbeit benötigt.

Im Folgenden wird eine vertex-treue Sphäre $X$ betrachtet.
\begin{bemerkung}
Diese Bemerkung beschreibt, wie sich die Orientierungen von Flächen in dem Schirm einer Ecke in $X$ verhalten.
 Seien deshalb $V\in X_0$ und $U(V)=((F_1,\ldots,F_n))$ der für Flächen $F_1,\ldots,F_n$ zu $V$ zugehörige Schirm. Im Folgenden nennen wir ein Tupel $(F_1,\ldots,F_n)\in U(V)$ ein \emph{Schirm-Tupel} von $V$. Weiterhin gibt es $V_1,\ldots,V_n\in X_0,$ die 
\[
X_0(F_i)=\{V,V_i,V_{i+1}\}
\] 
für $i=0,\ldots,n-1$ und 
\[
X_0(F_n)=\{V,V_{1},V_{n}\}
\] erfüllen.
\begin{figure}[H]
\begin{center}
\includegraphics[viewport=5cm 23cm 5cm 27cm]{bem}
\end{center}
%\caption{Kantendrehung}
\end{figure}
Dann gibt es Kanten $e_1,\ldots,e_n$ in $X,$ sodass wir 
\[
(V_1,e_1,V_2,e_2,\ldots,V_n,e_n)
\] 
als Ecken-Kanten-Pfad in $X$ erhalten. 
\begin{figure}[H]
\begin{center}
\includegraphics[viewport=5cm 23cm 5cm 27cm]{bem1}
\end{center}
%\caption{Kantendrehung}
\end{figure}
Wir bezeichnen die Orientierung einer Fläche $F\in X_2$ im Folgenden mit $o(F).$ Wir wollen nun Orientierungen der $F_i$ so angeben, dass die Orientierungen von je zwei benachbarten Flächen kohärent sind. 
Geben wir nun die Orientierung $o(F_1)=(V,V_1,V_2)$ vor, dann lässt sich, die Orientierung einer beliebigen Fläche in dem Schirm bestimmen. Diese ergibt sich nämlich durch $o(F_i)=(V,V_i,V_{i+1})$ bzw. $o(F_n)=(V,V_n,V_1).$ Die zwei von $V$ verschiedenen Ecken werden also in der Reihenfolge, wie sie auch im Ecken-Kanten Pfad vorkommen, gelesen. Beispielsweise ist die Orientierung von $F_4$ durch $(V,V_4,V_5)$ gegeben, da im Pfad $V_4$ vor $V_5 $ besucht wird. 
\begin{figure}[H]
\begin{center}
\includegraphics[viewport=5cm 23cm 5cm 27cm]{bem2}
\end{center}
%\caption{Kantendrehung}
\end{figure}
Analoges Vorgehen liefert uns ebenfalls im Fall $o(F_1)=(V,V_2,V_1)$ die Orientierung der restlichen Flächen des Schirmes. Diese ergibt sich nämlich durch $o(F_i)=(V,V_{i+1},V_{i})$ bzw. $o(F_n)=(V,V_1,V_n).$ Denn in diesem Fall lesen wir die beiden von $V$ verschiedenen Ecken in umgekehrter Reihenfolge. Beispielsweise ist die Orientierung von $F_3$ durch $(V,V_4,V_3)$ gegeben, da $V_4$ nach $V_3$ besucht wird.
\begin{figure}[H]
\begin{center}
\includegraphics[viewport=5cm 23cm 5cm 27cm]{bem3}
\end{center}
%\caption{Kantendrehung}
\end{figure}
Wir nennen den oben konstruierten Ecken-Kanten-Pfad einen \emph{Schirm-Pfad} von $V.$
Weiterhin nennen wir $\mathcal{O}_X=\{o(F)\mid F\in X_2\}$ eine Orientierung der Sphäre $X,$ falls die Orientierungen $o(F),o(F')\in \mathcal{O}$ von zwei benachbarten Flächen $F$ und $F'$ in $X$ kohärent sind. Falls $X$ aus dem Kontext klar ist, schreiben wir auch $\mathcal{O}$ für die Orientierung $\mathcal{O}_X.$ Da Sphären zusammenhängend sind, ist die  Orientierung einer Sphäre durch Angabe der Orientierung einer beliebigen Fläche der Sphäre eindeutig festgelegt. Durch eine Orientierung der Sphäre  entstehen zwei Arten von Schirm-Tupeln der Ecke $V\in X_0.$ Hierzu sei $U\in U(V)$ ein Schirm-Tupel von $V.$ Mit $i_1$ bezeichnen wir die Position von $F_1$ und mit $i_n$ die Position von $F_n$ in $U.$
\begin{itemize}
\item Sei
$(V,V_1,V_2)\in \mathcal{O}.$ Falls entweder $\{i_1,i_n\}\neq\{1,n\}$ und $i_1\geq i_n$ oder $(i_1,i_n)=(1,n)$ ist, dann nennen wir $U$ $\mathcal{O}$-positiv orientiert. Andernfalls nennen wir $\mathcal{O}$-negativ orientiert.
\item Sei $(V,V_2,V_1)\in \mathcal{O}.$ Falls entweder $\{i_1,i_n\}\neq \{1,n\}$ und $i_1\leq i_n$ oder $(i_1,i_n)=(n,1)$ ist, dann nennen wir $U$ $\mathcal{O}$-positiv orientiert. Andernfalls nennen wir  $\mathcal{O}$-negativ orientiert.
 \end{itemize}
Falls die Orientierung $\mathcal{O}$ im Kontext klar ist, heißt $U$ einfach nur positiv bzw. negativ orientiert.
Für $n=6$ zeigt die untenstehende Tabelle alle positiv und negativ orientierten Schirm-Tupel der Ecke $V$ im Fall $(V,V_1,V_2) \in \mathcal{O}$.\\
\begin{center}
\begin{tabular}{|c|c|}
\hline
$\mathcal{O}-$positiv orientiert & $\mathcal{O}-$negativ orientiert\\
\hline
$(F_1,\ldots,F_6)$&$(F_6,\ldots ,F_1)$\\
$(F_6,F_1,\ldots,F_5)$&$(F_1,F_6,\ldots ,F_2)$\\
$(F_5,F_6,F_1,\ldots,F_4)$&$(F_2,F_1,F_6,\ldots,F_3)$\\
$(F_4,F_5,F_6,F_1,F_2,F_3)$&$(F_3,F_2,F_1,F_6,F_5,F_4)$\\
$(F_3,F_4,F_5,F_6,F_1,F_2)$&$(F_4,F_3,F_2,F_1,F_6,F_5)$\\
$(F_2,F_3,F_4,F_5,F_6,F_1)$&$(F_5,F_4,F_3,F_2,F_1,F_6)$\\
\hline
\end{tabular}
\end{center}
\end{bemerkung}
  
\begin{bemerkung}
Seien $X$ eine vertex-treue Sphäre und $\mathcal{O}$ eine Orientierung von $X.$ Weiterhin seien $V_1,V_2$ zwei benachbarte Ecken in $X$ zusammen mit positiv orientierten Schirm-Tupeln $U_1\in U(V_1)$ und $U_2\in U(V_2).$ Für zwei benachbarte Flächen $F_1,F_2\in X_2$ mit 
\[
X_2(V_1)\cap X_2(V_2)=\{F_1,F_2\},
\]
erhalten wir den Zusammenhang, dass die beiden Flächen in $U_1$ und  $U_2$ in jeweils umgekehrter Reihenfolge auftauchen.\\
Genauer wird in diesem Zusammenhang Folgendes unter dem Auftauchen in umgekehrter Reihenfolge verstanden:\\ 
Bezeichnen wir die Position von $F_j$ in $U_l$ mit $i_j^l$ für $j,l=1,2,$ dann gilt:
\begin{align*}
i^1_1\leq i_2^1 \, oder \, (i_1^1,i^1_2)=(n_1,1)\Leftrightarrow i^2_1\geq i_2^2 \, oder \, (i_1^2,i^2_2)=(1,n_2),
\end{align*}
wobei $\deg(V_1)=n_1$ und $\deg(V_2)=n_2$ ist.
Zur Veranschaulichung betrachten wir für geeignete Flächen und Ecken folgenden Ausschnitt einer Sphäre. 
\begin{figure}[H]
\begin{center}
\includegraphics[viewport=5cm 21.5cm 5cm 27cm]{bem4}
\end{center}
%\caption{Kantendrehung}
\end{figure}
Sei nun $\mathcal{O}$ eine Orientierung von $X$ mit $(V_1,V_2,V_3)\in \mathcal{O}$, wobei $X_0(F_2)=\{V_1,V_2,V_3\}$ ist.
\begin{figure}[H]
\begin{center}
\includegraphics[viewport=5cm 21.5cm 5cm 27cm]{bem5}
\end{center}
%\caption{Kantendrehung}
\end{figure}

Dann erhalten wir durch 
\[
U_1=(F_1,F_2,F_3,F_4,F_5,F_6)
\]
bzw.
\[
U_2=(F_2,F_1,F_7,F_8,F_9,F_{10})
\]
zwei positiv orientierte Schirm-Tupel von $V_1$ bzw. $V_2.$ In diesen tauchen die Flächen $F_1,F_2$ in jeweils umgekehrter Reihenfolge auf. 
\end{bemerkung}
Zum tieferen Verständnis der oben eingeführten Definitionen betrachten wir als Beispiel einen Tetraeder, wobei an dieser Stelle auf die formale Definition verzichtet wird und uns die Definition durch die folgende Skizze genügt.
\begin{figure}[H]
\begin{center}
\includegraphics[viewport=4cm 23cm 5cm 27cm]{tetbem1}
\end{center}
%\caption{Kantendrehung}
\end{figure} 
Durch Angabe der Orientierung $o(F_1)=(V_2,V_4,V_3)$ erhalten wir
\[
\mathcal{O}=\{(V_2,V_4,V_3),(V_1,V_4,V_2),(V_1,V_2,V_3),(V_1,V_3,V_4)\}
\] 
als Orientierung des obigen Tetraeders.
 
\begin{figure}[H]
\begin{center}
\includegraphics[viewport=4cm 23cm 5cm 27cm]{tetbem}
\end{center}
%\caption{Kantendrehung}
\end{figure} 
Für zwei benachbarte Ecken, beispielsweise $V_1$ und $V_2,$ bilden $U_1=(F_3,F_4,F_2)$ und $U_2=(F_1,F_4,F_3)$ zwei positiv orientierte Schirm-Tupel in denen die zu beiden Ecken inzidenten Flächen $F_3$ und $F_4$ in jeweils umgekehrter Reihenfolge vorkommen.
\begin{lemma}\label{grad3}
Sei $X$ eine vertex-treue Sphäre und $V$ eine beliebige Ecke in $X,$ die $\deg(X)\geq 4$ erfüllt. Dann gibt es eine drehbare Kante in $X_1(V).$ 
\end{lemma}
\begin{proof}
Die Aussage wird per Widerspruch bewiesen und folgert dann, dass $X$ nicht orientierbar ist, also keine Sphäre sein kann.
Angenommen es existiert eine Ecke $V$ in $X$, sodass keine Kante in $X_1(V)$ drehbar ist. Dann gibt es für alle $e\in X_1(V)$ eine Kante $e'\in X_1\setminus{e},$ sodass 
\[
(X_0(X_2(e))-X_0(e))=X_0(e')
\]
ist.
Für $n\geq 4$ sei nun $U(V)=((F_1,\ldots,F_n))$ der Schirm von $V$ und $V_1,\ldots,V_5$ Ecken in $X,$ sodass  
\[
X_0(F_i)=\{V,V_i,V_{i+1}\}
\] 
für $i\in\{1,2,3,4\}$ ist.
\begin{figure}[H]
\begin{center}
\includegraphics[viewport=4cm 23cm 5cm 27cm]{beweis}
\end{center}
%\caption{Kantendrehung}
\end{figure} 
 Da $X$ vertex-treu ist, können wir die Kanten mit den inzidenten Ecken identifizieren. Da also $\{V,V_2\}$ und $\{V,V_4\}$ nicht drehbar sind, gibt es bereits die Kanten $\{V_1,V_3\}$ und $\{V_3,V_5\}$ in $X$. Für diese gibt es  Flächen $F^{1,3}_1,F^{1,3}_2,F^{3,5}_1,F^{3,5}_2\in X_2$ mit $X_2(\{V_1,V_3\})$ $=\{F^{1,3}_1,F^{1,3}_2\}$ und $X_2(\{V_3,V_5\})=\{F^{3,5}_1,F^{3,5}_2\}$. Wir wollen nun einen Flächen-Pfad ohne Wiederholung und eine Orientierung entlang dieses Pfades angeben, um dann den gewünschten Widerspruch zu erzeugen. Hierzu brauchen wir geeignete Schirm-Tupel der Ecken $V_1,V_3$ und $V_5.$ 
Da $X$ orientierbar ist, existiert eine Orientierung $\mathcal{O}_X$ mit $(V,V_1,V_2)\in \mathcal{O}_X.$
Den gewünschten Schirm-Tupel von $V_1$ erhalten wir durch folgende Überlegung: 
Aus obiger Voraussetzung wissen wir, dass $\{F_n,F_1,F_1^{1,3},F_2^{1,3}\}\subset X_2(V_1)$ ist und die Flächen $F_1$ und $F_2$ bzw. $F^{1,3}_1$ und $F^{1,3}_2$
benachbarte Flächen in $X$ sind. Da jedoch die $F_i$ und die $F_j^{1,3}$ im Allgemeinen nicht benachbart sind, gibt es also geeignete Flächen, sodass  
\[
U(V_1)=((F_n,F_1,F_a,\ldots,F_b,F^{1,3}_1,F^{1,3}_2,\ldots))
\]
den Schirm von $V_1$ bildet, woraus wir das negativ orientiere Schirm-Tupel
\[
U_1=(F_n,F_1,F_a,\ldots,F_b,F^{1,3}_1,F^{1,3}_2,\ldots)
\]  
konstruieren können.
Analog erhalten wir ebenfalls 
\[
U_5=(F_5,F_4,F_e\ldots,F_f,F^{3,5}_1,F^{3,5}_2,\ldots)
\] als positiv orientiertes Schirm-Tupel von $V_5$ in $X$. Da $\{F_2,F_3,F_1^{3,5},F_2^{3,5},F^{1,3}_1,F^{1,3}_2\}\subseteq X_2(V)$ ist,  erhalten wir wegen Bemerkung 0.2 ohne Einschränkung der Allgemeinheit für geeignete Flächen ein negativ-orientiertes Schirm-Tupel von $V_3$ durch
\[
U_3=(F_2,F_3,F_d,\ldots,F_c,F^{1,3}_2,F^{1,3}_1,\ldots,F_1^{3,5},F_2^{3,5},F_g,\ldots,F_h).
\]
%\begin{figure}[H]
%\begin{center}
%\includegraphics[viewport=4cm 20cm 5cm 27cm]{surf}
%\end{center}
%\caption{Kantendrehung}
%\end{figure}
Außerdem seien $V_i^{k,l}$ Ecken in $X,$ sodass für die obigen Flächen der Form $F_i^{k,l}$ die Gleichheit
\[
X_0(F^{k,l}_i)=\{V_k,V_l,V_i^{k,l}\}
\] für geeignete $i,j,k$ gilt. Also ist beispielsweise $X_0(F_1^{1,3})=\{V_1,V_3,V_1^{1,3}\}.$


%\begin{figure}[H]
%\begin{center}
%\includegraphics[viewport=5cm 18cm 5cm 27cm]{ma}
%\end{center}
%\caption{Kantendrehung}
%\end{figure}
Um nun den Pfad zu konstruieren, bewegen wir uns mithilfe der angegebenen Schirm-Tupel entlang der Schirme der jeweiligen Ecken. Zunächst bewegen wir uns auf dem Schirm von $V_1$ ausgehend von $F_1$ über die Flächen $F_a,\ldots,F_b$ zu den Flächen $F_1^{1,3},F_2^{1,3}$, wo wir dann auf den Schirm von $V_3$ wechseln können. Daraufhin gelangen wir über die Flächen $F_c,\ldots,F_d$ zu $F_3$ und dann zu $F_4,$ wodurch wir uns nun auf dem Schirm von $V_5$ befinden. Über die Flächen $F_e,\ldots,F_f$ gelingt uns nun wieder der Tausch zum Schirm von $V_3,$ weshalb die Flächen $F_g,\ldots,F_h$ uns schließlich das Erreichen der Fläche $F_2$ ermöglichen.  
Dadurch erhalten wir 
\[
(F_1,F_a,\ldots,F_b,F^{1,3}_1,F^{1,3}_2,F_c,\ldots,F_d,F_3,F_4,F_e,\ldots,F_f,F_1^{3,5},F_2^{3,5},F_g,\ldots,F_h,F_2)
\]
als geschlossenen Pfad.
Wegen der Orientierung $o(F_1)=(V,V_1,V_2)$ erhalten wir aufgrund der obigen Bemerkung 0.1
\[
o(F_1^{1,3})=(V_1,V_1^{1,3},V_3),
\] 
denn der Pfad 
\[
(V,\{V,V_2\},V_2,\ldots, V_1^{1,3},\{V_1^{1,3},V_3\},V_3,\{V_3,V_2^{1,3}\},\ldots)
\] bildet einen Schirm-Pfad von $V_1.$
Analoges Vorgehen liefert die Orientierungen:
\begin{align*}
&o(F_2^{1,3})=(V_1,V_3,V_2^{1,3})\\
\Rightarrow&o(F_3)=(V,V_3,V_4)\\
\Rightarrow&o(F_4)=(V,F_4,V_5)\\
\Rightarrow&o(F_1^{3,5})=(V_5,V_3,V_1^{3,5})\\
\Rightarrow&o(F_2^{3,5})=(V_3,V_5,V_2^{3,5})\\
\Rightarrow&o(F_2)=(V,V_3,V_2)
\end{align*}
Durch Betrachtung von $o(F_1)$ und $o(F_2)$ erhalten wir den gewünschten Widerspruch, denn diese sind nicht kohärent.
\end{proof}

Durch näheres Betrachten des obigen Beweises erhalten wir eine neue Charakterisierung der Orientierbarkeit einer vertex-treuen Sphäre.
\begin{comment}
\begin{bemerkung}
Um den folgenden Beweis einfacher zu Gestalten  führt man folgende Vereinfachung ein: Falls Sphären $X$ und $Y$ isomorph sind, nehmen wir $X=Y$ bezüglich ihrer Mengen und Relationen an. 
 
\end{bemerkung}
\begin{satz}
Sei $X$ eine Sphäre mit $n$ Flächen ohne Ecken vom Grad 2. Dann existiert eine Kantensequenz $E=(e_1,\ldots,e_m)$ in $X,$ sodass  $X^E$ zum Doppel-$n$-Gon isomorph ist. Zudem ist für alle  $1\leq i\leq k$ in der simplizialen Fläche $X^{(e_1,\ldots,e_i)}$ keine Ecke vom Grad 2 enthalten.
\end{satz}
\begin{proof}

Man führt den Beweis induktiv. 
Für $n=4,6$ ist nichts zu zeigen. Also sei $n>6$. Falls $X\cong (n)^2$ ist, ist nichts zu zeigen. Deshalb nimmt man an, das $X$ nicht zum $(n)^2$ isomorph ist. Ziel ist es eine Ecke vom Grad 3 durch Kantendrehungen zu erzeugen. Falls $X$ keine Ecke mit Flächengrad 3 hat, dann wählt man eine Ecke $V$ in $X$, die 
\[
deg_X(V)\leq deg_X(V') \text{ für alle }V'\in X_0
\]
erfüllt. Sei also hierzu $X_1(V)=\{e_1,\ldots,e_m\}$ die Menge der Kanten, die zu $V$ inzident sind. Dann ist $X^{e_1}$ eine simpliziale Fläche mit 
\[
deg_{X^{e_1}}(V')=deg_{X}(V')-1 \text{ für } V'\in X_0(e_1)
\]
Falls diese Kantendrehung keine Ecke vom Grad 3 erzeugt hat, wiederholt man die obige Prozedur mit einer der Kanten in $X^{e_1}_1(V)=\{e_2,\ldots,e_{m}\}$. Nach endlich vielen Schritten erhält man eine durch Kantendrehungen entstandene simpliziale Fläche Y mit einer Ecke $V^*$ vom Grad 3. Also kann man ohne Einschränkung der Allgemeinheit annehmen, dass $X$ eine Ecke vom Grad 3 hat. Außerdem gilt für $V'\in X_0(X_2(V^*))$, dass $deg(V')>3$ ist, da $Y$ nicht zum Tetraeder isomorph ist. Man definiert nun $Z$ als die simpliziale Fläche, die durch Entfernen des Tetraeders an $V^*$ entsteht, also $Z={}_{V*}T(X)$. Da $\vert Z_2 \vert =n-2$ ist, existiert nach Induktionsvoraussetzung eine Kantensequenz $E=(e_1,\ldots,e_m)$ in $Z\subset X$, sodass 
\[
Z^E\cong (n-1)^2
\]
ist. 

Man muss die Kantensequenz $E$ in $Z$ nun so in eine Kantensequenz $E'$ in $X$ abändern, dass Anhängen des Tetraeders und Anwenden einer Kantendrehung vertauschbar sind.
\[
bild 
\] 
Sei dafür die Sphäre $X^0:=X$ und $X^i$ und die Kantensequenz $E_i=(e_1,\ldots,e_l)$ in $X$ für $0\leq i \leq m$ schon konstruiert. Sei $F$ die Fläche, die beim Entfernen des Tetraeders den Tetraeder ersetzt. Man führt folgende Fallunterscheidung durch
\begin{itemize}
\item Falls die Kante $e_i$ der Kantensequenz $E$ nicht zu F inzident ist, so wähle $X^{i+1}:={(X^i)}^{e_i}$ und $E_{i+1}:=(e_1,\ldots,e_l,e_i)$. Damit ist 
\[
T(Z^{(e_1,\ldots,e_l,e_i)})^F\cong X^{E_{i+1}}
\]
und für alle Ecken $V\in X^{E_{i+1}}$ gilt $deg_{X^{E_{i+1}}}(V)\geq deg_S(V)\geq 3,$ wobei $S=T(Y^{(e_1,\ldots,e_i)})^F$ ist.
\item Falls die Kante $e_i$ der Kantensequenz $E$ zu F inzident ist, muss man Kantendrehungen anwenden, die die oben erwähnte Vertauschbarkeit liefern. 
Seien $F_1,F_2,F_3,$ die zu $V^*$ inzidenten Flächen. Sei ohne Einschränkung $e_i<F_1$ und $e$ die Kante, die $\vert {(X^i)^{e_i}}_2(e)\cup \{F_1,F_2,F_3\}\vert =1$ erfüllt.
\[
bild
\]

Dann definiert man $X^{i+1}:={X^i}^{(e_i,e)}$ und $E^{i+1}:=(e'_1,\ldots,e_l',e_i,e').$ Dann ist ebenfalls
\[
T(Y^{(e_1,\ldots,e_i)})^F\cong X^{E_{i+1}}
\] 
\end{itemize}
Nach endlich vielen Schritten erhält man also eine Kantensequenz $E^*$ in $X$ sodass 
\[X^{E^*}\cong T(Z^E)^F \cong T((n-1)^2)^F
\]ist. 
Sei $e$ die Kante, die die beiden Ecken mit Grad 5 verbindet. Dann ist $(X^{E^*})^e\cong (n)^2.$ Für alle $V\in X_0 \setminus \{V^*\}$ gilt 
\[
deg_{X^E_{i}}(V)=geq deg_{T(Z^{(e_1,\ldots,e_i)})}(V)\geq 3.
\] 
und Anwenden der beiden Kantendrehungen im zweitem Fall sorgen, dafür das $deg(V^*)$ erst um 1 erhöht und dann um 1 verringert wird. Damit ist die Aussage gezeigt.
\end{proof}



Sei $X$ eine vertex-treue Sphäre und $E=(e_1,\ldots ,e_n)$ eine Kantensequenz in $X$. Man nennt $E$ eine vertex-treue Kantensequenz, falls für alle $1\leq i \leq n$ die Sphäre $X^{(e_1,\ldots,e_i)}$ vertex-treu ist.
\end{definition}
\end{comment}
In Folgenden werden wir die Transitivität der Kantendrehung unter strikteren Voraussetzungen beweisen. Die Idee dieses Beweises ist die Vertauschbarkeit von Kantendrehungen und Tetraeder-Erweiterungen zu verwenden. Seien hierfür $X$ eine Sphäre mit einer Ecke vom Grad 3 und $Y=T_V(X)$. Weiterhin sei $F$ die Fläche in $Y,$ die den Tetraeder ersetzt. Unter der Verträglichkeit von Kantendrehungen ist gemeint, dass Kantendrehungen auf $Y$ anwenden und daraufhin den Tetraeder wieder anhängen zum selben Ergebnis führt wie die Kantendrehungen direkt auf $X$ anzuwenden. Kantendrehungen an Kanten der Fläche $F$ in $Y$ haben zur Folge, dass der Tetraeder an $V$ in $X$ wandern muss. Dies skizzieren wir im nachstehenden Beweis. Die nächste Skizze soll die erwähnte Verträglichkeit verdeutlichen. Hierfür sei $e$ eine drehbare Kante, die in $Y$ an der Fläche $F$ liegt. Also liegt $e$ in der Sphäre $X$ am Tetraeder an $V$. Wir wollen also eine drehbare Kantensequenz $E$ konstruieren, sodass Entfernen des Tetraeders in $X$ und Kante $e$ in der resultierenden Sphäre $e$ dasselbe Ergebnis herbeiführt wie die Kantensequenz $E$ auf $X$ anzuwenden. Das Konstruieren der drehbaren Kantensequenz $E$ soll im Beweis des Satzes genauer skizziert werden.
\begin{center}
\begin{figure}[H]
\includegraphics[scale=.95,viewport=0cm 17cm 19cm 27cm]{comute}
\caption{Verträglichkeit der Tetraeder-Erweiterung und Kantendrehung}
\end{figure}
\end{center}
\begin{satz}\label{3eck}
Sei $X$ eine vertex-treue Sphäre mit $n$ Flächen. Dann existiert eine drehbare Kantensequenz $E,$ sodass $X^E$ zum Doppel-$n$-Gon isomorph ist. 
\end{satz}
\begin{proof}
Wir weisen diese Aussage induktiv nach. Für $n=4$ ist nichts zu zeigen. Sei nun also $n>4$ und $X$ eine vertex-treue Sphäre, die nicht zum Doppel-$n$-Gon isomorph ist. 
Wegen \Cref{grad3} kann die Existenz einer Ecke $V$ vom Grad 3 in $X$ angenommen werden. Für $Y=T_V(X)$ gilt dann 
\[
\vert Y_2\vert=n-2.
\]
Deshalb existiert eine drehbare Kantensequenz $E=(e_1,\ldots e_m)$ in $Y,$ sodass $Y^E\cong (n-1)^2$ ist. Sei $F$ die Fläche in $Y$, sodass eine Tetraeder-Erweiterung an $F$ die Sphäre $X$ hervorbringt. Ziel ist es, aus der Kantensequenz in $Y$ eine Kantensequenz $E'=(e'_1,\ldots e'_k)$ in $X$ zu konstruieren, die Folgendes erfüllt:
\begin{itemize}
\item $E'$ ist eine drehbare Kantensequenz.
\item Die Kantensequenz erlaubt, dass die Entfernung des Tetraeders und die Durchführung der Kantensequenz vertauschbar sind. Damit ist gemeint, dass es für alle $1\leq i\leq m$ ein $1\leq l \leq k$ mit 
\begin{align*}
T^F(Y^{(e_1,\ldots, e_i)})&\cong X^{(e'_1,\ldots,e'_l)} \\
\Leftrightarrow Y^{(e_1,\ldots, e_i)}&\cong T_V(X^{(e'_1,\ldots,e'_l)})
\end{align*} 
existiert.
\end{itemize}
Sei dafür $X^0:=X$ und $E_0:=().$ Die Kantensequenz wird wie folgt konstruiert: Seien die Sphären $X^{i-1}$ und die Kantensequenz $E_{i-1}=(e_1',\ldots,e_l')$ in $X$ für $1\leq i \leq m$ bereits konstruiert. Wir führen nun folgende Fallunterscheidung durch:
\begin{itemize}
\item Falls die Kante $e_{i}$ der Kantensequenz $E$ in $Y^{(e_1,\ldots,e_{i-1})}$ nicht zu F inzident ist, so wählen wir $X^{i}:={(X^{i-1})}^{e_i}$ und $E_{i}:=(e_1',\ldots,e_l',e_i)$. Damit ist 
\[
T^F(Y^{(e_1,\ldots,e_i)})\cong X^{E_{i}}
\]
und in $X^{E_{i}}$ existiert keine 2-Taille, denn sonst wäre diese schon in $Y^{(e_1,\ldots,e_i)}$ enthalten. Damit ist $E_{i}$ drehbar.
\item Falls die Kante $e_i$ der Kantensequenz $E$ in $Y^{(e_1,\ldots,e_{i-1})}$ zu F inzident ist, müssen Kantendrehungen wie im Folgendem beschrieben angewendet werden. Da die Kante $e_i$ in $Y^{(e_1,\ldots,e_{i-1})}$ zu $F$ inzident ist, liegt die Kante in $X^{i-1}$ am Tetraeder an der Stelle $V.$
Weiterhin existieren Kanten $e',e''$, sodass die Kante $e''$ in $Y^{(e_1,\ldots,e_{i-1})}$ zu $F$ inzident ist und in $Y^{(e_1,\ldots,e_{i})}$ beide Kanten zu der Fläche $F$ inzident sind.
\begin{figure}[H]
\begin{center}
\includegraphics[scale=1,viewport=0cm 23cm 10cm 27cm]{3ecke3}
\end{center}
\caption{Kantendrehung in der Sphäre $Y^{(e_1,\ldots,e_{i-1})}$}
\end{figure}
Dann liegt $e''$ in $X^{(e_1',\ldots,e_l')}$ am Tetraeder an der Ecke $V.$ Diese Kante ist jedoch nicht zu $V$ inzident.  
\begin{figure}[H]
\begin{center}
\includegraphics[scale=1,viewport=0cm 24cm 6cm 27cm]{3ecke}
\end{center}
\caption{Ausschnitt der Sphäre $X^{i-1}$}
\end{figure}
In $X^{(e_1',\ldots,e_{l-1}',e_i)}$ gibt es genau eine Kante $e$, die zu $V$ inzident ist und 
 \[
Z_0(e)\cap Z_0(e')=Z_0(e)\cap Z_0(e'')=\emptyset
\]
erfüllt, wobei $Z=X^{(e_1',\ldots,e_{l-1}',e_i)}$ ist. Durch Drehen dieser Kante erhalten wir die geforderte Vertauschbarkeit.
\begin{comment}
Seien $F_1,F_2,F_3$ in $X^{i-1},$ die zu $V$ inzidenten Flächen. Dann liefert uns mindestens einer der Kanten in  $X_1(V)-\{e_i\}$ die oben geforderte Vertauschbarkeit. 
\end{comment}
Wir wählen also $X^{i}:={(X^{i-1})}^{(e_i,e)}$ und $E^{i}:=(e'_1,\ldots,e'_l,e_i,e).$ Dies liefert erneut
\[
T^F(Y^{(e_1,\ldots,e_i)})\cong X^{E_{i}}
\] 
\begin{figure}[H]
\begin{center}
\includegraphics[scale=0.9,viewport=0cm 18cm 14cm 26.5cm]{3ecke2}
\end{center}
\caption{Anwenden der Kantensequenz $(e_i,e)$ auf $X^{i-1}$}
\end{figure}
\end{itemize}
 Durch das Drehen der Kanten $e_i$ und $e$ können keine 2-Taillen entstehen, wodurch $E_{i}$ drehbar ist.
Nach endlich vielen Schritten erhalten wir also eine vertex-treue Kantensequenz $E'$ in $X,$ sodass 
\[X^{E'}\cong T^F(Y^E) \cong T^F((n-1)^2)
\]ist.
Sei $e$ nun die Kante in $X^{E'}$, die die beiden Ecken vom Grad 5 am eingesetzten Tetraeder verbindet. Dann ist $(X^{E'})^e\cong (n)^2.$
\end{proof}
\begin{bemerkung}
Seien $X$ und $Y$ vertex-treue Sphären und $\phi$ ein Isomorphismus von $X$ nach $Y.$ Für alle Kanten $e\in X_1$ und $e'\in Y_1$ mit $\phi(e)=e'$gilt dann
\[ 
X^e \cong Y^{e'} 
\]
Für eine Kantensequenz $E=(e_1,\ldots,e_n)$ in $X$ bedeutet dies 
\[
X^E\cong Y^{(\phi(e_1),\ldots,\phi(e_n))}.
\]
\end{bemerkung}
\begin{satz} \label{kantendrehung}
Seien $X$ und $Y$ vertex-treue Sphären mit $\vert X_2\vert=\vert Y_2\vert=n\in \mathbb{N}$.
Dann existiert eine drehbare Kantensequenz $E=(e_1,\ldots,e_n)$ in $X,$ sodass  $X^E$ zu $Y$ isomorph ist. 
\end{satz}
\begin{proof}
Nach vorherigem Satz existieren drehbare Kantensequenzen $E=(e_1,\ldots,e_m)$ in $X$ und $E'=(e'_1,\ldots,e'_{k})$ in $Y$, sodass 
\[
x^E\cong (n)^2 \cong Y^{E'}
\] ist.
Da $X^E$ und $Y^{E'}$ isomorph sind, existiert ein Isomorphismus 
\[
\phi: X^E\to Y^{E'}
\]
Somit bildet $E^*=(e_1,\ldots,e_m,\phi^{-1}(e'_{k}),\ldots,\phi^{-1}(e'_{1}))$ eine Kantensequenz in $X$ und es gilt:
\begin{align*}
X^{E^{*}} = &X^{(e_1,\ldots,e_m,\phi^{-1}(e'_{k}),\ldots,\phi^{-1}(e'_{1}))}\\
&\cong (X^{(e_1,\ldots,e_m)})^{(\phi^{-1}(e'_{k}),\ldots,\phi^{-1}(e'_{1}))}\\
&\cong ((n)^2)^{(\phi^{-1}(e'_{k}),\ldots,\phi^{-1}(e'_{1}))}\\
&\cong ((\phi^{-1}(Y^{E'})))^{(\phi^{-1}(e'_{k}),\ldots,\phi^{-1}(e'_{1}))}\\
&\cong Y .
\end{align*}
Da $E$ und $E'$ drehbar sind, gilt dies auch für Kantensequenz $E^*.$
\end{proof}
\begin{bemerkung}
 Um die Transitivität der Kantendrehung auf der Menge der Sphären zu erhalten, sind die Sphären mit 3-Taillen jedoch unerlässlich.
\begin{comment} 
 Für das tiefere Verständnis dieser Aussage betrachten wir die Sphären mit genau 12 Flächen. Bis auf Isomorphie gibt es genau 2 Sphären ohne 3- oder 2-Taille, nämlich den Doppel-6-Gon und die Sphäre, die durch das Symbol $(5)\overline{2}(5)$ beschrieben wird.
  Der Doppel-6-Gon und die Sphären, die durch eine Kantendrehung aus dem Doppel-6-Gon hervorgehen, sind uns bereits aus dem einführenden Beispiel zur Kantendrehung bekannt. Beim Betrachten der aus dem Doppel-6-Gon konstruierten Sphären ist zu erkennen, dass beide Sphären Ecken vom Grad 3, also 3-Taillen besitzen.
  \end{comment}
 Betrachten wir zum Nachweis für $n\in \mathbb{N}$ den Doppel-$n$-Gon. Im Doppel-$n$-Gon gibt es bis auf Isomorphie genau 2 Kanten. Es gibt Kanten, die zu einer Ecke vom Grad $n$ und zu einer Ecke vom Grad 4 inzident sind. Der andere Isomorphietyp sind die Kanten, die zu zwei Ecken vom Grad 4 inzident sind. Beide bilden drehbare Kanten im $(n)^2$ und beim Drehen dieser wird der Grad der inzidenten Ecken um 1 verringert. Also entstehen in beiden Fällen Ecken vom Grad 3 und somit auch 3-Taillen.
\end{bemerkung}
\begin{bemerkung}
Bisher wurden ausschließlich drehbare Kanten für die Durchführung von Kantendrehungen betrachtet. Es ist möglich Kantendrehungen auch allgemeiner zu formulieren. Falls die Kantendrehung an einer nicht drehbaren Kante durchgeführt wird, entsteht eine Sphäre mit einer 2-Taille und die Sphäre wird wie in \Cref{2waistk} in zwei Komponenten aufgeteilt.  Zum Durchführen der allgemeinen Kantendrehung an einer Kante $e$ der Sphäre $X,$ muss $\deg(V)>3$ für alle $V\in X_0(e)$ verlangt werden. Im Falle einer Kantensequenz $E=(e_1,\ldots,e_n)$ in $X,$ die nicht drehbar ist, schreiben wir $X^{[e_1,\ldots,e_n]}$.

 Führen wir diese allgemeine Kantendrehung am Beispiel des Tetraeders durch, indem wir das Symbol der resultierenden Sphäre angeben. Das Symbol des Tetraeders
\begin{align*}
\mu(T)=(4,6,4;&(\{1,2\}, \{1,3\},\{1,4\},\{2,3\},\{2,4\},\{2,4\},\{3,4\})\\
;&(\{4,5,6\},\{2,3,6\},\{1,3,5\},\{1,2,4\}))
\end{align*} ist aus vorherigen Beispielen bereits bekannt.
\begin{figure}[H]
\begin{center}
\includegraphics[viewport=1cm 24cm 5cm 27cm]{ET_Example1}
\end{center}
\caption{Tetraeder}
\end{figure}
Die Kante 1 des Tetraeders ist nicht drehbar, aber ist zu zwei Ecken inzident deren Grad 3 ist. Durch Drehen der Kante entsteht die Sphäre $T^{[1]}$, die durch folgendes Symbol beschrieben wird.
\begin{align*}
\mu ((T,<)):=(4,6,4;&(\{3,4\},\{1,3\},\{1,4\},\{2,3\},\{2,4\},\{3,4\})\\
;&(\{4,5,6\},\{2,3,6\},\{1,2,3\},\{1,4,5\}))
\end{align*}
\begin{figure}[H]
\begin{center}
\includegraphics[viewport=1cm 24cm 5cm 27cm]{ET_Example2}
\end{center}
\caption{Kantendrehung am Tetraeder}
\end{figure}

\end{bemerkung}

\begin{bemerkung}
Es stellt sich die Frage, ob das Ausschließen der Sphären mit 2-Taillen bei der Konstruktion der Kantensequenzen zur Folge hat, dass die minimale Anzahl an Kantendrehungen, die es braucht, um zwei Sphären ineinander zu transformieren, erhöht wird. Für Kantensequenzen bei denen Sphären ausschließlich triviale 2-Taillen, also Ecken vom Grad 2 erzeugt werden, kann gezeigt werden, dass es eine drehbare Kantensequenz derselben Länge gibt, die das selbe Ergebnis herbeiführt. Der allgemeinere Fall ist etwas schwieriger zu kontrollieren.
\end{bemerkung}
\subsection{Kantendrehungen am Schirmzeiger}
Ziel dieses Kapitels ist es die oben definierte Kantendrehung auf der Ebene der Schirmzeiger durchzuführen. Hierfür muss dieser zunächst eingeführt werden.

\begin{definition}
Sei $X$ eine Sphäre und $V\in X_0$ eine Ecke in $X.$ Dann wird der \emph{Schirmzeiger} von $X$ durch 
\[
U(X):=\{u_X(V) \mid V\in X_0\},
\]
wobei $u_X(V)$ der Schirm von $V\in X_0$ ist, definiert.
\end{definition}
\begin{bsp}
\begin{itemize}
\item 
Den Schirmzeiger eines Tetraeders bildet die Menge 
\[
U(T)=\{ ((1,2,3)),((1,2,4)),((1,3,4)),((2,3,4))\}.
\]
\item Der Janus-Kopf hat den Schirmzeiger 
\[
\{((1,2))\}
\]
\end{itemize}
\end{bsp}
\begin{bemerkung}
Falls der Minimalgrad einer Sphäre 3 ist, so lässt sich diese eindeutig aus dem Schirmzeiger rekonstruieren. Dieser Sachverhalt kann dem oben genannten Skript entnommen werden und wird deshalb hier nicht ausgeführt.
\end{bemerkung}
\begin{lemma}
Sei $X$ eine vertex-treue Sphäre, $U(X)$ der Schirmzeiger von $X$ und $e\in X_1$ eine drehbare Kante in $X$ mit $X_0(e)=\{F_1,F_2\}.$ Für eine Ecke $V\in X_0$ definieren wir  
\[
u_V:=
\Biggl\{
\begin{tabular}[l]{lcr}
\text{$u_X(V)$},&$(F_1,F_2)u_X(V)(F_1,F_2)=u_X(V)$ \\
$(F_1,F_2) u_X(V)$ ,& sonst \\

\end{tabular}.
\]
Dann ist der Schirmzeiger von $X^e$ gegeben durch 
\[
\{ u_V\mid V \in X_0\}.
\]
\end{lemma}
\begin{proof}
In der Sphäre $X$ gibt es genau zwei Ecken $V_1,V_2$, für die 
\[
X_2(V_1)\cap X_2(V_2)\cap \{F_1,F_2\}=\{F_1,F_2\}
\] 
gilt. Insbesondere sind diese beiden Ecken zur Kante $e$  inzident. Des Weiteren gelten für die beiden anderen Ecken $V_3,V_4$ in $X_0(X_2(e))$ die Gleichheiten 
\begin{align*}
&\{F_1,F_2\} \cap X_2(V_3)=\{F_1\},\\
&\{F_1,F_2\} \cap X_2(V_4)=\{F_2\}.
\end{align*}  
\begin{figure}[H]
\begin{center}
\includegraphics[viewport=0cm 23cm 5cm 27cm]{kantenumb}
\end{center}
\caption{Ausschnitt einer Sphäre $X$}
\end{figure}
Da $X$ vertex-treu ist, gibt es keine weitere Ecke, die zu $F_1$ oder $F_2$ inzident ist.
In der Sphäre $Y=X^e$ erhalten wir bis auf Isomorphie  
\begin{itemize}
\item $\{F_1,F_2\} \cap Y_0(V_1)=\{F_1\}$ 
\item $\{F_1,F_2\} \cap Y_0(V_2)=\{F_2\}$
\item und $\{F_1,F_2\} \cap Y_0(V_3)\cap Y_0(V_4)=\{F_1,F_2\}$.
\end{itemize}
\begin{figure}[H]
\begin{center}
\includegraphics[viewport=0cm 23cm 5cm 27cm]{kantenumb2}
\end{center}
\caption{Ausschnitt einer Sphäre $Y$}
\end{figure}
Beim Übertragen der obigen Beobachtungen auf den Schirmzeiger von $Y$ erhalten wir zunächst $u_Y(V)=u_X(V)=(F_1\,F_2)u_X(V) (F_1\,F_2)$
für alle $V\in X_0-\{V_1,V_2,V_3,V_4\}.$
Da die Konjugation mit der Transposition $(F_1\,F_2)$ den Schirm $u_X(V)$ für $V\in \{V_1,V_2,V_3,V_4\}$ nicht fixiert, müssen diese Schirme genauer betrachtet werden. 
Es gilt:
\begin{itemize} 
\item $u_{Y}(V_1)=(F_1,F,\ldots)=(F_1,F_2)(F_1,F_2)(F_1,F,\ldots)=(F_1,F_2)(F_2,F_1,F,\ldots)$ \\$=(F_1,F_2)u_X(V_1)$
\item $u_{Y}(V_2)=(F_2,F,\ldots)=(F_1,F_2)(F_1,F_2)(F_2,F,\ldots)=(F_1,F_2)(F_1,F_2,F\ldots)$\\$=(F_1,F_2)u_X(V_2)$
\item $u_{Y}(V_3)=(F_1,F_2,F,\ldots)=(F_1,F_2)(F_1,F,\ldots)=(F_1,F_2)u_X(V_3)$
\item $u_{Y}(V_4)=(F_2,F_1,F,\ldots)=(F_1,F_2)(F_4,F,\ldots)=(F_1,F_2)u_X(V_4)$
\end{itemize}
Somit folgt die Behauptung.
\end{proof}
%--------------------------------------------------------
\section{Multi-Tetraeder}\label{kapitelmultitetraeder}
In diesem Kapitel werden die sogenanntem \emph{Multi-Tetraeder} thematisiert. Diese bilden eine Klasse von vertex-treuen Sphären. Die hier einführenden Erkenntnisse sind ebenfalls dem oben genannten Skript zu entnehmen. Wir werden elementare Eigenschaften dieser Klasse von Sphären untersuchen und ebenfalls Invarianten in dieser angeben. Außerdem wird die Kaktus-Distanz einer vertex-treuen Sphäre als die minimale Anzahl an Kantendrehungen, um aus der Sphäre einen Multi-Tetraeder zu konstruieren, definiert. Es werden zwei Algorithmen zum Annähern der Kaktus-Distanz präsentiert und für einige Sphären die Kaktus-Distanz exakt bestimmt.

\subsection{Konstruktion und Klassifikation}\label{cactus}
Wir führen zunächst die Definition eines Multi-Tetraeders ein.
\begin{definition}\label{defcac}
Sei $X$ eine vertex-treue Sphäre mit $\vert X_0\vert \geq 6.$
\begin{enumerate}
\item Die simpliziale Fläche, die durch Entfernen aller Tetraeder entsteht, wird durch $X^{(1)}$ definiert. Für $i>1$ definieren wir analog 
\[
X^{(i)}:=(X^{(i-1)})^{(1)}.
\]
\item Wir nennen $X$ einen Multi-Tetraeder vom \emph{Grad} $k$, falls $X^{(k-1)}$ ein Tetraeder oder ein Doppel-Tetraeder ist. Hierfür bezeichne $a_0$ die Anzahl der Ecken vom Grad 3 in $X_0$ oder anders gesagt die Anzahl der Tetraeder, die von $X$ entfernt wurden und analog bezeichne $a_i$ die Anzahl der Ecken vom Grad 3 in $X^{(i)}$. Falls $X^{(i)}$ ein Tetraeder ist, so ist $a_i$ als 1 definiert. Das damit konstruierte Tupel $(a_0,a_1,\ldots,a_k)$ nennen wir den \emph{Typ} von $X$ und $T:=\sum_{i=0}^{k} a_i$ die \emph{Tetraeder-Zahl} von $X$.\\
\end{enumerate}
\end{definition}
\begin{bsp} \label{bspCactus}
\begin{itemize}
\item Per Definition bilden der Tetraeder und der Doppel-Tetraeder Multi-Tetraeder.
\item Bis auf Isomorphie gibt es genau einen Multi-Tetraeder mit 8 Flächen. Dieser ist ein Multi-Tetraeder vom Typ $(2,2,1)$ und wird durch den Flächen-Träger $\xi:=\Pot_3(\{1,2,3,4\})\Delta Pot_3(\{2,3,4,5\})\Delta \Pot_3(\{3,4,5,6\})$ beschrieben. 
\begin{figure}[H]
\begin{center}
\includegraphics[viewport=19cm 19cm 0cm 23cm]{Image_MultiTetraeder1}
\end{center}
\caption{Multi-Tetraeder mit 8 Flächen}
\end{figure}
\item Alle vertex-treuen Sphären ohne Ecken vom Grad 3 erfüllen $X^{(1)}=X$ und bilden damit keine Multi-Tetraeder.
\item Der simpliziale Parallelepiped $P$ ist eine Sphäre, beschrieben durch 
\begin{align*}
\xi=&\{ \{1, 2, 5 \}, \{ 1, 2, 7 \}, \{ 1, 3, 4 \}, \{ 1, 3, 7 \}, \{ 1, 4, 5 \},\{ 2, 3, 6 \},\\ &\{2, 3, 7 \}, 
\{ 2, 5, 6 \}, \{ 3, 4, 6 \}, \{ 4, 5, 8 \},\{ 4, 6, 8 \}, \{ 5, 6, 8 \} \}.
\end{align*}
\begin{figure}[H]
\begin{center}
\includegraphics[viewport=24cm 14cm 0cm 19cm]{Image_Parallelepiped}
\end{center}
\caption{Parallelepiped}
\end{figure}
Also ist $\deg_P(7)=\deg_P(8)=3.$ Aber $P$ bildet dennoch keinen Multi-Tetraeder, da $P^{(1)}$ ein Oktaeder ist und damit $P^{(2)}=P^{(1)}$ gilt.
\end{itemize}
\end{bsp}
Für einen Multi-Tetraeder erhalten wir also eine Kette 
\[
X^{(0)}\to \ldots \to X^{(t)}=X^{(t+1)},
\] 
sodass $X^{(t)}$ zum Tetraeder oder Doppel-Tetraeder isomorph ist.
Diese wird später von besonderem Interesse sein, wenn maximale Multi-Tetraeder thematisiert werden. Zunächst fassen wir aber in der folgenden Bemerkung elementare Eigenschaften der Multi-Tetraeder zusammen.
\begin{bemerkung}
\begin{itemize}
\item Sei die simpliziale Fläche $X$ für ein $k\in \mathbb{N}$ ein Multi-Tetraeder vom Typ $(a_0,a_1,\ldots,a_k),$ der nicht zum Doppel-Tetraeder oder zum Tetraeder isomorph ist. Dann ist $X$ eine Sphäre, denn beim Übergang von $X$ zu $Y=X^{(1)}$ werden $a_0$ Tetraeder durch Flächen ersetzt. Da beim Ersetzen eines Tetraeders durch eine Fläche die Flächenanzahl um 2, die Kantenanzahl um 3 und die Eckenanzahl um 1 verringert wird, erhalten wir  
\begin{align*}
\vert Y_0\vert =\vert X_0\vert-a_0\\
\vert Y_1\vert=\vert X_1\vert-3a_0\\
\vert Y_2\vert=\vert X_2\vert-2a_0\\
\end{align*}
Somit gilt für die Euler-Charakteristik
\begin{align*}
\chi(Y)=&\vert Y_0\vert -\vert Y_1\vert+\vert Y_2\vert\\
=&\vert X_0\vert-a_0-(\vert X_1\vert-3a_0)+\vert X_2\vert-2a_0\\
&=\chi (X)
\end{align*}
Also erhalten wir induktiv 
\[
\chi(X)=\chi(X^{(1)})=\ldots=\chi(X^{(k)})=2,
\]
da $X^{(k)}$ isomorph zum Tetraeder oder Doppel-Tetraeder ist.
\item Multi-Tetraeder enthalten Ecken vom Grad 3 und somit auch 3-Taillen. Vielmehr sind Multi-Tetraeder die Sphären mit maximaler Anzahl an 3-Taillen unter den Sphären mit derselben Flächenanzahl. Die Anzahl der 3-Taillen eines Multi-Tetraeders mit $n$ Flächen ist $\frac{n-4}{2}.$

\item
Da beim Anhängen von Tetraedern keine 2-Taillen entstehen, sind Multi-Tetraeder Sphären ohne 2-Taillen.
\item
Multi-Tetraeder sind vertex-treue Sphären.
\end{itemize}
\end{bemerkung}
\begin{bemerkung}
Seien $X$ ein Multi-Tetraeder, $(a_0,\ldots,a_k)$ der Typ von $X$ und $n=\sum_{i=1}^{k}a_i$ die Tetraeder-Zahl von $X$. Dann gelten folgende Aussagen: 
\begin{itemize}
\item $X$ hat genau $2(n+1)$ Flächen. 
\item Der Typ $(a_0,\ldots,a_k)$ von $X$ ist monoton fallend und es gilt $a_i\neq 1$ für $i=0,\ldots k-1.$
\item  Für alle $n \in \mathbb{N}$ existiert ein Multi-Tetraeder mit Tetraeder-Zahl $n$.
\end{itemize}
\end{bemerkung}
\begin{proof}
\begin{itemize}
\item klar
\item Sei $0\leq i \leq k-1$ mit $1<a_i <a_{i+1}.$ Es werden mindestens zwei Ecken vom Grad 3 in $X^{i}$ von demselben Tetraeder überdeckt. Damit muss aber $X^{i+1}$ zum Tetraeder isomorph sein, was $a_{i+1}=1$ impliziert und damit einen Widerspruch erzeugt.
 \item Der Tetraeder $T$ bildet den kleinsten Multi-Tetraeder  und durch iteratives Anheften eines Tetraeders folgt die Behauptung.
\end{itemize}
\end{proof}

Um unsere erste Erkenntnis über Multi-Tetraeder zu formulieren, führen wir folgende Definition ein.
\begin{definition}
Seien $X$ und $Y$ Multi-Tetraeder. Wir nennen $Y$ ein Kind von $X$, falls $T_V(Y)\cong X$ für ein $V\in Y_0$ ist.
\end{definition}
Also sind die Kinder eines Multi-Tetraeders $X$ genau die Multi-Tetraeder, die durch genau eine Tetraeder-Erweiterung aus $X$ hervorgehen.
Damit ist beispielsweise der Doppel-Tetraeder bis auf Isomorphie das einzige Kind des Tetraeders. \\\\
Zur Motivation der nächsten Fragestellung betrachten wir folgendes Beispiel in GAP:\\
Sei $X$ der bis auf Isomorphie eindeutige Multi-Tetraeder mit 8 Flächen, den wir bereits in \Cref{bspCactus} kennengelernt haben. 
\begin{center}
$\fbox{
\parbox{14cm}{
\textcolor{red}{gap$>$} \textcolor{blue}{L:=[[1,2,3],[1,2,4],[1,3,4],[2,3,5],[2,4,5],[3,4,6],[4,5,6],[3,5,6]];;}\newline
\textcolor{red}{gap$>$} \textcolor{blue}{s:=SimplicialSurfaceByVerticesInFaces(last);}\newline
simplicial surface (6 vertices, 12 edges, and 8 faces)\newline
\textcolor{red}{gap$>$} \textcolor{blue}{IsCactus(s);}\newline
true
}}$
\end{center}
Durch Tetraeder-Erweiterungen an den Flächen von $X$ können wir die Kinder von $X$ bestimmen.
\begin{center}
$\fbox{
\parbox{14cm}{
\textcolor{red}{gap$>$}\textcolor{blue}{ List(Faces(s),f-$>$AddTetraF(s,f));}\newline
[ simplicial surface (7 vertices, 15 edges, and 10 faces),\newline
simplicial surface (7 vertices, 15 edges, and 10 faces),\newline
 simplicial surface (7 vertices, 15 edges, and 10 faces),\newline
 simplicial surface (7 vertices, 15 edges, and 10 faces),\newline
 simplicial surface (7 vertices, 15 edges, and 10 faces),\newline
 simplicial surface (7 vertices, 15 edges, and 10 faces),\newline
  simplicial surface (7 vertices, 15 edges, and 10 faces)
    ,\newline
  simplicial surface (7 vertices, 15 edges, and 10 faces)
 ]
 }}$
 \end{center}
 Die durch die obigen Tetraeder-Erweiterungen konstruierten Multi-Tetraeder müssen jedoch nun auf Isomorphie untersucht werden.
 \begin{center} 
 $\fbox{
\parbox{14cm}{
\textcolor{red}{gap$>$} \textcolor{blue}{IsomorphismRepresentatives(last);}\newline
[ simplicial surface (7 vertices, 15 edges, and 10 faces),\newline
 simplicial surface (7 vertices, 15 edges, and 10 faces)
    ,
    \newline
  simplicial surface (7 vertices, 15 edges, and 10 faces)
 ]
}}$
\end{center}
Also hat $X$ bis auf Isomorphie genau 3 Kinder. \\\\
Es stellt sich also die Frage, wie viele Kinder ein Multi-Tetraeder bis auf Isomorphie haben kann. Denn das Erweitern durch Tetraeder an verschiedenen Flächen eines Multi-Tetraeders bringt nicht unbedingt nicht-isomorphe Kinder hervor. Hierfür muss jedoch etwas Vorarbeit geleistet werden.
\begin{bemerkung} 
Sei $X$ eine vertex-treue Sphäre und $G$ die Automorphismengruppe von $X$. Dann wird durch 
\[
\Phi_X:G \times X\mapsto X,(\phi, x)\to \phi(x)
\] eine Gruppenoperation definiert. Denn klarerweise ist
\begin{itemize}
\item id(x)=x für alle $x\in X$ und
\item $\phi_1(\phi_2 (x))=(\phi_1 \circ\phi_2)(x)$ für alle $\phi_1,\phi_2 \in G$ und $x\in X.$
\end{itemize}
Aber auch die Einschränkung   
\[
\Phi_{X_i}:G \times X_i\mapsto X_i,(\phi, x)\to \phi(x)
\] für $i\in \{0,1,2\}$ liefert eine Gruppenoperation, da für ein $x\in X_i$ und ein $\phi\in G$ stets $\phi(x)\in X_i$ gilt. Mithilfe dieser Gruppenoperation kann nun die Anzahl der Kinder eines Multi-Tetraeders bestimmt werden.
\end{bemerkung}
Zunächst betrachten wir aber folgende Beispiele. 
\begin{bsp}
\begin{itemize}
\item Sei $T$ ein Tetraeder. Dann ist die Operation der Automorphismengruppe auf $T$ und damit auch auf den Mengen $T_i$ transitiv.
\item Sei $X$ der bis auf Isomorphie eindeutige Multi-Tetraeder mit 8 Flächen. Nutzen wir GAP, um die Bahnen von $X_2$ unter der Automorphismengruppe von $X$ zu bestimmen.
\end{itemize}
\end{bsp}
\begin{center}
$\fbox{
\parbox{14cm} {
\textcolor{red}{gap$>$}\textcolor{blue}{ S;}\newline
simplicial surface (6 vertices, 12 edges, and 8 faces)\newline
\textcolor{red}{gap$>$} \textcolor{blue}{ IsCactus(S);}\newline
true\newline
\textcolor{red}{gap$>$}\textcolor{blue}{  VerticesOfFaces(S);}\newline
[ [ 1, 2, 3 ], [ 1, 2, 4 ], [ 1, 3, 4 ], [ 2, 3, 5 ], [ 2, 4, 5 ], [ 3, 4, 6 ], [ 3, 5, 6 ],\newline
  [ 4, 5, 6 ] ]\newline
\textcolor{red}{gap$>$}\textcolor{blue}{  FaceDegreesOfVertices(S);}\newline
[ 3, 4, 5, 5, 4, 3 ]\newline
\textcolor{red}{gap$>$} \textcolor{blue}{ G:=AutomorphismGroupOnFaces(S);}\newline
Group([ (1,2)(4,5)(7,8), (1,7)(2,8)(3,6) ])\newline
\textcolor{red}{gap$>$} \textcolor{blue}{ Orbits(G,Faces(S));}\newline
[ [ 3, 6 ], [ 4, 5 ], [ 1, 2, 7, 8 ] ]
}}$
\end{center}
Mithilfe dieser Gruppenoperation können wir nun die Anzahl der Kinder eines Multi-Tetraeders bestimmen.
\begin{lemma}
Sei $X$ ein Multi-Tetraeder. Dann ist die Anzahl der Kinder von $X$ die Anzahl der Bahnen der Gruppenoperation $\Phi_{X_2}.$
\end{lemma}
\begin{proof}
Wir müssen zunächst zeigen, dass das Erweitern durch Tetraeder an Flächen, die in einer Bahn unter $\Phi_{X_2}$ liegen isomorphe Kinder von $X$ hervorbringt. Seien also $F_1,F_2$ solche Flächen, dann existiert also ein $\phi \in \Aut(X),$ der $F_1$ auf $F_2$ abbildet. Dieser Isomorphismus lässt sich dann aber durch Ergänzen der fehlenden Bilder des zuletzt angehängten Tetraeders zu einem Isomorphismus 
\[
\phi':T^{F_1}(X)\mapsto T^{F_2}(X)
\]
 erweitern. \\
 Seien also nun $F_1$ und $F_2$ zwei Flächen aus verschieden Bahnen. Angenommen es existiert ein Isomorphismus $\phi':T^{F_1}(X)\mapsto T^{F_2}(X),$ dann erhalten wir durch leichtes Abändern einen Automorphismus, der $F_1$ auf $F_2$ abbildet, was ein Widerspruch ist. 
\end{proof}
\begin{definition}
Seien $X$ und $Y$ Multi-Tetraeder. Falls $X=Y^{(1)}$ ist, dann nennen wir $Y$ einen \emph{Enkel}  von $X.$ 
\end{definition}

Hier stellt sich nun die Frage, wie viele Enkel ein Multi-Tetraeder besitzt. 
Zur Motivation dieser Fragestellung schauen wir uns den Doppel-Tetraeder an.
Diesen erhalten wir durch:
\begin{center}
 $\fbox{
\parbox{14cm}{
\textcolor{red}{gap$>$} \textcolor{blue}{L:=[[1,2,3],[1,2,4],[1,3,4],[2,3,5],[2,4,5],[3,4,5]];;}\newline
\textcolor{red}{gap$>$} \textcolor{blue}{DT:=SimplicialSurfaceByVerticesInFaces(L);}\newline
simplicial surface (5 vertices, 9 edges, and 6 faces)\newline
\textcolor{red}{gap$>$} \textcolor{blue}{IsCactus(DT);}\newline
true
}}$
\end{center}
Ein Enkel des Doppel-Tetraeders $DT$ muss vom Typ $(a_0,2),$ wobei $2\leq a_0 \leq 6$ ist, sein.
Um einen Multi-Tetraeder $X$ mit $X^{(1)}\cong DT$ zu konstruieren, muss bei jeder Ecke vom Grad 3 mindestens eine inzidente Fläche durch einen Tetraeder ersetzt werden. Die Ecken vom Grad 3 in der Sphäre $DT$ sind die Ecken 1 und 5. 
Die Enkel von $DT$ vom Typ $(2,2)$ erhalten wir durch
\begin{center}
 $\fbox{
\parbox{14cm}{
\textcolor{red}{gap$>$} \textcolor{blue}{L:=[\,];;}\newline
\textcolor{red}{gap$>$} \textcolor{blue}{Cartesian(FacesOfVertex(DT,1),FacesOfVertex(DT,5));}\newline
[ [ 1, 4 ], [ 1, 5 ], [ 1, 6 ], [ 2, 4 ], [ 2, 5 ], [ 2, 6 ], [ 3, 4 ],
  [ 3, 5 ], [ 3, 6 ] ]\newline
\textcolor{red}{gap$>$} for g in last do\newline
\textcolor{red}{$>$} s:=DT;\newline
\textcolor{red}{$>$} for f in g do\newline
\textcolor{red}{$>$} s:=AddTetraF(s,f);\newline
\textcolor{red}{$>$} od;\newline
\textcolor{red}{$>$} Add(L,s);\newline
\textcolor{red}{$>$} od;\newline
\textcolor{red}{gap$>$} \textcolor{blue}{IsomorphismRepresentatives(L);}\newline
[ simplicial surface (7 vertices, 15 edges, and 10 faces)
    ,\newline
  simplicial surface (7 vertices, 15 edges, and 10 faces)
 ]
 }}$
 \end{center}
 Die Enkel vom Typ $(3,2)$ werden auf ähnliche Art und Weise konstruiert. Hierbei muss bei einer der beiden Ecken vom Grad 3 an zwei Flächen und bei der anderen Ecke eine Fläche durch eine Tetraeder-Erweiterung ersetzt werden.
 \begin{center}
 $\fbox{
\parbox{14cm}{
\textcolor{red}{gap$>$} \textcolor{blue}{L:=[\,];;}\newline
\textcolor{red}{gap$>$} \textcolor{blue}{Combinations(Faces(DT),3);;}\newline
\textcolor{red}{gap$>$} \textcolor{blue}{List(last,g-$>$List(g,h-$>$VerticesOfFace(DoubleTet,h)));;}\newline
\textcolor{red}{gap$>$} \textcolor{blue}{Filtered(last,g-$>$Intersection(Union(g),[1,5])$<>$[]);;}\newline
\textcolor{red}{gap$>$} \textcolor{blue}{List(last,g-$>$List(g,h$->$Position(VerticesOfFaces(DoubleTet),h)));}\newline
[ [ 1, 2, 3 ], [ 1, 2, 4 ], [ 1, 2, 5 ], [ 1, 2, 6 ], [ 1, 3, 4 ],
  [ 1, 3, 5 ], [ 1, 3, 6 ], \newline
  [ 1, 4, 5 ], [ 1, 4, 6 ], [ 1, 5, 6 ],
  [ 2, 3, 4 ], [ 2, 3, 5 ], [ 2, 3, 6 ], [ 2, 4, 5 ], 
 \newline
  [ 2, 4, 6 ], [ 2, 5, 6 ], [ 3, 4, 5 ], [ 3, 4, 6 ], [ 3, 5, 6 ], [ 4, 5, 6 ] ]\newline
\textcolor{red}{gap$>$} for g in last do\newline
\textcolor{red}{$>$} s:=DoubleTet;\newline
\textcolor{red}{$>$} for f in g do\newline
\textcolor{red}{$>$} s:=AddTetraF(s,f);\newline
\textcolor{red}{$>$} od;\newline
\textcolor{red}{$>$} Add(L,s);\newline
\textcolor{red}{$>$} od;\newline
\textcolor{red}{gap$>$} \textcolor{blue}{IsomorphismRepresentatives(last);}\newline
[ simplicial surface (8 vertices, 18 edges, and 12 faces)
    ,\newline
     simplicial surface (8 vertices, 18 edges, and 12 faces)
    ,\newline
  simplicial surface (8 vertices, 18 edges, and 12 faces)
 ]
}}$ 
\end{center}  
Durch analoge Vorgehensweise erhalten wir
\begin{center}
\begin{tabular}{|c|c|c|}
\hline
 Typ&Tetraeder-  & Anzahl\\
&Zahl &\\
\hline
$(4,2)$&6& 3\\
\hline
$(5,2)$&7& 1\\
\hline
$(6,2)$&8& 1\\
\hline
\end{tabular}
\end{center}
 Wir werden jedoch sehen, dass die Enkel eines Multi-Tetraeders mit sehr viel mehr Systematik bestimmt werden können. Dies wird zu einem späteren Zeitpunkt bei den Multi-Tetraedern vom Typ $(2,2,1)$ vorgerechnet. Zuerst fassen wir aber die derzeitigen Erkenntnisse über die Tetraeder-Zahl und Typen von Multi-Tetraeder in der folgenden Tabelle zusammen. 
\begin{center}
\begin{tabular}{|c|c|c|}
\hline
Tetraeder- & Typ & Anzahl\\
Zahl& &\\
\hline
1 &$(1)$ &1 \\
\hline
2 &(2) &1\\
\hline
3 &$(2,1)$ &1\\
\hline
4 &$(3,1)$ & 1 \\
 & $(2,2)$& 2\\
\hline
5 & $(4,1)$& 1\\
 &$(3,2)$ & 3\\
 & $(2,2,1)$& ?\\
\hline
\end{tabular}
\end{center}

\begin{bemerkung}\label{bemgruppe}
Sei $X$ eine Sphäre und G die Automorphismengruppe von $X$. Dann kann eine Gruppenoperation auf $\Pot_k(X_2)$ für $k \leq \vert X_2 \vert  $ durch 
\[
G\times \Pot_k(X_2) \to \Pot_k(X_2),(\phi , M)\mapsto \{\phi(x)\mid x\in M\}
\]
definiert werden. Die Wohldefiniertheit dieser Gruppenoperation folgt dabei direkt, da $\vert M\vert=\vert\{\phi (x)\mid x\in M\}\vert$ für ein $M\in \Pot_k(X_2)$ und ein $\phi\in G$ gilt.
\end{bemerkung}
\begin{definition}
Sei $X$ ein Multi-Tetraeder vom Typ $(a_0,\ldots a_k)$. Wir nennen eine Menge $M\subseteq X_2$ eine \emph{Überdeckung}, falls für alle Ecken $V\in X_0$ vom Grad 3 ein $F\in M$ mit $F\in X_2(V)$ existiert. Die Überdeckung $M$ ist \emph{minimal}, falls $\vert M\vert=a_k$ gilt.
Weiterhin definieren wir für $0\leq l\leq \vert X_2\vert -a_k$ die Mengen der $a_k+l$-elementigen Überdeckungen als
 $U_X^l.$ 
\end{definition}
\begin{bsp}
\begin{itemize}
\item Für einen Tetraeder $T$ bildet jede nicht-leere Menge $ M \subseteq \Pot(T_2)$ eine Überdeckung. Die minimalen Überdeckungen sind genau die einelementigen Teilmengen von $T_2.$
\item Für den Doppel-Tetraeder definiert durch den Flächen-Träger 
\[
\xi=\Pot_3(\{1,2,3,4\})\Delta \Pot_3(\{2,3,4,5\})
\]
liefert beispielsweise $\{\{1,2,3\},\{2,3,5\}\}$ eine minimale Überdeckung. Vielmehr liefern alle Mengen der Form $\{m,m'\}\subset \xi$ mit $1\in m$ und $5\in m'$ minimale Überdeckungen des Doppel-Tetraeders.
\end{itemize}
\end{bsp}
\begin{lemma}
Sei $X$ ein Multi-Tetraeder und $\phi \in \Aut(X).$ Für eine Überdeckung $M\in U_X^l$ mit  $0\leq l\leq \vert X_2\vert -a_k$ ist $\{\phi(x)\mid x\in M\}$ wieder eine Überdeckung. \end{lemma}
\begin{proof}
 Da $\phi$ ein Automorphismus ist, gilt
klarerweise $\vert M\vert =\vert\{\phi(m)\mid m\in M\}\vert$.
 Aufgrund der Tatsache, dass $F\in M$ zu einer Ecke vom Grad 3 inzident ist, muss dies auch für $\phi(F)$ gelten. Somit werden also genau $a_k$ Ecken vom Grad 3 durch die $a_k+l$-elementige Menge $M'=\{\phi(m)\mid m\in M\}$ überdeckt. Also ist $M'\in U^l_X.$
\end{proof}
\begin{bemerkung}
Sei $X$ ein Multi-Tetraeder vom Typ $(a_0,\ldots ,a_k)$ und $G$ seine Automorphismengruppe.
Durch die obige Erkenntnis kann nun die in $\Cref{bemgruppe}$ eingeführte Gruppenoperation leicht abgeändert werden, um so die Anzahl der Enkel eines Multi-Tetraeders zu bestimmen. Durch obiges Lemma ist für $0\leq l\leq \vert X_2 \vert -a_k$ die Abbildung
\[
\theta_l: G\times U_X^l \to U_X^l, (\phi, M)\mapsto \phi(M):=\{\phi(F)\mid F\in M\}
\] 
wohldefiniert und es lässt sich leicht nachprüfen, dass diese eine Gruppenoperation bildet.
\end{bemerkung}
\begin{satz}
Seien $X$ ein Multi-Tetraeder vom Typ $(a_0,\ldots,a_k)$ und $0\leq l\leq \vert X_2\vert -a_k.$ Weiterhin seien $\theta_l$ die Gruppenoperation auf $U_X^l$ und $u_l$ die Anzahl der Bahnen der Gruppenoperation. Dann gibt es $u_l$ Enkel vom Typ $(a_k+l,a_k,\ldots,a_1).$
\end{satz}
\begin{proof}
Um aus $X$ einen Multi-Tetraeder vom Typ $(r,a_k,\ldots,a_1)$ zu konstruieren, muss bei allen $a_k$ Ecken vom Grad 3 mindestens eine inzidente Fläche durch ein Tetraeder ersetzt werden. Damit bildet für $M\in U_X^l$ das Erweitern von $X$ durch Tetraeder-Erweiterungen an allen Flächen $F\in M$ einen Enkel von $X$. Seien nun $G$ die Automorphismengruppe von $X$ und $M_1,M_2$ zwei Überdeckung, die in derselben Bahn unter $\theta_l$ liegen. Somit existiert ein $\phi \in G$ mit $\phi(M_1)=M_2.$ Sei $Y_1$ bzw. $Y_2$ der Multi-Tetraeder, der durch Tetraeder-Erweiterungen an allen Flächen $F\in M_1$ bzw. $F \in M_2$ entstanden ist. Wir können den Automorphismus $\phi$ auf $X$ nun zu einem Isomorphismus zwischen $Y_1$ und $Y_2$ umstrukturieren, indem die fehlenden Bilder der Tetraeder so ergänzt werden, dass Inzidenzen berücksichtigt werden. \\
Falls aber $F_1,F_2$ in zwei verschiedenen Bahnen liegen, aber dennoch ein Isomorphismus $\phi'$ zwischen $Y_1$ und $Y_2$ existiert, dann kann dieser zu einem Automorphismus $\phi'$ auf $X$ mit $\phi'(M_1)=M_2$ abgeändert werden, was ein Widerspruch ist.
\end{proof}
Nun können wir also die Multi-Tetraeder vom Typ $(2,2,1)$ bestimmen. Sei dazu $X$ der Multi-Tetraeder mit 8 Flächen.
Damit ist die Anzahl der Multi-Tetraeder vom Typ $(2,2,1)$ die Anzahl der Bahnen auf $U^0_X.$ Durch Auslagern der Rechnung in GAP ergibt sich, dass es genau solcher 4 Bahnen gibt.
\begin{folgerung}
Sei $X$ eine Sphäre vom Typ $(a_k,\ldots, a_0)$ und $0\leq l\leq \vert X_2\vert -a_k.$ Zudem sei $u_l$ die Anzahl der Bahnen der Gruppenoperation $\theta_l$ auf $U_X^l$. Dann ist 
\[
\sum_{i=l}^{\vert X_2\vert -a_k} u_l
\] die Anzahl der Blätter von $X.$
\end{folgerung}

Mithilfe von GAP und obiger Folgerung können wir bestimmen, wie viele Multi-Tetraeder es mit einer bestimmten Flächenanzahl gibt.
Die untenstehende Tabelle beinhaltet die Anzahlen der Multi-Tetraeder mit bis zu 28 Flächen.
\begin{center}
\begin{tabular}[h]{|c|c|c|c|c|c|c|c|c|c|c|c|c|}
\hline
\textbf{ 4} &  \textbf{6}& \textbf{8} &\textbf{ 10} &\textbf{ 12} & \textbf{14}&\textbf{16}&\textbf{18}&\textbf{20}&\textbf{22}&\textbf{24}&\textbf{26}&\textbf{28}\\
\hline
 1& 1& 1& 3& 7& 24& 93& 434& 2110& 11003& 58598& 321726& 1614848
 \\
 \hline
\end{tabular}
\end{center}

Da Tetraeder leicht zu kontrollieren sind, kann mit geringem Aufwand eine Einbettung eines Multi-Tetraeders in den $\mathbb{R}^3$ berechnet werden.
Um die Multi-Tetraeder in den drei-dimensionalen reellen Raum einzubetten, muss jeder Ecke eine reelle Koordinatenspalte zugeordnet werden. Diese Vorgehensweise soll an dieser Stelle formuliert werden:\\
Sei $X$ ein Multi-Tetraeder. Zur Vereinfachung bezeichnen wir die zu $V\in X_0$ zugehörige Koordinatenspalte mit $x_V\in \mathbb{R}^3.$
 Wir sind im Folgenden nur an Einbettungen von $X$ interessiert, in der die Kanten alle dieselbe Länge haben. Das heißt, es existiert ein $d\in \mathbb{R}^+,$ sodass  
\[
\|x_V-x_{V'}\|=d
\]
für alle benachbarten Ecken $V,V'\in X_0$ ist. Weiterhin  fordern wir $x_V\neq x_{V'}$ für Ecken $V\neq V'$ in $X.$ Eine solche Einbettung werden wir in Kapitel \ref{eigenwerte} als \emph{vertex-treu} bezeichnen.
Für einen Tetraeder $T$ mit $T_1=\{V_1,V_2,V_3,V_4\}$ ist eine reelle vertex-treue Einbettung beispielsweise durch
\[
x_{V_1}=\left(\begin{tabular}{c}
1\\
1\\
-1\\
\end{tabular}\right),\,
x_{V_2}=\left(\begin{tabular}{c}
1\\
-1\\
-1\\
\end{tabular}\right),\,
x_{V_3}=\left(\begin{tabular}{c}
-1\\
1\\
1\\
\end{tabular}\right),\,
x_{V_4}=\left(\begin{tabular}{c}
1\\
-1\\
1\\
\end{tabular}\right)
\]
gegeben.\\
Seien $x_{V_1},\ldots,x_{V_n}$ die Koordinatenspalten der Ecken in $X.$
Durch eine Tetraeder-Erweiterung an dem Multi-Tetraeder an der Stelle $F\in X_0$ erhalten wir eine neue Ecke $V',$ dessen Koordinatenspalte noch zu bestimmen ist. 
Betrachten wir also $V_1,V_2,V_3\in X_0(F)$ mit den zugehörigen Koordinatenspalten $x_{V_1},x_{V_2},x_{V_3}.$ Es existiert eine Ecke $V\in X_0,$ die zu allen Ecken der Fläche $F$ inzident ist. Wir erhalten die Koordinatenspalte $x_{V'}$ durch Spiegeln der Koordinatenspalte $x_V$ an der durch $x_{1},x_{2},x_{3}$ aufgespannten Ebene. \\
Wir sagen, dass der Multi-Tetraeder $X$ sich selbst \emph{durchbohrt}, falls es eine Fläche $F$ in $X$ mit $X_0(F)=\{v_1,v_2,v_3\}$ und eine Kante $e\in X_1$ mit $X_0(e)=\{v,v'\}$ gibt, sodass die durch $x_{v_1},x_{v_2},x_{v_3}$ aufgespannte Ebene und die durch $x_v,v_{v'}$ definierte Gerade  einen Schnittpunkt haben. Zusätzlich muss dieser Schnittpunkt auf der Geraden echt zwischen $x_{v}$ und $x_{v'}$ liegen und sich als eine echte konvex Kombination von $x_{v_1},x_{v_2},x_{v_3}$ schreiben lassen. 
Mithilfe von Gap lässt leicht nachrechnen, ob die reelle vertex-treue Einbettung eines Multi-Tetraeders in den $\mathbb{R}^3$ sich selbst durchbohrt.
\begin{comment}
\textbf{In Gap:}
\begin{center}
$
   \fbox{
\parbox{13.4cm}{
\begin{tabbing}
\textcolor{blue}{gap$>$}Is\=Embeddible:=function(S)\\
\textcolor{red}{$>$}\>local temp, voe,voeD,voeE,voeDE,sym,vof,\\
\textcolor{red}{$>$}\>vofA,vofB,vofC,vofAC,vofAB,tempS,sol,coordinates,tempvof;\\
\textcolor{red}{$>$}\>tempS:=S;\\
\textcolor{red}{$>$}\>temp:=CoorMul(tempS);\\
\textcolor{red}{$>$}\>coordinates:=temp[2];\\
\textcolor{red}{$>$}\>tempS:=temp[1];\\
\textcolor{red}{$>$}\>for \=voe in VerticesOfEdges(tempS) do\\
\textcolor{red}{$>$}\>\> tempvof:=Filtered(VerticesOfFaces(tempS),
g-$>$ Intersection(voe,g)=[]);\\
\textcolor{red}{$>$}\>\>for \=vof in tempvof do\\
\textcolor{red}{$>$}\>\>\>vofA:=coordinates[vof[1]];\\
\textcolor{red}{$>$}\>\>\>vofB:=coordinates[vof[2]];\\
\textcolor{red}{$>$}\>\>\>vofC:=coordinates[vof[3]];\\
\textcolor{red}{$>$}\>\>\>voeD:=coordinates[voe[1]];\\
\textcolor{red}{$>$}\>\>\>voeE:=coordinates[voe[2]];\\
\textcolor{red}{$>$}\>\>\>vofAB:=vofB$-$vofA;\\
\textcolor{red}{$>$}\>\>\>vofAC:=vofC$-$vofA;\\
\textcolor{red}{$>$}\>\>\>voeDE:=voeE$-$voeD;\\
\textcolor{red}{$>$}\>\>\>sol:=SolutionMat([vofAB,vofAC,$-$voeDE],voeD$-$vofA);\\
\textcolor{red}{$>$}\>\>\>if n\=ot sol=fail then\\
\textcolor{red}{$>$}\>\>\>\>if (sol[1]$>$0. and sol[1]$<$1.) and (sol[2]$>$0. and sol[2]$<$1.) and \\
\textcolor{red}{$>$}\>\>\>\>(sol[3]$>$0.
and sol[3]$<$1.) and sol[1]+sol[2]$<$=1. \\
\textcolor{red}{$>$}\>\>\>\>and sol[1]+sol[2]$>$0. then\=\\
\textcolor{red}{$>$}\>\>\>\>\>return false;\\
\textcolor{red}{$>$}\>\>\>\>fi;\\
\textcolor{red}{$>$}\>\>\>fi;\\
\textcolor{red}{$>$}\>\>od;\\
\textcolor{red}{$>$}\>od;\\
\textcolor{red}{$>$}\>return true;\\
\textcolor{red}{$>$}end;
\end{tabbing}
}}$
\end{center}
\end{comment}
Die folgende Tabelle präsentiert die Anzahlen der Multi-Tetraeder mit geringer Flächenanzahl, die sich bei reeller vertex-treuer Einbettung nicht selbst durchbohren.  
\begin{center}
\begin{tabular}{|c|c|c|c|c|c|c|c|c|c|c|}
\hline
$\vert\textbf{X}_{\textbf{2}}\vert$&\textbf{4}&\textbf{6}&\textbf{8}&\textbf{10}&\textbf{12}&\textbf{14}&\textbf{16}&\textbf{18}&\textbf{20}&\textbf{22}\\
\hline
$\#$&1&1&1&3&7&23&89&398&1859&9161\\
\hline
\end{tabular}
\end{center} 
An dieser Stelle wollen wir ein Symbol für Tetraeder einführen, welches uns einen besseren Zugang zu dieser Klasse der Sphären liefern soll: \\
\begin{itemize}
\item Durch das leere Symbol $()$ wird der Tetraeder repräsentiert. 
\item Für $i \in \{1,2,3,4\}$ bezeichnen wir die folgende Sphäre mit $1_i$: 
Sei $T$ ein Tetraeder mit $T_2=\{1_1,1_2,1_3,1_4\}.$ Sei $Y=T^{1_i}(T)$ die Sphäre, die durch eine Tetraeder-Erweiterung an $T$ entsteht. Wir bezeichnen die neu entstandenen Flächen der Sphäre mit $2_j$ für $j\neq i.$
 Außerdem fordern wir, dass $1_j$ und $2_j$ benachbarte Flächen sind. Beispielsweise ist die Flächenmenge von dem Multi-Tetraeder $1_1$ durch $\{1_2,1_3,1_4,2_2,2_3,2_4\}$ gegeben, wobei für eine Kante $e\in Y_1,$ die inzident zu zwei Ecken vom Grad 4 ist, folgendes gilt:
\[
X_2(e)\in\{\{1_2,2_2\},\{1_3,2_3\},\{1_4,2_4\}\}
\] 
\item Sei $1_i\ldots m_j$ ein Symbol der Länge $n-1$ mit bereits konstruiertem Multi-Tetraeder $X$.
 Eine Fläche in $X$ liegt in der Gestalt $l_k$ für $l\leq n-1$ und $k\in \{1,2,3,4\}$ vor. Mit $1_i\ldots m_jl_k$ den Multi-Tetraeder der durch $T^{l_k}(X)$ entsteht. Die neuen Flächen des Tetraeders benennen wir mit $n_r$ für $1\leq i\leq 4, r\neq i,$ sodass die Flächen des zuletzt angehängten Tetraeders und die zugehörige Nachbar-Fläche in $X$ denselben Index haben. 
\end{itemize}
\begin{bemerkung}
\begin{itemize}
\item Es gibt bis auf Isomorphie genau einen Multi-Tetraeder mit 6 Flächen und einem Symbol der Länge 1.
\item Bis auf Isomorphie gibt es genau einen Multi-Tetraeder mit Flächen und einem Symbol der Länge 2.
\item Im obigen Kontext hat der Ausdruck $1_1$ zwei Bedeutungen. Zum Einen ist hiermit eine Fläche des Tetraeders $T$ gemeint und zum anderen der Doppel-Tetraeder, der durch eine Tetraeder-Erweiterung an der Fläche $1_1$ entsteht.  
\end{itemize}
\end{bemerkung}
Für Multi-Tetraeder mit einer geringen Anzahl an Flächen wollen wir nun elementare Eigenschaften angeben.
\begin{center}
{\footnotesize
\begin{tabular}[h]{|c|c|c|c|c|}
\hline
$\vert\textbf{X}_{\textbf{2}}\vert$ & \textbf{Symbol} & \textbf{Vertex-}& \textbf{Flächenzähler} & \textbf{Aut.} \\
 &&\textbf{zähler}&& \textbf{gruppe}\\
\hline
\textbf{4} & $()$ &$v_3^4$ & $f_{3^3}^4$ &$S_4$\\
\hline
\textbf{6} & $1_1$ & $v_3^2v_4^3$&$f^6_{3,4^2}$ &$C_2\times D_6$\\
\hline
\textbf{8} & $1_11_2$&$ v_3^2v_4^3$& $f^4_{3,4,5}f^2_{3,5^2}f^2_{4^2,5}$ & $D_4$\\
\hline  
  & $1_11_21_3$ & $v_3^3v_5^3v_6^1$& $f^3_{3,5^2}f^6_{3,5,6}f^1_{5^3}$ &$D_6$\\
\textbf{10}& $1_11_32_1$ &$v_3^2v_4^2v_5^2v_6^1$ & $f^2_{3,4,5}f^2_{3,4,6}f^2_{3,5,6}f^2_{4,5^2}f^2_{4,5,6}$ & $C_2$\\
  & $1_11_22_2$ &$v_3^2v_4^3v_6^2$& $f^4_{3,4,6}f^2_{3,6^2}f^4_{4^2,6}$ &$D_4$\\
\hline
  & $1_11_32_32_2$&$v_3^3v_4^1v_5^2v_6^1v_7^1$& $f^1_{3,4,6}f^1_{3,4,7}f^1_{3,5^2}f^1_{3,5,6}f^3_{3,5,7}f^2_{3,6,7}f^1_{4,5,6}f^1_{4,5,7}f^1_{5^2,6}$ &$\{id\}$\\
  & $1_11_21_31_4$& $v_3^4v_4^6$& $f^{12}_{3,6^2}$ &$S_4$\\
  & $1_12_41_32_2$&$v_3^3v_4^1v_5^1v_6^3$& $f^2_{3,4,6}f^4_{3,5,6}f^3_{3,6^2}f^2_{4,6^2}f^1_{5,6^2}$ & $C_2$\\
\textbf{12}& $1_21_12_43_2$&$v_3^2v_4^3v_5^1v_6^1v_7^1$& $f^1_{3,4,5}f^1_{3,4,6}f^2_{3,4,7}f^1_{3,5,7}f^1_{3,6,7}f^1_{4^2,6}f^1_{4^2,7}f^2_{4,5,6}f^1_{4,5,7}f^1_{4,6,7}$ &$\{id\}$\\
  & $1_21_12_43_3$& $v_3^2v_4^2v_5^2v_6^2$& $f^2_{3,4,5}f^2_{3,4,6}f^2_{3,5,6}f^2_{4,5,6}f^2_{4,6^2}f^2_{5^2,6}$&$C_2$\\
  & $1_31_22_43_4$& $v_3^2v_4^2v_5^3v_7^1$& $f^2_{3,4,5}f^2_{4,5,7}f^1_{5^3}f^1_{5^2,7}$&$C_2$\\
  & $1_31_22_23_3$& $v_3^2v_4^4v_7^2$&$f^4_{3,4,7}f^2_{3,7^2}f^6_{4^2,7}$ &$D_4$\\
 \hline
\end{tabular}
}
\end{center}
Beim Betrachten der Automorphismengruppen ist zu erkennen, dass für diese kleineren Beispiele die Automorphismengruppen auflösbar sind. Dieses Resultat lässt sich verallgemeinern. 
\begin{lemma}
Sei $X$ ein Multi-Tetraeder und $\Aut(X)$ die Automorphismengruppe von $X$. Dann ist $\Aut(X)$ auflösbar.
Weiterhin seien $l,l'$ minimal mit der Eigenschaft, dass
\[
\vert \Aut(X)^l\vert=\vert \Aut(X^{(1)})^{l'}\vert =1
\] ist. Dann gilt $l\leq l'$.
\end{lemma}
\begin{proof}
Wir führen den Beweis induktiv. Die Automorphismengruppe des Tetraeders ist die symmetrische Gruppe der Ordnung 24 und die des Doppel-Tetraeders ist das direkte Produkt einer $C_2$ und der symmetrischen Gruppe der Ordnung 6, also insgesamt Ordnung 12. Bei beiden Gruppen lässt sich leicht nachrechnen, dass sie auflösbar sind.

Sei $X$ nun ein Multi-Tetraeder mit mehr als  6 Flächen und $G=\Aut(X)$. Für die Ecken $V_1,\ldots,V_k$ vom Grad $3$ definieren wir zudem 
\[
M_t:=\{V_t\} \cup X_1(V_t) \cup X_2(V_t).  
\] 
Da Ecken unter einem Automorphismus auf Ecken mit demselben Grad abgebildet werden, müssen die Ecken vom Grad 3 untereinander permutiert werden. Vielmehr gilt 
\[
\phi(M_i)=M_{\pi(i)} 
\]
für alle $\phi \in G,\,i\in \{1,\ldots,k\}$ und eine geeignete Permutation $\pi.$ Sei weiterhin $Y=X^{(1)}$, die Sphäre, die durch Entfernen aller Tetraeder von $X$ entsteht und $F_1,\ldots,F_k$ die Flächen, die die Tetraeder an den Stellen $V_1,\ldots,V_k$ ersetzen. Wir betrachten nun den Homomorphismus $\psi:G\mapsto \Aut(Y),$ der durch  
\[
\psi(\phi)(x)=\Biggl\{
\begin{tabular}[l]{lcr}
$F_j$,&\textcolor{black}{ falls  $x\in M_j$} \\
x,& sonst\\
\end{tabular}
\]
definiert wird. Es ist leicht nachzurechnen, dass $\psi(G)$ eine Untergruppe von $\Aut(Y)$ bildet. Da nach Induktionsvoraussetzung $\Aut(Y)$ auflösbar ist, ist $\psi(G)$ als Untergruppe einer auflösbaren Gruppe ebenfalls auflösbar. 
Sei also nun $l$ minimal mit der Eigenschaft, dass $(\Aut(\psi(G))^l=\{id\}$ ist. Dann gilt für alle $x\in X-\bigcup_{t=1}^k M_t$ die Gleichheit 
\[
\phi_l(x)=\psi(\phi_l)(x)=x
\] für alle $\phi_l \in G^l$. Da $\phi_l$ die Inzidenzen in der simplizialen Fläche $X$ respektiert, gilt $\phi_l(M_i)=M_i$ und genauer sogar $\phi_l=id.$ Damit ist $G^l =\{id\}$.
  
\end{proof}



\subsection{Kaktus-Distanz}
In diesem Abschnitt werden wir wie bereits erwähnt die Kaktus-Distanz genauer betrachten und einige Beispiele der Berechnung der Kaktus-Distanz von Sphären vorstellen.
Vorerst führen wir aber die Definition ein.
\begin{definition}
Sei $X$ eine vertex-treue Sphäre. Die minimale Anzahl an Kantendrehungen, die es braucht, um aus $X$ einen Multi-Tetraeder zu  konstruieren, nennen wir die \emph{Kaktus-Distanz} $\zeta(X)$ von $X$.
\end{definition}
\begin{bsp}
\begin{itemize}
\item Für jeden Multi-Tetraeder $X$ gilt $\zeta(X)=0.$
\item Der Oktaeder ist eine vertex-treue Sphäre mit Kaktus-Distanz $1.$ Dies zeigt sich durch eine simple Rechnung in GAP:
\begin{center}
$\fbox{
\parbox{13cm}{
\textcolor{red}{gap$>$} \textcolor{blue}{IsCactus(O);}\\
false\\
\textcolor{red}{gap$>$}\textcolor{blue}{ O;}\\
simplicial surface (6 vertices, 12 edges, and 8 faces)\\
\textcolor{red}{gap$>$} \textcolor{blue}{EdgeTurn(O,1);}\\
simplicial surface (6 vertices, 12 edges, and 8 faces)\\
\textcolor{red}{gap$>$} \textcolor{blue}{IsCactus(last);}\\
true
}}$
\end{center}
\end{itemize}
\end{bsp}
\begin{lemma}
Für alle Sphären ist die Kaktus-Distanz endlich.
\end{lemma}
\begin{proof}
\begin{comment}
\item Man führt den Beweis per vollständiger Induktion.
Für $n=4$ hat man den Tetraeder als Multi-Tetraeder. Man nimmt nun an, es gibt Multi-Tetraeder $X$ mit $\vert X_2\vert=n$ und will nun die Existenz eines Multi-Tetraeders $Y$ mit $\vert Y_2\vert =n+2$ nachweisen.Da Multi-Tetraeder vertex-treu sind existiert ein Flächen-Träger $\zeta \subseteq \{1,\ldots,\vert X_0\vert\}$. Man definiert nun den Flächen-Träger eines Tetraeders $\xi:=\Pot_3(\{\vert X_0\vert +1\}\cup A)$ für ein $A\in \zeta.$ Dann ist $\zeta \Delta \xi$ der Flächen-Trägers des Multi Tetraeders $\mathcal{S}(\zeta \Delta \xi)$ vom Typ $(a_0+1,\ldots,a_k)$ oder $(1,a_0,\ldots,a_k)$, wobei $(a_0+1,\ldots,a_k)$ der Typ von $X$ ist. Die Flächenanzahl von $Y$ ist
\[
\vert Y_2 \vert=\vert X_2 \vert+4-2=n+2.
\]
\end{comment}
Sei $X$ eine Sphäre mit $n$ Flächen. Für jedes $n$ existiert ein  Multi-Tetraeder $Y$ mit $\vert Y_2\vert =n .$ Wegen \Cref{kantendrehung} existiert eine Kantensequenz $E=(e_1,\ldots,e_n)$ mit $X^E\cong Y$. Daraus folgt $\zeta(X)\leq n<\infty.$ 

\end{proof}
Es ist klar, dass die simplizialen Flächen mit Kaktus-Distanz $0$ genau die Multi-Tetraeder sind. Über die Sphären der Kaktus-Distanz 1 wurde ebenfalls bereits im oben genannten Skript eine Erkenntnis erlangt.

\begin{lemma}
Sei $X$ eine vertex-treue Sphäre ohne 3-Taillen, die  sich durch eine Kantendrehung in einen Multi-Tetraeder umformen lässt. Dann ist $X$  zum Doppel-$n$-Gon isomorph.
\end{lemma}
\begin{bemerkung}
Wenn die Voraussetzung fallen gelassen wird, dass $X$ eine Sphäre ohne 3-Taillen ist, so ist die Aussage falsch. 
\begin{itemize}
\item Der simpliziale Parallelepiped $P$ aus \Cref{bspCactus} ist eine vertex-treue Sphäre mit Kaktus-Distanz 1, denn das Drehen der Kante $e\in P_1$ mit 
\[
P_0(e)\in\{\{1,4\},\{1,5\}\{2,5\},\{2,6\},\{3,4\},\{3,6\},\}
\] liefert einen Multi-Tetraeder vom Typ $(2,2,1).$
\item Nutzen wir GAP zum Erzeugen einer Sphäre mit Kaktus-Distanz 1, die nicht isomorph zum Doppel n-Gon isomorph ist. \\\\
\fbox{
\parbox{13.4cm}{
\textcolor{red}{$gap>$} \textcolor{blue}{$L:=[[ 2, 3, 5 ], [ 2, 4, 5 ], [ 3, 4, 5 ], [ 1, 3, 6 ], [ 1, 4, 6 ], 
  [ 3, 4, 6 ], [ 1, 7, 8 ], [ 1, 4, 7 ],\newline [ 2, 4, 7 ], [ 2, 7, 8 ], [ 1, 3, 8 ], [ 2, 3, 8 ] ];;$}
  }}\\\\
  Durch den Flächen-Träger $L$ definieren wir die simpliziale Fläche $S$.\\\\
  \fbox{
\parbox{13.4cm}{
\textcolor{red}{gap$>$} \textcolor{blue}{SimplicialSurfaceByVerticesInFaces(L);}\newline
simplicial surface (8 vertices, 18 edges, and 12 faces)\newline
\textcolor{red}{gap$>$} \textcolor{blue}{$S:=last;$}\newline
simplicial surface (8 vertices, 18 edges, and 12 faces)\newline
\textcolor{red}{gap$>$}\textcolor{blue}{ FaceDegreesOfVertices(S);}\newline
[ 5, 5, 6, 6, 3, 3, 4, 4 ]\newline
\textcolor{red}{gap$>$} \textcolor{blue}{$IsCactus(S);$}\newline
$false$
}}\\\\
\begin{figure}[H]
\begin{center}
\includegraphics[viewport=0cm 22.cm 10cm 26cm]{TetExtOct}
\end{center}
\caption{Parallelepiped}
\end{figure}
Durch das Drehen der Kante, die zu den beiden Ecken vom Grad 4 inzident ist, erhalten wir einen Multi-Tetraeder.\\\\
$\fbox{
\parbox{13.4cm}{
\textcolor{red}{gap$>$}\textcolor{blue}{ VerticesOfEdges(S);}\newline
$[ [ 1, 3 ], [ 1, 4 ], [ 1, 6 ], [ 1, 7 ], [ 1, 8 ], [ 2, 3 ], [ 2, 4 ], [ 2, 5 ], [ 2, 7 ], [ 2, 8 ], [ 3, 4 ], [ 3, 5 ], [ 3, 6 ], \newline
 [ 3, 8 ],
  [ 4, 5 ], [ 4, 6 ], [ 4, 7 ], [ 7, 8 ] ]$\newline
\textcolor{red}{gap$>$} \textcolor{blue}{EdgeTurn(S,18);}\newline
simplicial surface (8 vertices, 18 edges, and 12 faces)\newline
\textcolor{red}{gap$>$} \textcolor{blue}{IsCactus(last);}\newline
$true$
}}$
\end{itemize}
Die obigen Sphären sind nicht isomorph und gehen aus dem Oktaeder durch das Durchführen von genau zwei Tetraeder-Erweiterungen an verschiedenen Flächen hervor.  
\end{bemerkung}
Wir nutzen die Doppel-$n$-Gons, um eine erste Abschätzung der Kaktus-Distanz einer Sphäre zu erhalten. Wir können nämlich durch gezielte Kantendrehungen aus jeder Sphäre den Doppel-$n$-Gon konstruieren. Doch vorher betrachten wir 

\begin{satz}\label{ngon}
Sei $(X,<)$ eine vertex-treue Sphäre und $V_1,V_2$ zwei nicht benachbarte Ecken, die  $X_0(X_1(V_1))\cap X_0(X_1(V_2))\neq \emptyset$ und $\deg(V_1)=\deg(V_2)= n$ erfüllen. Dann ist $X$ isomorph zum Doppel-$n$-Gon.
\end{satz}
\begin{proof}
Wir  führen den Beweis per vollständiger Induktion. Für $n=3$ finden wir den Doppel-3-Gon als einzige Sphäre, die die Behauptung erfüllt. Nehmen wir nun an, dass es eine vertex-treue Sphäre $X$ und Ecken $V_1,V_2\in X_0$ gibt, die die Voraussetzung erfüllen. Somit gibt es eine Ecke $V$ in $X$ und Kanten $e_1\in X_1(V_1)$ und $e_2 \in X_1(V_2)$ in $X$, die $X_0(e_1)\cap X_0(e_2)=\{V\}$ erfüllen.
 Ziel ist es, $X\cong (n+1)^2$ zu zeigen. Hierfür nutzen wir die zuvor eingeführte Schmetterlings-Entfernung. Wegen der fehlenden Adjazenz von $V_1$ und $V_2$ und $\deg(V_1)=\deg(V_2)= n,$ ist jede Fläche $F\in X_2$ entweder zu $V_1$ oder $V_2$ inzident. Somit existiert also eine Kante $e\in X_1$, sodass $\vert X_0(e)\cap X_0(e_1)\cap X_0(e_2)\vert  =\{V\}$
 ist und $V_1,V_2$ in $X_0(X_2(e))$ enthalten sind, woraus die Wohldefiniertheit der Sphäre ${{}^e\beta(X)}$ folgt. In dieser Sphäre gibt es zwei nicht benachbarte Ecken $V_1',V_2'$, die durch die Schmetterlings-Entfernung aus $V_1$ und $V_2$ hervorgehen und deshalb die Grade $\deg(V_1')=\deg(V_2')=n+1-1=n$ besitzen. Damit gilt ${{}^e\beta(X)}\cong (n)^2$ und dies führt schließlich dazu, dass 
 \[
X\cong (n+1)^2 
 \]
 gefolgert werden kann.
\end{proof}
\begin{comment}
\begin{center}
$\fbox{
\parbox{13.4cm}{
\begin{tabbing}
\textcolor{blue}{gap}\textcolor{red}{$>$}Schr\=anke:=function(S)\\
\textcolor{red}{$>$}\> local g,tempV;\\
\textcolor{red}{$>$}\> tempV:=[ ];\\
\textcolor{red}{$>$}\> vert:=VerticesOfEdges(S);\\
\textcolor{red}{$>$}\> for \=v in vert do\\
\textcolor{red}{$>$}\>\> temp:=Filtered(vert,g$->$ Length(Intersection(v,g))=1);\\
\textcolor{red}{$>$}\>\> temp:=Filtered(temp,g$->$not Union(g,v) in VerticesOfFaces(S));\\
\textcolor{red}{$>$}\>\> temp:=List(temp,g$->$ Difference(Union(g,v),Intersection(v,g)));\\
\textcolor{red}{$>$}\>\> Append(tempV,temp);\\
\textcolor{red}{$>$}\> od;\\
\textcolor{red}{$>$}\> tempV:=Set(tempV);\\
\textcolor{red}{$>$}\> tempV:=List(tempV,v$->$\\ \textcolor{red}{$>$}\>\>AbsoluteValue(NumberOfFaces(S)/2-FaceDegreeOfVertex(S,v[1]))+\\
\textcolor{red}{$>$}\>\> AbsoluteValue(NumberOfFaces(S)/2-FaceDegreeOfVertex(S,v[2])));\\
\textcolor{red}{$>$}\> return Minimum(tempV)+1;\\
\textcolor{red}{$>$}end;\\
\end{tabbing}
}}$
\end{center}
\end{comment}
Diese Beobachtung wird nützlich sein, um erste Beobachtungen über die Kaktus-Distanz einer vertex-treuen Sphäre anzustellen. Hierfür definieren wir anlehnend an den obigen Satz die Menge $D$ als die Menge aller Tupel $(V_1,V_2)\in X_0 \times X_0$, die nicht benachbart sind und $ X_0(X_1(V_1))\cap X_0(X_1(V_2))\neq \emptyset$ erfüllen.
\begin{satz}
Sei $(X,<)$ eine vertex-treue Sphäre, die kein Multi-Tetraeder ist. Dann ist 
\[
\zeta(X)\leq m+1,
\] wobei $m:=min_{(V,V')\in D}\{\vert\frac{\vert X_2 \vert}{2}-\deg_X(V)\vert +\vert \frac{\vert X_2 \vert}{2}-\deg_X(V')\vert\}$
 ist.
\end{satz}
\begin{proof}
Seien $V_1$ und $X_2$ Ecken in $X,$ sodass $(V_1,V_2)\in D$ ein Ecken-Paar mit 
$\vert\frac{\vert X_2 \vert}{2}-\deg_X(V_1)\vert +\vert \frac{\vert X_2 \vert}{2}-\deg(V_2)_X\vert=m$ ist.
Wir führen nun für $V \in \{V_1,V_2\}$ und $l:=\frac{\vert X_2 \vert}{2}-\deg_X(V)$ folgende Fallunterscheidung durch:
\begin{itemize}
\item Falls $l=0$ ist, ist nichts zu tun.
\item Sei $l>0$. Dann existieren Flächen $F_1,F_2 \in X_2(V)$ und eine Kante $e\in X_1(V)$, die zu den Flächen $F_1$ und $F_2$ inzident ist. Damit ist $V$ eine Ecke, die in der Sphäre $X^e$ die Gleichung $\deg_{X^e}(V)=l-1$ erfüllt. Das $l-1$-fache Anwenden der obigen Prozedur liefert uns eine durch eine Kantensequenz entstandene Sphäre $Y$, in der  $\deg_Y(V)=0$ erfüllt ist. 
 \item Falls $l<0$ der Fall ist, gibt es eine Kante $e\in X_1$ und Flächen $F_1,F_2$ in $X$ mit $X_2(e)=\{F_1,F_2\}$. Zudem gilt $F_1\in X_2(V)$ und $V_1,V_2 \notin X_0(F_2)$. Somit liefert die Kantendrehung an der Kante $e$ eine Sphäre $X^e,$ in der $\deg_{X^e}(V)=l+1$ gilt. Hier liefert uns erneut das $l+1$-fache Anwenden dieser Prozedur eine durch eine Kantensequenz entstandene Sphäre $Y,$ in der  $\deg_Y(V)=0$ erfüllt ist.
\end{itemize}  
 Anwenden der in der Fallunterscheidung vorgestellten Prozeduren auf $V_1$ und $V_2$ bringt uns also eine durch eine Kantensequenz entstandene simpliziale Fläche $Z$, in der $\vert\frac{\vert X_2 \vert}{2}-\deg_Z(V)\vert +\vert \frac{\vert X_2 \vert}{2}-\deg_Z(V')\vert=0$ ist. Mit obigem Satz kann also $Z\cong (n)^2$ gefolgert werden. Deshalb ist $\zeta(Z)=1,$ was also $\zeta(X)\leq m+1$ bedeutet. 
\end{proof}

Für kleine natürliche Zahlen $n$ liefert der oben skizzierte Algorithmus die exakte Kaktus-Distanz, wie folgende Folgerung zusammenfasst. 
\begin{folgerung}
Seien $X$ eine Sphäre, die kein Multi-Tetraeder ist und $m$ definiert wie im obigem Satz. Dann gilt 
\[
\zeta(X)= m+1
\]
 für $\vert X_2 \vert \in \{4,6,8\}$ 
\end{folgerung}
\begin{proof}
Für $\vert X_2 \vert \in \{4,6,8\}$ gibt es bis auf Isomorphie nur einen Multi-Tetraeder $Y$ mit $\vert Y_2\vert=\vert X_2\vert.$
\end{proof}

\begin{bemerkung}
Beim Betrachten des Terms $m$ ergibt sich
\begin{align*}
m=&min_{(V,V')\in D}\{\vert\frac{\vert X_2 \vert}{2}-\deg_X(V)\vert +\vert \frac{\vert X_2 \vert}{2}-\deg_X(V')\vert\}\\
\leq& \vert\frac{\vert X_2 \vert}{2}-\vert X_2\vert\vert +\vert \frac{\vert X_2 \vert}{2}-\vert X_2\vert\vert \\
=&\frac{\vert X_2 \vert}{2}+\frac{\vert X_2 \vert}{2}\\
=&\vert X_2 \vert.
\end{align*}
Bei genauerem Hinschauen ist zu erkennen, dass für diese Abschätzung die Gleichheit $\deg_X(V)=\deg_X(V')=\vert X_2 \vert$ verwendet wurde. Dies ist aber genau dann der Fall, wenn $X$ isomorph zu der im Folgenden definierte $n-$fachen Tasche ist. 
Damit ist $\zeta (X)< \vert X_2 \vert$ für alle vertex-treuen Sphären $X$. 
\end{bemerkung}

\begin{definition}
Sei $n\geq 1$. Wir definieren die simpliziale Fläche $n$-fache Tasche durch folgende Konstruktion:
\begin{itemize}
\item Für $n=1$ ist der $OB^1=OB$, wobei OB die ist.
\item Für $n=2$ ist $OB^2$, die durch die Kantendrehung $e\in T_1$ entstandene Sphäre $T^e$.
\item Für $n\geq 3$ nutzen wir den Doppel-$n$-Gon, um die gewünschte Sphäre zu konstruieren. Seien $V_1,V_2$ die nicht benachbarten Ecken vom Grad $n$ und $e_1,\ldots, e_n$ die Kanten, die weder zu $V_1$ oder $V_2$ inzident sind. Dann ist Sphäre $OB^n$ gegeben durch
\[
{((n)^2)}^{[e_1,\ldots,e_n]}
\]
\end{itemize}
\begin{bemerkung}
Für $n\geq 2$ besitzt die Sphäre $OB^n$ genau 2 Ecken mit Flächengrad $2n$ und die restlichen $n$ Ecken haben alle den Grad 2. Damit ist $OB^n$ nicht vertex-treu, da sie $n$ 2-Taillen enthält.
\end{bemerkung}
Wenn wir den Begriff der Kaktus-Distanz an dieser Stelle auf nicht vertex-treue Sphären ausweiten, erhalten wir hier ein erstes Beispiel für eine Klasse von Sphären, deren Kaktus-Distanz wir bestimmen können. 
\begin{comment}
Sei $(OB^i,$ nun für $i\in \{2,\ldots,n-1\}$ konstruiert. Dann definiert man $OB^{i+1}$ wie folgt:
Man schaut sich zunächst die simpliziale Fläche $(X,<)$ mit
\begin{align*}
X_0:=OB^i_0 \cup Y_0\\
X_1:=OB^i_1\cup Y_1\\
X_2:=OB^i_2 \cup Y_2,
\end{align*}
wobei $Y$ eine simpliziale Fläche ist, die isomorph zur offenen Tasche ist und $Y \cap OB^i=\emptyset $ erfüllt. Sei $(e_1,e_2)$ die 2-Taille in $Y\subset X$ 
mit zugehörigen Knoten $V_1,V_2 \in Y_0\subset X_0$ und $e=\{ e',e''\}\in OB^i_1\subset X_1$ eine Kante mit $X_1(e)=\{V,V'\}$ und $3 \notin \{deg_X(V),deg_X(V')\}$.
Man führt nun folgende Mender und Mutteroperationen durch:
\begin{enumerate}
\item Man wendet den Krater Cut $W:=C^C_{\{e',e''\}}(X)$ an und erhält die Kanten $e',e''\in W_1$ mit 
\begin{align*}
W_0(e') = W_0(e'')=\{V_1,V_2\}
\end{align*}
\item Durch den Split Mender entstehen die simpliziale Fläche  $Z:=S^m_{(V_1,e_1),(V,e')}(W)$, in der die Knoten $\{V_1,V\},\{V_2,V'\}$ und Kante $\{e_1,e'\}$ folgende Inzidenzen erfüllen:
\begin{align*}
Z_0(\{e_1,e'\}) = Z_0(e_2) = Z_0(e'')=\{\{V_1,V\},\{V_2,V'\}\}
\end{align*}
\item Schlussendlich gilt in $OB^{i+1}:=C^m_{e_2,e''}(Z)$, dass
\[
OB^{i+1}_0(\{e_1,e'\})=OB^{i+1}_0(\{e_2,e''\})=\{\{V_1,V\},\{V_2,V'\}\}
\]
ist.
\end{enumerate}
\end{comment}
\end{definition}


\begin{lemma}
Sei $X$ eine Sphäre, die für ein $n>1$ zu $OB^n$ isomorph ist, dann ist $\zeta(X)=n-1.$
\end{lemma}
\begin{proof}
Um die Aussage zu zeigen, orientieren wir uns an der Konstruktion der Sphäre und nutzen die Distanz von $OB^n$ zum Doppel-$n$-Gon. Seien $V_1,V_2\in X_0$ mit $\deg(V_1)=\deg(V_2)=2n$. Hierzu existieren $n$ Kanten $e_1,\ldots,e_n$, sodass 
\[
X_0(e_1)=X_0(e_2)=\ldots=X_0(e_n)=\{V_1,V_2\}
\] gilt.
In der Sphäre $X^{e_1}$ gilt nun $\deg(V_1)=\deg(V_2)=2n-1$. Also erhalten wir durch die Kantensequenz $E=(e_1,\ldots,e_{n-1})$ die Sphäre $Y$, die 
\[
Y^{e_n}\cong (n)^2
\] erfüllt. Die Kante $e_n$ ist in $Y^{e_n}$ eine Kante, die zwei Kanten vom Grad 4 verbindet. Also muss schon ${(Y^{e_n})}^{e_n}\cong Y$ ein Multi-Tetraeder sein und
 es folgt $\zeta(X)\leq n-1$. \\
Da eine Sphäre mit einem Ecken von Grad 2 kein Multi-Tetraeder sein kann, muss der Grad der $n$ Ecken vom Grad 2 durch Kantendrehungen angehoben werden. Es braucht mindestens $n-1$ Kantendrehungen, um aus $X$ eine Sphäre zu erhalten, die keine Ecken von Grad 2 besitzt. Somit folgt $\zeta(X)=n-1.$
\end{proof}
\begin{folgerung}
Für jedes $n \in \mathbb{N}$ gibt es eine Sphäre mit $X$ mit $\zeta(X)=n.$
\end{folgerung}
\begin{proof}
Die $n+1$-fache Tasche liefert die Behauptung, da $\zeta(OB^{n+1})=n+1-1=n$ ist.
\end{proof}

\begin{lemma}
Sei $n\in \mathbb{N}$ und $0\leq i \leq n.$ Dann ist 
\[
\zeta((n)\overline{2l}(n))\leq 2l+1
\]
\end{lemma}
\begin{proof}
Wir zeigen diese Aussage, indem wir die Sphäre $X=(n)\overline{2l}(n)$ mit $2l$ Kantendrehungen in die Sphäre $(n+l)^2$ umformen. Seien $V_1$ und $V_2$ die inneren Ecken der beiden $n$-Gons, also $\deg(V_1)=\deg(V_2)=n$. Sei für geeignete Kanten $(e_1,\ldots,e_l,e,e_l',\ldots,e_1')$ ein Kantenpfad in $X$ ist, sodass keine der Kanten zu $V_1$ und $V_2$ inzident ist und folgende Relationen gelten:
\begin{itemize}
\item $X_1(X_2(V_1))-(X_1(V_1)\cup X_1(V_2))=\{e_1,\ldots,e_l,e,e'\}$
\item $X_1(X_2(V_2))-(X_1(V_2)\cup X_1(V_1))=\{e'_1,\ldots,e'_l,e,e'\}$
\end{itemize}
\begin{figure}[H]
\begin{center}
\includegraphics[viewport=8cm 22cm 5cm 26.5cm]{n2ln}
\end{center}
\caption{die Sphäre $(6)\overline{6}(6)$}
\end{figure}
Diese Kanten bilden also die Randkanten des eingesetzten $2l$-Strips.
Es gilt also $\{e_1,\ldots,e_l\}\cap\{e_1',\ldots,e_l'\}=\emptyset.$
Durch Anwenden der drehbaren Kantensequenz $(e_1,\ldots,e_l)$ auf $X$ erhalten wir eine Sphäre $Y$ mit $\deg_Y(V_1)=n+l.$ Dadurch erhalten wir durch drehen der Kanten $e_1',\ldots,e_l'$ die Sphäre $Z$ mit 
\[
\deg_Z(V_1)=\deg_Z(V_2)=n+l.
\]
Da für alle $V\in Z_0-\{V_1,V_2\}$ die Gleichheit 
\[
\deg_Z(V)=4
\]
gilt und $V_1,V_2$ in $Z$ nicht benachbart sind, muss schon $Z\cong (n+l)^2$ gelten. Daraus folgt die Behauptung.
\end{proof}
\begin{bemerkung}
Als ein Nebenprodukt des Beweises erhalten wir für $3\leq l\leq n$ die Abschätzung 
\[
\zeta((n+1)\overline{2l-1}(n))\leq 2l-1,
\]
da die Sphäre $(n+1)\overline{2(l-1)}(n+1)$ durch eine Kantendrehung aus der Sphäre $(n+1)\overline{2l-1}(n)$ hervorgeht.
\end{bemerkung}
In der Formulierung des Lemmas gilt aber keine Gleichheit.
Denn für die Sphäre $X=(4)\overline{2}(4)$ ergibt sich die Schranke $\zeta(X)\leq 3.$ Die Sphäre kann jedoch mithilfe von zwei Kantendrehungen in einen Multi-Tetraeder umgeformt werden. Dies wird im Verlauf dieser Arbeit noch mithilfe eines weiteres Algorithmus gezeigt. Doch vorerst führen wir das folgende Lemma ein.
\begin{lemma}
Sei $X$ eine vertex-treue Sphäre mit einer Ecke $V$ vom Grad 3,  $Y$ die Sphäre die durch das Entfernen des Tetraeders entsteht, also $Y=T_V(X),$ und $\zeta:=\zeta(Y)$ die Kaktus-Distanz von $Y.$ Weiterhin seien $e_1,\ldots,e_{\zeta}$ drehbare Kanten in $Y,$ sodass $Y^{(e_1,\ldots,e_{\zeta})}$ ein Multi-Tetraeder ist. Dann gilt
\[
\zeta(X)\leq \vert\{e_i\mid \, e_i \notin M\}\vert+2*\vert \{e_j\mid \, e_j \in M\}\vert,
\]
wobei $M=X_1(X_2(V))-X_1(V)$ ist.
\end{lemma}
Die wesentliche Idee für den Beweis wurde bereits im Beweis von \Cref{3eck} präsentiert. Diese muss nur auf diese Situation angepasst werden.
\begin{proof}
Wir weisen diese Aussage nach, indem wir die drehbare Kantensequenz $E:=(e_1,\ldots , e_{\zeta})$
in $Y$ in eine drehbare Kantensequenz in $X$ umstrukturieren, die aus $X$ einen
Multi-Tetraeder hervorbringt. Dabei halten wir fest, wie viele Kantendrehungen angewendet wurden, um das Übersetzen der Kantensequenz zu ermöglichen.
Wir wollen also aus der Kantensequenz in $Y$ eine Kantensequenz $E'=(e'_1,\ldots e'_k)$ in $X$ konstruieren, die Folgendes erfüllt:
\begin{itemize}
\item $E'$ ist eine drehbare Kantensequenz.
\item Die Kantensequenz erlaubt, dass Entfernen des Tetraeders an  $V$ und Kantendrehungen miteinander vereinbar sind. Damit meinen wir, dass es für alle $1\leq i\leq \zeta$ ein $1\leq l \leq k$ mit 
\begin{align*}
T^F(Y^{(e_1,\ldots, e_i)})&\cong X^{(e'_1,\ldots,e'_l)} \\
\Leftrightarrow Y^{(e_1,\ldots, e_i)}&\cong T_V(X^{(e'_1,\ldots,e'_l)})
\end{align*} 
existiert.
\end{itemize}
Seien  hierfür $X^0:=X$ und $E_0:=().$ Die Kantensequenz wird wie folgt konstruiert: Seien die Sphären $X^{i-1}$ und die Kantensequenz $E_{i-1}=(e_1',\ldots,e_l')$ in $X$ für $1\leq i \leq \zeta$ bereits konstruiert. Wir führen nun folgende Fallunterscheidung durch:
\begin{itemize}
\item Falls die Kante $e_{i}$ der Kantensequenz $E$ in $Y^{(e_1,\ldots,e_{i-1})}$ nicht zu F inzident ist, so wählen wir $X^{i}:={(X^{i-1})}^{e_i}$ und $E_{i}:=(e_1',\ldots,e_l',e_i)$. Damit ist 
\[
T^F(Y^{(e_1,\ldots,e_i)})\cong X^{E_{i}}
\]
und in $X^{E_{i}}$ existiert keine 2-Taille, denn sonst wäre diese schon in $Y^{(e_1,\ldots,e_i)}$ enthalten. Damit ist $E_{i}$ drehbar.
\item Falls die Kante $e_i$ der Kantensequenz $E$ in $Y^{(e_1,\ldots,e_{i-1})}$ zu F inzident ist, müssen folgende Kantendrehungen angewendet werden. Da die Kante $e_i$ in $Y^{(e_1,\ldots,e_{i-1})}$ zu $F$ inzident ist, liegt die Kante in $X^{i-1}$ am Tetraeder an der Stelle $V.$
Des Weiteren existieren Kanten $e',e''$, sodass die Kante $e''$ in $Y^{(e_1,\ldots,e_{i-1})}$ zu $F$ inzident ist und in $Y^{(e_1,\ldots,e_{i})}$ beide Kanten zu der Fläche $F$ inzident sind.
Dann liegt $e''$ in $X^{(e_1',\ldots,e_l')}$ am Tetraeder an der Ecke $V.$ Diese Kante ist jedoch nicht zu $V$ inzident.  
Für $Z=X^{(e_1',\ldots,e_{l-1}',e_i)}$ gibt es in $X^{(e_1',\ldots,e_{l-1}',e_i)}$ gibt es genau eine Kante $e$, die zu $V$ inzident ist und 
 \[
Z_0(e)\cap Z_0(e')=Z_0(e)\cap Z_0(e'')=\emptyset
\]
erfüllt. Durch Drehen dieser Kante erhalten wir die geforderte Vertauschbarkeit. Die Kantensequenz $(e_1',\ldots,e_l',e_i,e)$ ist dann ebenfalls drehbar.
\end{itemize}
Wir erhalten also nach $\zeta$ Schritten eine Kantensequenz $E'$ in $X,$ sodass $X^{E'}$ isomorph zu $T^F(Y^{E})$ ist. Da $Y^{E}$ ein Multi-Tetraeder ist, gilt dies auch für $X^{E'}.$ Für Kanten, die nicht am Tetraeder an der Stelle $V$ liegen, gelingt uns das obige Übersetzen durch eine einfache Kantendrehung. Falls dies jedoch nicht der Fall ist gelangen wir in den zweiten Teil der Fallunterscheidung und es müssen zwei Kantendrehungen durchgeführt werden, um das Übertragen der Kantensequenz zu ermöglichen.
\end{proof}
\begin{bemerkung}
Dieses Lemma bildet die Grundidee für den folgenden Algorithmus, der die Kaktus-Distanz einer Sphäre annähert. Und zwar wählen wir eine Ecke mit minimalen Grad. An dieser Ecke führen wir solange Kantendrehungen durch, bis eine Ecke vom Grad 3 erzeugt wird. Dann entfernen wir diesen Tetraeder und führen diese Prozedur erneut aus. Durch Iterieren dieses Vorgehens tritt nach endlich vielen Schritten der Fall ein, dass die konstruierte Sphäre isomorph zu einem Multi-Tetraeder ist. Durch obiges Lemma werden die in den Zwischenschritten des beschriebenen Verfahrens konstruierten Kantensequenzen zu einer Kantensequenz in der ursprünglichen Sphäre übersetzt. Somit liefert diese Kantensequenz einen Multi-Tetraeder und wir erhalten die gewünschte Abschätzung der Kantensequenz.
\end{bemerkung}
\begin{comment}
\begin{lemma}\textcolor{red}{weg\,and mix with the other one}
Seien $l,n\in \mathbb{N}$ und $l \leq n.$ Dann ist $\zeta((n)\overline{2l}(n))\leq l+1$
\end{lemma}
\begin{proof}
Sei $X$ die Sphäre mit dem Symbol $(n)\overline{2l}(n).$
Die Sphäre $X$ lässt sich bekanntlicher Weise aus zwei $n$-gons und einem $2l$-Streifen zusammensetzen. Mit $V$ und $V'$ bezeichnen wir die die inneren Ecken der beiden $n$-gons. Also ist $\deg_X(V)=\deg_X(V')=n.$ Beim $2l$-Streifen gibt es genau zwei Randecken $V_1,V_1'$, die Flächengrad 1 haben. ZU $V_1$ bzw. $V_1'$ gibt es genau eine benachbarte Randecke $V_2$ bzw. $V_2'$, die den Flächengrad 2 hat.
Beim Zusammensetzen der Sphäre $X$ erhalten wir also 
\begin{align*}
\deg_X(V_1)=\deg_X(V_1')=5,\\
\deg_X(V_2)=\deg_X(V_2')=4.
\end{align*}
Es gibt genau eine Nachbar-Ecke $\tilde{V}$ von $V_1,$ die ungleich $V,V'$ ist und $\deg_X(\tilde{V})=4$ erfüllt. Die Kante, die zu $V_1$ und $\tilde{V}$ inzident ist, bezeichnen wir mit $e.$ Damit ist $e$ eine drehbare Kante und beim Drehen dieser Erhalten wir die Sphäre $Y=(n-1)\overline{2l-1}(n).$ Analoge Argumentation bei der Ecke $V_1'$ liefert uns eine Kante $e',$ wodurch eine Kantendrehung die Sphäre $Y^{e'}\cong (n-1)\overline{2(l-1)}(n-1).$ $l-1-$maliges Wiederholen liefert uns  
\end{proof}
\end{comment}
Für den Ikosaeder ergibt sich nach Vorgehensweise des ersten Algorithmus 10 als Schranke der Kaktus-Distanz.
Bei der Implementierung des zweiten Algorithmus erhalten wir 8 als verbesserte Schranke. Diese ist jedoch nicht gut genug, da der Ikosaeder mit sechs Kantendrehungen in einen Multi-Tetraeder verwandelt werden kann. Dies kann mit GAP verifiziert werden.\\
Durch Implementierung der beiden Algorithmen zum Annähern der Kaktus-Distanz können wir diese miteinander und mit der Kaktus-Distanz der Sphären vergleichen. Die Resultate werden in der folgenden Tabelle für die vertex-treuen Sphären ohne 3-Taillen mit bis zu 14 Flächen dargestellt. 
\begin{center}
\begin{tabular}{|c|c|c|c|}
\hline

$\textbf{X}$&$\textbf{$\zeta$(X)}$&\textbf{Algorithmus 1}&\textbf{Algorithmus 2}\\
\hline
$T$&0&0&0\\
\hline
$(4)^2$&1&1&1\\
\hline
$(5)^2$&1&1&1\\
\hline
$(6)^2$&1&1&1\\
\hline
$(5)\overline{2}(5)$&2&3&2\\
\hline
$(6)\overline{2}(6)$&2&3&2\\
\hline
$(7)^2$&1&1&1\\
\hline
$(6)\overline{3}(5)$&2&4&2\\
\hline
$(5)\overline{4}(5)$&3&5&3\\
\hline
\end{tabular} 
\end{center}
Damit ist der erste Algorithmus nur für Kaktus-Distanz 1 exakt, wohingegen der zweite die Sphären mit Kaktus-Distanz 2 verifizieren kann. Bei den vertex-treuen Sphären mit 18 Flächen gibt es erstmals eine Sphäre deren Kaktus-Distanz echt kleiner als die mit dem zweitem Algorithmus berechnete Schranke.
Um den zweiten Algorithmus zu optimieren muss, die Wahl der Kanten genauer betrachtet werden. Beispielsweise kann vermieden werden, dass Kanten, die vor dem Entfernen eines Tetraeders an einem Tetraeder lagen, gedreht werden. Dies führt nämlich dazu, dass zwei Kantendrehungen angewendet werden müssen, um die Verträglichkeit von Kantendrehung und Tetraeder-Erweiterung zu gewährleisten.\\
Die Tatsache, dass der zweite Algorithmus eine Verbesserung darstellt geht ebenfalls aus der Tabelle hervor. Dies sollte aber nicht verwunderlich sein, da bei der ersten Vorgehensweise nur die Distanz zum Doppel-$n$-Gon  ausgenutzt wird. Bei dem zweiten Algorithmus jedoch, muss der konstruierte Multi-Tetraeder nicht derjenige sein, der eine Kantendrehung vom $(n)^2$ entfernt ist.
\begin{satz}
Sei $n \geq 4.$ Dann ist $\zeta((n)\overline{2}(n))=2.$
\end{satz}
\begin{proof}
Wir nutzen den vorgestellten Algorithmus zum Nachweis dieser Aussage.
Die Sphäre $(n)\overline{2}(n)$ entsteht durch eine Schmetterlings-Einsetzung am Doppel-$n$-Gon an zwei adjazenten Kanten, die jeweils zu zwei Ecken vom Grad 4 inzident sind. Seien $V_1,V_2,V_3,V_4$ die Ecken des eingesetzten Schmetterlings. Dann gilt ohne Einschränkung 
\begin{align*}
deg_X(V_1)=deg_X(V_2)=5\\
deg_X(V_3)=deg_X(V_4)=4
\end{align*}
Durch Drehen der Kante $\{V_3,V_4\}$ entstehen Tetraeder an $V_3$ und $V_4.$ 
  
\begin{figure}[H]
\begin{center}
\includegraphics[viewport=10cm 23cm 5cm 27cm]{zeta2}
\end{center}
\caption{skizzierte Kantendrehung am Beisiel $(4)\overline{2}(4)$}
\end{figure}
Dann ist $Y=T_{V_3}(T_{V_4}(X))$ eine Sphäre, die isomorph zum Double-$(n-1)$-Gon ist. Es gilt also $\zeta (Y)=1$ und da $X$ wegen $n\geq 4$ die Sphäre Kanten besitzt, die nicht keinem der beiden Tetraeder liegen, gilt $\zeta (X)\leq 2.$ Da $x$ eine vertex-treue Sphäre ist, die nicht zum Double-$n$-Gon isomorph ist, muss $\zeta(X)\geq 2$ gelten, wodurch die Behauptung folgt. 
\end{proof}
Beim genaueren Betrachten des Beweises ist zu erkennen, dass wir weitere Sphären mit Kaktus-Distanz 2 angeben können.
Dies halten wir in der nachstehenden Folgerung fest.
\begin{folgerung}
Sei $n\geq 4.$ Dann ist $\zeta ((n+1)\overline{3}(n))=2.$
\end{folgerung}
Bei der Konstruktion der Sphäre $(n+1)\overline{3}(3)$ wird der 3-Strip verwendet. Es gibt zwei Ecken des Eingesetzten 3-Strips, die Grad 4 haben. Durch Drehen der Kante, die zu diesen Ecken inzident ist und Entfernen der beiden resultierenden Tetraeder erhalten wir den Doppel-$(n)$-Gon, wodurch die Aussage gezeigt ist.
\begin{bemerkung}
Wenn wir den Begriff der Kaktus-Distanz verallgemeinern, erhalten wir eine Metrik auf der Menge der Isomorphieklassen der Sphären ohne 2-Taillen. Für vertex-treue Sphären $X$ und $Y$ definieren wir die Distanz $d(X,Y)$ als die minimale Anzahl an Kantendrehungen, um $X$ isomorph in $Y$ umzuformen. 
\begin{itemize}
\item Für zwei vertex-treue Sphäre $X$ und $Y$ ist die Distanz $d(X,Y)$ eine nicht negative ganze Zahl.
\item Die Nulltreue ist klarerweise erfüllt.
\item Für alle vertex-treuen Sphären gilt $d(X,Y)=d(Y,X).$
\item Sei $Z$ eine weitere vertex-treue Sphäre. Dann existiert eine Kantensequenz, die bei Anwenden auf $X$ die Sphäre $Z$ hervorbringt und es existiert eine weitere Kantensequenz, die durch Anwenden auf $Y$ die Sphäre $Z$ erzeugt. Diese Kantensequenzen können wir dann zu einer Kantensequenz, die $X$ in $Y$ umformt, zusammensetzen. Also muss $d(X,Y)\leq d(X,Z)+d(Y,Z)$ gelten.
\end{itemize}
\end{bemerkung}

\subsection{Tetraeder-Zerlegung}
\begin{definition}
Sei $X$ eine vertex-treue Sphäre und $D\subseteq \Pot_4(X_0)$. Wir nennen $D$ eine Tetraeder-Zerlegung von $X$, falls $D$ folgende Eigenschaften erfüllt:
\begin{itemize}
\item Für jedes $F\in X_2$ gibt es genau ein $d\in D$, sodass $X_0(F) \subseteq D$ ist.
\item Für jedes $N\in \Pot_3(X_0)-\{X_0(F)\mid F\in X_2\}$ gibt es entweder kein oder genau zwei $d_1,d_2\in D$ mit $N\subseteq d_1,d_2.$
\end{itemize}
$D$ ist eine minimale Tetraeder-Zerlegung, falls $\vert D \vert\leq \vert D' \vert$ für jede weitere Tetraeder-Zerlegung $D'$ von $X$ gilt.
\end{definition}


Nutzen wir GAP, um Beispiele für Tetraeder-Zerlegungen zu konstruieren. Betrachten wir als vertex-treue Sphäre den Oktaeder.
\begin{figure}[H]
\begin{center}
\includegraphics[viewport=17cm 17cm 5cm 20.7cm]{Image_Octahedron}
\end{center}
\caption{Oktaeder}
\end{figure}
\begin{center}
$\fbox{
\parbox{15cm}{
\textcolor{red}{gap$>$} \textcolor{blue}{O;}\newline 
simplicial surface (6 vertices, 12 edges, and 8 faces) \newline
\textcolor{red}{gap$>$} \textcolor{blue}{VerticesOfFaces(O);}\newline
[ [ 1, 2, 3 ], [ 2, 5, 6 ], [ 1, 2, 5 ], [ 2, 3, 6 ], [ 1, 4, 5 ],  [ 3, 4, 6 ], [ 1, 3, 4 ], [ 4, 5, 6 ] ]
}}$
\end{center}
Tetraeder-Zerlegungen können nun auf folgende Art und Weise  erzeugt werden.
Seien $V_1,V_2$ zwei nicht benachbarte Ecken des Oktaeders.
Die vierelementigen Teilmengen der Tetraeder-Zerlegung sind genau die Schmetterlinge, die $V_1$ und $V_2$ enthalten. Dadurch erhalten wir die folgenden Mengen als Kandidaten für Tetraeder-Zerlegungen.
\begin{center}
$\fbox{
\parbox{15cm}{
\textcolor{red}{gap$>$}  \textcolor{blue}{D1:=[[1,2,3,6],[1,2,5,6],[1,4,5,6],[1,3,4,6]];}\newline
[ [ 1, 2, 3, 6 ], [ 1, 2, 5, 6 ], [ 1, 4, 5, 6 ], [ 1, 3, 4, 6 ] ]\newline
\textcolor{red}{gap$>$}  \textcolor{blue}{D2:=[[2,3,1,4],[2,4,5,6],[2,1,5,4],[2,3,4,6]];}\newline
[ [ 2, 3, 1, 4 ], [ 2, 4, 5, 6 ], [ 2, 1, 5, 4 ], [ 2, 3, 4, 6 ] ]
}}$
\end{center}
Mit GAP lässt sich leicht verifizieren, dass $D_1$ und $D_2$  Tetraeder-Zerlegungen sind.
\begin{center}
$\fbox{
\parbox{15cm}{
\textcolor{red}{gap$>$}  \textcolor{blue}{IsTetrahedrialDecomposition(O,D1);}\newline
 true \newline
\textcolor{red}{gap$>$}  \textcolor{blue}{IsTetrahedrialDecomposition(O,D2);}\newline
 true
 }}$ 
 \end{center}
Vielmehr sind beide minimal. Angenommen es existiert eine Tetraeder-Zerlegung $D'$ des Oktaeders mit $\vert D' \vert\leq 3$. Dann muss es ein $d\in D'$ geben, welches die Eckenmenge von drei paarweise verschiedenen Flächen als Teilmengen enthält. Dies bedeutet aber, dass es im Oktaeder eine Ecke vom Grad 3 geben muss, was ein Widerspruch ist. 
\begin{comment}
\begin{bsp}
$\fbox{
\parbox{15cm}{
\textcolor{blue}{gap$>$} \textcolor{red}{D5g:=Doublengon(5);}\\
simplicial surface (7 vertices, 15 edges, and 10 faces)\\
\textcolor{blue}{gap$>$} \textcolor{red}{VerticesOfFaces(D5g);}\\
$[ [ 1, 2, 6 ], [ 2, 6, 7 ], [ 1, 2, 3 ], [ 2, 3, 7 ], [ 1, 3, 4 ], [ 3, 4, 7 ], [ 1, 4, 5 ], [ 4, 5, 7 ], [ 1, 5, 6 ], [ 5, 6, 7 ] ]$\\
}}$
\[
bild
\]
Um nachzuprüfen, ob wir wirklich Tetraeder-Zerlegungen konstruiert haben, benötigen wir die dreielementigen Teilmengen der Ecken, die keine Eckenmenge von einer beliebigen Fläche bilden.  \\
$\fbox{
\parbox{15cm}{
\textcolor{blue}{gap$>$} \textcolor{red}{Combinations([1..7],3);}\\
$[ [ 1, 2, 3 ], [ 1, 2, 4 ], [ 1, 2, 5 ], [ 1, 2, 6 ], [ 1, 2, 7 ], [ 1, 3, 4 ], [ 1, 3, 5 ], [ 1, 3, 6 ], [ 1, 3, 7 ], [ 1, 4, 5 ]$,
  $[ 1, 4, 6 ], [ 1, 4, 7 ], [ 1, 5, 6 ], [ 1, 5, 7 ], [ 1, 6, 7 ], [ 2, 3, 4 ], [ 2, 3, 5 ], [ 2, 3, 6 ], [ 2, 3, 7 ], [ 2, 4, 5 ]$, 
  $[ 2, 4, 6 ], [ 2, 4, 7 ], [ 2, 5, 6 ], [ 2, 5, 7 ], [ 2, 6, 7 ], [ 3, 4, 5 ], [ 3, 4, 6 ], [ 3, 4, 7 ], [ 3, 5, 6 ], [ 3, 5, 7 ]$, 
  $[ 3, 6, 7 ], [ 4, 5, 6 ], [ 4, 5, 7 ], [ 4, 6, 7 ], [ 5, 6, 7 ] ]$\\
\textcolor{blue}{gap$>$}\textcolor{red}{ C:=last;;}\\
\textcolor{blue}{gap$>$} \textcolor{red}{Diff:=Difference(C,VerticesOfFaces(D5g));}\\
$[ [ 1, 2, 4 ], [ 1, 2, 5 ], [ 1, 2, 7 ], [ 1, 3, 5 ], [ 1, 3, 6 ], [ 1, 3, 7 ], [ 1, 4, 6 ], [ 1, 4, 7 ], [ 1, 5, 7 ], [ 1, 6, 7 ]$, 
$  [ 2, 3, 4 ], [ 2, 3, 5 ], [ 2, 3, 6 ], [ 2, 4, 5 ], [ 2, 4, 6 ],  [ 2, 4, 7 ], [ 2, 5, 6 ], [ 2, 5, 7 ], [ 3, 4, 5 ], [ 3, 4, 6 ]$, 
 $ [ 3, 5, 6 ], [ 3, 5, 7 ], [ 3, 6, 7 ], [ 4, 5, 6 ], [ 4, 6, 7 ] ]$\\
}}$ 
Man kann nun auf zweierlei Weisen Tetraeder-Zerlegungen erzeugen.
\begin{itemize}
\item Die vierelementigen Teilmengen der Tetraeder-Zerlegung sind die Ecken jener Schmetterlinge, die genau zwei Ecken vom Grad 4 und genau zwei Ecken vom Grad 5 haben.\\
$\fbox{
\parbox{15cm}{
\textcolor{blue}{gap$>$} \textcolor{red}{D2:=$[[2,6,1,7],[2,3,1,7],[3,4,1,7],[4,5,1,7],[1,5,6,7]]$;}\\
$[ [ 2, 6, 1, 7 ], [ 2, 3, 1, 7 ], [ 3, 4, 1, 7 ], [ 4, 5, 1, 7 ], [ 1, 5, 6, 7 ] ]$\\
\textcolor{blue}{gap$>$} \textcolor{red}{List(D2,r-$>$Filtered(Diff,g-$>$IsSubset(r,g)));}\\
$[ [ [ 1, 2, 7 ], [ 1, 6, 7 ] ], [ [ 1, 2, 7 ], [ 1, 3, 7 ] ], 
  [ [ 1, 3, 7 ], [ 1, 4, 7 ] ], [ [ 1, 4, 7 ], [ 1, 5, 7 ] ], 
  [ [ 1, 5, 7 ], [ 1, 6, 7 ] ] ]$\\
  \textcolor{blue}{gap$>$} \textcolor{red}{List(D2,r-$>$Length(Filtered(Diff,g-$>$IsSubset(r,g))));}\\
$[ 2, 2, 2, 2, 2 ]$
}}$
\item Man kann die Schmetterlinge des Doppel-5-Gon so einteilen, dass vier Schmetterlinge inzident zu einer Ecke vom Grad 5 sind und der letzte Schmetterling wieder zu beiden Ecken vom Grad 5 inzident sind. 
\end{itemize}
$\fbox{
\parbox{15cm}{
\textcolor{blue}{gap$>$} \textcolor{red}{D1:=$[[1,2,3,6],[1,3,4,5],[1,5,6,7],[2,7,6,3],[3,4,5,7]]$;}\\
$[ [ 1, 2, 3, 6 ], [ 1, 3, 4, 5 ], [ 1, 5, 6, 7 ], [ 2, 7, 6, 3 ], [ 3, 4, 5, 7 ] ]$\\
\textcolor{blue}{gap$>$} \textcolor{red}{List(D1,r-$>$Filtered(Diff,g-$>$IsSubset(r,g)));}\\
$[ [ [ 1, 3, 6 ], [ 2, 3, 6 ] ], [ [ 1, 3, 5 ], [ 3, 4, 5 ] ], 
  [ [ 1, 5, 7 ], [ 1, 6, 7 ] ], [ [ 2, 3, 6 ], [ 3, 6, 7 ] ], 
  [ [ 3, 4, 5 ], [ 3, 5, 7 ] ] ]$\\
\textcolor{blue}{gap$>$} \textcolor{red}{List(D1,r-$>$Length(Filtered(Diff,g-$>$IsSubset(r,g))));}\\
$[ 2, 2, 2, 2, 2 ]$\\
 }}$ 
 Insbesondere hat man mit den obigen zwei Tetraeder-Zerlegungen Beispiele für Tetraeder-Zerlegungen gesehen, die minimal sind. Denn angenommen es existiert eine Tetraeder-Zerlegung $D'$ mit $\vert D' \vert =4.$ Da es für jede Fläche des $(5)^2$ genau ein $d\in D'$ geben muss, das die Eckenmenge der Fläche enthält, muss es ein $d' \in D$ mit 
 \[
\vert \{F\in (5)^2_2\mid X_0(F)\in d'\}\vert \geq 3 
 \]
 geben. Da der Doppel-5-Gon jedoch vertex-treu ist, muss es eine Ecke vom Grad 3 geben, was ein Widerspruch ist.
 \end{bsp}
 \end{comment}
\begin{lemma}\label{zerlegung}
Sei $X$ ein Multi-Tetraeder. Dann besitzt $X$ eine Tetraeder-Zerlegung. 
\end{lemma}
\begin{proof}
Wir beweisen die Aussage induktiv. Falls $X$ ein Tetraeder mit $X_0=\{1,2,3,4\}$ ist, dann bildet die Menge $\{\{1,2,3,4\}\}$ eine Tetraeder-Zerlegung, die insbesondere minimal ist.
Sei $X$ ein Multi-Tetraeder mit $\vert X_2\vert =n > 4.$ Sei zudem $V\in X_0$ eine Ecke vom Grad 3 und $F_1,F_2,F_3$ die drei Flächen, die $X_2(V)=\{F_1,F_2,F_3\}$ erfüllen.  Dann bildet $Y=T_V(X)$ einen Multi-Tetraeder mit $\vert Y_2 \vert =n-2.$ Deshalb existiert eine Tetraeder-Zerlegung $D$ von $Y.$ Sei $F$ die Fläche, die in $Y$ den Tetraeder ersetzt. Es muss nun nachgewiesen werden, dass $D'=D\cup \{X_0(X_2(V))\}$ eine Tetraeder-Zerlegung von $X$ ist.
\begin{itemize}
\item Für alle $F\in X_2= Y_2 \Delta\{F,F_1,F_2,F_3\}$ gibt es genau ein $d\in D'$ mit 
\[
X_0(F)\subseteq d.
\] 
\item Sei nun $N\in \Pot_3(X_0)-\{X_0(F)\mid F\in X_2\}.$ Falls die Ecke $V$ vom Grad 3 in $N$ enthalten ist, 
dann gibt es kein $d\in D$ mit $N\subseteq d$ und damit auch kein $d'\in D'$ mit dieser Eigenschaft. Für $N=Y_0(F)$ existiert genau ein $d \in D$ mit $N\subseteq d.$ Mit $\{X_0(X_2(V))\}$ gibt es also genau zwei Elemente in $D'$ mit dieser Eigenschaft. Falls $V\notin N$ ist, dann ist $N\in \Pot_3(Y)$ und die Behauptung folgt direkt, da $D$ eine Tetraeder-Zerlegung von $Y$ ist.
\end{itemize}
\end{proof}
\[
\textcolor{red}{fehler}
\]
  \begin{lemma}\label{tzer}
 Sei $X$ ein Multi-Tetraeder und $e$ eine drehbare Kante in $X$. Dann besitzt $X^e$ eine Tetraeder-Zerlegung. 
 \end{lemma}
 \begin{proof}
 Wegen \Cref{zerlegung} besitzt $X$ eine Tetraeder-Zerlegung $D.$ Seien $F_1,F_2\in X_2$ Flächen mit $X_2(e)=\{F_1,F_2\}$ und $Y:=X^e.$ Seien zudem $V_1,\ldots,V_4$ paarweise verschiedene Ecken in $X$, die $X_0(F_1)=\{V_1,V_3,V_4\}$ und $X_0(F_2)=\{V_2,V_3,V_4\}$ erfüllen.
 \begin{figure}[H]
\begin{center}
\includegraphics[viewport=30cm 15cm 0cm 21cm]{Image_ET}
\end{center}
\caption{Ausschnitt einer simplizialen Fläche}
\end{figure}
 
 Dann gelten in $Y$ bis auf Isomorphie die Relationen ${Y}_0(F_1)=\{V_1,V_2,V_3\}$ und ${Y}_0(F_2)=\{V_1,V_2,V_4\}.$
 \begin{figure}[H]
\begin{center}
\includegraphics[viewport=30cm 15cm 0cm 21cm]{Image_ET1}
\end{center}
\caption{Ausschnitt einer simplizialen Fläche}
\end{figure}
Für die Mengen ${X}_0(F_1)$ und ${X}_0(F_1)$ können nun genau zwei Fälle auftreten.
\begin{enumerate}
\item Entweder es existiert genau ein $d\in D,$ dass die obigen beiden Mengen als Teilmengen enthält
\item oder es existieren genau zwei $d_1,d_2,$ sodass $X_0(F_1)$ in $d_1$ enthalten und $X_0(F_2)$ in $d_2$ enthalten ist.
\end{enumerate}
\begin{enumerate}
\item Falls dieser Fall eintritt, dann bildet $D':=D-\{d\}$ eine Tetraeder-Zerlegung von $Y.$ Zunächst ist nachzuprüfen, dass für jedes $F\in {Y}_2=X_2$ genau ein $d'\in D'$ mit ${Y}_0(F)\subseteq d'$ existiert. Da diese Aussage für alle $F\in Y_2-\{ F_1,F_2\}$ bereits erfüllt ist, reicht es $F_1,F_2$ zu betrachten. Da ${Y}_0(F_1)=\{V_1,V_2,V_3\}\subseteq d$ und  $Y_0(F_1) \in \Pot_3(X_0)-\{X_0(F)\mid F\in X_2\}$ ist, gibt es genau ein weiteres $d'\in D,$ das ${Y}_0(F_1)$ als Teilmenge enthält. Daraus folgt direkt, dass es genau ein $d'\in D',$ das $\{V_1,V_2,V_3\}$ enthält. Analog geht man auch für ${Y}_0(F_2)=\{V_1,V_3,V_4\}$ vor.
Nun muss ebenfalls nachgewiesen werden, dass es für jedes $N$ in der Menge 
\begin{align*}
 &\Pot_3({Y}_0)-\{{Y}_0(F)\mid F\in {Y}_2\}\\
 =& (\Pot_3({X}_0)-\{{X}_0(F)\mid F\in {X}_2\})\cup \{X_0(F_1),X_0(F_2)\}-\{{Y}_0(F_1),{Y}_0(F_2)\}
 \end{align*}
  entweder kein $d'\in D'$ mit $N\subseteq d'$ 
  oder genau zwei $d_1',d_2'\in D'$ mit $N\subseteq d_1',d_2'$ gibt. 
  Es reicht den Fall $X_0(F_1)$ zu diskutieren. Da $X_0(F_1)\subseteq d$ ist und $d$ als einziges Element in $D$ die Menge $X_0(F_1)$ als Teilmenge enthält, folgert man, dass es kein $d'\in D'=D-\{d\}$ mit dieser Eigenschaft geben kann. Analog geht man für $X_0(F_2) $ vor.
\item In diesem Fall bildet $D'=D\cup\{X_0(X_2(e))\}$ eine Tetraeder-Zerlegung von $Y.$ Klarerweise gibt es für alle $F\in {Y}_2$ genau ein $d'\in D^e$ mit ${X^e}_0(F)\subseteq d'.$ Nun muss wieder nachgewiesen werden, dass es für jedes $n\in \Pot_3({X^e}_0)-\{{X^e}_0(F)\mid F\in {X^e}_2\}=(\Pot_3({X}_0)-\{{X}_0(F)\mid F\in {X}_2\})\cup \{X_0(F_1),X_0(F_2)\}-\{{X^e}_0(F_1),{X^e}_0(F_2)\}$ entweder kein oder genau zwei $d'\in D^e$ gibt, die $n$ enthalten. Es reicht $X_0(F_1)$ zu betrachten. Da $X_0(F_1)$ in genau einem $d\in D$ enthalten ist, gibt es genau zwei $d'\in D^e,$ sodass $X_0(F_1)$ eine Teilmenge von den $d'$ ist. Analog geht man für $X_0(F_2) $ vor.
\end{enumerate}
\end{proof}
\begin{folgerung}
Sei $X$ eine vertex-treue Sphäre. Dann besitzt $X$ eine Tetraeder-Zerlegung.
\end{folgerung}
\begin{proof}
Es existiert ein Multi-Tetraeder $Y$ mit $\vert X_2\vert =\vert Y_2 \vert .$ Nach obigem Lemma hat $Y$ eine Tetraeder-Zerlegung $D$. Da die Kantendrehung transitiv ist, existiert eine Kantensequenz $E,$ sodass $Y^E$ isomorph zu $X$ ist. Durch iteratives Anwenden von \Cref{tzer}, erhält meine eine Tetraeder-Zerlegung $D'$ von $X$. 
\end{proof}
%\begin{bsp}
Wir nutzen diese Erkenntnis, um Tetraeder-Zerlegungen von komplizierteren Sphären zu konstruieren.\newline
$\fbox{
\parbox{15cm}{
\textcolor{red}{gap $>$}\textcolor{blue}{$S;$}\\
simplicial surface (8 vertices, 18 edges, and 12 faces)\\
\textcolor{red}{gap $>$} \textcolor{blue}{VerticesOfFaces(S);}\\
$[ [ 1, 4, 6 ], [ 1, 4, 8 ], [ 1, 6, 7 ], [ 1, 7, 8 ], [ 2, 3, 5 ],
[ 2, 3, 6 ], [ 2, 4, 5 ], [ 2, 4, 6 ], [ 3, 5, 7 ],$\newline
$ [ 3, 6, 7 ],
[ 4, 5, 8 ], [ 5, 7, 8 ] ]$\newline
\textcolor{red}{gap $>$}\textcolor{blue}{ FaceDegreesOfVertices(S);}\\
$[ 4, 4, 4, 5, 5, 5, 5, 4 ]$
 }}$\\
 \begin{figure}[H]
\begin{center}
\includegraphics[viewport=0cm 22.cm 10cm 26cm]{Tetzer}
\end{center}
\caption{vertex-treue Sphäre mit 12 Flächen}
\end{figure}
Durch den oben eingeführten Algorithmus zum Annähern der Kaktus-Distanz, lässt sich in diesem Fall sogar die exakte Kaktus-Distanz bestimmen. Anhand der Grade der Ecken erkennen wir, dass $S$ nicht isomorph zum Doppel-6-Gon ist. Also ist die Kaktus-Distanz mindestens zwei.
\begin{center}
 \fbox{
\parbox{15cm}{
\textcolor{red}{gap$>$}\textcolor{blue}{ AlgorithmCactus(S);}\\
2\newline
\textcolor{red}{gap$>$}\textcolor{blue}{EdgeTurn(S,4);}\newline 
simplicial surface (8 vertices, 18 edges, and 12 faces)\newline 
\textcolor{red}{gap$>$}\textcolor{blue}{EdgeTurn(last,5);}\newline 
simplicial surface (8 vertices, 18 edges, and 12 faces)\newline 
\textcolor{red}{gap$>$}\textcolor{blue}{C:=last;}\newline
simplicial surface (8 vertices, 18 edges, and 12 faces)\newline 
\textcolor{red}{gap$>$} \textcolor{blue}{IsCactus(C);}\newline 
true
 }}
 \end{center}
Da es sich bei der Sphäre $C$ um einen Multi-Tetraeder handelt, liefert \Cref{zerlegung} eine Tetraeder-Zerlegung von $C.$
\begin{center}
 $\fbox{
\parbox{15cm}{
\textcolor{red}{gap$>$}\textcolor{blue}{ VerticesOfFaces(s);}\newline 
$[ [ 1, 4, 6 ], [ 1, 4, 7 ], [ 1, 6, 7 ], [ 4, 7, 8 ], [ 2, 5, 6 ],
[ 3, 5, 6 ],[ 2, 4, 5 ], [ 2, 4, 6 ], [ 3, 5, 7 ],$ \newline
$  [ 3, 6, 7 ],
[ 4, 5, 8 ], [ 5, 7, 8 ] ]$\newline
\textcolor{red}{gap $>$}\textcolor{blue}{d:=[ [1, 4, 6, 7 ], [ 3, 5, 6, 7 ], [ 2, 4, 5, 6 ],[ 8, 4, 5, 7 ], [ 4, 5, 6, 7 ]];}\newline
\textcolor{red}{gap $>$} \textcolor{blue}{IsTetrahedrialDecomposition(C,d);}\newline
true
 }}$
 \end{center}
 Ausgehend von der Tetraeder-Zerlegung des Multi-Tetraeders $C$ können wir gezielt durch Anwenden der Kantendrehungen an den Kanten $4$ und $5$ und der im Beweis von \Cref{tzer} präsentierten Fallunterscheidung gezielt eine Tetraeder-Zerlegung der Sphäre $S$ konstruieren.
 \begin{center}
 $\fbox{
\parbox{15cm}{
\textcolor{red}{gap$>$} \textcolor{blue}{FacesOfEdge(C,4);;}\newline 
\textcolor{red}{gap$>$}\textcolor{blue}{Union(VerticesOfFace(C,last[1]),VerticesOfFace(C,last[2]));}
\newline
 [ 1, 4, 7, 8 ]\newline 
\textcolor{red}{gap$>$}\textcolor{blue}{dd:=[ [1, 4, 6, 7 ], [ 3, 5, 6, 7 ], [ 2, 4, 5, 6 ], [ 8, 4, 5, 7 ],$ \newline $ [ 4, 5, 6, 7 ], [ 1,
4, 7, 8 ]];}\newline 
\textcolor{red}{gap$>$}\textcolor{blue}{ IsTetrahedrialDecomposition(s1,dd);}\newline 
true\newline 
\textcolor{red}{gap$>$} \textcolor{blue}{FacesOfEdge(s1,4);;}\newline 
\textcolor{red}{gap$>$}\textcolor{blue}{Union(VerticesOfFace(s1,last[1]),VerticesOfFace(s1,last[2]));}
\newline [ 2, 3, 5, 6 ]\newline 
\textcolor{red}{gap$>$}\textcolor{blue}{ddd:=[[1,4,6,7],[3,5,6,7],[2,4,5,6],[8,4,5,7],[4,5,6,7],[4,7,1,8],[2,3,5,6]];;}\newline 
\textcolor{red}{gap$>$}\textcolor{blue}{IsTetrahedrialDecomposition(S,ddd);}\newline 
true
 }}$
 \end{center}
%\end{bsp}
\[
\textcolor{red}{fehler}
\]
\begin{lemma}
Seien $X$ und $Y$ Sphären und $\phi:X\to Y$ ein Isomorphismus von $X$ nach $Y$. Für eine Tetraeder-Zerlegung $D$ von $X$ ist \[
\phi(D):=\{\phi(d)\mid d\in D\}
\] eine Tetraeder Zerlegung von $Y$.
\end{lemma}
\begin{proof}
Sei $F$ eine Fläche in $X$, dann existiert genau eine Fläche $F'$ mit $\phi(F')=F.$
 Da es zu $F'$ genau ein $d\in D$ mit $X_0(F) \subseteq d$ gibt, folgt $X_0(F)\subseteq \phi(d).$
  Angenommen es existiert ein weiteres $d'\in \phi(D)$ mit $X_0(F)\subseteq d'$. Dann folgt direkt $X_0(F)\in \phi^{-1}(d').$
   Da $d$ eindeutig mit der Eigenschaft $X_0(F')\subseteq d,$ muss also $\phi^{-1}(d')=d$ sein. Dies bedeutet aber $d'=\phi(d).$ Sei nun $n\in \Pot_3(X_2)-\{X_0(F)\mid \,F \in X_2\}.$ Angenommen es existiert ein $d\in \phi(D)$ mit $n\subseteq d.$ Somit erhalten wir $\phi^{_1}(n)\in d_1:=\phi^{-1}(d)\in D.$ Da  $D$ eine Tetraeder-Zerlegung ist existiert ein weiteres $d_2$ mit $\phi^{n}\subseteq d_2,$ woraus wir $n\in subseteq \phi(d_1)\phi(d_2).$ Die Aussage, dass es genau zwei Elemente in $\phi(D)$ mit dieser Eigenschaft gibt erhalten wir mit analoger Begründung wie oben.  
\end{proof}
Mithilfe der Tetraeder-Zerlegung erhalten wir eine neue Anschauung der vertex-treuen Sphären. Durch eine Tetraeder-Zerlegung erhalten wir nämlich eine Repräsentation unserer Sphären durch eine Menge vierelementiger Mengen. Den Übergang von dieser zu der Sphäre liefert uns die symmetrische Differenz. Hierfür definieren wir die symmetrische Differenz $\Delta D$ einer Tetraeder-Zerlegung $D:$
\begin{itemize}
\item Falls $\vert D\vert =1,$ also $D=\{d\}$ ist, dann wählen wir $\Delta D:=\Pot_3(d).$
\item Falls $\vert D\vert >1$ ist, dann hat die Tetraeder-Zerlegung für ein $n>1$ die Gestalt $D=\{d_1,\ldots,d_n\}.$
Wir  erhalten $\Delta D$ in diesem Fall durch
\[
\Pot_3(d_1)\Delta \ldots \Delta \Pot_3(d_n).
\]
Dies liefert uns die Gestalt 
\[
\Delta D=\bigcup^n_{i=1} \Pot_3(d_i)\setminus  (\bigcup^n_{i,j=1,i\neq j}\Pot_3(d_i)\cap \Pot_3(d_j))
\]
\end{itemize} 
Beispiele hierfür haben wir bereits gesehen.
\begin{itemize}
\item $\Pot_3(\{1,2,3,4\})$ bildet die Tetraeder-Zerlegung eines Tetraeders und gleichzeitig auch den Flächen-Träger des Tetraeders.
\item Eine Tetraeder-Zerlegung des Oktaeders aus obigen Beispiels lautet
 \[
D=\{\{1,2,3,6\},\{1,2,5,6\},\{1,4,5,6\},\{1,3,4,6\}\}. 
 \]
 Aus dieser erhalten wir 
 \begin{align*}
\Delta D=&\Pot_3(\{1,2,3,6\},\{1,2,5,6\},\{1,4,5,6\},\{1,3,4,6\})\\
=&\{\{1,2,3\},\{2,5,6\},\{1,2,5\},\{2,3,6\},\\
&\{1,4,5\},\{3,4,6\},\{1,3,4\},\{4,5,6\}\}
\end{align*}
und dies bildet erneut einen Flächen-Träger des Oktaeders.
\end{itemize} 
Diese Beispiele dienen zur Motivation folgendes Lemmas.
\begin{lemma}
Sei $X$ eine vertex-treue Sphäre und $D$ eine Tetraeder-Zerlegung von $X.$ Dann ist $\Delta D$ ein Flächen-Träger von $X.$
\end{lemma}
\begin{proof}
Sei $D=\{d_1,\ldots,d_n\}$ für ein $n>1$
und $F$ eine beliebige Fläche in $X.$ Dann existiert genau ein $i$ mit $X_0(F)\subseteq d_i.$ Damit ist $X_0(F) \in \Delta D.$ Betrachten wir nun $N\in \Pot_3(X_0)-\{X_0(F)\mid F\in X_2\}.$ Da $D$ eine Tetraeder-Zerlegung ist, gibt es entweder kein $d\in D$ mit $N\subseteq d$ oder genau zwei $d_1,d_2$ mit $N\subseteq d_1,d_2.$ Falls ersteres der Fall ist, folgt direkt $N\notin \Delta D.$ Im zweiten Fall erhalten wir $N\in \bigcup \Pot_3(d_i),$ aber auch $N\in \bigcup Pot_3(d_1)\cap \Pot_3(d_2),$ wodurch wieder $N\notin \Delta D$ folgt. 
Also ist insgesamt $\Delta D=\{X_0(F)\mid F\in X_2\}.$ 
\end{proof}
\subsection{wilde Färbungen auf Multi-Tetraedern}
\begin{satz}\label{wild}
Sei $X$ ein Multi-Tetraeder. Dann besitzt $X$ eine wilde Färbung.
\end{satz}
\begin{proof}
Wir weisen diese Aussage per Induktion nach. Für einen Tetraeder mit dem Flächen-Träger
\[
\{\{1,2,3\},\{1,2,4\},\{1,3,4\},\{2,3,4\}\}
\] erhalten wir eine wilde Färbung durch die Abbildung
\begin{align*}
\omega(e)=\Biggl\{
\begin{tabular}[l]{lcr}
a, falls e=\{1,2\},\{3,4\}\\
b, falls e=\{1,3\},\{2,4\}\\
c, falls e=\{1,4\},\{2,3\}
\end{tabular}.
\end{align*}
Da der Tetraeder eine vertex-treue Sphäre ist, wurden an dieser Stelle die Kanten mit den inzidenten Ecken identifiziert.
\begin{figure}[H]
\begin{center}
\includegraphics[viewport=0cm 23.6cm 6cm 27cm]{wildcTet}
\end{center}
\caption{wild gefärbter Tetraeder}
\end{figure}
Sei nun $X$ ein Multi-Tetraeder mit $n=\vert X_2 \vert >4$ und $V\in X_0$ eine Ecke vom Grad 3. Dann erhalten wir durch Entfernen des Tetraeders an der Stelle $V$ den Multi-Tetraeder $Y=T_V(X)$ mit $\vert Y_2 \vert=n-2.$ Deshalb erhalten wir für $Y$ eine wilde Färbung $\omega:Y_1\to \{a,b,c\}.$ Diese kann auf genau eine Art und Weise zu einer wilden Färbung auf $X$ erweitert werden.
Denn seien $F_1,F_2,F_3$ die Flächen des angehängten Tetraeders. Dann gilt $X_1(X_2(V))=\{e_1,e_2,e_3\}\cup \{e_a,e_b,e_c\},$ wobei $X_1(V)=\{e_1,e_2,e_3\}$ ist und $e_a,e_b,e_c$ geeignete Kanten in $Y_1$ sind. Da die letzteren Kanten zu derselben Fläche $F\in Y_2$ inzident sind, müssen diese Kanten unter $\omega$ paarweise verschieden gefärbt sein, also  muss ohne Einschränkung $
\omega(e_a)=a,$ $\omega(e_b)=b$ und $\omega(e_c)=c$
gelten.
\begin{figure}[H]
\begin{center}
\includegraphics[viewport=0cm 23.7cm 5cm 26.7cm,scale=1.1]{colouredTriangle}
\end{center}
\caption{Ausschnitt der wild-gefärbten Sphäre $Y$}
\end{figure}
Die Flächen des Tetraeders an der Stelle $V$ sind zu genau einer Kante aus der Menge $\{e_a,e_b,e_c\}$ und genau zwei Kanten aus der Menge $\{e_1,e_2,e_3\}$ inzident. Deshalb erhalten wir ohne Einschränkung 
\begin{align*}
&X_1(F_1)=\{e_1,e_2,e_a\},\\
&X_1(F_2)=\{e_2,e_3,e_b\}\\
& X_1(F_3)=\{e_1,e_3,e_a\}.\\
\end{align*}
\begin{figure}[H]
\begin{center}
\includegraphics[scale=0.9,viewport=0cm 19.5cm 10cm 26.cm]{notcol3gon}
\end{center}
\caption{Ausschnitt der Sphäre $X$}
\end{figure}
Die wilde Färbung $\omega$ auf $Y$ muss nun durch Ergänzen der fehlenden Bilder zu einer wilden Färbung $\omega_X$ auf $X$ erweitert werden. Für $e\in Y_1\subseteq X_1$ setzen wir  also $\omega_X(e)=\omega(e).$  
\begin{figure}[H]
\begin{center}
\includegraphics[scale=0.9,viewport=0cm 19.5cm 10cm 27cm]{col3gon}
\end{center}
\caption{Ausschnitt der teilweise wild-gefärbten Sphäre $X$}
\end{figure}
Da $e_1 \in X_1(F_1)\cap X_1(F_3)$ ist und die Gleichheiten $\omega_X(e_c)=\omega(e_c)=c$ und $\omega_X(e_a)=\omega(e_a)=a$ gelten, muss $\omega_X(e_1)=b$ sein, da sonst keine wilde Färbung zustande kommt. Analog erhalten wir $\omega_X(e_2)=c$ und $\omega_X(e_3)=a$ und schlussendlich eine Färbung der Sphäre $X$.
\begin{figure}[H]
\begin{center}
\includegraphics[scale=0.9,viewport=0cm 19.5cm 10cm 27cm]{col3gon2}
\end{center}
\caption{Ausschnitt der wild-gefärbten Sphäre $X$}
\end{figure}
\end{proof}
Durch analoge Beweisführung kann ebenfalls gezeigt werden, dass für Multi-Tetraeder eine gleichschenklige Färbung existiert. 

\begin{satz}
Sei $X$ ein Multi-Tetraeder. Dann besitzt $X$ bis auf Permutation der Farben $a,b,c$ genau eine wilde Färbung.
\end{satz}
\begin{proof}
Angenommen obige Aussage gilt nicht. Dann gibt es einen bezüglich der Flächenanzahl minimalen Multi-Tetraeder $X$, sodass $\omega_1$ und $\omega_2$ zwei wilde Färbungen mit 
\begin{align} \label{eq}
\{\omega_1^{-1}(\{a\}),\omega_1^{-1}(\{b\}),\omega_1^{-1}(\{c\})\}\neq \{\omega_2^{-1}(\{a\}),\omega_2^{-1}(\{b\}),\omega_2^{-1}(\{c\}\}
\end{align}
sind. Wir können $\vert X_2\vert >4$ annehmen, da der Tetraeder bis auf Permutation der Farben genau eine wilde Färbung besitzt. Sei $V$ eine Ecke vom Grad 3 in $X.$ Da jeweils zwei der drei Kanten in $X_1(V)$ inzident zu derselben Fläche sind, müssen die Kanten in $X_1(V)$ unter den Färbungen $\omega_1$ und $\omega_2$ paarweise verschiedene Farben erhalten. Deshalb können wir ohne Einschränkung 
\[
\omega_1(e)=\omega_2(e)
\] 
für alle $e\in X_1(V)$ annehmen. Es sei angemerkt, dass hierdurch die Bedingung (\ref{eq}) nicht verletzt wird. Durch Entfernen des Tetraeders an der Stelle $V$ erhalten wir den Multi-Tetraeder $Y=T_V(X)$ und durch Einschränken der Färbungen $\omega_1$ und $\omega_2$ auf $Y_1=X_1-X_1(V)$ erhalten wir wilde Färbungen auf $Y$. Da $X$ minimal mit der Eigenschaft der nicht eindeutigen Färbung war, muss
\[
\omega_1(e)=\omega_2(e)
\]
für alle $e\in Y_1$ gelten, woraus direkt $\omega_1=\omega_2$ folgt. Somit erhalten wir den gewünschten Widerspruch.
\end{proof}
Jedoch sind gleichschenklige Färbungen von Multi-Tetraedern nicht eindeutig. Dieser Sachverhalt lässt sich am Beispiel des Doppel-Tetraeders erkennen.
\begin{center}
$\fbox{
\parbox{14cm}{
\textcolor{red}{gap$>$} \textcolor{blue}{ DT;}\\
simplicial surface (5 vertices, 9 edges, and 6 faces)
}}$
\end{center}
Die gleichschenklig gefärbten Doppel-Tetraeder erhalten wir durch Aufrufen des Befehls 
\textsf{AllIsoscelesColouredSurfaces}.
\begin{center}
$\fbox{
\parbox{14cm}{
\textcolor{red}{gap$>$} \textcolor{blue}{AllIsoscelesColouredSurfaces(DT);}\newline
[ isosceles coloured surface (5 vertices, 9 edges and 6 faces),\newline
  isosceles coloured surface (5 vertices, 9 edges and 6 faces) ]
}}$
\end{center}
Dabei zeigt die nächste Abbildung die erste Färbung des Doppel-Tetraeders 
\begin{figure}[H]
\begin{center}
\includegraphics[viewport=0cm 22cm 5cm 27cm]{dt1}
\end{center}
\caption{gleichschenklig gefärbter Doppel-Tetraeder}
\end{figure}
und folgendes Bild die zweite berechnete Färbung des Doppel-Tetraeders.
\begin{figure}[H]
\begin{center}
\includegraphics[viewport=0cm 22.cm 5cm 27cm]{dt2}
\end{center}
\caption{gleichschenklig gefärbter Doppel-Tetraeder}
\end{figure}
\begin{lemma}
Sei $X$ ein Multi-Tetraeder. Falls $X$ eine zahme Färbung besitzt, muss diese eine $rrr$-Struktur sein.
\end{lemma}
\begin{proof}
Sei $\omega$ eine zahme Färbung auf $X$ und $V$ eine Ecke vom Grad 3 in $X.$ Weiterhin seien $F_1,F_2,F_3\in X_2(V)$ und $e_1,e_2,e_3,e_a,e_b,e_c\in X_1$ paarweise verschieden, sodass 
\begin{align*}
&X_1(F_1)=\{e_1,e_2,e_a\},\\
&X_1(F_2)=\{e_2,e_3,e_b\}, \\
&X_1(F_3)=\{e_1,e_3,e_c\},
\end{align*}
gilt.
\begin{figure}[H]
\begin{center}
\includegraphics[scale=0.8,viewport=0cm 19.5cm 10cm 26.cm]{notcol3gon}
\end{center}
\caption{Ausschnitt der Sphäre $X$}
\end{figure}
Da jeweils zwei der drei Kanten $e_1,e_2,e_3$ inzident zu derselben Fläche sind, muss ohne Einschränkung $\omega(e_1)\neq\omega(e_2)\neq\omega(e_3)$ gelten. Durch setzen der Farben $\omega(e_1)=b,\omega(e_2)=c,\omega(e_3)=a$ folgt direkt $\omega(e_a)=a,\omega(e_b)=b,\omega(e_c)=c.$  Durch Betrachten der Typen der Kanten $e_1,e_2,e_3$ folgt die Behauptung.
\begin{figure}[H]
\begin{center}
\includegraphics[scale=0.8,viewport=0cm 19.5cm 10cm 27cm]{col3gon2}
\end{center}
\caption{Ausschnitt der wild-gefärbten Sphäre $X$}
\end{figure}
\end{proof}
Im Folgenden wollen wir untersuchten, welche Multi-Tetraeder eine zahme Färbung besitzen. Mit dem Tetraeder haben wir bereits ein Beispiel eines zahm gefärbten Multi-Tetraeders gesehen. Mit der Definition des \emph{Sterns}, erhalten wir ein weiteres Beispiel.
\begin{definition}
Der \emph{Stern} $S$ ist der Multi-Tetraeder, der durch das Symbol $1_11_21_31_4$ konstruiert wird.
\begin{figure}[H]
\begin{center}
\includegraphics[scale=0.7,viewport=0cm 2.5cm 8cm 8cm]{star}
\end{center}
\caption{Stern}
\end{figure}
\end{definition}
Der Stern ist ein weiterer Multi-Tetraeder mit einer zahmen Färbung. Dieser entsteht durch Tetraeder-Erweiterungen an allen Flächen des Tetraeders. 
\begin{center}
$\fbox{\parbox{14cm}{
\textcolor{red}{gap$>$} \textcolor{blue}{MultiTetraederBySymbol([[1,1],[1,2],[1,3],[1,4]]);}\newline
simplicial surface (8 vertices, 18 edges, and 12 faces)\newline
\textcolor{red}{gap$>$}\textcolor{blue}{ AllTameColouredSurfaces(last);}\newline
[ tame coloured surface (RRR with 8 vertices, 18 edges and 12 faces)
 ]
}}$
\end{center}
Wir wollen nun anlehnend an den Stern \emph{maximale} Multi-Tetraeder charakterisieren.
\begin{definition}
Sei $X$ ein Multi-Tetraeder. Durch das iterative Entfernen aller Tetraeder entsteht im Sinne von Definition \ref{defcac} eine Kette 
\[
X=X^{(0)}\to X^{(1)}\to \ldots \to X^{(t)}\cong X^{(t+1)}
\]
Wir nennen einen Multi-Tetraeder \emph{maximal}, falls $X^{(t)}\cong T$ ist und für alle $0\leq i< t$ die Gleichheit
\[
\vert \{V\in X_0^{(i)}\,\mid \, \deg(V)=3\}\vert=\vert X^{(i+1)}_2\vert 
\]
gilt.
\end{definition}
Damit sind der Tetraeder und der oben definierte Stern maximal. 
Denn für den Stern ergibt sich die Kette 
\[
S\to S^{(1)}=T.
\]
und es gilt 
\[
\vert \{V\in S_0\,\mid \, \deg(V)=3\}\vert=4=\vert T_2\vert .
\]
Der Multi-Tetraeder $X$, den wir durch das Symbol $1_11_21_3$ erhalten, ist jedoch nicht maximal. Es gilt zwar
\[
X\to X^{(1)}=T,
\] 
aber zweite Teil der Definition wird von der Sphäre $X$ nicht erfüllt, denn 
\[
\vert \{V\in X_0\,\mid \, \deg(V)=3\}\vert=3\neq 4=\vert T_2\vert.
\]
\begin{lemma}\label{max}
Sei $X$ ein maximaler Multi-Tetraeder. Dann existiert eine zahme Färbung auf $X.$
\end{lemma}
\begin{proof}
Wir weisen die Aussage induktiv nach. Bei dem Tetraeder und dem Stern kann nachgerechnet werden, dass eine zahme Färbung existiert und somit diese Multi-Tetraeder $rrr$-Strukturen bilden. Sei nun $X$ ein maximaler Multi-Tetraeder, der nicht zum Stern oder Tetraeder isomorph ist.
 Dann ist $Y=X^{(1)}$ ebenfalls maximal und nach Induktionsvoraussetzung erhalten wir eine zahme Färbung $\omega$ auf $Y,$ die eine $rrr$-Struktur bildet. Seien nun $F_1$ und $F_2$ zwei beliebige benachbarte Flächen in $Y$, die $Y_1(F_1)=\{e_a,e_b,e_c\}$ und $Y_1=\{e_a,e_b',e_c'\}$ und $Y_0(e_b)\cap Y_0(e_c')\neq \emptyset$ für geeignete Kanten erfüllen.
Da $Y$ eine zahme Färbung besitzt und diese eine $rrr$-Struktur bildet, gilt ohne Einschränkung
\begin{align*}
&\omega(e_a)=a\\
&\omega(e_b)=b=\omega(e_b')\\
&\omega(e_c)=c=\omega(e_c')
\end{align*}
\begin{figure}[H]
\begin{center}
\includegraphics[scale=1,viewport=0cm 22cm 12cm 26.5cm]{tamcol2}
\end{center}
\caption{Ausschnitt der wild-gefärbten Sphäre $Y$}
\end{figure}
In $X$ werden die Flächen $F_1$ und $F_2$ durch Tetraeder ersetzt. Und die wilde Färbung auf $Y$ muss nun durch Ergänzen der fehlenden Farben der neuen Kanten zu einer wilden Färbung auf $X$ ergänzt werden. Seien nun also $f_1,f_2,f_3$ die Flächen des Tetraeders, der $F_1$ ersetzt und $f'_1,f'_2,f'_3$ die Flächen des Tetraeders der $F_2$ ersetzt, sodass für geeignete Kanten folgende Relationen erfüllt sind:
\begin{align*}
&X_1(f_1)=\{e_a,e_1,e_2\}\\
&X_1(f_2)=\{e_b,e_2,e_3\}\\
&X_1(f_3)=\{e_c,e_1,e_3\}\\
&X_1(f'_1)=\{e_a,e_1',e_2'\}\\
&X_1(f'_2)=\{e_b',e_2',e'_3\}\\
&X_1(f'_3)=\{e_c',e'_1,e'_3\}
\end{align*}
\begin{figure}[H]
\begin{center}
\includegraphics[scale=1,viewport=0cm 22cm 12cm 26.5cm]{tamcol3}
\end{center}
\caption{Ausschnitt der wild-gefärbten Sphäre $X$}
\end{figure}
Dann erhalten wir durch analoges Vorgehen wie im Beweis von \Cref{wild}, eine wilde Färbung $\omega_X$ auf $X$ als eine eindeutige Erweiterung der Färbung auf $Y,$ welche insbesondere die $rrr-$Struktur auf dem Multi-Tetraeder $X$ fortfuhrt. Für $e\in X_1$ ergibt sich diese durch 
\begin{align*}
\omega_X(e)=\Biggl\{\begin{tabular}[l]{lcr}
a, falls $e=e_3,e_3'$\\
b, falls $e=e_1,e_1'$\\
c, falls $e=e_2,e_2'$\\
$\omega(e)$, sonst
\end{tabular}.
\end{align*}

\begin{figure}[H]
\begin{center}
\includegraphics[scale=1,viewport=0cm 22cm 12cm 26.5cm]{tamcol}
\end{center}
\caption{Ausschnitt der wild-gefärbten Sphäre $X$}
\end{figure}
\end{proof}
Wenn der Beweis genauer betrachtet wird, kann die nachstehende Folgerung formuliert werden.
\begin{folgerung}
Sei $X$ ein Multi-Tetraeder mit einer zahmen Färbung. Sei $Y$ der Multi-Tetraeder, der dadurch entsteht, dass an allen Flächen von $X$ Tetraeder-Erweiterungen durchgeführt werden. Dann existiert eine zahme Färbung auf $Y.$
\end{folgerung}
Die Rückrichtung von \Cref{max} gilt jedoch nicht, dies halten wir mit folgendem Beispiel fest:\\
Die Sphäre, die wir betrachten ist der Multi-Tetraeder mit 20 Flächen, der durch das Symbol $1_21_12_42_32_13_43_33_2$ beschrieben wird.
\begin{center}
$\fbox{
\parbox{14cm}{
\textcolor{red}{gap$>$} \textcolor{blue}{s;}\newline
simplicial surface (12 vertices, 30 edges, and 20 faces)\newline
\textcolor{red}{gap$>$} \textcolor{blue}{GetSymbol(s);}\newline
[ [ 1, 2 ], [ 1, 1 ], [ 2, 4 ], [ 2, 3 ], [ 2, 1 ], [ 3, 4 ], [ 3, 3 ],
  [ 3, 2 ] ]\newline
\textcolor{red}{gap$>$}\textcolor{blue}{ Length(AllTameColouredSurfaces(s));}\newline
1
}}$
\end{center}
Dieser besitzt also eine zahme Färbung. Jedoch ist $s$ nicht maximal wie durch folgende Rechnung gezeigt werden kann.
\begin{center}
$\fbox{
\parbox{14cm}{
\textcolor{red}{gap$>$} \textcolor{blue}{RemoveAllTetra(s);}\newline
simplicial surface (6 vertices, 12 edges, and 8 faces)\newline
\textcolor{red}{gap$>$}\textcolor{blue}{ Length(Filtered(Vertices(s),g-$>$FaceDegreeOfVertex(s,g)=3));}\newline
6
}}$
\end{center}
\subsection{Flächengraph von Multi-Tetraedern}
In Kapitel 3 haben wir uns den Flächengraphen simplizialer Flächen gewidmet. In diesem Abschnitt betrachten wir im Genauen die Flächengraphen von Multi-Tetraedern. Es lässt sich nämlich ein Zusammenhang zwischen dem Flächengraphen eines Multi-Tetraeders und der Sphäre, die durch eine Tetraeder-Erweiterung konstruiert wird, erkennen.\\\\
Den Flächengraphen des Tetraeders haben wir bereits kennengelernt. Dieser wird bis auf Isomorphie durch $G_T=(V,E)$ mit $V=\{F_1,F_2,F_3,F_4\}$ und $E=\Pot_2(V)$ dargestellt.
\begin{figure}[H]
\begin{center}
\includegraphics[viewport=1.5cm 20.5cm 18cm 23cm]{Image_FaceGraphTetraeder}
\end{center}
\caption{Flächengraph des Tetraeders}
\end{figure}

Durch eine Tetraeder-Erweiterung am Tetraeder wird der Doppel-Tetraeder mit zugehörigem Flächengraph $G_{DT}$ konstruiert. Diesen enthalten wir bis auf Isomorphie durch die Knoten $V=\{F_1,\ldots,F_6\}$ und die Kanten 
\begin{align*}
E=\{&\{F_1,F_2\},\{F_1,F_3\},\{F_1,F_4\},\{F_2,F_4\}\\
&\{F_2,F_5\},\{F_3,F_5\},\{F_3,F_6\},\{F_4,F_6\},\{F_5,F_6\}\}.
\end{align*}
\begin{figure}[H]
\begin{center}
\includegraphics[viewport=2cm 20.5cm 14cm 22.2cm]{Image_FaceGraphdoubleTetraeder}
\end{center}
\caption{Flächengraph des Doppel-Tetraeders}
\end{figure}
Durch genaueres Hinschauen lässt sich erkennen, dass der zu der Fläche $F_3$ zugehörige Knoten im ursprünglichen Graphen des Tetraeders durch die Erweiterung unter Berücksichtigung der Inzidenzen in drei neue Knoten aufgeteilt wird. Im Allgemeinen ist dieses Phänomen weiterhin erkennbar, weshalb wir dieses an dieser Stelle beschreiben wollen.\\
Sei $X$ ein Multi-Tetraeder mit zugehörigem Flächengraph $G_X$ und $F\in X_2$ eine Fläche mit $X_2(X_1(F))=\{F,F_1,F_2,F_3\}$ für geeignete Flächen. 
Da für die Skizzierung des erwähnten Zusammenhangs nur die Knoten der Flächen $F,F_1,F_2,F_3$ relevant sind, wird in den folgenden Abbildungen auch nur dieser Ausschnitt des Flächengraphen dargestellt. Der Flächengraph kann mehr Knoten und Inzidenzen enthalten, diese sind aber für unsere Zwecke nicht von Bedeutung.
\begin{figure}[H]
\begin{center}
\includegraphics[viewport=3cm 19.6cm 14cm 23cm]{Image_fg1}
\end{center}
\caption{Ausschnitt eines Flächengraphen eines Multi-Tetraeders}
\end{figure}
Auf Ebene der simplizialen Flächen wird bei einer Tetraeder-Erweiterung die Fläche $F$ entfernt und durch den 3-gon mit den Flächen $\{F_a,F_b,F_c\}$ so ersetzt, dass $F_1$ und $F_a$, $F_2$ und $F_b$ bzw. $F_3$ und $F_c$ benachbarte Flächen in der konstruierten Sphäre sind. Dieses Vorgehen muss nun nur noch auf der Ebene der Flächengraphen nachgeahmt werden.\\
Bei einer Tetraeder-Erweiterung wird in einem ersten Schritt der Knoten $F$ durch die Knoten $F_a,F_b,F_c$ ersetzt, wobei $F_a,F_b,F_c$ die Flächen des angehängten Tetraeders sind. Dann werden die Kanten $\{F,F_1\},\{F,F_2\}$ und $\{F,F_3\}$ in dem Graphen gelöscht.
\begin{figure}[H]
\begin{center}
\includegraphics[viewport=3cm 19.7cm 14cm 23cm]{Image_fg2}
\end{center}
\caption{Graph, der aus dem Flächengraph des Multi-Tetraeders hervorgeht}
\end{figure}
Daraufhin werden die Inzidenzen 
\[
\{F_a,F_b\},\{F_a,F_c\},\{F_b,F_c\},\{F_1,F_a\},\{F_2,F_b\},\{F_3,F_c\}
\] 
in dem letzterem Graphen hergestellt, um so schließlich den Flächengraph des Multi-Tetraeders, der durch die Tetraeder-Erweiterung entstanden ist, zu erzeugen.
\begin{figure}[H]
\begin{center}
\includegraphics[viewport=3cm 19.7cm 14cm 23cm]{Image_fg3}
\end{center}
\caption{Ausschnitt eines Flächengraphen eines Multi-Tetraeders}
\end{figure}
\begin{bsp}
Im folgenden Beispiel steht die Verdeutlichung der obigen Prozedur im Vordergrund. Deshalb soll auf eine genaue Definition der zugehörigen Flächengraphen durch Knoten und Kanten verzichtet werden. Wir geben uns an dieser Stelle mit den Abbildungen der jeweiligen Graphen zufrieden. \\
Den Flächengraph des Tetraeders haben wir bereits in dem einführenden Beispiel gesehen.
\begin{figure}[H]
\begin{center}
\includegraphics[scale=0.8,viewport=10cm 19.5cm 5cm 22.5cm]{mttry1}

\end{center}
\caption{Ausschnitt eines Flächengraph eines Multi-Tetraeders}
%\caption{Kantendrehung}
\end{figure}
Durch eine Tetraeder-Erweiterung an der Fläche 1 erhalten wir die neuen Flächen 5,6,7 und den zugehörigen Graphen unter Beachtung der neuen Inzidenzen.
\begin{figure}[H]
\begin{center}
\includegraphics[viewport=10cm 19.5cm 5cm 23.cm]{Mttry4}
\end{center}
\caption{Flächengraph des Doppel-Tetraeders}
\end{figure}
Durch eine Tetraeder-Erweiterung am Doppel-Tetraeder an der Fläche 2 erhalten wir bis auf Isomorphie den Multi-Tetraeder mit 8 Flächen, der folgenden Flächengraphen besitzt. 
\begin{figure}[H]
\begin{center}
\includegraphics[scale=0.5,viewport=16cm 14.5cm 5cm 24cm]{bsp9}
\end{center}
\caption{Flächengraph eines Multi-Tetraeders}
\end{figure}
Anhängen eines Tetraeders an der Fläche 3 liefert uns einen Multi-Tetraeder mit 10 Flächen und folgendem Flächengraph.
\begin{figure}[H]
\begin{center}
\includegraphics[scale=0.6,viewport=15.5cm 14.cm 5cm 23cm]{bsp10}
\end{center}
\caption{Flächengraph eines Multi-Tetraeders}
\end{figure}
Schließlich kommt durch eine Erweiterung an der Fläche 7 folgender Flächengraph zustande.
\begin{figure}[H]
\begin{center}
\includegraphics[scale=0.6,viewport=16cm 14cm 5cm 23cm]{bsp11}
\end{center}
\caption{Flächengraph eines Multi-Tetraeders}
\end{figure}
\end{bsp}
\subsection{Eigenwerte von Multi-Tetraedern}\label{eigenwerte}
In diesem Abschnitt wollen wir uns eine interessante Invariante der Multi-Tetraeder anschauen, nämlich das \emph{Tetraeder-Polynom}. Hierfür werden wir den Begriff der Einbettung definieren und dann als Ziel beobachten, wie sich die Multi-Tetraeder unter dieser Invariante verhalten.
\begin{definition}
Sei $X$ ein Multi-Tetraeder. Wir nennen eine Abbildung 
\[
c:X_0\to \mathbb{R}^+
\]
eine reelle Einbettung von $X,$ falls ein $d\in \mathbb{R}^+$ existiert, sodass für alle benachbarten $V,V'$ in $X$ die Gleichung
\[
\| c(V)-c(V')\|=d
\] 
gilt. Wir nennen $c$ eine vertex-treue Einbettung, falls $c(V)\neq c(V')$ für $V\neq V'$ ist.
\end{definition}
Dies ist nur eine Umformulierung aus dem in Kapitel \ref{cactus} vorgestellten Resultat. An dieser Stelle sei angemerkt, das nicht klar ist, dass stets eine vertex-treue Einbettung eines Multi-Tetraeder existiert.
\begin{definition}
Sei $X$ ein Multi-Tetraeder mit $n=\vert X_0\vert $ und $c:X_0\to \mathbb{R}^3$ eine reelle Einbettung von $X.$ Die Matrix $K\in \mathbb{R}^{3\times n}$ definiert durch
\[
(K^c_X)_{-,i}=c(V_i)
\] 
nennen wir die Koordinatenmatrix von $X$ und den Vektor $S^c_X$ definiert durch 
\[
\frac{1}{n}\sum_{i=1}^nc(V_i)
\] den Schwerpunkt von $X.$
Falls $c$ aus dem Kontext heraus klar ist, schreiben wir nur $S_X$ und $K_X.$
Weiterhin sei $1_{3\times n}\in \mathbb{R}^{3\times n}$ die Matrix, in der jeder Eintrag gleich 1 ist. Die \emph{Tetraeder-Matrix} $\mathcal{T}^c_X$ von $X$ ist durch
\[
(K_X-S_X1_{3\times n})(K_X-S_X1_{3\times n})^T
\] 
gegeben. 
\end{definition} 
Wir können ohne Einschränkung der Allgemeinheit annehmen, dass $d=\sqrt{2}$ ist. 
Für die in Kapitel \ref{cactus} vorgestellte reelle  Einbettung des Tetraeders ist
\[
K_T=\left(\begin{tabular}{cccc}
1&1&-1&1\\
1&-1&1&-1\\
-1&-1&1&1\\
\end{tabular}\right)
\] 
eine Koordinatenmatrix des Tetraeders.
Den zugehörigen Schwerpunkt erhalten wir durch 
\[
\frac{1}{n}\sum_{i=1}^nc(V_i)=
K_T=\left(\begin{tabular}{c}
0\\
0\\
0\\
\end{tabular}\right)
\]
Damit erhalten wir also 
\[
\mathcal{T}_X=K_T=\left(\begin{tabular}{ccc}
4&0&0\\
0&4&0\\
0&0&4\\
\end{tabular}\right)
\]
\begin{bemerkung}
Die Koordinatenmatrix und damit auch der Schwerpunkt  eines Multi-Tetraeders sind nicht eindeutig. Selbst bei Festlegung einer vertex-treuen Einbettung können die Spalten der Matrix permutiert werden. Für die Tetraeder-Matrix kann jedoch die Eindeutigkeit gezeigt werden.
\end{bemerkung}
\begin{lemma}
Seien $X$ ein Multi-Tetraeder und $c,c':X_0\to \mathbb{R}^3$ zwei reelle vertex-treue Einbettungen. Dann gilt 
\[
\mathcal{T}^c_X=\mathcal{T}_X^{c'}.
\]
\end{lemma}
\begin{proof}
Wenn wir annehmen, dass der Abstand von zwei benachbarten Ecken unter beiden Einbettungen gleich ist, muss $c'$ durch eine Komposition einer Translation oder einer orthogonalen Abbildung mit $c$ entstanden sein. 
\begin{itemize}
\item Angenommen es existiert ein $x\in \mathbb{R}^3,$ sodass $c'(V)=c(V)+x$ für alle $V\in X_0$ ist. Dann gilt 
\begin{align*}
&(K_X^{c'}-S_X^{c'}1_{3\times n})_{-,i}=c'(V_i)-S_X^{c'}\\
=&c(V_i)+x-S_X^c-x=(K_X^{c}-S_X^{c}1_{3\times n})_{-,i}
\end{align*}
Also folgt direkt $\mathcal{T}_X^c=\mathcal{T}_X^{c'}$
\item Falls eine orthogonale Matrix $O\in \mathbb{R}^{3\times 3}$ mit $c'(V)=Oc(V)$ existiert dann ist $S_X^{c'}=OS_X^c.$ Da wir die Einträge der Matrix $\mathcal{T}_X^{c'}$ über die Skalarprodukte der verschiedenen Zeilen der Matrix $\mathcal{T}_X^{c'}$ erhalten, und orthogonale Matrizen, das Skalarprodukt unverändert lassen, folgt die Behauptung.
\end{itemize} 
\end{proof}
Also können wir von nun an den Bezug zur Einbettung vernachlässigen und nur noch $\mathcal{T}_X$ schreiben. 
\begin{bemerkung}
Es ist möglich, das obige Lemma in einem allgemeineren Kontext zu formulieren, nämlich für Einbettungen von vertex-treuen Sphären. Da wir aber an dieser Stelle nur an einer Invariante für Multi-Tetraeder interessiert sind, wird hier darauf verzichtet. 
\end{bemerkung}
Führen wir nun das oben erwähnte Tetraeder-Polynom ein.
\begin{definition}
Sei $X$ ein Multi-Tetraeder und $c:X_0\to \mathbb{R}^3$ eine vertex-treue Einbettung. Dann nennen wir das charakteristische Polynom der Matrix $\mathcal{T}_X$ das \emph{Tetraeder-Polynom} von $X.$
\end{definition}
Beispielsweise ist das Tetraeder-Polynom des Tetraeders durch $(x-4)^3$ gegeben.
Es stellt sich heraus, dass das Tetraeder-Polynom eine interessante Invariante der Multi-Tetraeder ist. Für geringere Anzahlen an Flächen geben wir die Anzahl der Klassen an, in die die Menge der Multi-Tetraeder unter dieser Invariante aufgeteilt wird.
\begin{center}
\begin{tabular}{|c|c|c|c|c|c|c|c|c|c|c|}
\hline
\textbf{Flächenanzahl} &4&6&8&10&12&14&16&18&20\\
\hline
\textbf{Anzahl der}&1&1&1&3&7&24&93&434&2110\\
\textbf{Multi-Tetraeder}&&&&&&&&& \\
\hline
\textbf{Anzahl der}&1&1&1&3&7&24&93&433&2106\\
 \textbf{Klassen}&&&&&&&&&\\
\hline
\end{tabular}
 \end{center}
 An den letzteren Zahlenbeispielen ist zu erkennen, dass die Invariante nicht trennend ist. Die reellen Einbettungen der Multi-Tetraeder, die dasselbe Tetraeder-Polynom haben ähneln sich jedoch sehr. Dies kann durch plotten der Multi-Tetraeder beobachtet werden.\\
 Für die Multi-Tetraeder mit bis zu 10 Flächen geben wir nun, die oben eingeführten Eigenschaften an.\\
 \begin{large}
 \textbf{Multi-Tetraeder $\textbf{1}_\textbf{1}$}
 \end{large}\\
\underline{Koordinaten-Matrix} 
 \[
K_{DT}=\left(\begin{tabular}{ccccc}
1&-1&-1&1&-$\frac{5}{3}$ \\
1&-1&1&-1&-$\frac{5}{3}$\\
-1&-1&1&1&$\frac{5}{3}$\\
\end{tabular}\right)
\] 
\underline{Tetraeder-Matrix}
 \[
K_{DT}=\left(\begin{tabular}{ccc}
$\frac{56}{9}$&$\frac{20}{9}$&$-\frac{20}{9}$ \\
$\frac{20}{9}$&$\frac{56}{9}$&$-\frac{20}{9}$\\
-$\frac{20}{9}$&$-\frac{20}{9}$&$\frac{56}{9}$\\
\end{tabular}\right)
\] 
\underline{Tetraeder-Polynom}
\[
x^3-\frac{56}{3}x^2+\frac{304}{3}-\frac{512}{3}
\]
 \begin{large}
 \textbf{Multi-Tetraeder $\textbf{1}_\textbf{1}\textbf{1}_\textbf{2}$}
 \end{large}\\
 \underline{Koordinaten-Matrix}
\[ 
K_{X}=\left(\begin{tabular}{cccccc}
1&-1&-1&1&-$\frac{5}{3}$&$\frac{5}{3}$ \\
1&-1&1&-1&-$\frac{5}{3}$&-$\frac{5}{3}$\\
-1&-1&1&1&$\frac{5}{3}$&-$\frac{5}{3}$\\
\end{tabular}\right)
\] 
\underline{Tetraeder-Matrix}
\[
K_T=\left(\begin{tabular}{ccc}
$\frac{208}{27}$&0&0\\
0&$\frac{86}{9}$&$\frac{50}{9}$\\
0&$\frac{50}{9}$&$\frac{86}{9}$\\
\end{tabular}\right)
\] 
\underline{Tetraeder-Polynom}
\[
x^3-\frac{724}{27}x^2+\frac{50464}{243}x-\frac{113152}{242}
\]
\begin{large}
 \textbf{Multi-Tetraeder $\textbf{1}_\textbf{1}\textbf{1}_
 \textbf{2}\textbf{1}_
 \textbf{3}$}
\end{large} \\
 \underline{Koordinaten-Matrix}
 \[ 
K_{X}=\left(\begin{tabular}{ccccccc}
1&-1&-1&1&$\frac{5}{3}$&-$\frac{5}{3}$&-$\frac{5}{3}$ \\
1&-1&1&-1&-$\frac{5}{3}$&-$\frac{5}{3}$&$\frac{5}{3}$\\
-1&-1&1&1&-$\frac{5}{3}$&$\frac{5}{3}$&-$\frac{5}{3}$\\
\end{tabular}\right)
\] 
\underline{Tetraeder-Matrix}
\[
K_T=\left(\begin{tabular}{ccc}
$\frac{752}{63}$& $-\frac{200}{63}$&$ -\frac{200}{63}$ \\
 $-\frac{200}{63}$&$ \frac{752}{63}$&$ -\frac{200}{63}$\\
 $ -\frac{200}{63}$& $-\frac{200}{63}$& $\frac{752}{63}$ \\
\end{tabular}\right)
\]
\underline{Tetraeder-Polynom}
\begin{align*}
&x^3-\frac{6768}{189}x^2+\frac{2026944}{5103}x-\frac{6510592}{5103}\\
=&x^3-\frac{752}{21}x^2+\frac{25024}{63}x-\frac{6510592}{5103}
\end{align*}
 \begin{large}
 \textbf{Multi-Tetraeder $\textbf{1}_\textbf{1}\textbf{1}_\textbf{2}\textbf{2}_\textbf{2}$}
 \end{large}\\
 \underline{Koordinaten-Matrix}
\[
K_T=\left(\begin{tabular}{ccccccc}
 1&-1&-1&1&$-\frac{5}{3}$&$\frac{5}{3}$&$-\frac{1}{9}$\\
 1&-1&1&-1&$-\frac{5}{3}$&$\frac{5}{3}$&$-\frac{1}{9}$\\
-1&-1&1&1&$\frac{5}{3}$&$\frac{5}{3}$&$\frac{31}{9}$\\
\end{tabular}\right)
\] 
\underline{Tetraeder-Matrix}
\[
K_T=\left(\begin{tabular}{ccc}
$\frac{1808}{189}$& $\frac{1052}{189}$& $-\frac{52}{189}$ \\
 $\frac{1052}{189}$& $\frac{1808}{189}$& $-\frac{52}{189}$ \\
 $-\frac{52}{189}$& $-\frac{52}{189}$& $\frac{104}{7}$ \\ 
\end{tabular}\right)
\]
\underline{Tetraeder-Polynom}
\[
  x^3-\frac{6424}{189}x^2+\frac{1758640}{5103}x-\frac{4585984}{5103}
\]
 \begin{large}
 \textbf{Multi-Tetraeder $\textbf{1}_\textbf{1}\textbf{1}_\textbf{2}\textbf{2}_\textbf{4}$}
 \end{large}\\
\underline{Koordinaten-Matrix} 
 \[
K_T=\left(\begin{tabular}{ccccccc}
 1&-1&-1&1&$-\frac{5}{3}$&$\frac{5}{3}$&$-\frac{31}{9}$\\
 1&-1&1&-1&$-\frac{5}{3}$&$\frac{5}{3}$&$-\frac{1}{9}$\\
-1&-1&1&1&$\frac{5}{3}$&$\frac{5}{3}$&$\frac{1}{9}$\\
\end{tabular}\right)
\] 
\underline{Tetraeder-Matrix}
\[
\mathcal{T}=\left(\begin{tabular}{ccc}
 $\frac{3728}{189}$&$\frac{1112}{189}$& $\frac{248}{189}$\\
 $\frac{1112}{189}$& $\frac{1808}{189}$& $\frac{8}{189}$ \\
 $\frac{248}{189}$& $\frac{8}{189}$&$\frac{ 496}{63}$ \\
\end{tabular}\right)
\]
\underline{Tetraeder-Polynom}
\[
  x^3-\frac{7024}{189}x^2+\frac{1954240}{5103}x-\frac{6109184}{5103} 
\] 
 \section{2-Taillen-freie Sphären}
In diesem Abschnitt der Arbeit werden wir die Sphären mit 3-Taillen genauer betrachten. 
\subsection{vertex-treue Sphären mit genau einem 3-Taille}
Wir führen zunächst das Zusammensetzen zweier Sphären an einer 3-Taille ein. In 
\begin{definition}
Seien $X$ und $Y$ vertex-treue Sphären, die durch die Flächen-Träger $\xi_X$ bzw. $\xi_Y$ dargestellt werden. Um $X$ und $Y$ durch eine 3-Taille an Flächen $F\in X_2$ und $F'\in Y_2$ zu verbinden, muss die Annahme getroffen werden, dass
$X_0(F)=Y_0(F')$ und $X_0\setminus X_0(F)\cap Y_0\setminus Y_0(F')=\emptyset.$ Dann bildet die Sphäre $X\#Y$ repräsentiert durch $\xi_W=\xi_X \Delta \xi_Y$ eine wohldefinierte simpliziale Fläche.
\end{definition}
\begin{bemerkung}\label{3waist}
Obige Definition lässt sich leicht verallgemeinern. Falls $X_0$ und $Y_0$ disjunkt sind, kann das Zusammensetzen der Sphären mithilfe einer Permutation $\phi=(v_1v_1')(v_2v_2')(v_3v_3')$ für $X_0(F)=\{v_1,v_2,v_3\}$ und $Y_0(F')=\{v_1',v_2',v_3'\}$ durchführt werden. Wir identifizieren $\xi_Y$ mit der Menge  
\[
\{\phi (y)\mid y\in \xi_Y \}.
\]
und bezeichnen die Sphäre, die durch das Zusammensetzen  entsteht mit $X\#_{\phi}Y.$\\
Das Unterteilen einer vertex-treuen Sphäre mit einer 3-Taille in die disjunkte Vereinigung zweier vertex-treuer Sphären kann in ähnlicher Form skizziert werden. Sei $W\subseteq X_1$ eine 3-Taille. Nach Definition \ref{2waistk} kann die Flächenmenge $X_2$ in die 3-Taillen Komponenten $M_1,M_2$ bezüglich $W$ aufgeteilt werden. Weiterhin seien $\{V_1,V_2,V_3\}=X_0(W)$ die Ecken der 3-Taille und $\{P_1,P_2,P_3\}\cap X_0=\emptyset,$ dann erhalten wir die Sphäre $X^W$ durch die Ecken-Flächen-Inzidenzen 
\begin{align*}
\xi_W=&(\{\phi(X_0(m_1))\mid \, m_1\in M_1\}\cup \{\{P_1,P_2,P_3\}\}) \cup\\
 &(\{X_0(m_2)\mid \, m_2\in M_2\}\cup \{\{V_1,V_2,V_3\}\}),
\end{align*}
wobei $\phi=(V_1P_1)(V_2P_2)(V_3P_3)$ ist.
\end{bemerkung}
Seien $X$ und $Y$ zwei Sphären. Weiterhin seien  $F$ eine Fläche in $X$ und $F'$ eine Fläche in $Y$ zusammen mit einer Bijektion $\phi :X_0(F)\to X_0(F').$ Beim Übergang von den Sphären $X$ und $Y$ zu der simplizialen Fläche $X\#_\phi Y$ werden die zwei Flächen $F,F'$ aus der konstruierten Fläche herausgenommen. Außerdem werden jeweils eine Ecke in $X$ und eine Ecke in $Y$ in $X\#_\phi Y$ zu einer Ecke zusammengeführt. Selbiges gilt für die Kanten. Es werden jeweils eine Kante in $X$ und eine Kante in $Y$ zu einer Kante in $X\#_\phi Y$ zusammengefasst. Deshalb ergibt sich für die Euler-Charakteristik
\begin{align*}
\chi(X\#_\phi Y)=&\vert(X\#_\phi Y)_0\vert -\vert (X\#_\phi Y)_1\vert +\vert (X\#_\phi Y)_2\vert\\
&=(\vert X_0\vert+\vert Y_0\vert-3)-(\vert X_1\vert+\vert Y_1\vert-3)+(\vert X_2\vert+\vert Y_2\vert-2)\\ 
&=\vert X_0\vert-\vert X_1\vert+\vert X_2\vert+\vert X_0\vert-\vert X_1\vert+\vert X_2\vert-2\\
&2+2-2.
\end{align*}
Also ist die simpliziale Fläche $(X\#_\phi Y)$ eine Sphäre. 
\begin{bsp}
Der zweite Teil der obigen Bemerkung wird nun am Beispiel des Doppel-Tetraeders veranschaulicht.
Zur Erinnerung geben wir an dieser Stelle den Flächen-Träger eines Doppel-Tetraeders an:
\[
\xi_{DT}\{\{1,2,3\},\{1,2,4\},\{1,3,4\},\{5,2,3\},\{5,2,4\},\{5,3,4\}\}
\]
Dann bilden die Kanten $\{2,3\},\{3,4\}$ und $\{2,4\}$ eine 3-Taille. Also sind $M_1=\{\{1,2,3\},\{1,2,4\},\{1,3,4\}\}$ und $M_2=\{\{5,2,3\},\{5,2,4\},\{5,3,4\}\}$ die 3-Taillen Komponenten bezüglich $W,$ wobei die Flächen durch die inzidenten Flächen repräsentiert werden. Durch Einführen von neuen Ecken 6,7,8 und der Permutation $\phi=(2\, 6)(3\, 7)(4\,8)$ erhalten wir nun durch die simpliziale Fläche $DT^W$ als disjunkte Vereinigung von zwei Tetraedern mit zugehörigem Flächen-Träger
\begin{align*}
\xi_W=&(\{\phi(X_0(m_2))\mid \, m_2\in M_2\}\cup \{\{\{6,7,8\}\}\}) \cup \\
&(\{X_0(m_1)\mid \, m_1\in M_1\}\cup \{\{2,3,4\}\})\\
=&\Pot_3(\{1,2,3,4\})\cup \Pot_3(\{5,6,7,8\})
\end{align*}
\end{bsp}
\begin{bemerkung}
Wenn wir uns die vertex-treue Sphären ohne 3-Taillen als konvexe Figuren, eingebettet in den $\mathbb{R}^3$ vorstellen, dann kommt die obige Konstruktion dem Zusammensetzen oder vielmehr dem Zusammenkleben der beiden Sphären an Flächen dieser Sphären gleich. Dafür gibt es genau 3 Möglichkeiten. Auf kombinatorischer Ebene jedoch, gibt es jedoch 6 Möglichkeiten, zwei Sphären $X$ und $Y$ auf obige Weise an 
Flächen $F\in X_2$ und $F'\in Y_2$ zusammenzusetzen. Denn dies entspricht der Anzahl der Bijektionen $\phi:X_0(F)\to Y_0(F'),$ wobei verschiedene Bijektionen isomorphe Sphären hervorbringen können.
\end{bemerkung}
\begin{bemerkung}
Sei $X$ eine vertex-treue Sphäre und $W_1,\ldots,W_n$ 3-Taillen in $X.$ Durch  $n$-maliges Anwenden der $\Cref{3waist}$ erhalten wir eine simpliziale Fläche $Y,$ die aus $n+1$ Zusammenhangskomponenten besteht. Es existieren also vertex-treue Sphären $Z_1,\ldots,Z_{n+1}$ ohne 3-Taillen, sodass
\[
Y_i=\bigcup_{j=1}^{n+1}(Z_j)_i 
\]
für $i=0,1,2$ ist. Wir nennen $Z_1,\ldots,Z_n$ die \emph{Blöcke} von $X$ und definieren durch 
\[
\prod_{i=1}^{n+1} B_i^{t_i},
\]
wobei $t_i=\vert\{j \mid 1\leq j\leq n+1,\,(Z_j)_2=i\}\vert$ ist, den \emph{Blocktyp} von $X.$
\end{bemerkung}
\begin{lemma}
Seien $X$ bzw. $Y$ vertex-treue Sphären mit zugehörigen Blocktypen $\prod B_i^{t_i}$ bzw. $\prod B_j^{s_j}..$ Falls $X\cong Y$ ist, dann gibt es Permutationen $\pi_1,\pi_2,$ sodass 
\[
\prod B_i^{t_i}=\prod B_{\pi_1(i)}^{\pi_2(t_i)}
\]  
ist.
\end{lemma}
Diese Aussage ist nur eine Umformung eines Resultates des Skriptes \emph{Simplicial Surfaces of Congruent Triangles} und der zugehörige Beweis kann diesem entnommen werden. Die Umkehrung gilt jedoch nicht, wie am Beispiel der Multi-Tetraeder zu erkennen ist. Diese Bilden vertex-treue Sphären vom Typ $B_4^n.$ Und für $n=4$ gibt es 3 nicht isomorphe Multi-Tetraeder mit demselben Blocktyp.

Mithilfe des SimplicialSurfaces Paket können die Sphären mit mindestens einer 3-Taille berechnet werden. Die folgende Tabelle zeigt die Anzahlen der Sphären mit mindestens einer 3-Taille und $n$ Flächen, die keine Multi-Tetraeder sind. 
\begin{center}
\begin{tabular}{|c|c|c|c|c|c|c|c|c|c|c|c|c|c|c|}
\hline
\textcolor{blue}{$\#$ 3-Taillen }&\textbf{4}& \textbf{6}& \textbf{8}& \textbf{10}& \textbf{12}& \textbf{14}& \textbf{16}& \textbf{18}& \textbf{20}& \textbf{22}& \textbf{24}& \textbf{26}\\
\hline
\textcolor{blue}{1} &0& 1& 0& 1& 1& 4& 14& 52& 237& 1132& 5729& 30100\\
\hline
\textcolor{blue}{2} &0& 0& 0 &0& 4& 7& 30& 120& 550& 2785& 14803& 92604\\
\hline
\textcolor{blue}{3}& 0& 0& 0& 0& 0& 11& 29& 164& 837& 4598& 26551& 156029\\
\hline
\textcolor{blue}{4}& 0& 0& 0& 0& 0& 0& 57& 184& 1126& 6358& 103576& 236964\\
\hline
\textcolor{blue}{5}& 0& 0& 0& 0& 0& 0& 0& 270& 1084& 7422& 46175& 299906\\
\hline
\textcolor{blue}{6} &0& 0& 0& 0& 0& 0& 0& 0 &1564& 6825& 54405& 331985\\
\hline
\textcolor{blue}{7}& 0& 0& 0& 0& 0& 0& 0& 0& 0& 9128& 42535& 335990\\
\hline
\textcolor{blue}{8}& 0& 0& 0& 0& 0& 0& 0& 0& 0& 0& 55288& 267548\\
\hline
\textcolor{blue}{9} &0& 0& 0& 0& 0& 0& 0& 0& 0& 0& 0& 337437\\
\hline
\end{tabular}
\end{center}
\begin{comment}
\begin{center}
\begin{tabular}[h]{|c|c|c|c|c|c|c|c|c|c|c|c|c|c|}
\hline
\textbf{$\vert X_2\vert$}& \textbf{4} &  \textbf{6}& \textbf{8} & \textbf{10} & \textbf{12} & \textbf{14}&\textbf{16}&\textbf{18}&\textbf{20}&\textbf{22}&\textbf{24}&\textbf{26}&\textbf{28}\\
\hline
 \textbf{$k_n$}  &0& 1& 0& 1 &1& 4& 14& 52 &237& 1132& 5729& 30100& 162410\\
 \hline
\end{tabular}
\end{center}
\end{comment}
Für die Sphären mit genau einer 3-Taille und geringer Flächenanzahl listen wir nun den Eckenzähler, den Flächenzähler und die Automorphismengruppe auf.
\begin{center}
\begin{tabular}[h]{|c|c|c|c|c|}
\hline
n &Zshngs.-& Vertex- & Flächenzähler & Aut.\\
&komp.&zähler&& gruppe\\
 \hline
 6& $T,T$ & $v_3^2v_4^3$&$f^6_{3,4^2}$& $C_2\times D_6$\\
 \hline
10& $T,O$ & $v_3^1v_4^4v_5^3$& $f^3_{3,5^2}f^1_{4^3}f^3_{4^2,5}f^3_{4,5^2}$ &$S_3$\\
 \hline
12& $(5)^2,T$ &$v_3^1v_4^3v_5^3v_6^1$& $f^1_{3,5^2}f^2_{3,5,6}f^2_{4^2,5}f^2_{4^6}f^2_{4,5^2}f^2_{4,5,6}f^1_{5^3}$&$C_2$\\
 \hline
  & $(6)^2,T $& $ v_3^1v_4^4v_5^2v_6^1v_7^1$& $f^1_{3,5^2}f^2_{3,5,7}f^3_{4^2,6}f^3_{4^2,7}f^2_{4,5,6}f^2_{4,5,7}f^1_{5^2,6}$ &$C_2$\\
14& $S,T$& $ v_3^1v_4^3v_5^3v_6^2$& $f^2_{3,5,6}f^1_{3,6^2}f^1_{4^2,5}f^1_{4^2,6}f^3_{4,5^2}f^4_{4,5,6}f^1_{4,6^2}f^1_{5^2,6}$ &$\{id\}$\\
  & $O,O$ & $v_4^6v_6^3$& $f^2_{4^3}f^6_{4^2,6}f^6_{4,6^2} $ & $D_{12}$\\
 \hline
  &$(7)^2,T$ &$v_3^1v_4^4v_5^2v_6^2v_7^1$&$f^1_{3,5,6}f^1_{3,5,7}f^1_{3,6,7}f^2_{4^2,6}f^2_{4^2,7}f^1_{4,5^2}f^3_{4,5,6}$ &$\{id\}$ \\
  &&&$f^1_{4,5,7}f^2_{4,6^2}f^1_{4,6,7}f^1_{5^2,7}$&\\
  & $(5)^2,O$& $v_3^1v_4^5v_5^2v_7^1v_8^1$&$f^1_{3,5^2}f^2_{3,5,8}f^4_{4^2,7}f^4_{4^2,8}f^2_{4,5,7}f^2_{4,5,8}f^1_{5^2,7}$ &$C_2$\\
16& $(5)^2,O$&$ v_3^1v_4^3v_5^4v_6^1v_7^1$&$f^1_{3,5,6}f^1_{3,5,7}f^1_{3,6,7}f^1_{4^2,5}f^1_{4^2,7}f^3_{4,5^2}f^1_{4,5,6}$ &$\{id\}$\\
&&&$f^3_{4,5,7}f^1_{4,6,7}f^1_{5^3}f^2_{5^2,6}$&\\
  &          & $v_3^1v_4^2v_5^5v_6^2$& $f^2_{3,5,6}f^1_{3,6^2}f^4_{4,5^2}f^4_{4,5,6}f^2_{5^3}f^2_{5^2,6}f^1_{5,6^2}$ &$C_2$\\
  &          &$v_4^6v_5^1v_6^2v_7^1$ &$f^1_{4^3}f^2_{4^2,5}f^2_{4^2,6}f^3_{4^7}f^2_{4,5,6}f^1_{4,6^2}f^4_{4,6,7}f^1_{5,6^2}$ &$C_2$\\
 \hline
\end{tabular}
\end{center}
Beim Genaueren Hinschauen ist zu erkennen, das die Sphären, die einen Tetraeder als Zusammenhangskomponente haben, eine Automorphismengruppe von kleiner Ordnung haben. Eine genauere Erkenntnis halten wir in der nächsten Bemerkung fest.
\begin{bemerkung}
Sei $X$ eine Sphäre ohne Ecken vom Grad 3 und $F$ eine Fläche in $X$. Dann ist $\Aut(T^F(X))$ eine Untergruppe von $S_3.$ 
\end{bemerkung}
\begin{proof}
Sei $V$ die Ecke vom Grad 3 in $T^F(X).$
Da die Sphäre $X$ keine Ecken vom Grad 3 besitzt, muss
\begin{itemize}
\item $\phi(V)=V$
\item $\phi (X_1(V))=X_1(V)$
\item $\phi (X_2(V))=X_2(V)$
\end{itemize}
für jeden Isomorphismus $\phi:\Aut({T}^F(X))\mapsto \Aut({T}^F(X))$ gelten. Falls $\phi(x)^k=x$ für ein $k \in \mathbb{N}$ und alle $x \in X_1(X_2(V))$ ist, muss $\phi(y)=y$ für alle $y\in T^F(X)$ gelten. Für solch einen Isomorphismus kommen nur die Elemente der Gruppe $S_3$ in Frage. 
\end{proof}
Diese Beobachtung kann verallgemeinert werden. Denn beim Genauerem Hinschauen ist zu erkennen, dass 
\[
S_3\cong\{\phi\in \Aut(T)\mid \phi(F)=F\}=Stab(F)
\] für eine Fläche $F\in T_2$ ist.
\begin{bemerkung}
Sobald die Voraussetzung fallen gelassen wird, dass $X$ eine Sphäre ohne Ecke vom Grad 3 ist, ist die Aussage falsch, denn die Automorphismengruppe des Doppel-Tetraeders ist isomorph zu der Gruppe $C_2\times S_3$ und wie bereits bekannt, geht der Doppel-Tetraeder durch eine Tetraeder-Erweiterung aus dem  Tetraeder hervor.
\end{bemerkung}
\begin{lemma}
Seien $X$ und $Y$ zwei nicht isomorphe vertex-treue Sphären ohne 3-Taillen mit Flächen $F$ in $X$ bzw. $F'$ in $Y$. Sei zudem $\phi$ eine bijektive Abbildung von $X_0(F)$ nach $Y_0(F').$ Dann ist $\Aut(X\#_\phi Y)$ isomorph zu einer echten Untergruppe von $Stab_X(F)$ und zu einer echten Untergruppe von $Stab_Y(F').$
\end{lemma}
\begin{proof}
Sei $W=(e_1,e_2,e_3)$ die durch das Zusammensetzen entstandene 3-Taille und $M_1,M_2$ zugehörigen 3-Taillen Komponenten. Es gilt $M_1=X_2-\{F\}$ und $M_2=Y_2-\{F'\}.$ Da $W$ die einzige 3-Taille ist, bildet ein Isomorphismus $\Phi:X\#_{\phi}Y \mapsto X\#_{\phi}Y$  eine Kante der 3-Taille wieder auf eine Kante des 3-Taille ab. Daraus folgt also entweder $\Phi(M_i)=M_i$ oder $\Phi(M_i)=M_{3-i}.$ Da $X$ und $Y$ nicht isomorph sind, muss also der erste Fall eintreten. Somit lässt sich \textsc{$\Phi_{\mid M_1}$} zu einem Isomorphismus $\Phi_X$ auf $X$ erweitern, der $F$ fixiert. Analog kann $\Phi_{\mid M_2}$ zu einem Isomorphismus $\Phi_Y,$ der $F'$ auf sich selber abbildet, erweitert werden.
\end{proof}
\begin{bemerkung}
An dieser Stelle lässt sich wieder erkennen, dass das Abschwächen
der Voraussetzung zu einem anderem Ergebnis führt.
\end{bemerkung}


\begin{lemma}
Seien $X$ und $Y$ vertex-treue Sphären. Weiterhin operiert $\Aut(Y)$ transitiv auf $Y.$ Dann ist die Anzahl der Sphären mit $X$ und $Y$ als 3-Taillen Komponenten die Anzahl der Bahnen der Gruppenoperation $\Phi_{X_2}.$
\end{lemma}
\begin{proof}
Die Aussage wird analog zu \textcolor{red}{...} bewiesen.
\end{proof}
\subsection{Flächengraphen von Sphären mit 3-Taillen}
Wir wollen nun die Auswirkungen der obigen Konstruktion auf die zugehörigen Flächengraphen der Sphären untersuchen.
Hierfür werden im Folgenden erneut Ausschnitte von Flächengraphen präsentiert. Wir verwenden in diesen Abbildungen zwei unterschiedliche Farben für die Knotenmenge des Graphen, der sich durch diese Konstruktion ergibt, verwenden. Damit wollen wir andeuten das Knoten einer Farbe jeweils zu einer Sphäre gehören, die zum Zusammensetzen an der 3-Taille verwendet wurden.\\
Seien $X$ mit $F\in X_2$ und $Y$ mit $F'\in Y_2$ zwei vertex-treue Sphären. Weiterhin sei die Abbildung $\phi:X_2(F)\to Y_2(F')$ bijektiv. An dieser Stelle wollen wir beschreiben, wie sich der Flächengraph der Sphäre $X\#_{\phi}Y$ aus den Flächengraphen der Sphären $X$ und $Y$ ergibt. Seien hierzu $F_1,F_2,F_3$ bzw. $F_1',F_2',F_3'$ die Nachbar-Flächen von $F$ in $X$ bzw. von $F'$ in $Y.$ Durch das Zusammensetzen der Sphären an den Flächen $F$ und $F'$ erhalten wir zum einen die konstruierte Sphäre $Z=X \#_\phi Y$ und eine 3-Taille $W=(e_1,e_2,e_3)$ in $Z,$ sodass ohne Einschränkung 
\[
Z_2(e_i)=\{F_i,F_i'\}
\] für $i=1,2,3$ gilt. Also sind $F_i$ und $F_i'$ benachbarte Flächen in $Z$.
\begin{figure}[H]
\begin{center}
\includegraphics[viewport=4cm 23.5cm 5cm 27.5cm]{3waist}
\end{center}
\caption{Ausschnitt der Sphäre $Z$}
\end{figure}   
 Weiterhin sei nun $G_X=(V_X,E_X)$ bzw. $G_Y=(V_Y,E_Y)$ der zu $X$ bzw. $Y$ gehörige Flächengraph. Den Flächengraph $G_Z$ erhalten wir durch folgende Konstruktion:
Sei $G=(V_X\cup V_Y,E_X \cup E_Y)$ die Vereinigung der Graphen $G_X$ und $G_Y.$
\begin{figure}[H]
\begin{center}
\includegraphics[viewport=14cm 18.5cm 5cm 23cm]{Image_fg10}
\end{center}
\caption{Ausschnitt der Vereinigung der Graphen $G_X$ und $G_Y$}
\end{figure}
Als erstes müssen die zu den Flächen $F$ und $F'$ gehörigen Knoten in $G$ und damit auch die inzidenten Kanten dem Graphen entnommen werden. 
\begin{figure}[H]
\begin{center}
\includegraphics[viewport=14cm 18.5cm 5cm 23cm]{Image_fg11}
\end{center}
\caption{Ausschnitt eines aus dem Graphen $G$ konstruiertem Graphen}
\end{figure}
Da in $Z$ die Flächen $F_i$ und $F_i'$ benachbart sind, erhalten wir den zugehörigen Flächengraphen, indem wir nun die Kanten $\{F_1,F_1'\},\{F_1,F_1'\},\{F_1,F_1'\}$ dem zuvor konstruierten Graphen hinzufügen. 
\begin{figure}[H]
\begin{center}
\includegraphics[viewport=14cm 18.5cm 5cm 22.5cm]{Image_fg12}
\end{center}
\caption{Ausschnitt des Graphen $G_Z$}
\end{figure}
 Seien $X$ bzw. $Y$ zwei Oktaeder mit $X_2=\{1_X,\ldots,8_X\}$ bzw. $Y_2=\{1_Y,\ldots,8_Y\}.$
Weiterhin seien die ordinalen Symbole der Sphären durch
\begin{align*}
&\mu(Z)= \mu(Y)=\\
 &(6,12,8;(\{1,2\},\{1,3\},\{1,4\},\{1,5\},\{2,3\},\{2,5\},\\
 &\{2,6\},\{3,4\},\{3,6\},\{4,5\},\{4,6\},\{5,6\});\\
 &(\{1,2,5\},\{2,3,8\},\{3,4,10\},\{1,4,6\},\{5,7,9\},\\&\{8,9,11\},\{10,11,12\},\{6,7,12\})
 \end{align*}
 gegeben. Die zugehörigen Graphen erhalten wir deshalb durch 
\begin{align*}
G_X&=(V_X,E_X)=(X_2,\{X_2(e)\mid e\in X_1\})\\G_Y&=(V_Y,E_Y)=(Y_2,\{Y_2(e)\mid e\in Y_1\}).\\
\end{align*}
 \begin{figure}[H]
\begin{center}
\includegraphics[scale=1.1,viewport=0cm 23cm 12cm 27cm]{facegraphoct}
\end{center}
\caption{Ausschnitt der Sphäre $Z$}
\end{figure}   
Da $X_0(1_1)=\{V_1,V_2,V_3\}$ und $Y_0(3_2)=\{V_1',V_4',V_5'\}$ ist, können wir mit der Bijektion $\phi=(V_1,V_1')(V_2,V_4')(V_3,V_5')$ die Sphäre $Z=X\#_\phi Y$ konstruieren. In $Z$ bilden $(2_1,2_2)(4_1,4_2)$ und $(5_1,7_2)$ Paare benachbarter Flächen in $Z.$
Die Knotenmenge des Graphen $G_Z$ bildet die Menge 
\[
V_Z=(V_X \cup V_Y)-\{1_1,3_2\}
\]
mit der zugehöriger Kantenmenge 
\begin{align*}
E_Z=&(E_X \cup E_Y\cup \{\{2_1,2_2\},\{5_1,7_2\},\{4_1,4_2\}\})-\\
&\{\{1_1,4_1\},\{1_1,5_1\},\{1_1,2_1\},\{3_2,4_2\},\{2_2,3_2\},\{3_2,7_2\}\}.
\end{align*}
\begin{figure}[H]
\begin{center}
\includegraphics[scale=1.1,viewport=0cm 23cm 12cm 27cm]{facegraphoct2}
\end{center}
\caption{Ausschnitt der Sphäre $Z$}
\end{figure}   
\subsection{Färbungen auf Sphären mit 3-Taillen}
In der \Cref{bemf} wollen wir zunächst eine Erkenntnis über die Färbung einer Sphäre, die sich durch Zusammensetzen zweier Sphären mithilfe einer 3-Taille ergibt, festhalten.
\begin{bemerkung}\label{bemf}
Seien $X$ mit $F\in X_2$ und $Y$ mit $F'\in Y_2$ vertex-treue Sphären zusammen mit einer Bijektion $\phi:X_0(F)\mapsto Y_0(F')$. Weiterhin seien $\omega_X$ eine wilde Färbung auf $X$ und $\omega_Y$ eine wilde Färbung auf $Y.$
In dieser Bemerkung wird festgehalten, dass eine wilde Färbung auf der Sphäre $Z=X\#_{\phi}Y$ durch geeignetes Zusammensetzen der Färbungen $\omega_X$ und $\omega_Y$ konstruiert werden kann. Sei deshalb $W=(e_1,e_2,e_3)$ die 3-Taille der Sphäre $Z.$
 Dann gibt es Kanten $e_1^X,e_2^X,e_3^X$ in $X$ und Kanten $e_1^Y,e_2^Y,e_3^Y$ in $Y,$ sodass die Kanten $e_i^X$ und $e_i^Y$ bei der Konstruktion von $Z$ zu der Kante $e_i$ zusammengesetzt werden.
  In $X$ gibt es 
   Flächen $F_i,$ die $X_2(e_i^X)=\{F,F_i\}$ erfüllen.
   Analog gibt es also auch Flächen $F_i' \in Y_2,$ sodass die Gleichheit $Y_2(e_i^Y)=\{F_i,F_i'\}$ gilt. Also sind $F_i$ und $F_i'$ benachbarte Flächen in $Z.$
\begin{figure}[H]
\begin{center}
\includegraphics[viewport=0cm 23.5cm 9cm 28cm]{3waist}
\end{center}
\caption{Ausschnitt der Sphäre $Z$}
\end{figure}   
    Es reicht im Folgenden die Kanten $e_1^X$ und $e_1^Y$ für die Argumentation zu betrachten. In $X$ gibt es weitere Kanten $e_X,e_X'\in X_1(F_1)$ mit der Eigenschaft
\[
X_0(e_X)\cap X_0(e_1^X)\cap X_0(e_2^X)\neq \emptyset 
\] 
und 
\[
  X_0(e'_X)\cap X_0(e_1^X)\cap X_0(e_3^X)\neq \emptyset   
\]
In $Y$ finden wir ebenfalls Kanten $e_Y,e_Y',$ die  
\[
   Y_0(e_Y)\cap Y_0(e_1^Y)\cap Y_0(e_2^Y)\neq \emptyset 
\]
und
\[
  Y_0(e'_Y)\cap Y_0(e_1^Y)\cap Y_0(e_3^Y)\neq \emptyset   
\]
erfüllen. Durch die Konstruktion von $Z$ sind also $e_X$ und $e_Y$ zu derselben Ecke in $Z$ inzident. Selbiges gilt auch für die Kanten $e_X'$ und $e_Y'.$ 
\begin{figure}[H]
\begin{center}
\includegraphics[scale=0.9,viewport=0cm 18.cm 19cm 26.5cm]{konst3waist}
\end{center}
\caption{Übergang von den sphären $X$ und $Y$ zu $Z$}
\end{figure}   
Schauen wir uns nun die Färbungen auf den jeweiligen Sphären an. Wir können ohne Einschränkung $\omega_X(e_1^X)=\omega_Y(e_1^Y)=a$ annehmen. Aufgrund dessen muss dann $
a\neq \omega_X(e_X) \neq\omega_X(e_X')$
und $
a\neq \omega_Y(e_Y) \neq\omega_Y(e_Y')$
gelten, wodurch wir ebenfalls $
b= \omega_X(e_X) =\omega_Y(e_Y)$ und $
c= \omega_X(e_X') =\omega_Y(e_Y')$ annehmen dürfen.
\begin{figure}[H]
\begin{center}
\includegraphics[scale=0.9,viewport=0cm 23.7cm 19cm 27cm]{konst3waist2}
\end{center}
\caption{Ausschnitt der Sphären $X$ und $Y$}
\end{figure}
Wird nun zusätzlich $\omega_X(e_2^X)=\omega_Y(e_2^Y)=b$ und $\omega_X(e_3^X)=\omega_Y(e_3^Y)=c$ angenommen, dann erhalten wir durch die Abbildung 
\[
\omega: Z_1 \{a,b,c\},e\mapsto
 \begin{cases}
 a, &e=e_1\\
 b,& e=e_2 \\
 c, & e=e_3\\
 \omega_X(e), &e \in X_1\cap Z_1\\
 \omega_Y(e), & e\in Y_1 \cap Z_1
 \end{cases}
\]
eine wilde Färbung auf $Z.$
\begin{figure}[H]
\begin{center}
\includegraphics[scale=0.9,viewport=0cm 23.8cm 8cm 27cm]{konst3waist3}
\end{center}
\caption{Ausschnitt der Sphäre $Z$}
\end{figure}
\end{bemerkung}
Bei genauerer Beobachtung können wir sogar Aussagen darüber treffen, wie sich die Strukturen der einzelnen Sphären auf die zusammengesetzte Sphäre überträgt.
\begin{lemma}
Seien $X$ und $Y$ vertex-treue Sphären und $\omega_X$ eine zahme Färbung auf $X$ bzw. $\omega_Y$ eine zahme Färbung auf $Y.$ Weiterhin seien $F\in X_2$ und $F'\in Y_2$ Flächen und $\phi:X_0(F)\to Y_0(F')$ eine bijektiven Abbildung. 
Falls $X$ und $Y$ eine mmm-Struktur besitzen, dann existiert eine mmm-Struktur auf $X\#_{\phi}Y$.
\end{lemma}
\begin{proof}
 Sei  $(e_1,e_2,e_3)$ eine 3-Taille in $Z=X\#_{\phi}Y$. Dann existieren Kanten $e^X_1,e^X_2,e_3^X$ in $X$ und $e^Y_1,e^Y_2,e_3^Y$ in $Y,$ sodass bei der Konstruktion von $Z$ die Kanten $e_i^X$ und $e_i^Y$ zu der Kante $e_i$ zusammengeführt werden. Insbesondere sind die Kanten $e_i^X$ zu $F$ und die Kanten $e_i^Y$ zu $F'$ inzident. Wir wollen eine zahme Färbung auf $Z$ durch eine geeignete Fortsetzung der bereits vorhandenen zahmen Färbungen konstruieren.  
Wir können ohne Einschränkung 
\begin{align*}
\omega_X(e_1^X)=a=\omega_Y(e_1^Y),\\
 \omega_X(e_2^X)=b=\omega_Y(e_2^Y),\\
 \omega_X(e_3^X)=c=\omega_Y(e_3^Y)
\end{align*}
annehmen. Es reicht im Folgenden die Kante $e_1,$ die durch Zusammensetzen der Kanten $e_1^X$ und $e_1^Y$ entsteht, zu betrachten. Da $e_1^X$ eine Kante in $X$ ist, existiert eine Fläche $F_X$ in $X$ mit $X_2(e_1^X)=\{F,F_X\}$. Weiterfinden gelten für die zwei anderen Kanten $e_X,e_X'\in X_1(F_X) $ die Gleichheiten
\[
X_0(e_1^X)\cap X_0(e_X) \cap X_0(e_2^X)\neq \emptyset
\]
und 
\[
X_0(e_1^X)\cap X_0(e_X') \cap X_0(e_3^X)\neq \emptyset.
\]
 Analog erhalten wir eine Fläche $F_Y$ in $Y$ mit $Y_2(e_1^Y)=\{F',F_Y\}$ und zu $e_1^Y$ verschiedene Kanten $e_Y,e_Y'\in Y_1(F_Y)$ mit 
\begin{align*}
Y_0(e_1^Y)\cap Y_0(e_Y) \cap Y_0(e_2^Y)\neq \emptyset
\end{align*}
und 
\[
Y_0(e_1^Y)\cap Y_0(e_Y') \cap Y_0(e_3^Y)\neq \emptyset.
\]
Somit gibt es in $Z$ eine Ecke, die zu $e_X$ und $e_Y$ inzident ist und es existiert eine Ecke in $Z$ sodass  $e_X'$ und $e_Y'$ zu dieser inzident sind.
Da $e_1^X$ und $e_1^Y$ beides m-Kanten sind, muss 
\begin{align*}
\omega_X(e_X)=b=\omega_Y(e_Y),\\
\omega_X(e_X')=c=\omega_Y(e_Y')
\end{align*}
gelten.
Wählen wir nun als Kandidaten für unsere zahme Färbung die Abbildung
\[
\omega: Z_1 \{a,b,c\},e\mapsto
 \begin{cases}
 a, &e=e_1\\
 b,& e=e_2 \\
 c, & e=e_3\\
 \omega_X(e), &e \in X_1\cap Z_1\\
 \omega_Y(e), & e\in Y_1 \cap Z_1
 \end{cases}
\]
dann bilden $e_1,e_2,e_3$ wieder m-Kanten. Da alle anderen Kanten und Flächen bei der Konstruktion von $Z$ unberührt bleiben, bilden die Kanten der Sphäre Kanten vom Typ $m$ und wir erhalten schlussendlich eine $mmm$-Struktur auf $Z.$  

\end{proof}
Salopp formuliert werden Spiegelungskanten unter der obigen Konstruktion wieder zu Spiegelungskanten.
Dieselbe Aussage für Sphären mit einer $rrr-$Struktur zu formulieren, liefert kein richtiges Ergebnis. Denn beispielsweise der Tetraeder besitzt eine $rrr$-Struktur, jedoch existiert auf dem Doppel-Tetraeder, der sich bekanntlicher Weise aus zwei Tetraedern zusammensetzt, keine zahme Färbung.
\subsection{Der Sphären-Graph}
Mithilfe der drehbaren Kantendrehungen und Kantensequenzen erhalten wir eine neue Anschauung der vertex-treuen Sphären. In diesem Abschnitt wollen wir die Menge der Sphären durch Einführen einer geeigneten Kantenmenge als ungerichteten Graphen auffassen. Daraufhin werden dann elementare Eigenschaften des sogenannten \emph{Sphären-Graphen} zusammengefasst und kleinere Beispiele dieses Graphen dargelegt. 

\begin{definition}
Sei $n\in \mathbb{N}$ gerade. Die Menge der Isomorphieklassen der Sphären ohne 2- oder 3-Taillen bezeichnen wir mit $\mathcal{S}_n.$ Weiterhin bezeichnen wir die Menge der Isomorphieklassen der Sphären ohne 2-Taillen mit $\mathcal{S}_n^3.$
\end{definition} 
 Es gilt $\mathcal{S}_2=\mathcal{S}_2^3=\emptyset.$ 
 Für $n=4$ ist $\mathcal{S}_n^3=\mathcal{S}_n=\{T\},$ wobei $T$ ein Tetraeder ist.
 Falls $n\geq 6$ ist, dann herrscht eine echte Teilmengenbeziehung zwischen $\mathcal{S}_n$ und $\mathcal{S}_n^3.$ Beispielsweise ist 
 \[
 \mathcal{S}_6=\emptyset \subset \{DT\}=\mathcal{S}_6^3.
 \]
 Diese Mengen können wir nun nutzen, um den Sphären-Graph zu definieren. 
 \begin{definition}
Sei $n\in \mathbb{N}$ gerade. Dann definieren wir den \emph{Sphären-Graph} $\mathcal{G}^3_n$ durch die Knotenmenge $\mathcal{S}^3_n.$ Zwei Knoten $V,V'$ sind in diesem Graphen durch eine ungerichtete Kante verbunden, falls die zu $V$ zugehörige Sphäre durch genau eine Kantendrehung aus der zu $V'$ zugehörigen Sphäre hervorgeht. 
 \end{definition}
 \begin{bsp}
 \begin{itemize}
 \item Für $n=4$ besteht die Knotenmenge des Sphären-Graphen aus genau einem Knoten $V_T,$ der zum Tetraeder $T$ zugehörig ist. Da alle Kanten des Tetraeders nicht drehbar sind, ist die Kantenmenge des Graphen $\mathcal{G}_4^3$ leer. 
 \item Für $n=6$ ist die Knotenmenge durch $V=\{V_{DT}\}$, wobei $DT$ der Doppel-Tetraeder ist, gegeben. Somit kommt für die Kantenmenge nur $E=\emptyset$ oder $E=\{\{V_{DT},V_{DT}\}\}$ in Frage. Der Doppel-Tetraeder enthält Kanten, die zu zwei Ecken vom Grad 4 inzident und drehbar sind. Durch das Drehen dieser Kante erhalten wir jedoch erneut den Doppel-Tetraeder und dies liefert uns $E=\{\{V_{DT},V_{DT}\}\}.$ 
 \item Die Knotenmenge des Graphen $\mathcal{G}^3_8$ ist zweielementig. Denn es gibt genau 2 Sphären mit 8 Flächen, nämlich den Oktaeder und den Multi-Tetraeder, der durch das Symbol $1_11_2$, entsteht.
 Da $1_11_2$ jedoch durch eine Kantendrehung aus dem Oktaeder hervorgeht, sind die zugehörigen Knoten durch eine Kante verbunden. Da es Kanten in $1_11_2$ gibt, sodass Drehen dieser wieder die Sphäre $1_11_2$ hervorbringt, erhalten wir zudem einen nicht einfachen Graphen.
 \begin{figure}[H]
\begin{center}
\includegraphics[viewport=0cm 26.5cm 5cm 27.5cm]{spgr}
\end{center}
\caption{Der Graph $\mathcal{G}_8^3$}
\end{figure}
 \item In dem Sphären-Graph $\mathcal{G}_{12}^3$ gibt es 14 Knoten. Nämlich 
 \begin{itemize}
 \item die Multi-Tetraeder beschrieben durch die Symbole \begin{align*}
&1_11_21_31_4,\,1_11_32_22_3,\,1_11_32_42_2,\,1_21_32_33_2,\\
&1_21_32_43_2,\,1_21_32_43_4,\,1_21_32_43_1
\end{align*}
\item das simpliziale Parallelepiped $P$
 \item die Sphäre $OT^1$ die durch zwei Tetraeder-Erweiterungen an zwei benachbarten Flächen des Oktaeders entsteht,
 \item die Sphäre $OT^2,$ die durch zwei Tetraeder-Erweiterungen an zwei Flächen des Oktaeders, die genau eine Ecke gemeinsam haben
 \item den Doppel-6-Gon $(6)^2,$
 \item die Sphäre $(5)\overline{2}(5)$
 \item die Sphäre $(5)^2\#T,$ die durch eine Tetraeder-Erweiterung am Doppel-5-Gon entsteht,
 \item die Sphäre $O\# DT$, die durch Zusammensetzen des Oktaeders und des Doppel-Tetraeders durch eine 3-Taille entsteht.
 \end{itemize}
 Wenn wir die Sphären in der Reihenfolgen durchnummerieren, in der sie präsentiert wurden und dementsprechend Knoten in dem Graphen zuordnen, können wir die Inzidenzmatrix des  Sphären-Graphen $\mathcal{G}^3_{12}$ angeben. Beispielsweise ist der Knoten $V_{14}$ zur Sphäre $O\#DT$ gehörig und der Knoten $V_8$ zum simplizialen Parallelepiped. Durch diese Festlegung erhalten wir die Inzidenzmatrix des Sphären-Graphen $\mathcal{G}^3_{12}.$
 \[
\left( \begin{array}{rrrrrrrrrrrrrr}
 0&0&0&0&0&0&0&0&1&0&0&0&0&0\\
 0& 1& 1& 1& 1& 0& 0& 0& 1& 0& 0& 0& 1& 1\\
 0& 1& 0& 0& 1& 0& 0& 0& 1& 1& 0& 0& 0& 1\\
 0& 1& 0& 0& 1& 0& 0& 0& 0& 0& 1& 0& 0& 0 \\
 0& 1& 1& 1& 1& 1& 1& 0& 0& 0& 0& 0& 1& 1 \\
 0& 0& 0& 0& 1& 0& 1& 0& 0& 1& 0& 0& 0& 1\\
 0& 0& 0& 0& 1& 1& 1& 1& 0& 0& 0& 0& 0& 1\\
 0& 0& 0& 0& 0& 0& 1& 0& 0& 0& 0& 0& 1& 0\\
 1& 1& 1& 0& 0& 0& 0& 0& 0& 0& 0& 1& 1& 0\\
 0& 0& 1& 0& 0& 1& 0& 0& 0& 0& 0& 0& 1& 0\\
 0& 0& 0& 1& 0& 0& 0& 0& 0& 0& 0& 0& 1& 0\\
 0& 0& 0& 0& 0& 0& 0& 0& 1& 0& 0& 1& 1& 1\\
 0& 1& 0& 0& 1& 0& 0& 1& 1& 1& 1& 1& 0& 1 \\
 0& 1& 1& 0& 1& 1& 1& 0& 0& 0& 0& 1& 1& 1 \\
\end{array}
\right)
 \]
 Es sei angemerkt, dass die $i$-te Zeile bzw. Spalte zum Knoten $V_i$ zugehörig sind.
 \end{itemize}
 \end{bsp}
 \begin{bemerkung}
 \begin{itemize}
 \item Der Sphären-Graph ist zusammenhängend. Dies folgt direkt aus der Transitivität der Kantendrehung.
 \item die Anzahl der Einsen in einer Zeile bzw. Spalte gibt die Anzahl der Isomorphieklassen der drehbaren Kanten einer Sphäre an.
 \item Für alle $n\geq 5$ ist der Knoten, der zum Doppel-$n$-
 Gon zugehörig ist ein Knoten vom Grad 2.
 \item An $\mathcal{S}^3_6$ und $\mathcal{S}_{18}^3$ ist zu erkennen, dass Sphären-Graphen im Allgemeinen nicht einfach sind. 
 \item Die minimale Anzahl an Kantendrehungen, um eine Sphäre $X$ isomorph in eine Sphäre $Y$ umzuformen, ist die Länge des kürzesten Weges der zugehörigen Knoten im Sphären-Graphen.   
 \item Wenn der Graph $\mathcal{G}^3_n$ auf die Knotenmenge $\mathcal{S}_n$ eingeschränkt wird, dann ist der Graph nicht zusammenhängend. Denn beim Drehen einer Kante des Doppel-$\frac{n}{2}$-Gons entsteht mindestens eine Ecke vom Grad 3, also eine 3-Taille. Damit ist der zum Doppel-$\frac{n}{2}$-Gon zugehörige Knoten in dem einschränkten Graphen isoliert.
 \end{itemize}
 \end{bemerkung}
 Da wir besonders an den Sphären ohne 2- oder 3-Taillen interessiert sind, können wir durch das Einführen von gewichteten Kanten einen Graphen für die Menge $\mathcal{S}_n$ aufstellen.
 \begin{definition}
Sei $n\in \mathbb{N}$ gerade. Dann definiert das Tripel $\mathcal{G}_n=(V,E,c)$ den \emph{taillen-freien Sphären-Graph}. Die Kantenmenge erhalten wir durch $V=\mathcal{S}_n.$ Zwei Knoten $V_X,V_Y$ mit zugehörigen vertex-treuen Sphären $X,Y$ sind durch eine Kante verbunden, falls $X^e\cong Y$ für eine Kante $e\in X_1$ ist oder es eine drehbare Kantensequenz $E=(e_1,\ldots,e_m)$ in $X$ gibt, sodass $X^E\cong Y$ ist und für alle $1\leq i < m $ die Sphäre $X^{(e_1,\ldots,e_i)}$ eine 3-Taille besitzt. Die Gewichtsfunktion $c:E\to\mathbb{Z}^+$  erhalten wir folgendermaßen: Sei $\{V_X,V_Y\}=e\in E.$ Das Gewicht $c(e)$ ist die minimale Länge einer Kantensequenz, die $X$ isomorph in $Y$ umwandelt und obige Bedingung erfüllt.
 \end{definition}
 Betrachten wir nun einmal den Graphen $\mathcal{G}_{14}.$
Dieser besitzt vier Knoten, da es bis auf Isomorphie genau vier Sphären ohne 2- oder 3-Taillen gibt. Diese sind durch die Symbole $(7)^2,(5)\overline{4}(5),(5)\overline{4}(4)$ und $(6)\overline{2}(6)$ gegeben. Zur Vereinfachung bezeichnen wir die Knoten im Graphen mit denselben Symbolen. Mit etwas Hilfe von Gap können wir den Graphen genau bestimmen. An dieser Stelle geben wir uns mit einer bildlichen Veranschaulichung des Graphen zufrieden.
 \begin{figure}[H]
\begin{center}
\includegraphics[scale=0.7,viewport=1cm 15.5cm 24cm 22cm]{spheregraph}
\end{center}
\caption{der taillen-freie Sphären-Graph $\mathcal{G}_{14}$}
\end{figure}
 \begin{bemerkung}
 \begin{itemize}
 \item Der zum Doppel-$n$-Gon zugehörige Knoten ist zu allen Knoten inzident, deren Sphären eine Ecke vom Grad 4 besitzen.
 \item Der taillen-freie Sphären-Graph ist im Allgemeinen nicht einfach.
 \item Der taillen-freie Sphären-Graph ist im Allgemeinen nicht vollständig. Denn in $\mathcal{G}_{20}$ gibt es beispielsweise genau eine Sphäre, die bis auf Isomorphie durch eine Kantendrehung aus dem Ikosaeder hervorgeht. Diese enthält keine 2- oder 3-Taillen. Also ist der Grad des zum Ikosaeder zugehörigen Knoten in dem taillen-freien Sphären-Graph gleich 1. Und die von diesem Knoten ausgehende Kante hat die Gewichtung 1.
 \end{itemize}
 \end{bemerkung}
\subsection{Multi-Sphären}
In diesem Kapitel dieser Arbeit wollen wir die in Kapitel 6 vorgestellte Konstruktion von Multi-Tetraedern durch Tetraeder-Erweiterungen für beliebige vertex-treue Sphären ohne 3-Taillen erweitern.   
\begin{definition}
Sei $X$ eine vertex-treue Sphäre ohne 3-Taillen. Wir nennen $X$ eine \emph{Multi-Sphäre}, falls es natürliche Zahlen $k,n$ gibt, sodass $X$ eine Sphäre vom Blocktyp $B_n^k$ ist und weiterhin
\[
Z_1\cong \ldots\cong Z_k
\]
für die Blöcke $Z_1,\ldots,Z_k$ gilt. 
\end{definition}
\begin{bemerkung}
\begin{itemize}
\item Jede vertex-treue Sphäre ohne 3-Taillen ist eine Multi-Sphäre vom Blocktyp $B_n^1$ für ein $n\in \mathbb{N}.$
\item Multi-Tetraeder sind Multi-Sphären vom Blocktyp $B_4^k.$
\end{itemize}
\end{bemerkung}
\begin{comment}
In Analogie zu dem Tetraeder-Symbol wollen wir nun ein Symbol für Multi-Sphären einführen. 
Hierfür schauen wir uns als motivierendes Beispiel Multi-Sphären bestehend aus Oktaedern an. Die Bezeichnung der Ecken, Kanten und Flächen ist hier von großer Bedeutung, deshalb geben wir die des Oktaeders durch folgendes Symbol an:
\begin{align*}
 \mu((O,<))=&(6,12,8;(\{1,2\},\{1,3\},\{1,4\},\{1,5\},\{2,3\},\{2,5\},\\
 &\{2,6\},\{3,4\},\{3,6\},\{4,5\},\{4,6\},\{5,6\});\\
 &(\{1,2,5\},\{6,7,12\},\{1,4,6\},\{5,7,9\},\{3,4,10\},\\&\{8,9,11\},\{2,3,8\},\{10,11,12\}))
 \end{align*}
 An dieser Stelle wurde bewusst auf die Disjunktheit der Mengen $O_0,O_1$ und $O_2$ verzichtet.
  Wir wollen einen Multi-Oktaeder vom Blocktyp $B_8^2$ konstruieren. Hierfür betrachten wir die Oktaeder $Y$ und $Z,$ die durch die Symbole 
\begin{align*}
 \mu((Y,<))=&(6,12,8;(\{1_1,2_1\},\{1_1,3_1\},\{1_1,4_1\},\{1_1,5_1\},\{2_1,3_1\},\{2_1,5_1\},\\
 &\{2_1,6_1\},\{3_1,4_1\},\{3_1,6_1\},\{4_1,5_1\},\{4_1,6_1\},\{5_1,6_1\});\\
 &(\{1_1,2_1,5_1\},\{6_1,7_1,{12}_1\},\{1_1,4_1,6_1\},\{5_1,7_1,9_1\},\{3_1,4_1,{10}_1\},\\&\{8_1,9_1,{11}_1\},\{2_1,3_1,8_1\},\{{10}_1,{11}_1,{12}_1\}))
 \end{align*}
und 
\begin{align*}
 \mu((Z,<))=&(6,12,8;(\{1_2,2_2\},\{1_2,3_2\},\{1_2,4_2\},\{1_2,5_2\},\{2_2,3_2\},\{2_2,5_2\},\\
 &\{2_2,6_2\},\{3_2,4_2\},\{3_2,6_2\},\{4_2,5_2\},\{4_2,6_2\},\{5_2,6_2\});\\
 &(\{1_2,2_2,5_2\},\{6_2,7_2,{12}_2\},\{1_2,4_2,6_2\},\{5_2,7_2,9_2\},\{3_2,4_2,{10}_2\},\\&\{8_2,9_2,{11}_2\},\{2_2,3_2,8_2\},\{{10}_2,{11}_2,{12}_2\}))
 \end{align*}
 sind. Wir betrachten also exakte Kopien des Oktaeders $O,$ wobei wir zur Unterscheidung der Sphären eine Indexschreibweise einführen.
 Dadurch bilden die Abbildungen $\phi_1:Y_0\cup Y_1\to X,j_1 \mapsto j$ und $\phi_2:Z_0\cup Z_1\to X,j_2 \mapsto j$ unter der Berücksichtigung, das Ecken auf Ecken und Kanten auf Kanten abgebildet werden, Abbildungen, die zu Isomorphismen $Y\to X $ bzw. $Z\to X$ erweitert werden können. Die Fläche $1\in X_2$ mit $X_0(1)=\{1,2,3\}$ in $X$ kann also mit der Fläche $1_1$ in $Y$ bzw. mit der Fläche $1_2$ in $Z$ identifiziert werden. Durch $\phi=(1_1\, 1_2) (2_1\, 2_2) (3_1\, 3_2)$ erhalten wir also den Doppel-Oktaeder $Y\#_{\phi} Z.$ Bezeichnen wir die Flächenmenge des Oktaeders $Y$ bzw. $Z$ mit $\{1_1,\ldots,1_8\}$ bzw. $\{1_1,\ldots,1_8\}$ dann gilt
 \[
(Y\#_{\phi} Z)_2=\{1_2,\ldots,1_8,2_2,\ldots 2_8\}. 
 \]
 Dieses Vorgehen wollen wir nun iterieren und für beliebige Sphären ohne 3-Taillen formulieren.
 Sei deshalb $X$ eine vertex-treue Sphäre ohne 3-Taillen mit $n$ Flächen. Zur Vereinfachung nehmen wir an, dass $X_2$ durch $\{1,\ldots,n\},$ $X_1$ durch $\{1,\ldots,k\}$ und $X_0$ durch $\{1,\ldots,m\}$ gegeben ist. Es wird an dieser Stelle und in dem nachstehendem Absatz bewusst auf die Disjunktheit der Ecken-,Kanten- und Flächenmengen verzichtet, um das generelle Vorgehen möglichst einfach zu skizzieren. Mit $l$ bezeichnen wir die Länge des Symbols. Damit ist $l$  ebenfallls die Anzahl der 3-Taillen in der konstruierten Sphäre. Dieses Symbol werden wir wie folgt aufbauen:
\begin{itemize}
\item Mit $()_X$ bezeichnen wir das leere Symbol und die dadurch konstruierte Sphäre ist die uns bekannte Sphäre $X.$
\item 
Für $l=1$ wollen wir also ein Symbol für eine Sphäre vom Typ $B_n^2$ aufstellen.
 Konstruieren wir zunächst die besagte Sphäre. Hierfür benötigen wir die Sphären $Y$ und $Z$ mit $X\cong Y\cong Z.$ 
Die Ecken-, Kanten- und Flächenmengen müssen nun zum Wohle der Konstruktion genauer angegeben werden.
An dieser Stelle sei wieder angemerkt, dass bewusst auf die Disjunktheit der einzelnen Mengen einer Sphäre verzichtet wurde.
Wir nehmen also an, dass die Flächenmengen durch
\begin{align*}
Y_2=\{1_1,\ldots,n_1\},\\ 
Z_2=\{1_2,\ldots,n_2\},
\end{align*}
die Kantenmengen durch
\begin{align*}
Y_1=\{1_1,\ldots,m_1\}, \\
Z_1=\{1_2,\ldots,m_2\},
\end{align*}
und die Eckenmengen durch 
\begin{align*}
Y_0=\{1_1,\ldots,1_k\}, \\
Z_0=\{2_1,\ldots,2_k\},
\end{align*}
dargestellt werden. 
 Wir betrachten also exakte Kopien der Sphäre $X,$ wobei wir zur Unterscheidung der Sphären eine Indexschreibweise einführen.
Außerdem nehmen wir an, dass $\phi_1:Y\to X,j_1 \mapsto j$ und $\phi_2:Z\to X,j_2 \mapsto j$ unter der Berücksichtigung, dass Ecken auf Ecken, Kanten auf Kanten und Flächen wieder auf Flächen abgebildet werden, Isomorphismen bilden.
Seien nun $f,f'\in \{1,\ldots,n\},$ sodass ${f}_1\in Y_2$ und ${f'}_2\in Z_2$ ist.
 Mithilfe einer Bijektion $\phi:Y_0({f_1})\to Z_0({f'}_2)$ erhalten wir dann die gewünschte Sphäre durch $Y\#_{\phi}Z.$ Um diese Sphäre zu charakterisieren, beschreiben wir diese durch das Symbol $(1_{f_1\, {f'}_2}^\phi)_X.$ Falls $X$ aus dem Kontext heraus bekannt ist, schreiben wir nur $1_{f_1\, f_2}^\phi.$ Durch das Zusammensetzen der Sphären entstehen unter anderem drei neue Ecken. Diese bezeichnen wir für ein $x\in Y_0(f_1)$ mit
 \[
\{x,\phi(x)\}^V. 
 \]
  Falls der Kontext es erlaubt können jedoch auch die alten Bezeichnungen der Sphären Sphären $Y$ bzw. $Z$ verwendet werden. Das heißt, falls klar ist, dass im Kontext die Sphäre $Y\#_{\phi}Z$ gemeint ist, machen wir beispielsweise zwischen $x,\,\phi(x)$ und $\{x,\,\phi(x)\}$ keine Unterscheidung und sprechen damit immer dieselbe Ecke in der Sphäre $Y\#_{\phi}Z$ an.  
\item Sei nun $l> 1$ und $(1_{f_1g_2}^{\phi_1}\ldots p_{f_{l-1}g_{l-2}}^{\phi_{l-1}})_X$ ein Symbol der Länge $l-1$ mit zugehöriger Sphäre $Y.$ Durch Anhängen der Sphäre $X$ an $Y$ durch eine 3-Taille, erhalten wir eine Multi-Sphäre mit zu $X$ isomorphen Blöcken und wollen nun wieder ein Symbol für diese angeben.
Sei $Z$ eine Sphäre mit $Z\cong X$, sodass die Eckenmenge durch 
\[
Z_0=\{1_l,\ldots,k_l\},
\]
die Kantenmenge durch 
\[
Z_1=\{1_l,\ldots,m_l\},
\]
und die Flächenmenge durch 
\[
Z_2=\{1_l,\ldots,n_l\},
\]
gegeben sind. Es sei hier wieder angemerkt, dass $Z$ eine Kopie von $X$ ist. Die Abbildung $\phi:Z\to X,i_l\mapsto i$ soll unter der Berücksichtigung, dass Ecken auf Ecken, Kanten auf Kanten und Flächen wieder auf Flächen abgebildet werden, einen Isomorphismus bilden. Sei nun $F_1\in Y_2$ und $F_2\in Z_2.$ Somit gilt  $F_1=f_r$ und $F_2={f'}_l$  für $r\in \{1,\ldots,l-1\}$ und $f,f'\in \{1,\ldots,n \}$. Dann erhalten wir für eine Bijektion $\phi:Y_0(f_r)\to Z_0({{f'}_l})$ die gewünschte Sphäre durch $Y\#_{\phi } Z.$ Diese umschreiben wir mit dem Symbol $(1_{f_1}^{\phi_1}\ldots p_{f_{l-1}}^{\phi_{l-1}},r^\phi_{f})_X.$ Falls $X$ Kontext im klar ist, schreiben wir nur $1_{f_1}^{\phi_1}\ldots p_{f_{l-1}}^{\phi_{l-1}},r^\phi_{f}.$ Falls $X$
\end{itemize}
\end{comment}
\[
\textcolor{red}{comment}
\]
Schauen wir uns beispielsweise Multi-Oktaeder an.
\begin{itemize}
\item Der Oktaeder bildet den kleinsten Multi-Oktaeder.
\begin{center}
$\fbox{
\parbox{13cm}{
\textcolor{red}{gap$>$} \textcolor{blue}{O;}\newline 
simplicial surface (6 vertices, 12 edges, and 8 faces) \newline
\textcolor{red}{gap$>$} \textcolor{blue}{VerticesOfFaces(O);}\newline
[ [ 1, 2, 3 ], [ 1, 3, 4 ], [ 1, 4, 5 ], [ 1, 2, 5 ], \newline
[ 2, 3, 6 ], [ 3, 4, 6 ], [ 4, 5, 6 ], [ 2, 5, 6 ] ]
}}$
\end{center}

\item Bis auf Isomorphie gibt es genau einen Doppel-Oktaeder. Diesen erhalten wir durch folgende Rechnung in GAP.
\begin{center}
$\fbox{
\parbox{13cm}{
\textcolor{red}{gap$>$} \textcolor{blue}{C3W2([O],[O])};\newline
[ simplicial surface (9 vertices, 21 edges, and 14 faces) ]
}}$
\end{center}
\item Bis auf Isomorphie gibt es genau drei Multi-Oktaeder die sich aus drei Oktaedern Zusammensetzen.
\begin{center}
$\fbox{
\parbox{13cm}{
\textcolor{red}{gap$>$}\textcolor{blue}{ C3W2(last,[O]);}\newline
[ simplicial surface (12 vertices, 30 edges, and 20 faces)
    ,\newline
     simplicial surface (12 vertices, 30 edges, and 20 faces)
    ,\newline
  simplicial surface (12 vertices, 30 edges, and 20 faces)
 ]
}}$
\end{center}
\end{itemize}
\begin{definition}
Sei $X$ eine vertex-treue Sphäre ohne 3-Taillen.
\begin{itemize}
\item Für eine vertex-treue Sphäre $Y$ mit $\vert Y_2 \vert > 2\vert X_2 \vert-2$ definieren wir $Y^{(1)_X}$ wie folgt:
Seien $W_1,\ldots,W_k$ die 3-Taillen der Sphäre $Y$ und $Z^0:=Y.$ 
Zudem sei für $0\leq i < k$ die Sphäre $Z^i$ bereits konstruiert.
Falls die simplziale Fläche $X^{W_{i+1}}$ eine Zusammenhangskomponente $Z'$ mit $Z'\cong X$ hat, dann definieren wir $Z^{i+1}$ durch die Flächenmenge 
\[
(Z^{i+1})_2=((Z^{i})^{W_{i+1}})_2-Z'_2,
\]
die Kantenmenge
\[
(Z^{i+1})_1=((Z^{i})^{W_{i+1}})_1-Z'_1
\]
und die Eckenmenge 
\[
(Z^{i+1})_0=((Z^{i})^{W_{i+1}})_0-Z'_0.
\]
und die zugehörige Inzidenz erhalten wir durch Einschränkung der Inzidenz auf $((Z^{i})^{W_{i+1}})$
Falls jedoch $X^{W_{i+1}}$ keine Zusammenhangskomponente besitzt, die isomorph zu $X$ ist, dann definieren wir $Z^{i+1}=Z^i.$ Wir setzen also $Y^{(1)_X}:=Z^{k+1}.$ 
Für $j>1$ definieren wir analog 
\[
Y^{(i)_X}:={(Y^{(i-1)_X})}^{(1_X)}
\]
\item Wir nennen $Y$ eine Multi-Sphäre mit zu $X$ isomorphen Blöcken vom Grad $k,$ falls $Y^{(k)_X}$ isomorph zu $X$ ist oder genau einen 3-Waist $W$ hat und die beiden Zusammenhangskomponenten von $(Y^{(k)_X})^W$ isomorph zu $X$ sind.
\end{itemize}
\end{definition}
In Kapitel \ref{cactus} haben wir nachgewiesen, dass die Automorphismengruppen von Multi-Tetraedern auflösbar sind. Der nächste Satz formuliert eine Verallgemeinerung dieses Resultats. 
\begin{satz}
Sei $Y$ eine Multi-Sphäre mit zu der Sphäre $X$ isomorphen Blöcken. Falls $\Aut(Y)$ auflösbar ist, dann folgt die Auslösbarkeit von $\Aut(X).$ 
\end{satz}
\begin{proof}
Wir führen den Beweis induktiv. Sei $Y$ eine Multi-Sphäre mit zu $X$ isomorphen Blöcken. Wir können annehmen, dass die Anzahl der 3-Taillen in $Y$ mindestens 2 ist. Weiterhin sei $G$ die Automorphismengruppe von $Y$ und $Z:=Y^{(1)_X}.$ Dann gibt es 3-Taillen $W_1,\ldots,W_k,$ sodass das k-malige Anwenden der zweiten Konstruktion in Bemerkung \ref{3waist} uns die Sphäre $Z$ hervorbringt.
 Für $1\leq i \leq k$ sei $Y^i$ eine 3-Taillen Komponente mit $\vert Y^i\vert =\vert X_2\vert -1.$ Für diese definieren wir die Mengen 
\[
M_t:=Y^t\cup Y_1(Y^t)\cup Y_0(Y^t).
\] Da eine der 3-Taille $W_i$ unter einem Automorphismus auf eine der obigen 3-Taillen $W_j$ abgebildet werden muss, werden die Mengen $M_t$ in $Y$ bei einer Anwendung eines Automorphismus untereinander permutiert. In $Z$ werden die 3-Taillen Komponenten durch Flächen $F_1,\ldots,F_k$ ersetzt.
Wir betrachten nun in Analogie zum oben genannten Beweis die Abbildung $\psi:G\to \Aut(Z),$ der durch
\[
\Psi(\phi)(x)=\Biggl\{
\begin{tabular}[l]{lcr}
$F_j$,&\textcolor{black}{ falls  $x\in Y^j$} \\
x,& sonst\\
\end{tabular}
\]
für alle $\phi\in G$ definiert ist.
$\Psi(G)$ bildet eine Untergruppe von $\Aut(Z)$ und ist somit als Untergruppe einer auflösbaren Gruppe ebenfalls auflösbar. Sei also $l\in \mathbb{N}$ mit der Eigenschaft $\Psi(G)^l=\{id\},$ dann gilt für alle $x\in Y-\bigcup_{t=1}^kM_t$ und für alle $\phi_l \in G^l$ die Gleichheit
\[
\phi_l(x)=\Psi(\phi_l)(x)=x.
\]
Also muss schon $\phi(M_t)=M_t$ gelten. Es müssen also nur noch die Komponenten $Y^1,\ldots,Y^k$ betrachtet werden. Da für alle $1\leq t\leq k$ die Menge $G_t:=\{\phi_{\mid M_t}\mid \phi\in G^l\}$ eine Untergruppe von $\Aut(X)$ ist erhalten wir die Auflösbarkeit von $G_t$ als Untergruppe einer auflösbaren Gruppe und damit insgesamt die Auflösbarkeit der Gruppe $G.$
 \end{proof}
Die Umkehrung des obigen Satzes gilt jedoch nicht. Hierfür betrachten wir den Ikosaeder und führen folgende Rechnung in GAP an:
\begin{center}
$\fbox{
\parbox{14cm}{
\textcolor{red}{gap$>$}\textcolor{blue}{ S:=Icosahedron();}\newline
simplicial surface (12 vertices, 30 edges, and 20 faces)\newline
\textcolor{red}{gap$>$} \textcolor{blue}{IsSolvableGroup(AutomorphismGroup(S));}\newline
false
}}$
\end{center}
Der Ikosaeder hat also eine Automorphismengruppe, die nicht auflösbar ist. Wir konstruieren nun Multi-Sphären die aus drei Ikosaedern zusammengesetzt sind.
\begin{center}
$\fbox{
\parbox{14cm}{
\textcolor{red}{gap$>$} \textcolor{blue}{C3W2([S],[S]);}\newline
[ simplicial surface (21 vertices, 57 edges, and 38 faces)
 ]\newline
\textcolor{red}{gap$>$} \textcolor{blue}{C3W2(last,[S]);}\newline
[ simplicial surface (30 vertices, 84 edges, and 56 faces)
    ,\newline simplicial surface (30 vertices, 84 edges, and 56 faces)
    , \newline simplicial surface (30 vertices, 84 edges, and 56 faces)
    ,\newline simplicial surface (30 vertices, 84 edges, and 56 faces)
    ,\newline
  simplicial surface (30 vertices, 84 edges, and 56 faces)
 ]\newline
\textcolor{red}{gap$>$}\textcolor{blue}{ List(last,g-$>$Order(AutomorphismGroup(g)));}\newline
[ 4, 2, 2, 4, 12 ]
}}$
\end{center}
In dieser Menge gibt es also Sphären mit der $C_2$ als Automorphismengruppe, welche auflösbar ist.
\subsection{Beispiele}

\section{GAP}
An verschiedenen Stellen dieser Arbeit tauchen Rechnungen in GAP auf. Dieses Kapitel soll die Funktionsweise der Funktionen erläutern, die bei der Behandlung der angeführten Thematiken unerlässlich sind. 
\begin{comment}
\begin{satz}
Sei $(X,<)$ eine vertex-treue  Sphäre und $e\in X_1$ eine Kante, die folgendes erfüllt 
\[
deg(V)=4 \forall V\in X_0(X_2(e)).
\]
 Dann ist $X \cong O$, wobei $(O,<_O)$ der bereits bekannte Oktaeder ist. 
\end{satz}
\[
\textcolor{red}{Anzahl}
\]
\begin{proof}
Für den Beweis nutzen wir die zuvor definierte Schmetterlings-Entfernung. Sei $X$ eine vertex-treue Sphäre mit obiger Eigenschaft. Dann bildet $Y=\textcolor{red}{e\beta(X)}$ eine simpliziale Fläche mit den Knoten $V,V',V''$ und Kanten $e',e''$, wobei
\begin{align*}
&deg(V)=4\\
&deg(V')=deg(V'')=3\\
&X_0(e')=\{V,V'\}\\
&X_0(e')=\{V,V''\}
\end{align*}
gilt.  Ein erneutes Anwenden der Schmetterlings-Entfernung liefert und die Sphäre $Z=\textcolor{red}{e\beta(Y)}$ mit $\overline{V}\in Z_0$ und $\overline{e}\in Z_1$, die $X_0(\overline{e})=\{\overline{V},V''\}$ erfüllen. Damit ist $Z\cong T$, da $\overline{V}$ und $V''$ benachbarte Knoten vom Grad $3$ sind.
Damit ist $Y$ isomorph zum Doppel-Tetraeder $DT$ und schlussendlich erhält man die obige Behauptung.
\end{proof}
\begin{bemerkung}
Falls bei der obigen Formulierung des Satzes auf die Voraussetzung der Vertex-Treue verzichtet wird, so ist die Aussage falsch. Denn die simpliziale Fläche, die dadurch entsteht, dass man sich eine beliebige Kante des Oktaeders nimmt, einen Cratercut an dieser durchführt und an den neu entstandenen Randkanten eine offene Tasche anheftet, erfüllt dann die Voraussetzung der schwächeren Umformulierung. Diese ist jedoch nicht zum Oktaeder isomorph.
\end{bemerkung}
\end{comment}

\section*{Appendix}

  
\begin{bemerkung}
Sei $X$ eine vertex-treue Sphäre und $G$ seine Automorphismengruppe. Dann definieren wir die Menge $FV_X$ durch   
\[
\{(F,V_1,V_2,V_3)\mid \, F\in X_2,\, X_0(F)=\{V_1,V_2,V_3\}\}
\]
In Analogie zu \textcolor{red}{...} kann mithilfe dieser Menge die Gruppenoperation
\[
\Phi:G\times FV_x\to FV_X,(g,(F,V_1,V_2,V_3))\mapsto (\phi(F),\phi(V_1),\phi(V_2),\phi(V_3))
\] 
aufgestellt werden. Diese ist vom besonderem Interesse für die Bestimmung der Anzahl der Sphären $X\#Y,$ die bis auf Isomorphie aus $X$ und $Y$ hervorgehen.
\end{bemerkung}
\begin{bsp}
\begin{itemize}
\item 
Sei $T$ der Tetraeder. Dann operiert $G$ transitiv auf $FV_T.$
\item

\end{itemize}
\end{bsp}
\textcolor{red}{!!!!!!!!!!!!!!!!!!!!!}
\begin{lemma}
Seien $X$ und $Y$ vertex-treue Sphären. Weiterhin operiere $
\Aut(Y)$ transitiv auf $FV_Y.$ Die Anzahl der Sphären der Form $X\#Y$ erhalten wir dann durch die Anzahl der Bahnen der Operation $ \Phi_{X_2}.$ 
\end{lemma}
\begin{proof}
Dieses Lemma bildet eine Verallgemeinerung von \textcolor{red}{...} und daher erhalten wir die Behauptung durch eine analoge Beweisführung.
\end{proof}
\pagestyle{empty}
\end{document}

