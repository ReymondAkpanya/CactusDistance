\documentclass{beamer} 
\usetheme{Warsaw}             % Falls Ihnen das Layout nicht gefällt, können Sie hier
                              % auch andere Themes wählen. Ein Verzeichnis der möglichen 
                              % Themes finden Sie im Kapitel 15 des beameruserguide.

\usepackage[utf8]{inputenc}
\usepackage[ngerman]{babel}
\usepackage{amsmath}
\usepackage{amsfonts}
\usepackage{amssymb}
\usepackage{float}
\usepackage{graphicx}
\usepackage{pdfpages}
\newcommand{\gelb}{0.550000011920929}
\usepackage{pgf,tikz,pgfplots}
\pgfplotsset{compat=1.15}
\usetikzlibrary{arrows}
\AtBeginSection[]{\frame<beamer>{\frametitle{Übersicht} \tableofcontents[current]}}

\newcommand{\defin}[1]{\textit{\color{blue}#1}}

% ========== Abkürzungen ==========
\newcommand{\N}{\mathbb{N}}
\newcommand{\Z}{\mathbb{Z}}
\newcommand{\Q}{\mathbb{Q}}
\newcommand{\R}{\mathbb{R}}
\newcommand{\C}{\mathbb{C}}

\author{Reymond Akpanya}
\title{Manipulation diskreter simplizialer Flächen}
\date{19.10.2018 \\[.5\baselineskip] Vortrag zur Bachelorarbeit}

\begin{document}
\frame{\maketitle}
%\frame{\tableofcontents[currentsection]}
\begin{frame}{Gliederung}
\tableofcontents
\end{frame}
%\maketitle

\section{Wanderinghole}
\begin{frame}
\begin{block}{Bemerkung 4.1}
Sei $(X,<)$ eine simpliziale Fläche.
Für $x\in X$ und $m \in \mathbb{N}$ schreibt man $x=\{x_1,\ldots,x_m\}$ und dies wird wie folgt interpretiert:\\ Man identifiziert $X$ mit der isomorphen simplizialen Fläche $Y \in \mathcal{M}(n \Delta)$, wobei $n\in \mathbb{N}$ die Anzahl der Flächen in $X$ ist. Es gilt $x_1,\ldots,x_m \in n\Delta$ und $\{x_1,\ldots,x_m\}\in Y$. Außerdem gilt $\beta(\{x_1,\ldots,x_m\})=x$ für einen  Isomorphismus $\beta: Y \to X$.  

\end{block}
\end{frame}
\begin{frame}{Beispiel 1}
Simpliziale Fläche $X$:\pause 
\begin{center}
\definecolor{qqqqff}{rgb}{0.,0.,1.}
\definecolor{ffffqq}{rgb}{1.,1.,0.}
\definecolor{ududff}{rgb}{0.30196078431372547,0.30196078431372547,1.}
\definecolor{xdxdff}{rgb}{0.49019607843137253,0.49019607843137253,1.}
\begin{tikzpicture}[line cap=round,line join=round,>=triangle 45,x=1.5cm,y=1.5cm]
%\begin{axis}[
x=1.0cm,y=1.0cm,
axis lines=middle,
ymajorgrids=true,
xmajorgrids=true,
xmin=-1.5211726268822376,
xmax=6.403358908477344,
ymin=-1.538156922171471,
ymax=4.6112795492675716,
xtick={-1.0,0.0,...,6.0},
ytick={-1.0,0.0,...,4.0},]
\clip(-0.5211726268822376,-0.338156922171471) rectangle (4.403358908477344,2.426112795492675716);
\fill[line width=2.pt,color=ffffqq,fill=ffffqq,fill opacity=\gelb] (2.,0.) -- (2.,2.) -- (0.2679491924311226,1.) -- cycle;
\fill[line width=2.pt,color=ffffqq,fill=ffffqq,fill opacity=\gelb] (2.,2.) -- (2.,0.) -- (3.7320508075688776,1.) -- cycle;
\draw [line width=2.pt] (2.,0.)-- (2.,2.);
\draw [line width=2.pt] (2.,2.)-- (0.2679491924311226,1.);
\draw [line width=2.pt] (0.2679491924311226,1.)-- (2.,0.);
%\draw [line width=2.pt,color=ffffqq] (2.,2.)-- (2.,0.);
\draw [line width=2.pt] (2.,0.)-- (3.7320508075688776,1.);
\draw [line width=2.pt] (3.7320508075688776,1.)-- (2.,2.);
\begin{scriptsize}
\draw [fill=blue] (2.,0.) circle (2.5pt);
\draw[color=black] (2.0712816691474387,-0.19995699458406646) node {$V_3$};
\draw [fill=blue] (2.,2.) circle (2.5pt);
\draw[color=black] (2.0712816691474387,2.20969389414946834) node {$V_2$};
\draw[color=black] (1.30697720713558868,0.980705685914867) node {$F_1$};
\draw[color=black] (1.80697720713558868,.980705685914867) node {$e_1$};
\draw[color=black] (1.0697720713558868,1.650690142385852654) node {$e_3$};
\draw[color=black] (1.0697720713558868,0.350799571088334387) node {$e_2$};
\draw [fill=blue] (0.2679491924311226,1.) circle (2.5pt);
\draw[color=black] (0.0033845077341547725,0.980705685914867) node {$V_1$};
\draw[color=black] (2.7214706544948127,0.980705685914867) node {$F_2$};
\draw[color=black] (2.90867021027061,0.30799571088334387) node {$e_5$};
\draw[color=black] (2.90867021027061,1.650690142385852654) node {$e_4$};
\draw [fill=blue] (3.7320508075688776,1.) circle (2.5pt);
\draw[color=black] (3.95041125648794005,0.983164947015802) node {$V_4$};
\end{scriptsize}
%\end{axis}
\end{tikzpicture}
%\caption{isomorphe simpliziale Fläche in $n\Delta$}
\end{center}\pause
Man schreibt: $e_1=\{e_1^1,e_1^2\}$ und meint damit die zu $X$ isomorphe simpliziale Fläche $Y\in \mathcal{M}(2\Delta)$:
\end{frame}

\begin{frame}{Beipiel 1}
simpliziale Fläche $Y$\pause
\begin{center}
\definecolor{qqqqff}{rgb}{0.,0.,1.}
\definecolor{ffffqq}{rgb}{1.,1.,0.}
\definecolor{ududff}{rgb}{0.30196078431372547,0.30196078431372547,1.}
\definecolor{xdxdff}{rgb}{0.49019607843137253,0.49019607843137253,1.}
\begin{tikzpicture}[line cap=round,line join=round,>=triangle 45,x=1.5cm,y=1.5cm]
%\begin{axis}[
x=1.0cm,y=1.0cm,
axis lines=middle,
ymajorgrids=true,
xmajorgrids=true,
xmin=-1.5211726268822376,
xmax=6.403358908477344,
ymin=-1.538156922171471,
ymax=4.6112795492675716,
xtick={-1.0,0.0,...,6.0},
ytick={-1.0,0.0,...,4.0},]
\clip(-0.5211726268822376,-0.338156922171471) rectangle (4.403358908477344,2.526112795492675716);
\fill[line width=2.pt,color=ffffqq,fill=ffffqq,fill opacity=\gelb] (2.,0.) -- (2.,2.) -- (0.2679491924311226,1.) -- cycle;
\fill[line width=2.pt,color=ffffqq,fill=ffffqq,fill opacity=\gelb] (2.,2.) -- (2.,0.) -- (3.7320508075688776,1.) -- cycle;
\draw [line width=2.pt] (2.,0.)-- (2.,2.);
\draw [line width=2.pt] (2.,2.)-- (0.2679491924311226,1.);
\draw [line width=2.pt] (0.2679491924311226,1.)-- (2.,0.);
%\draw [line width=2.pt,color=ffffqq] (2.,2.)-- (2.,0.);
\draw [line width=2.pt] (2.,0.)-- (3.7320508075688776,1.);
\draw [line width=2.pt] (3.7320508075688776,1.)-- (2.,2.);
\begin{scriptsize}
\draw [fill=blue] (2.,0.) circle (2.5pt);
\draw[color=black] (2.0712816691474387,-0.19995699458406646) node {$\{V_3^1,V_3^2\}$};
\draw [fill=blue] (2.,2.) circle (2.5pt);
\draw[color=black] (2.0712816691474387,2.2969389414946834) node {$\{V_2^1,V_2^2\}$};
\draw[color=black] (1.30697720713558868,0.980705685914867) node {$F_1$};
\draw[color=black] (1.60697720713558868,1.380705685914867) node {$\{e_1^1,e_1^2\}$};
\draw[color=black] (.959771842857143,1.70690142385852654) node {$\{e_3^1\}$};
\draw[color=black] (0.801414699689985814,0.40799571088334387) node {$\{e_2^1\}$};
\draw [fill=blue] (0.2679491924311226,1.) circle (2.5pt);
\draw[color=black] (0.033845077341547725,1.293164947015802) node {$\{V_1^1\}$};
\draw[color=black] (2.8214706544948127,0.980705685914867) node {$F_2$};
\draw[color=black] (3.164867021027061,0.40799571088334387) node {$\{e_3^2\}$};
\draw[color=black] (3.164867021027061,1.70690142385852654) node {$\{e_2^2\}$};
\draw [fill=blue] (3.7320508075688776,1.) circle (2.5pt);
\draw[color=black] (3.8041125648794005,1.293164947015802) node {$\{V_1^2\}$};
\end{scriptsize}
%\end{axis}
\end{tikzpicture}
%\caption{isomorphe simpliziale Fläche in $n\Delta$}
\end{center}
\end{frame}
\begin{frame}
Sei $(X,<)$ eine geschlossene simpliziale Fläche und für $i,j\in \{1,2,3\}$ seien $V_i \in X_0$ Knoten in $X$, $e_j \in X_1$ Kanten in $X$ und $F\in X_2$ eine Fläche in $X$, sodass folgendes gilt\pause 
\begin{itemize}
 \item $\vert X_{2}\vert \geq 4$,\pause
 \item $e_{i} < F$ für alle $i \in \{1,2,3\}$,\pause
 \item $V_{i}<e_{j}$ für alle $i \in \{1,2,3\}$ und $j \in \{1,2,3\} \setminus\{i\}$.
 \end{itemize}
 Zudem seien $f,g\in X_1$, $V_4\in X_0$ und $F'\in X_2$ mit\pause
\begin{itemize}
\item $e_3<F'$,\pause
\item $f,g <F'$, \pause
\item $V_1,V_4<f$ und $V_2,V_4<g$.
\end{itemize}
\end{frame}  
%-------------------------------
%-------------------------------
\begin{frame}
simpliziale Fläche $X$\\
\begin{figure}[H]
\definecolor{uuuuuu}{rgb}{0.26666666666666666,0.26666666666666666,0.26666666666666666}
\definecolor{ududff}{rgb}{0.30196078431372547,0.30196078431372547,1.}
\definecolor{ffffqq}{rgb}{1.,1.,0.}
\begin{tikzpicture}[line cap=round,line join=round,>=triangle 45,x=1.5cm,y=1.5cm]
%\begin{axis}[
x=1.5cm,y=1.5cm,
axis lines=middle,
ymajorgrids=true,
xmajorgrids=true,
xmin=-4.3,
xmax=7.0600000000000005,
ymin=-2.46,
ymax=6.3,
xtick={-4.0,-3.0,...,7.0},
ytick={-2.0,-1.0,...,6.0},]
\clip(-3.5,-0.45) rectangle (3.06,4.3);
\fill[line width=2.pt,color=ffffqq,fill=ffffqq,fill opacity=\gelb] (-2.,0.) -- (2.,0.) -- (2.,4.) -- (-2.,4.) -- cycle;
%\fill[line width=2.pt,color=ffffqq,fill=ffffqq,fill opacity=\gelb] (-1.,2.) -- (1.,2.) -- (0.,3.7320508075688776) -- cycle;
%\fill[line width=2.pt,color=ffffqq,fill=ffffqq,fill opacity=\gelb] (1.,2.) -- (-1.,2.) -- (0.,0.2679491924311226) -- cycle;
\draw [line width=2.pt] (-1.,2.)-- (1.,2.);
\draw [line width=2.pt] (1.,2.)-- (0.,3.7320508075688776);
\draw [line width=2.pt] (0.,3.7320508075688776)-- (-1.,2.);
\draw [line width=2.pt] (1.,2.)-- (-1.,2.);
\draw [line width=2.pt] (-1.,2.)-- (0.,0.2679491924311226);
\draw [line width=2.pt] (0.,0.2679491924311226)-- (1.,2.);
\begin{scriptsize}
\draw [fill=ududff] (-1.,2.) circle (2.5pt);
\draw[color=black] (-1.24,2.07) node {$V_1$};
\draw [fill=ududff] (1.,2.) circle (2.5pt);
\draw[color=black] (1.24,2.07) node {$V_2$};
\draw[color=black] (0.,2.75) node {$F$};
\draw[color=black] (0.06,1.85) node {$e_3$};
\draw[color=black] (0.67,2.96) node {$e_1$};
\draw[color=black] (-0.67,2.96) node {$e_2$};
\draw [fill=ududff] (0.,3.7320508075688776) circle (2.5pt);
\draw[color=black] (0.,3.91) node {$V_3$};
\draw[color=black] (0.,1.31) node {$F'$};
\draw[color=black] (-0.7,1.16) node {$f$};
\draw[color=black] (0.7,1.16) node {$g$};
\draw [fill=ududff] (0.,0.2679491924311226) circle (2.5pt);
\draw[color=black] (0.,0.1) node {$V_4$};
\end{scriptsize}
%\end{axis}
\end{tikzpicture}
\caption{Ausschnitt einer simplizialen Fläche}
\end{figure}

\end{frame}
\begin{frame}
Für die folgende Konstruktion nutzt man Bemerkung 4.1 und identifiziert $X$ mit der zu $X$ isomorphen simplizialen Fläche $Y \in \mathcal{M}(n\Delta) $, wobei folgendes gilt\pause
\begin{itemize}
\item $e_i=\{e_i^1,e_i^2\}$ für $i=1,2,3$\pause
\item $V_j=\{V_j^1,V_j^2\}$ für $j=2,3$\pause
\item $V_1=\{V_1^1,V_1^2,V_1^3\}$\pause
\item $f=\{f^1,f^2\}$\pause
\item $g=\{g\}$\pause
\item und $V_4=\{V_4\}$.
\end{itemize}
\end{frame}

\begin{frame}
\begin{center}
\begin{figure}[H]
\definecolor{uuuuuu}{rgb}{0.26666666666666666,0.26666666666666666,0.26666666666666666}
\definecolor{ududff}{rgb}{0.30196078431372547,0.30196078431372547,1.}
\definecolor{ffffqq}{rgb}{1.,1.,0.}
\definecolor{qqqqff}{rgb}{0.,0.,1.}
\begin{tikzpicture}[line cap=round,line join=round,>=triangle 45,x=1.5cm,y=1.5cm]
%\begin{axis}[
x=1.5cm,y=1.5cm,
axis lines=middle,
ymajorgrids=true,
xmajorgrids=true,
xmin=-4.3,
xmax=7.0600000000000005,
ymin=-2.46,
ymax=6.3,
xtick={-4.0,-3.0,...,7.0},
ytick={-2.0,-1.0,...,6.0},]
\clip(-3.8,-0.46) rectangle (8.06,4.3);
\fill[line width=2.pt,color=ffffqq,fill=ffffqq,fill opacity=\gelb] (-2.2,-0.1) -- (2.,-0.1) -- (2.,4.1) -- (-2.2,4.1) -- cycle;
%\fill[line width=2.pt,color=ffffqq,fill=ffffqq,fill opacity=\gelb] (-1.,2.) -- (1.,2.) -- (0.,3.7320508075688776) -- cycle;
%\fill[line width=2.pt,color=ffffqq,fill=ffffqq,fill opacity=\gelb] (1.,2.) -- (-1.,2.) -- (0.,0.2679491924311226) -- cycle;
\draw [line width=2.pt] (-1.,2.)-- (1.,2.);
\draw [line width=2.pt] (1.,2.)-- (0.,3.7320508075688776);
\draw [line width=2.pt] (0.,3.7320508075688776)-- (-1.,2.);
\draw [line width=2.pt] (1.,2.)-- (-1.,2.);
\draw [line width=2.pt] (-1.,2.)-- (0.,0.2679491924311226);
\draw [line width=2.pt] (0.,0.2679491924311226)-- (1.,2.);
\begin{scriptsize}
\draw [fill=qqqqff] (-1.,2.) circle (2.5pt);
\draw[color=black] (-1.6,2.2) node {$\{V_1^1,V_1^2,V_1^3\}$};
\draw [fill=qqqqff] (1.,2.) circle (2.5pt);
\draw[color=black] (1.54,2.07) node {$\{V_2^1,V_2^2\}$};
\draw[color=black] (0.,2.75) node {$F$};
\draw[color=black] (0.06,1.8) node {$\{e_3^1,e_3^2\}$};
\draw[color=black] (0.87,2.96) node {$\{e_1^1,e_1^2\}$};
\draw[color=black] (-0.87,2.96) node {$\{e_2^1,e_2^2\}$};
\draw [fill=qqqqff] (0.,3.7320508075688776) circle (2.5pt);
\draw[color=black] (0.,3.91) node {$\{V_3^1,V_3^2\}$};
\draw[color=black] (0.,1.31) node {$F'$};
\draw[color=black] (-1.,1.16) node {$\{f^1,f^2\}$};
\draw[color=black] (0.8,1.16) node {$\{g\}$};
\draw [fill=qqqqff] (0.,0.2679491924311226) circle (2.5pt);
\draw[color=black] (0.,0.05) node {$\{V_4\}$};
\end{scriptsize}
%\end{axis}
\end{tikzpicture}
\caption{Ausschnitt eines Mendings einer simplizialen Fläche}\label{abb16}
\end{figure}
\end{center}
\end{frame}

\subsection{Prozedur $P^1$}
%\subsection{Prozedur P^1}
\begin{frame}
\center{\huge{\bf{Prozedur $P^1$}}}
\end{frame}
\begin{frame}
Anwenden des $Cratercuts$ $C^{c}_{\{e_{1}^1,e_{1}^2\}}$:\\
\begin{figure}[H]
\definecolor{ffffff}{rgb}{1.,1.,1.}
\definecolor{qqqqff}{rgb}{0.,0.,1.}
\definecolor{ududff}{rgb}{0.30196078431372547,0.30196078431372547,1.}
\definecolor{ffffqq}{rgb}{1.,1.,0.}
\begin{tikzpicture}[line cap=round,line join=round,>=triangle 45,x=1.4cm,y=1.38cm]
%\begin{axis}[
x=1.0cm,y=1.0cm,
axis lines=middle,
ymajorgrids=true,
xmajorgrids=true,
xmin=-4.3,
xmax=18.7,
ymin=-5.34,
ymax=6.3,
xtick={-4.0,-3.0,...,18.0},
ytick={-5.0,-4.0,...,6.0},]
\clip(-4.1,-0.) rectangle (3.7,4.3);
\fill[line width=2.pt,color=ffffqq,fill=ffffqq,fill opacity=\gelb] (-2.2,0.) -- (2.,0.) -- (2.,4.2) -- (-2.2,4.2) -- cycle;
\fill[line width=2.pt,color=ffffqq,fill=ffffqq,fill opacity=\gelb] (-1.,2.) -- (1.,2.) -- (0.,3.7320508075688776) -- cycle;
\fill[line width=2.pt,color=ffffqq,fill=ffffqq,fill opacity=\gelb] (1.,2.) -- (-1.,2.) -- (0.,0.2679491924311226) -- cycle;
\draw [line width=2.pt] (0.,3.7320508075688776)-- (-1.,2.);
\draw [line width=2.pt] (1.,2.)-- (-1.,2.);
\draw [line width=2.pt] (-1.,2.)-- (0.,0.2679491924311226);
\draw [line width=2.pt] (0.,0.2679491924311226)-- (1.,2.);
\draw [rotate around={-60.:(0.5,2.8660254037844513)},line width=2.pt,color=ffffff,fill=ffffff,fill opacity=1.0] (0.5,2.8660254037844513) ellipse (1.4633824013732526cm and 0.1641459454658895cm);
\draw [rotate around={-60.:(0.5,2.866025403784439)},line width=2.pt] (0.5,2.866025403784439) ellipse (1.463382401373216cm and 0.16414594546590086cm);
\begin{scriptsize}
%\draw[color=black] (0.48,2.17) node {$Vieleck1$};
\draw [fill=qqqqff] (-1.,2.) circle (2.5pt);
\draw[color=black] (-1.56,2.27) node {$\{V_1^1,V_1^2,V_1^3\}$};
\draw [fill=qqqqff] (1.,2.) circle (2.5pt);
\draw[color=black] (1.49,2.12) node {$\{V_2^1,V_2^2\}$};
\draw[color=black] (-0.1,2.37) node {$F$};
\draw[color=black] (-0.,1.8) node {$\{e_3^1,e_3^2\}$};
\draw [fill=qqqqff] (0.,3.7320508075688776) circle (2.5pt);
\draw[color=black] (0.14,4.01) node {$\{V_3^1,V_3^2\}$};
\draw[color=black] (0.06,1.47) node {$F'$};
\draw [fill=qqqqff] (0.,0.2679491924311226) circle (2.5pt);
\draw[color=black] (0.76,1.13) node {$\{g\}$};
\draw[color=black] (-0.94,1.13) node {$\{f^1,f^2\}$};
\draw[color=black] (0.36,0.13) node {$\{V_4\}$};
\draw[color=black] (0.21,2.67) node {$\{e_1^1\}$};
\draw[color=black] (0.84,3.13) node {$\{e_1^2\}$};
\draw[color=black] (-0.86,2.97) node {$\{e_2^1,e_2^2\}$};
\end{scriptsize}
%\end{axis}
\end{tikzpicture}
%\caption{simpliziale Fläche nach einem Cratercut}
\end{figure}


\end{frame}
\begin{frame}
Anwenden des $Ripcuts$ $R^{c}_{\{e^1_{2},e^2_{2}\}}$:\\
 \begin{figure}[H]
\definecolor{xdxdff}{rgb}{0.49019607843137253,0.49019607843137253,1.}
\definecolor{ffffff}{rgb}{1.,1.,1.}
\definecolor{qqqqff}{rgb}{0.,0.,1.}
\definecolor{ffffqq}{rgb}{1.,1.,0.}
\begin{tikzpicture}[line cap=round,line join=round,>=triangle 45,x=.78cm,y=.78cm]

x=1.cm,y=1.cm,
axis lines=middle,
xmin=-4.0,
xmax=14.0,
ymin=-3.3,
ymax=5.34,
xtick={-9.0,-8.0,...,14.0},
ytick={-5.0,-4.0,...,6.0},]
\clip(-5.1,-3.0) rectangle (4.,5.3);
\fill[line width=2.pt,color=ffffqq,fill=ffffqq,fill opacity=\gelb] (4.,-3.) -- (4.,5.) -- (-4.,5.) -- (-4.,-3.) -- cycle;   
\fill[line width=2.pt,color=ffffff,fill=ffffff,fill opacity=1.0] (-2.,1.) -- (2.,1.) -- (0.,4.464101615137755) -- cycle;
\fill[line width=2.pt,color=ffffqq,fill=ffffqq,fill opacity=0.1] (-2.,1.) -- (0.,-2.44) -- (1.9791273890184695,1.012050807568877) -- cycle;
\fill[line width=2.pt,color=ffffqq,fill=ffffqq,fill opacity=0.550000011920929] (-2.,1.) -- (0.,3.48) -- (2.,1.) -- cycle;
\draw [line width=2.pt] (-2.,1.)-- (2.,1.);
\draw [line width=2.pt] (2.,1.)-- (0.,4.464101615137755);
\draw [line width=2.pt] (0.,4.464101615137755)-- (-2.,1.);
\draw [line width=2.pt] (-2.,1.)-- (0.,-2.44);
\draw [line width=2.pt] (0.,-2.44)-- (1.9791273890184695,1.012050807568877);
\draw [line width=2.pt] (1.9791273890184695,1.012050807568877)-- (-2.,1.);
\draw [line width=2.pt] (-2.,1.)-- (0.,3.48);
\draw [line width=2.pt] (0.,3.48)-- (2.,1.);
\draw [line width=2.pt] (2.,1.)-- (-2.,1.);
\begin{scriptsize}
\draw[color=black] (0.12,0.617) node {$\{e_3^1,e_3^2\}$};
\draw [fill=qqqqff] (-2.,1.) circle (2.5pt);
\draw[color=black] (-3.01,1.22) node {$\{V_1^1,V_1^2,V_1^3\}$};
\draw [fill=qqqqff] (2.,1.) circle (2.5pt);
\draw[color=black] (2.59,1.42) node {$\{V_2^1,V_2^2\}$};
%\draw[color=ffffff] (0.48,2.33) node {$Vieleck1$};
%\draw[color=black] (0.07,1.5) node {$e_3$};
\draw[color=black] (1.37,3.12) node {$\{e_1^2\}$};
\draw[color=black] (-1.27,3.12) node {$\{e_2^2\}$};
\draw [fill=qqqqff] (0.,4.464101615137755) circle (2.5pt);
\draw[color=black] (0.42,4.8) node {$\{V_3^2\}$};
\draw [fill=qqqqff] (0.,-2.44) circle (2.5pt);
\draw[color=black] (0.09,-2.84) node {$\{V_4\}$};
\draw[color=black] (0.1,-0.13) node {F'};
\draw[color=black] (-1.74,-0.71) node {$\{f^1,f^2\}$};
\draw[color=black] (1.42,-0.69) node {$\{g\}$};
\draw [fill=qqqqff] (0.,3.48) circle (2.5pt);
\draw[color=black] (0.,2.8) node {$\{V_3^1\}$};
\draw[color=black] (0.06,2.01) node {F};
\draw[color=black] (-0.95,1.76) node {$\{e_2^1\}$};
\draw[color=black] (0.85,1.76) node {$\{e_1^1\}$};
\end{scriptsize}

\end{tikzpicture}
\end{figure}

\end{frame}
\begin{frame}
 Anwenden des $Splitcuts$ $ S^{c}_{\{e^1_{3},e^2_{3}\}}$:\\
 \begin{figure}[H]
\definecolor{qqqqff}{rgb}{0.,0.,1.}
\definecolor{ffffff}{rgb}{1.,1.,1.}
\definecolor{ududff}{rgb}{0.30196078431372547,0.30196078431372547,1.}
\definecolor{ffffqq}{rgb}{1.,1.,0.}
\begin{tikzpicture}[line cap=round,line join=round,>=triangle 45,x=.7cm,y=.7cm]
x=.50cm,y=.50cm,
axis lines=middle,
ymajorgrids=true,
xmajorgrids=true,
xmin=-3.5,
xmax=10.0,
ymin=-3.0,
ymax=5.2,
xtick={-9.0,-8.0,...,14.0},
ytick={-5.0,-4.0,...,6.0},]
\clip(-5.,-3.3) rectangle (14.,5.34);
%----------
%\fill[line width=2.pt,color=ffffqq,fill=ffffqq,fill opacity=0.550000011920929] (-4.,-3.) -- (-4.,5.) -- (-3.,5.) -- (-3.,-3.) -- cycle;

%--------------
\fill[line width=2.pt,color=ffffqq,fill=ffffqq,fill opacity=\gelb] (-3.5,-3.) -- (-3.5,5.) -- (3.,5.) -- (3.,-3.) -- cycle;
\fill[line width=2.pt,color=ffffff,fill=ffffff,fill opacity=1.0] (-2.,1.) -- (2.,1.) -- (0.,4.464101615137755) -- cycle;
\fill[line width=2.pt,color=ffffqq,fill=ffffqq,fill opacity=0.20000000298023224] (-2.,1.) -- (0.,-2.44) -- (1.9791273890184695,1.012050807568877) -- cycle;
\fill[line width=2.pt,color=ffffqq,fill=ffffqq,fill opacity=\gelb] (5.,1.) -- (9.,1.) -- (7.,4.464101615137755) -- cycle;
%\draw [line width=2.pt,color=ffffqq] (-3.5,-3.)-- (-3.5,5.);
%\draw [line width=2.pt,color=ffffqq] (-3.,5.)-- (3.,5.);
%\draw [line width=2.pt,color=ffffqq] (3.,5.)-- (3.,-3.);
%\draw [line width=2.pt,color=ffffqq] (3.,-3.)-- (-3.,-3.);
\draw [line width=2.pt] (-2.,1.)-- (2.,1.);
\draw [line width=2.pt] (2.,1.)-- (0.,4.464101615137755);
\draw [line width=2.pt] (0.,4.464101615137755)-- (-2.,1.);
\draw [line width=2.pt] (-2.,1.)-- (0.,-2.44);
\draw [line width=2.pt] (0.,-2.44)-- (1.9791273890184695,1.012050807568877);
\draw [line width=2.pt] (1.9791273890184695,1.012050807568877)-- (-2.,1.);
\draw [line width=2.pt] (5.,1.)-- (9.,1.);
\draw [line width=2.pt] (9.,1.)-- (7.,4.464101615137755);
\draw [line width=2.pt] (7.,4.464101615137755)-- (5.,1.);
\begin{scriptsize}
\draw[color=black] (0,0.7) node {$\{e_3^2\}$};
%\draw[color=black] (7,1.4) node {$e_3^1$};

\draw[color=black] (1.28,2.93) node {$\{e_1^2\}$};
%\draw[color=black] (8.28,2.93) node {$e_1^2$};

\draw[color=black] (-1.28,2.93) node {$\{e_2^2\}$};
\draw[color=black] (5.7,2.93) node {$\{e_2^1\}$};
\draw[color=black] (8.3,2.93) node {$\{e_1^1\}$};

\draw[color=black] (-1.88,-0.6) node {$\{f^1,f^2\}$};
\draw[color=black] (1.68,-0.6) node {$\{g\}$};
\draw [fill=qqqqff] (-2.,1.) circle (2.5pt);
\draw[color=black] (-2.551,1.42) node {$\{V_1^2,V_1^3\}$};
\draw [fill=qqqqff] (2.,1.) circle (2.5pt);
\draw[color=black] (2.29,1.42) node {$\{V_2^2\}$};
\draw[color=ffffff] (0.48,2.33) node {$Vieleck1$};
\draw [fill=qqqqff] (0.,4.464101615137755) circle (2.5pt);
\draw[color=black] (0.43,4.7) node {$\{V_3^2\}$};
\draw [fill=qqqqff] (0.,-2.44) circle (2.5pt);
\draw[color=black] (0.14,-2.75) node {$\{V_4\}$};
\draw[color=black] (0.1,0.03) node {F'};
\draw [fill=qqqqff] (5.,1.) circle (2.5pt);
\draw[color=black] (5.08,0.63) node {$\{V_1^1\}$};
\draw [fill=qqqqff] (9.,1.) circle (2.5pt);
\draw[color=black] (9.1,0.67) node {$\{V_2^1\}$};
\draw[color=black] (7.06,2.33) node {$F$};
\draw[color=black] (7.06,0.70) node {$\{e_3^1\}$};
\draw [fill=qqqqff] (7.,4.464101615137755) circle (2.5pt);
\draw[color=black] (7.14,4.83) node {$\{V_3^1\}$};
\end{scriptsize}

\end{tikzpicture}
%\caption{simpliziale Fläche nach einem Splitcut}
\end{figure}
\end{frame}
\begin{frame}
Man bezeichnet die simpliziale Fläche, die nach der Anwendung der Prozedur $P^1$ ensteht, mit \emph{$P_F^1(X):=(S^c_{\{e_3^1,e_3^2\}}\circ R^c_{\{e_2^1,e_2^2\}}\circ C^c_{\{e_1^1,e_1^2\}})(X)$}.
\end{frame}
\subsection{Prozedur $P^2$}
\begin{frame}
\center{\huge{\bf{Prozedur $P^2$}}}
\end{frame}
\begin{frame}
Anwenden des $Ripcuts$ $R^{c}_{\{f^1,f^2\}}$\\
\begin{figure}
\definecolor{ududff}{rgb}{0.30196078431372547,0.30196078431372547,1.}
\definecolor{ffffff}{rgb}{1.,1.,1.}
\definecolor{sqsqsq}{rgb}{0.12549019607843137,0.12549019607843137,0.12549019607843137}
\definecolor{ffffqq}{rgb}{1.,1.,0.}
\definecolor{qqqqff}{rgb}{0.,0.,1.}
\begin{tikzpicture}[line cap=round,line join=round,>=triangle 45,x=.750cm,y=.750cm]

x=1.0cm,y=1.0cm,
axis lines=middle,
ymajorgrids=true,
xmajorgrids=true,
xmin=-4.5,
xmax=10.0,
ymin=-5.0,
ymax=5.2,
xtick={-9.0,-8.0,...,14.0},
ytick={-5.0,-4.0,...,6.0},]
\clip(-4.34966779911168,-10.336419420914822) rectangle (10.164271214115164,4.25368189432285);
%\fill[line width=2.pt,color=ffffqq,fill=ffffqq,fill opacity=\gelb] (-2.,0.) -- (0.,-3.481320628255737) -- (2.014912102788271,-0.008609506558991509) -- cycle;
\fill[line width=2.pt,color=ffffqq,fill=ffffqq,fill opacity=\gelb] (-3.06633318691027,3.9892123475876073) -- (-3.0412642843067688,-3.8824230699117557) -- (2.6995144118949965,-3.9826986803257602) -- (2.6744455092914956,3.9892123475876073) -- cycle;
\fill[line width=2.pt,color=ffffff,fill=ffffff,fill opacity=1.0] (-2.,0.) -- (2.,0.) -- (0.,3.4641016151377553) -- cycle;
\fill[line width=2.pt,color=ffffff,fill=ffffff,fill opacity=1.0] (0.,-3.481320628255737) -- (-2.,0.) -- (-1.010683173423175,0.028325736234424664) -- cycle;
\fill[line width=2.pt,color=ffffqq,fill=ffffqq,fill opacity=0.44999998807907104] (4.,0.) -- (8.,0.) -- (6.,3.4641016151377553) -- cycle;
\draw [line width=2.pt,color=sqsqsq] (-2.,0.)-- (0.,-3.481320628255737);
\draw [line width=2.pt] (0.,-3.481320628255737)-- (2.014912102788271,-0.008609506558991509);
\draw [line width=2.pt,color=sqsqsq] (2.014912102788271,-0.008609506558991509)-- (-1.,0.);
\draw [line width=2.pt,color=ffffff]  (-3.06633318691027,3.9892123475876073)-- (-3.0412642843067688,-3.8824230699117557);
\draw [line width=2.pt,color=ffffff] (2.6995144118949965,-3.9826986803257602)-- (2.6744455092914956,3.9892123475876073);
\draw [line width=2.pt,color=sqsqsq] (2.,0.)-- (0.,3.4641016151377553);
\draw [line width=2.pt] (0.,3.4641016151377553)-- (-2.,0.);
\draw [line width=2.pt,color=sqsqsq] (0.,-3.481320628255737)-- (-2.,0.);
\draw [line width=2.pt,color=ffffff] (-1.5,0.)-- (-1.010683173423175,0.028325736234424664);
\draw [line width=2.pt,color=sqsqsq] (-1.010683173423175,0.028325736234424664)-- (0.,-3.481320628255737);
\draw [line width=2.pt,color=ffffff] (-2.,0.)-- (-1.010683173423175,0.028325736234424664);
\draw [line width=2.pt,color=sqsqsq] (4.,0.)-- (8.,0.);
\draw [line width=2.pt,color=sqsqsq] (8.,0.)-- (6.,3.4641016151377553);
\draw [line width=2.pt,color=sqsqsq] (6.,3.4641016151377553)-- (4.,0.);
\begin{scriptsize}
\draw [fill=qqqqff] (-2.,0.) circle (2.5pt);
\draw[color=black] (-2.2815938522964825,0.30408366487293736) node {$\{V_1^3\}$};
\draw [fill=qqqqff] (-0.,-3.481320628255737) circle (2.5pt);
\draw[color=black] (0.4555720184676383,-3.6314584334627392) node {$\{V_4\}$};
\draw[color=black] (0.5937265932008994,-0.9368270140003698) node {$F'$};
\draw[color=black] (1.4210003791164376,-1.7139629947089057) node {$\{g\}$};
\draw[color=sqsqsq] (0.3555720184676383,-0.27222548459358455) node {$\{e_3^2\}$};
\draw [fill=qqqqff] (2.014912102788271,-0.108609506558991509) circle (2.5pt);
\draw[color=black] (2.260808616333726,0.3) node {$\{V_2^2\}$};
\draw[color=ffffff] (-3.2295566642470322,0.32915256747643856) node {$\{f^1\}$};
%\draw[color=ffffff] (3.21342691526677,0.30408366487293736) node {$h_1$};
%\draw[color=ffffff] (0.5937265932008994,1.3695120255217366) node {$Vieleck1$};
\draw[color=sqsqsq] (1.258879343435694,2.209320262739025) node {$\{e_1^2\}$};
\draw[color=black] (-1.199251163777443,2.284526970549529) node {$\{e_2^2\}$};
\draw [fill=qqqqff] (0.,3.4641016151377553) circle (2.5pt);
\draw[color=black] (0.23022750545013249,3.7892123475876073) node {$\{V_3^2\}$};
\draw [fill=ududff] (-1.010683173423175,0.028325736234424664) circle (2.5pt);
\draw[color=black] (-0.7725285986899139,0.3547726909079489) node {$\{V_1^2\}$};
\draw[color=sqsqsq] (-1.0352008551986667,-1.0120337218108733) node {$\{f^2\}$};
\draw[color=ffffff] (-1.4368545176826946,-0.15969103329183398) node {$d$};
\draw[color=sqsqsq] (-1.5363032968546852,-1.5635495790878988) node {$\{f^1\}$};
%\draw[color=ffffff] (-1.4368545176826946,-0.15969103329183398) node {$l$};
\draw [fill=qqqqff] (4.,0.) circle (2.5pt);
\draw[color=black] (4.090838506389312,-0.3477078028180926) node {$\{V_1^1\}$};
\draw [fill=qqqqff] (8.,0.) circle (2.5pt);
\draw[color=black] (8.352276338570503,-0.2725010950075892) node {$\{V_2^1\}$};
\draw[color=black] (6.108885165971154,1.4196498307287388) node {$F$};
\draw[color=sqsqsq] (5.96626325083409,-0.22208767938658227) node {$\{e_3^1\}$};
\draw[color=sqsqsq] (7.450347065672471,2.209320262739025) node {$\{e_1^1\}$};
\draw[color=sqsqsq] (4.842905584494346,2.209320262739025) node {$\{e_2^1\}$};
\draw [fill=qqqqff] (6.,3.4641016151377553) circle (2.5pt);
\draw[color=black] (6.321970838100914,3.9892123475876073) node {$\{V_3^1\}$};
\end{scriptsize}

\end{tikzpicture}
\caption{simpliziale Fläche nach einem Ripcut}
\end{figure}
\end{frame}
\begin{frame}
Anwenden des $Ripmenders$ $R^{m}_{\{e^2_{1}\},\{e^2_{3}\}}$ an
\begin{figure}[H]
\definecolor{ffffff}{rgb}{1.,1.,1.}
\definecolor{qqqqff}{rgb}{0.,0.,1.}
\definecolor{ffffqq}{rgb}{1.,1.,0.}
\begin{tikzpicture}[line cap=round,line join=round,>=triangle 45,x=1.2cm,y=1.2cm]
%\begin{axis}[
x=1.0cm,y=1.0cm,
axis lines=middle,
ymajorgrids=true,
xmajorgrids=true,
xmin=-3.583376623376623,
xmax=16.330043290043285,
ymin=-4.489177489177493,
ymax=5.588744588744593,
xtick={-3.0,-2.0,...,16.0},
ytick={-4.0,-3.0,...,5.0},]
\clip(-2.8383376623376623,-0.289177489177493) rectangle (16.330043290043285,4.0588744588744593);
\fill[line width=2.pt,color=ffffqq,fill=ffffqq,fill opacity=\gelb] (-2.2,0.) -- (2.2,0.) -- (2.2,4.) -- (-2.2,4.) -- cycle;
\fill[line width=2.pt,color=white,fill=ffffff,fill opacity=1.0] (0.,1.) -- (0.,3.) -- (-1.7320508075688776,2.) -- cycle;
\fill[line width=2.pt,color=ffffqq,fill=ffffqq,fill opacity=0.] (0.,3.) -- (0.,1.) -- (1.7320508075688776,2.) -- cycle;
\fill[line width=2.pt,color=ffffqq,fill=ffffqq,fill opacity=\gelb] (3.,1.) -- (5.594112554112552,0.9696969696969713) -- (4.323299471110349,3.231415856986091) -- cycle;
\draw [line width=2.pt] (0.,1.)-- (0.,3.);
\draw [line width=2.pt] (0.,3.)-- (-1.7320508075688776,2.);
\draw [line width=2.pt] (-1.7320508075688776,2.)-- (0.,1.);
\draw [line width=2.pt] (0.,3.)-- (0.,1.);
\draw [line width=2.pt] (0.,1.)-- (1.7320508075688776,2.);
\draw [line width=2.pt] (1.7320508075688776,2.)-- (0.,3.);
\draw [line width=2.pt] (3.,1.)-- (5.594112554112552,0.9696969696969713);
\draw [line width=2.pt] (5.594112554112552,0.9696969696969713)-- (4.323299471110349,3.231415856986091);
\draw [line width=2.pt] (4.323299471110349,3.231415856986091)-- (3.,1.);
\begin{scriptsize}
\draw[color=black] (0.6993073593073588,2.051515151515153) node {$F'$};
\draw [fill=qqqqff] (0.,1.) circle (2.5pt);
\draw[color=black] (0.12225108225108178,0.7203463203463215) node {$\{V_4\}$};
\draw [fill=qqqqff] (0.,3.) circle (2.5pt);
\draw[color=black] (0.10225108225108178,3.2116883116883145) node {$\{V_1^2,V_3^2\}$};
%\draw[color=ffffff] (-0.17212121212121256,2.151515151515153) node {$Vieleck2$};
\draw [fill=qqqqff] (-1.7320508075688776,2.) circle (2.5pt);
\draw[color=black] (-1.8593506493506495,2.2246753246753267) node {$\{V_1^3\}$};
%\draw[color=ffffqq] (0.9880519480519474,2.151515151515153) node {$Vieleck3$};
\draw [fill=qqqqff] (1.7320508075688776,2.) circle (2.5pt);
\draw[color=black] (1.893852813852813,2.2246753246753267) node {$\{V_2^2\}$};
\draw [fill=qqqqff] (3.,1.) circle (2.5pt);
\draw[color=black] (2.717922077922077,1.1246753246753267) node {$\{V_1^1\}$};
\draw [fill=qqqqff] (5.594112554112552,0.9696969696969713) circle (2.5pt);
\draw[color=black] (5.9153246753246725,1.1246753246753267) node {$\{V_2^1\}$};
\draw[color=black] (4.31099567099567,1.8961038961038983) node {$F$};
\draw [fill=qqqqff] (4.323299471110349,3.231415856986091) circle (2.5pt);
\draw[color=black] (4.45125541125541,3.5584415584415625) node {$\{V_3^1\}$};
\draw[color=black] (4.31125541125541,0.745584415584415625) node {$\{e_3^1\}$};
\draw[color=black] (3.31099567099567,2.1961038961038983) node {$\{e_2^1\}$};
\draw[color=black] (5.31099567099567,2.1961038961038983) node {$\{e_1^1\}$};
\draw[color=black] (-0.26099567099567,2.01961038961038983) node {$\{f^2\}$};
\draw[color=black] (-1.01099567099567,1.2961038961038983) node {$\{f^1\}$};
\draw[color=black] (-1.01099567099567,2.6961038961038983) node {$\{e_2^2\}$};
\draw[color=black] (1.01099567099567,1.2961038961038983) node {$\{g\}$};
\draw[color=black] (1.11099567099567,2.6961038961038983) node {$\{e_1^2,e_3^2\}$};
\end{scriptsize}

%\end{axis}
\end{tikzpicture}
%\caption{simpliziale Fläche nach Anwendung eines Ripmernders}
\end{figure}

\end{frame}
\begin{frame}
Für eine nach der Prozedur $P^1$ entstandene simpliziale Fläche $Z=P^1_F(X)$  bezeichnet man die simpliziale Fläche, die nach Anwendung der zweiten Prozedur entsteht, mit \emph{$P^2_f(Z)$}$:=(R^m_{\{e_1^2\},\{e_3^2\}}\circ R^c_{\{f^1\},\{f^2\}})(Z)$. \pause Das heißt, $P^2_f(P^1_F(X))$ ist die Fläche, die nach der Anwendung der beiden Prozeduren $P^1$ und $P^2$ auf die gegebene simpliziale Fläche $X$ entsteht.
\end{frame}
\subsection{Prozedur $P^3$}
\begin{frame}
\center{\huge{\bf{Prozedur $P^3$}}}
\end{frame}
\begin{frame}
Anwenden des $SplitMender$ $S^{m}_{(\{V^1_{1}\},\{e^1_{3}\}),(\{V_{4}\},\{f^2\})}$:\\
\begin{figure}[H]
%\begin{comment}
\definecolor{ududff}{rgb}{0.30196078431372547,0.30196078431372547,1.}
\definecolor{ffffff}{rgb}{1.,1.,1.}
\definecolor{qqqqff}{rgb}{0.,0.,1.}
\definecolor{ffffqq}{rgb}{1.,1.,0.}
\begin{tikzpicture}[line cap=round,line join=round,>=triangle 45,x=1.5cm,y=1.5cm]
%\begin{axis}[
x=1.0cm,y=1.0cm,
axis lines=middle,
ymajorgrids=true,
xmajorgrids=true,
xmin=-4.3,
xmax=7.0600000000000005,
ymin=-2.46,
ymax=6.3,
xtick={-4.0,-3.0,...,7.0},
ytick={-2.0,-1.0,...,6.0},]
\clip(-3.8,-0.) rectangle (7.06,4.);
\fill[line width=2.pt,color=ffffqq,fill=ffffqq,fill opacity=\gelb] (-2.,0.) -- (2.1,0.) -- (2.1,4.) -- (-2.,4.) -- cycle;
\fill[line width=2.pt,color=ffffqq,fill=ffffqq,fill opacity=0.1499999940395355] (0.,3.) -- (0.,1.) -- (1.7320508075688776,2.) -- cycle;
\fill[line width=2.pt,color=ffffff,fill=ffffff,fill opacity=1.0] (0.,1.) -- (0.,3.) -- (-1.7320508075688776,2.) -- cycle;
\fill[line width=2.pt,color=ffffqq,fill=ffffqq,fill opacity=\gelb] (0.,3.) -- (-1.,2.) -- (0.,1.) -- cycle;
\draw [line width=2.pt] (0.,3.)-- (0.,1.);
\draw [line width=2.pt] (0.,1.)-- (1.7320508075688776,2.);
\draw [line width=2.pt] (1.7320508075688776,2.)-- (0.,3.);
\draw [line width=2.pt] (0.,1.)-- (0.,3.);
\draw [line width=2.pt] (0.,3.)-- (-1.7320508075688776,2.);
\draw [line width=2.pt] (-1.7320508075688776,2.)-- (0.,1.);
\draw [line width=2.pt] (0.,3.)-- (-1.,2.);
\draw [line width=2.pt] (-1.,2.)-- (0.,1.);
\begin{scriptsize}
\draw[color=black] (-1.04,1.37) node {$\{f^1\}$};
\draw [fill=qqqqff] (0.,3.) circle (2.5pt);
\draw[color=black] (0.14,3.17) node {$\{V_2^1,\{V_1^2,V_3^2\}\}$};
\draw[color=black] (0.1,0.77) node {$\{V_1^1,V_4\}$};
\draw [fill=qqqqff] (0.,1.) circle (2.5pt);
\draw[color=black] (0.64,1.97) node {$F'$};
\draw[color=black] (0.34,2.27) node {$\{e_3^1,f^2\}$};
\draw[color=black] (1.04,2.71) node {$\{e_1^2,e_3^2\}$};
\draw[color=black] (0.94,1.31) node {$\{g\}$};
\draw [fill=qqqqff] (1.7320508075688776,2.) circle (2.5pt);
\draw[color=black] (1.88,2.17) node {$\{V_2^2\}$};
\draw[color=black] (-0.3,2.01) node {$F$};
\draw [fill=qqqqff] (-1.7320508075688776,2.) circle (2.5pt);
\draw[color=black] (-1.8,2.22) node {$\{V_1^3\}$};
\draw [fill=ududff] (-1.,2.) circle (2.5pt);
\draw[color=black] (-1.26,2.0) node {$\{V_3^1\}$};
\draw[color=black] (-1.03,2.66) node {$\{e_2^2\}$};
\draw[color=black] (-0.33,2.37) node {$\{e_1^1\}$};
\draw[color=black] (-0.33,1.62) node {$\{e_2^1\}$};
\end{scriptsize}
%\end{axis}
\end{tikzpicture}
%\end{comment}
%\caption{simpliziale Fläche nach Anwendung eines Splitmenders}
\end{figure}
\end{frame}

\begin{frame}
Anwenden des $Ripmenders$ $R^m_{\{e^1_{2}\},\{f^1\}}$ \\
\begin{figure}[H]
\definecolor{ududff}{rgb}{0.30196078431372547,0.30196078431372547,1.}
\definecolor{ffffff}{rgb}{1.,1.,1.}
\definecolor{qqqqff}{rgb}{0.,0.,1.}
\definecolor{ffffqq}{rgb}{1.,1.,0.}
\begin{tikzpicture}[line cap=round,line join=round,>=triangle 45,x=1.4cm,y=1.4cm]
%\begin{axis}[
x=1.5cm,y=1.5cm,
axis lines=middle,
ymajorgrids=true,
xmajorgrids=true,
xmin=-4.3,
xmax=7.0600000000000005,
ymin=-2.46,
ymax=6.3,
xtick={-4.0,-3.0,...,7.0},
ytick={-2.0,-1.0,...,6.0},]
\clip(-4.3,-0.46) rectangle (7.06,4.3);
\fill[line width=2.pt,color=ffffqq,fill=ffffqq,fill opacity=\gelb] (-2.5,0.) -- (2.2,0.) -- (2.2,4.) -- (-2.5,4.) -- cycle;
\fill[line width=2.pt,color=ffffqq,fill=ffffqq,fill opacity=0.] (0.,3.) -- (0.,1.) -- (1.7320508075688776,2.) -- cycle;
\fill[line width=2.pt,color=ffffff,fill=ffffff,fill opacity=1.0] (0.,1.) -- (0.,3.) -- (-1.7320508075688776,2.) -- cycle;
\fill[line width=2.pt,color=ffffqq,fill=ffffqq,fill opacity=0.5] (0.,3.) -- (-1.74,2.) -- (0.,1.) -- cycle;
\draw [line width=2.pt] (0.,3.)-- (0.,1.);
\draw [line width=2.pt] (0.,1.)-- (1.7320508075688776,2.);
\draw [line width=2.pt] (1.7320508075688776,2.)-- (0.,3.);
\draw [line width=2.pt] (0.,1.)-- (0.,3.);
\draw [line width=2.pt] (-1.7320508075688776,2.)-- (0.,1.);
\draw [line width=2.pt] (-1.74,2.)-- (0.,1.);
\draw [rotate around={29.886526940424037:(-0.87,2.5)},line width=2.pt,color=ffffff,fill=ffffff,fill opacity=1.0] (-0.87,2.5) ellipse (1.41989009757162764cm and 0.22745816720870826cm);
\draw [rotate around={29.886526940424037:(-0.87,2.5)},line width=2.pt] (-0.87,2.5) ellipse (1.41989009757163032cm and 0.2274581672087103cm);
\begin{scriptsize}
\draw[color=black] (-0.5,1.9) node {$F$};
\draw [fill=qqqqff] (0.,3.) circle (2.5pt);
\draw[color=black] (0.14,3.23) node {$\{V_2^1,\{V_1^2,V_3^2\}\}$};
\draw [fill=qqqqff] (0.,1.) circle (2.5pt);
\draw[color=black] (0.14,0.75) node {$\{V_1^1,V_4\}$};
\draw[color=black] (1.04,2.75) node {$\{e_1^2,e_3^2\}$};
\draw [fill=qqqqff] (1.7320508075688776,2.) circle (2.5pt);
\draw[color=black] (1.88,2.22) node {$\{V_2^2\}$};
\draw[color=black] (0.8,2.) node {$F'$};
\draw[color=black] (0.4,2.32) node {$\{e_3^1,f^2\}$};
\draw [fill=qqqqff] (-1.7320508075688776,2.) circle (2.5pt);
\draw[color=black] (-2.1,2.22) node {$\{V_1^3,V_3^1\}$};
\draw [fill=qqqqff] (-1.74,2.) circle (2.5pt);
\draw[color=black] (-1.0,2.87) node {$\{e_2^2\}$};
\draw[color=black] (-1.0,1.26) node {$\{e_2^1,f^1\}$};
%\draw[color=black] (-1.16,2.25) node {$c$};
\draw[color=black] (-0.65,2.2) node {$\{e^1_1\}$};
\draw[color=black] (0.94,1.31) node {$\{g\}$};
\end{scriptsize}
%\end{axis}
\end{tikzpicture}
\caption{simpliziale Fläche nach Anwendung eines Ripmenders}
\end{figure}

\end{frame}
\begin{frame}
 Anwenden des $Crater Menders$ $R^{m}_{\{e_1^1\},\{e^2_{2}\}}$:\\
 \begin{figure}[H]
\definecolor{ududff}{rgb}{0.30196078431372547,0.30196078431372547,1.}
\definecolor{ffffff}{rgb}{1.,1.,1.}
\definecolor{qqqqff}{rgb}{0.,0.,1.}
\definecolor{ffffqq}{rgb}{1.,1.,0.}
\begin{tikzpicture}[line cap=round,line join=round,>=triangle 45,x=1.4cm,y=1.4cm]
%\begin{axis}[
x=1.3cm,y=1.3cm,
axis lines=middle,
ymajorgrids=true,
xmajorgrids=true,
xmin=-4.3,
xmax=7.0600000000000005,
ymin=-2.46,
ymax=6.3,
xtick={-4.0,-3.0,...,7.0},
ytick={-2.0,-1.0,...,5.0},]
\clip(-4.3,-0.) rectangle (7.06,4.);
\fill[line width=2.pt,color=ffffqq,fill=ffffqq,fill opacity=\gelb] (-2.5,0.) -- (2.2,0.) -- (2.2,4.) -- (-2.5,4.) -- cycle;
\fill[line width=2.pt,color=ffffqq,fill=ffffqq,fill opacity=0.] (0.,3.) -- (0.,1.) -- (1.7320508075688776,2.) -- cycle;
\fill[line width=2.pt,color=ffffff,fill=ffffff,fill opacity=0.] (0.,1.) -- (0.,3.) -- (-1.7320508075688776,2.) -- cycle;
\fill[line width=2.pt,color=ffffqq,fill=ffffqq,fill opacity=0.] (0.,3.) -- (-1.74,2.) -- (0.,1.) -- cycle;
\draw [line width=2.pt] (0.,3.)-- (0.,1.);
\draw [line width=2.pt] (0.,1.)-- (1.7320508075688776,2.);
\draw [line width=2.pt] (1.7320508075688776,2.)-- (0.,3.);
\draw [line width=2.pt] (0.,1.)-- (0.,3.);
\draw [line width=2.pt] (-1.7320508075688776,2.)-- (0.,1.);
\draw [line width=2.pt] (0.,3.)-- (-1.74,2.);
\draw [line width=2.pt] (-1.74,2.)-- (0.,1.);
\begin{scriptsize}
%\draw[color=ffffqq] (-0.84,1.27) node {$Vieleck1$};
%\draw [fill=qqqqff] (0.,3.) circle (2.5pt);
%\draw[color=qqqqff] (0.14,3.37) node {$E$};
%\draw [fill=qqqqff] (0.,1.) circle (2.5pt);
%\draw[color=qqqqff] (0.14,1.37) node {$F$};
%\draw[color=ffffqq] (1.04,3.11) node {$Vieleck2$};
%\draw [fill=qqqqff] (1.7320508075688776,2.) circle (2.5pt);
%\draw[color=qqqqff] (1.88,2.37) node {$G$};
%\draw[color=ffffff] (-0.1,2.17) node {$Vieleck3$};
%\draw [fill=qqqqff] (-1.7320508075688776,2.) circle (2.5pt);
%\draw[color=qqqqff] (-1.6,2.37) node {$H$};
%\draw [fill=ududff] (-1.74,2.) circle (2.5pt);
%\draw[color=ududff] (-1.6,2.37) node {$I$};
%\draw[color=black] (-0.93,3.) node {$f_1$};
%\draw[color=black] (-0.98,1.41) node {$e$};
%\end{comment}
%-----------------------------
%\begin{figure}
\draw[color=black] (-0.6,2.) node {$F$};
\draw [fill=qqqqff] (0.,3.) circle (2.5pt);
\draw[color=black] (0.14,3.23) node {$\{V_2^1,\{V_1^2,V_3^2\}\}$};
\draw [fill=qqqqff] (0.,1.) circle (2.5pt);
\draw[color=black] (0.14,0.75) node {$\{V_1^1,V_4\}$};
\draw[color=black] (1.04,2.75) node {$\{e_1^2,e_3^2\}$};
\draw [fill=qqqqff] (1.7320508075688776,2.) circle (2.5pt);
\draw[color=black] (1.88,2.22) node {$\{V_2^2\}$};
\draw[color=black] (0.8,2.) node {$F'$};
\draw[color=black] (0.4,2.32) node {$\{e_3^1,f^2\}$};
\draw [fill=qqqqff] (-1.7320508075688776,2.) circle (2.5pt);
\draw[color=black] (-2.1,2.22) node {$\{V_1^3,V_3^1\}$};
\draw [fill=qqqqff] (-1.74,2.) circle (2.5pt);
\draw[color=black] (-1.0,2.77) node {$\{e_1^1,e_2^2\}$};
\draw[color=black] (-1.0,1.26) node {$\{e_2^1,f^1\}$};
%\draw[color=black] (-1.16,2.25) node {$c$};
%\draw[color=black] (-0.65,2.2) node {$\{e^1_1\}$};
\draw[color=black] (0.94,1.31) node {$\{g\}$};
\end{scriptsize}
%\end{axis}
\end{tikzpicture}
%\tiny{\caption{simpliziale Fläche nach einem Cratermender}}
\end{figure}
 \end{frame}
 
 \begin{frame}
Für eine durch die Prozedur $P^1$ oder $P^2$ entstandene simpliziale Fläche $Z$ bezeichnet man die simpliziale Fläche, die aus der Prozedur $P^3$ hervorgeht, mit \emph{$P^3_F(Z)$}$:=(S^m_{(\{V_1^1\},\{e_3^1\}),(\{V_4\},\{f^2\})} \circ R^m_{\{e_2^1\},\{f^1\}} \circ C^m_{\{e_1^1\},\{e_2^2\}})(Z)$. \\ \pause
Somit ist $P^3_F(P^2_f(P^1_F(X)))$ die simpliziale Fläche, die aus $X$ nach der Anwendung der drei Prozeduren hervorgeht.\\ 
\end{frame}


 \begin{frame}
 Die entstandene simpliziale Fläche bezeichnet man mit $X^H_{(F,f)}$, die Kante $f$ des Tupels $(F,f)$ nennt man eine Wanderkante von F. Man definiert die Menge der Wanderkanten von einer Fläche $F \in X_2$ als $\mathcal{W}_F(X)$.
\end{frame}

\subsection{Verallgemeinerung}
\begin{frame}
\center{\huge{\bf{Verallgemeinerung}}}
\end{frame}
\begin{frame}
\begin{block}{Definition 4.6}
Seien $(X,<)$ eine geschlossene simpliziale Fläche, $F \in X_2$ eine Fläche und $n\in \mathbb{N}$. Sei außerdem $(F,f_1,\ldots,f_n)$ ein Tupel mit
\begin{itemize}
%\item $F \in X_2$
\item $f_1 \in \mathcal{W}_F(P_F^1(X))$,
%\item $f_2 \in \mathcal{W}(P^2_{f_1}(P_F^1(X))))$
\item $f_i \in \mathcal{W}_F((P^2_{f_{i-1}} \circ \ldots \circ P^2_{f_1}\circ P_F^1 )(X)),$ für $i=2,\ldots ,n$.
\end{itemize}
So ist $X_{(F,f_1,\ldots,f_n)}^H$ definiert durch 
\[
X^{H}_{(F,f_1,\ldots,f_n)}:=(P_F^3\circ P^2_{f_n} \circ \ldots \circ P^2_{f_1}\circ P_F^1)(X).
\]
Das Tupel $(F,f_1,\ldots,f_n)$ nennt man eine \emph{Lochwanderung von $X$} und $X^H_{(F,f_1,\ldots,f_n)}$, die durch die Lochwanderung $(F,f_1,\ldots,f_n)$ entstandene simpliziale Fläche.
\end{block}
\end{frame}
\begin{frame}
\begin{block}{Definition 4.6}
Für eine geschlossene simpliziale Fläche $(X,<)$ und eine Lochwanderung $\sigma_1$ von $X$ führt man folgende Konstruktion durch:
\begin{itemize}
\item $X^H_{(\sigma_1)}:=X^H_{\sigma_1}$
\item Für eine Lochwanderung $\sigma_2$ von $X^H_{\sigma_1}$ ist 
\[
X_{(\sigma_1,\sigma_2)}^H:=(X_{\sigma_1}^H)^H_{\sigma_2}.
\]
\item Seien nun  $X_{(\sigma_1,\ldots, \sigma_i)}^H$ für $i\in \mathbb{N}\setminus\{1\}$ und Lochwanderungen $\sigma_j$ von $X^H_{(\sigma_1,\ldots,\sigma_{j-1})}$ mit $2 \leq j \leq i$ schon konstruiert und $\sigma_{i+1}$ eine Lochwanderung von $X_{(\sigma_1,\ldots, \sigma_i)}^H$, so ist
\[
X_{(\sigma_1,\ldots, \sigma_{i+1})}^H:=(X_{(\sigma_1,\ldots, \sigma_i)}^H)^H_{\sigma_{i+1}}.
\]
\end{itemize}
\end{block}
 \end{frame}
 \begin{frame}
 \begin{block}{Definition 4.6}
 \begin{itemize}
\item Für obige Lochwanderungen $\sigma_i$ und $n \in \mathbb{N}$ nennt man $(\sigma_1, \ldots,\sigma_n)$ eine \emph{Lochwanderungssequenz von $X$} und $X^H_{(\sigma_1,\ldots,\sigma_n)}$ die durch die Lochwanderungssequenz $(\sigma_1,\ldots,\sigma_n)$ entstandene simpliziale Fläche.
\end{itemize}
\end{block}
\end{frame}
\section{Transitivität der Operation Wanderinghole}
\subsection{Beweisidee}
\begin{frame}
\center{\huge{\bf{Beweisidee}}}
\end{frame}
\begin{frame}
 Seien $(X,<)$ und $(Y,\prec)$ geschlossene simpliziale Flächen mit $\chi(X)=\chi(Y)$ und $\vert X_2 \vert =\vert Y_2 \vert $.\\\pause
 \textbf{Ziel}: Durch Anwenden der Operation Wanderinghole aus $X$ eine simpliziale Fläche konstruieren, die dieselben Nachbarschaften wie $Y$ aufweist.
\end{frame}
\begin{frame}
simpliziale Fläche $X$:\\
\includegraphics[scale=0.8, viewport=-1.5cm 20cm 8cm 28cm]{surfacey}
\end{frame}
\begin{frame}
simpliziale Fläche $Y$:\\
\includegraphics[scale=0.8, viewport=-1.5cm 20cm 8cm 27cm]{surfacex}
\end{frame}
\begin{frame}{Beispiel 2}
\begin{columns}
    \column{0.5\textwidth}
   \center{simpliziale Fläche $Y$}\\
    \includegraphics[scale=0.6, viewport=.5cm 18cm 9cm 27.5cm]{surfacex}\\
 
    \column{0.5\textwidth}
    \center{simpliziale Fläche $X$}\\
    \includegraphics[scale=0.6, viewport=1.5cm 18cm 9cm 27.5cm]{surfacey}
    
\end{columns}
\end{frame}
\begin{frame}{Beispiel}
\begin{columns}
    \column{0.5\textwidth}
   \center{simpliziale Fläche $Y$}\\
    \includegraphics[scale=0.6, viewport=.5cm 18cm 9cm 27.5cm]{surfacex}
 
    \column{0.5\textwidth}
    \center{simpliziale Fläche $X_{(F_9,e_7)}^H$}\\
    \includegraphics[scale=0.6, viewport=2.5cm 18cm 8cm 28cm]{BAWH1}
\end{columns}
\end{frame}
\begin{frame}{Beispiel}
\begin{columns}
    \column{0.5\textwidth}
   \center{simpliziale Fläche $Y$}\\
    \includegraphics[scale=0.6, viewport=0.5cm 18cm 9cm 27.5cm]{surfacex}
 
    \column{0.5\textwidth}
    \center{simpliziale Fläche $(X_{(F_9,e_7)}^H)^H_{(F_9,e_{6})}$}\\
    \includegraphics[scale=0.6, viewport=0cm 18cm 8cm 27.5cm]{BHWH2}
\end{columns}
\end{frame}
\begin{frame}{Beispiel}
\begin{columns}
    \column{0.5\textwidth}
   \center{simpliziale Fläche $Y$}\\
    \includegraphics[scale=0.6, viewport=.5cm 18cm 9cm 27.5cm]{surfacex}
 
    \column{0.5\textwidth}
    \center{simpliziale Fläche $((X_{(F_9,e_7)}^H)^H_{(F_9,e_{6})})_{(F_9,e_{13})}^H$}\\
    \includegraphics[scale=0.6, viewport=0cm 18cm 8cm 27.2cm]{BAWH3}
\end{columns}
\end{frame}
%------------------------------------------
\begin{frame}
\center{\textbf{Anwenden von Lochwanderungen auf $X$}}\\
\begin{columns}
\column{0.5\textwidth}
\begin{itemize}
\item $X$\\
\includegraphics[scale=0.3, viewport=1cm 21cm 9cm 27.5cm]{surfacey}
\item $(X^H_{(F_9,e_7)})^H_{(e_9,e_{6})}$\\
\includegraphics[scale=0.3, viewport=0cm 21cm 8cm 27.5cm]{BHWH2}
\end{itemize}
\column{0.5\textwidth}
\begin{itemize}
\item $X^H_{(F_9,e_7)}$\\
\includegraphics[scale=0.3, viewport=0cm 21cm 8cm 27.5cm]{BAWH1}
\item $((X^H_{(F_9,e_7)})^H_{(e_9,e_{6})})^H_{(F_9,e_{13})}$\\
\includegraphics[scale=0.3, viewport=0cm 21cm 8cm 27.5cm]{BAWH3}
\end{itemize}
\end{columns}
\end{frame}
\begin{frame}
\begin{itemize}
\item Ist es immer möglich in $X$ die Nachbarschaften aus der simplizialen Fläche $Y$ nachzuahmen?\pause
\item Können durch das Anwenden von Lochwanderungssequenzen auf $X$ schon erfolgreich konstruierte Nachbarschaften zerstört werden?\pause
\item Ist es immer möglich die Operation Wanderinghole wie oben skizziert entlang eines Pfades anzuwenden? \pause
\item Ist die konstruierte simpliziale Fläche, deren Nachbarschaften mit denen der simplizialen Fläche $Y$ übereinstimmen, isomorph zu $Y$? 
\end{itemize}
\end{frame}

\begin{frame}
\center{\huge{\bf{Beweis}}}
\end{frame}

\begin{frame}
\begin{block}{Definition 4.10}
\begin{itemize}
\item Sei $(X,<)$ eine simpliziale Fläche. Ein \emph{Flächenpfad} von $S\in X_{2}$ nach $T \in X_{2}$ in $X$ ist eine Sequenz $(F_1:=S,F_{2},\ldots,F_{k}:=T)$ für ein $k \in \mathbb{N}$ so, dass $F_{i} $ und $F_{i+1}$ für $i=1,\ldots,k-1$ benachbarte Flächen in $X$ sind.
\end{itemize}
\end{block}
\end{frame}
\begin{frame}
\begin{block}{Definition 4.10}
\begin{itemize}
\item Man nennt eine Menge $M\subseteq X_2$  \emph{stark-zusammenhängend}, falls für beliebige $S,T \in M$ ein Flächenpfad $(F_{1}:=S,F_{2},\ldots,F_{k}:=T)$ mit $F_i \in M$ für $1\leq i \leq k$ existiert. 
\end{itemize}
\end{block}
\end{frame}

\begin{frame}
\begin{block}{Defintion 4.10}
\begin{itemize}
\item Sei $(X,<)$ eine simpliziale Fläche. Man nennt die Menge $M \subseteq X_2$ \emph{Jordan-zusammenhängend} in $X$, falls $M=X_2$ stark-zusammenhängend ist oder  $M \subsetneq X_2$ gilt und die Mengen $M$ und $X_2\setminus M$ stark-zusammenhängend sind. Falls $X_2$ stark zusammenhängend ist, so nennt man die simpliziale Fläche $(X,<)$ Jordan-zusammenhängend.
\end{itemize}
\end{block}
\end{frame}
\begin{frame}
\begin{block}{Lemma 4.22}
Seien $(X,<)$  eine geschlossene Jordan-zusammenhängende simpliziale Fläche und $M \subsetneq X_2$ eine Jordan-zusammenhängende Menge. Dann existiert ein $F\in X_2\setminus M$ so, dass die Menge $M \cup \{F\}$ Jordan-zusammenhängend ist.
\end{block}
\end{frame}
\begin{frame}
\begin{block}{Lemma 4.24}
Seien $(X,<)$ eine geschlossene Jordan-zusammenhängende simpliziale Fläche und $(S,F_1,F_2,\ldots,F_n,F_{n+1}:=T)$ für $n\geq 1$ ein $S$-$T$-Weg in $X$ ohne Flächenwiederholung, wobei $F_i \in X_2$ für $1 \leq i \leq n$ ist und $S,T\in X_2$ zwei nicht benachbarte Flächen sind. 
Dann existiert eine Lochwanderungssequenz $\Sigma$ so, dass $(S,F_2, \ldots,F_n,T)$ ein $S$-$T$-Weg ohne Flächenwiederholung in $X^H_{\Sigma}$ ist.
\end{block}
\end{frame}

\begin{frame}
\begin{block}{Lemma 4.27}
 Seien $(X,<)$ eine geschlossene Jordan-zusammenhängende simpliziale Fläche mit $\chi(X)=2$, $M \subsetneq X_2$ eine Jordan-zusammenhängende Teilmenge der Flächen und $F \in X_2\setminus M$ eine Fläche in $X$. Dann existiert eine Lochwanderungssequenz $\Sigma$ so, dass $M \cup \{F\}$ Jordan-zusammenhängend in $X^H_{\Sigma}$ ist.

\end{block}
\end{frame}
\subsection{Beweis}
\begin{frame}
\begin{block}{Satz 4.29}
Sei $(X,<)$ eine geschlossene Jordan-zusammenhängende simpliziale Fläche. Dann ist die Anwendung von Wanderinghole auf $X$ transitiv, das heißt, für alle geschlossenen Jordan-zusammenhängenden simplizialen Flächen $(Y,\prec)$, die keine Knoten vom Grad 2 enthalten, wobei $\vert X_2 \vert = \vert Y_2 \vert$ und $\chi(X)=\chi(Y)=2$ ist, existiert eine Lochwanderungssequenz $\Sigma$ mit $X^H_{\Sigma} \cong Y$.
\end{block}
\end{frame}
\begin{frame}
\begin{block}{Satz 4.30}
Sei $G=(V,E)$ ein 3-fach zusammenhängender planarer Graph. Dann lässt sich G eindeutig in die Sphäre einbetten.
\end{block}
\end{frame}
\section{Fazit}

%----------------------------------------------


\end{document}
