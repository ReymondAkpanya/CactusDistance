
\documentclass[12pt,titlepage,twoside,cleardoublepage]{article}
\usepackage[ngerman]{babel}
\usepackage[utf8]{inputenc}
\usepackage[a4paper,lmargin={4cm},rmargin={2cm},
tmargin={2.5cm},bmargin = {2.5cm}]{geometry}
\usepackage{amsmath}
\usepackage{amssymb}
\usepackage{pdfpages} 
%\usepackage[pdftex,article]{geometry}
\usepackage{amsthm}
%\usepackage{ngerman,amsthm}
\usepackage{lineno} 
\usepackage{lineno, blindtext} 
\usepackage{cleveref}
\usepackage{enumerate}
\usepackage{float}
\usepackage{thmtools}
\usepackage{tabularx}
\linespread{1.25}
\usepackage{color}
\usepackage{verbatim}
\newcommand{\gelb}{0.550000011920929}
\usepackage{pgf,tikz,pgfplots}
\pgfplotsset{compat=1.15}
\usepackage{mathrsfs}
\usepackage{mathrsfs}
\usetikzlibrary{arrows}
%\numberwithin{equation}{chapter}
%\usepackage{scrheadings}
\pagestyle{headings}
\usepackage{titlesec}     
\usepackage{tikz}   
\usepackage{graphicx}        % für Kontrolle der Abschnittüberschriften

%\begin{comment}
\newtheorem{zahl}{}[section]
%\setcounter{zahl}{1}
%\newtheorem{section}{section}[section]
\newtheorem{definition}[zahl]{Definition}
\newtheorem{vor}[zahl]{Vorüberlegung}
\newtheorem{lemma}[zahl]{Lemma}
\newtheorem{folgerung}[zahl]{Folgerung}
\newtheorem{bsp}[zahl]{Beispiel}
\newtheorem{herleitung}[zahl]{Herleitung}
\newtheorem{bemerkung}[zahl]{Bemerkung}
\newtheorem{satz}[zahl]{Satz}
\newtheorem{beweisidee}[zahl]{Beweisidee}
\numberwithin{equation}{section}


%-----------------------------------------------

%\end{comment}
 %Nummerierung mit Kapitelnummern
%-------------------------
%\newcommand{\secnumbering}[1]{% 
 % \setcounter{chapter}{0}% 
  %\setcounter{section}{0}% 
  %\renewcommand{\thechapter}{\csname #1\endcsname{chapter}.}% nach Duden gehört 
                                  % der Punkt hier hin bei gemischten Zählungen 
%  \renewcommand{\thesection}{\thechapter\csname #1\endcsname{section}}% 
%}
%------------------------------
\begin{document}
\begin{satz}\label{wild}
Sei $X$ ein Multi-Tetreader. Dann besitzt $X$ eine wilde Färbung.
\end{satz}
\begin{proof}
Wir weisen diese Aussage per Induktion nach. Für einen Tetraeder mit dem Flächenträger
\[
\{\{1,2,3\},\{1,2,4\},\{1,3,4\},\{2,3,4\}\}
\] erhalten wir eine wilde Färbung durch die Abbildung
\begin{align*}
\omega(e)=\Biggl\{
\begin{tabular}[l]{lcr}
a, falls e=\{1,2\},\{3,4\}\\
b, falls e=\{1,3\},\{2,4\}\\
c, falls e=\{1,4\},\{2,3\}
\end{tabular}.
\end{align*}
Da der Tetraeder eine vertex-treue Sphäre ist, wurden an dieser Stelle die Kanten mit den inzidenten Ecken identifiziert.
\begin{figure}[H]
\begin{center}
\includegraphics[viewport=0cm 23.6cm 6cm 27cm]{wildcTet}
\end{center}
\caption{wild gefärbter Tetraeder}
\end{figure}
Sei nun $X$ ein Multi-Tetraeder mit $n=\vert X_2 \vert >4$ und $V\in X_0$ eine Ecke vom Grad 3. Dann erhalten wir durch Entfernen des Tetraeders an der Stelle $V$ den Multi-Tetraeder $Y=T_V(X)$ mit $\vert Y_2 \vert=n-2.$ Deshalb erhalten wir für $Y$ eine wilde Färbung $\omega:Y_1\to \{a,b,c\}.$ Diese kann auf genau eine Art und Weise zu einer wilden Färbung auf $X$ erweitert werden.
Denn seien $F_1,F_2,F_3$ die Flächen des angehängten Tetraeders. Dann gilt $X_1(X_2(V))=\{e_1,e_2,e_3\}\cup \{e_a,e_b,e_c\},$ wobei $e_1,e_2,e_3$ geeignete Kanten in $X_1-Y_1$ und $e_a,e_b,e_c$ geeignete Kanten in $Y$ sind. Da die letzteren Kanten zu derselben Fläche $F\in Y_2$ inzident sind, müssen diese Kanten paarweise verschieden gefärbt sein, also  muss ohne Einschränkung $
\omega(e_a)=a,\omega(e_b)=b$ und $\omega(e_c)=c$
gelten.
\begin{figure}[H]
\begin{center}
\includegraphics[viewport=0cm 24.5cm 5cm 27cm]{colouredTriangle}
\end{center}
\caption{Ausschnitt der wild-gefärbten Sphäre $Y$}
\end{figure}
Die Flächen des Tetraeders an der Stelle $V$ sind zu genau einer Kante aus der Menge $\{e_a,e_b,e_c\}$ und genau zwei Kanten aus der Menge $\{e_1,e_2,e_3\}$ inzident. Deshalb erhalten wir ohne Einschränkung 
\begin{align*}
&X_1(F_1)=\{e_1,e_2,e_a\},\\
&X_1(F_2)=\{e_2,e_3,e_b\}\\
& X_1(F_3)=\{e_1,e_3,e_a\}.\\
\end{align*}
\begin{figure}[H]
\begin{center}
\includegraphics[scale=0.9,viewport=0cm 19.5cm 10cm 26.cm]{notcol3gon}
\end{center}
\caption{Ausschnitt der Sphäre $X$}
\end{figure}
Die wilde Färbung $\omega$ auf $Y$ muss nun durch Ergänzen der fehlenden Bilder zu einer wilden Färbung $\omega_X$ auf $X$ erweitert werden. Für $e\in Y_1\subseteq X_1$ gilt also $\omega_X(e)=\omega(e).$  
\begin{figure}[H]
\begin{center}
\includegraphics[scale=0.9,viewport=0cm 19.5cm 10cm 27cm]{col3gon}
\end{center}
\caption{Ausschnitt der teilweise wild-gefärbten Sphäre $X$}
\end{figure}
Da $e_1 \in X_1(F_1)\cap X_1(F_3)$ ist und die Gleichheiten $\omega_X(e_c)=\omega(e_c)=c$ und $\omega_X(e_a)=\omega(e_a)=a$ gelten, muss $\omega_X(e_1)=b$ sein, da sonst keine wilde Färbung zustande kommt. Analog erhalten wir $\omega_X(e_2)=c$ und $\omega_X(e_3)=a$ und schlussendlich eine Färbung der Sphäre $X$.
\begin{figure}[H]
\begin{center}
\includegraphics[scale=0.9,viewport=0cm 19.5cm 10cm 27cm]{col3gon2}
\end{center}
\caption{Ausschnitt der wild-gefärbten Sphäre $X$}
\end{figure}
\end{proof}
Durch analoge Beweisführung kann ebenfalls gezeigt werden, dass für Multi-Tetraeder eine gleichschenklige Färbung existiert. 

\begin{satz}
Sei $X$ ein Multi-Tetraeder. Dann besitzt $X$ bis auf Permutation der Farben $a,b,c$ genau eine wilde Färbung.
\end{satz}
\begin{proof}
Angenommen obige Aussage gilt nicht. Dann gibt es einen bezüglich der Flächenanzahl minimalen Multi-Tetraeder $X$, sodass $\omega_1$ und $\omega_2$ zwei wilde Färbungen mit 
\begin{align} \label{eq}
\{\omega_1^{-1}(\{a\}),\omega_1^{-1}(\{b\}),\omega_1^{-1}(\{c\})\}\neq \{\omega_2^{-1}(\{a\}),\omega_2^{-1}(\{b\}),\omega_2^{-1}(\{c\}\}
\end{align}
sind. Wir können $\vert X_2\vert >4$ annehmen, da die wilde Färbung des Tetraeders eindeutig ist. Zudem sei $V$ eine Ecke vom Grad 3 in $X.$ Da jeweils zwei der drei Kanten in $X_1(V)$ inzident zu derselben Fläche sind, müssen die Kanten in $X_1(V)$ unter den Färbungen $\omega_1$ und $\omega_2$ paarweise verschiedene Farben erhalten. Deshalb können wir ohne Einschränkung 
\[
\omega_1(e)=\omega_2(e)
\] 
für alle $e\in X_1(V)$ annehmen. Beachte, dass hierdurch die Bedingung (\ref{eq}) nicht verletzt wird. Durch Entfernen des Tetraeders an der Stelle $V$ erhalten wir den Multi-Tetraeder $Y=T_V(X)$ und durch Einschränken der Färbungen $\omega_1$ und $\omega_2$ auf $Y_1-X_1(V)$ erhalten wir wilde Färbungen auf $Y$. Da $X$ minimal mit der Eigenschaft der nicht eindeutigen Färbung war, muss
\[
\omega_1(e)=\omega_2(e)
\]
für alle $e\in Y_1$ gelten, woraus direkt $\omega_1=\omega_2$ folgt. Somit erhalten wir den gewünschten Widerspruch.
\end{proof}
Jedoch sind gleichschenklige Färbungen von Multi-Tetraedern nicht eindeutig. Dieser Sachverhalt lässt sich am Beispiel des Doppel-Tetraeders erkennen.
\begin{center}
$\fbox{
\parbox{14cm}{
\textcolor{red}{gap$>$} \textcolor{blue}{ DT;}\\
simplicial surface (5 vertices, 9 edges, and 6 faces)
}}$
\end{center}
Die gleichschenklig gefärbten Doppel-Tetraeder erhalten wir durch Aufrufen des Befehls 
\textsf{AllIsoscelesColouredSurfaces}.
\begin{center}
$\fbox{
\parbox{14cm}{
\textcolor{red}{gap$>$} \textcolor{blue}{AllIsoscelesColouredSurfaces(DT);}\newline
[ isosceles coloured surface (5 vertices, 9 edges and 6 faces),\newline
  isosceles coloured surface (5 vertices, 9 edges and 6 faces) ]
}}$
\end{center}
Dabei zeigt die nächste Abbildung die erste Färbung des Doppel-Tetraeders 
\begin{figure}[H]
\begin{center}
\includegraphics[viewport=0cm 22cm 5cm 27cm]{dt1}
\end{center}
\caption{gleichschenklig gefärbter Doppel-Tetraeder}
\end{figure}
und folgendes Bild die zweite berechnete Färbung des Doppel-Tetraeders.
\begin{figure}[H]
\begin{center}
\includegraphics[viewport=0cm 22.cm 5cm 27cm]{dt2}
\end{center}
\caption{gleichschenklig gefärbter Doppel-Tetraeder}
\end{figure}
\begin{lemma}
Sei $X$ ein Multi-Tetraeder. Falls $X$ eine zahme Färbung besitzt, muss diese eine $rrr$-Struktur sein.
\end{lemma}
\begin{proof}
Sei $\omega$ eine zahme Färbung auf $X$ und $V$ eine Ecke vom Grad 3 in $X.$ Weiterhin seien $F_1,F_2,F_3\in X_2(V)$ und $e_1,e_2,e_3,e_a,e_b,e_c\in X_1$ paarweise verschieden, sodass 
\begin{align*}
&X_1(F_1)=\{e_1,e_2,e_a\},\\
&X_1(F_2)=\{e_2,e_3,e_b\}, \\
&X_1(F_3)=\{e_1,e_3,e_c\},
\end{align*}
gilt.
\begin{figure}[H]
\begin{center}
\includegraphics[scale=0.8,viewport=0cm 19.5cm 10cm 26.cm]{notcol3gon}
\end{center}
\caption{Ausschnitt der Sphäre $X$}
\end{figure}
Da jeweils zwei der drei Kanten $e_1,e_2,e_3$ inzident zu derselben Fläche sind, muss ohne Einschränkung $\omega(e_1)\neq\omega(e_2)\neq\omega(e_3)$ gelten. Durch setzen der Farben $\omega(e_1)=b,\omega(e_2)=c,\omega(e_3)=a$ folgt direkt $\omega(e_a)=a,\omega(e_b)=b,\omega(e_c)=c.$  Durch Betrachten der Typen der Kanten $e_1,e_2,e_3$ folgt die Behauptung.
\begin{figure}[H]
\begin{center}
\includegraphics[scale=0.8,viewport=0cm 19.5cm 10cm 27cm]{col3gon2}
\end{center}
\caption{Ausschnitt der wild-gefärbten Sphäre $X$}
\end{figure}
\end{proof}
Im Folgenden wollen wir untersuchten, welche Multi-Tetraeder eine zahme Färbung besitzen. Mit dem Tetraeder haben wir bereits ein Beispiel gesehen. Mit der Definition des \emph{Sterns}, erhalten wir ein weiteres Beispiel.
\begin{definition}
Der \emph{Stern} $S$ ist der Multi-Tetraeder, der durch das Symbol $1_11_21_31_4$ konstruiert wird.
\begin{figure}[H]
\begin{center}
\includegraphics[scale=0.01]{stern}
\end{center}
\caption{Stern}
\end{figure}
\end{definition}
\begin{definition}
Sei $X$ ein Multi-Tetraeder. Durch das iterative Entfernen aller Tetraeder entsteht im Sinne von Definition ?? eine Kette 
\[
X=X^{(0)}\to X^{(1)}\to \ldots \to X^{(t)}\cong X^{(t+1)}
\]
Wir nennen einen Multi-Tetraeder \emph{maximal}, falls $X^{(t)}\cong T$ ist und für alle $0\leq i< t$ die Gleichheit
\[
\vert \{V\in X_0^{(i)}\,\mid \, deg(V)=3\}\vert=\vert X^{(i+1)}_2\vert 
\]
gilt.
\end{definition}
Damit sind der Tetraeder und der oben definierte Stern maximal. 
Denn für den Stern ergibt sich die Kette 
\[
S\to S^{(1)}=T.
\]
und es gilt 
\[
\vert \{V\in S_0\,\mid \, deg(V)=3\}\vert=4=\vert T_2\vert .
\]
Der Multi-Tetraeder $X$, den wir durch das Symbol $1_11_21_3$ erhalten, ist jedoch nicht maximal. Es gilt zwar
\[
X\to X^{(1)}=T,
\] 
aber zweite Teil der Definition wird von der Sphäre $X$ nicht erfüllt, denn 
\[
\vert \{V\in X_0\,\mid \, deg(V)=3\}\vert=2\neq 4=\vert T_2\vert.
\]
\begin{lemma}\label{max}
Sei $X$ ein maximaler Multi-Tetraeder. Dann existiert eine zahme Färbung auf $X.$
\end{lemma}
\begin{proof}
Wir weisen die Aussage induktiv nach. Bei dem Tetraeder und dem Stern kann nachgerechnet werden, dass eine zahme Färbung existiert und das diese eine $rrr$-Struktur bildet. Sei nun $X$ ein maximaler Multi-Tetraeder, der nicht isomorph zum Stern oder Tetraeder ist.
 Dann ist $Y=X^{(1)}$ ebenfalls maximal und nach Induktionsvoraussetzung erhalten wir eine zahme Färbung $\omega$ auf $Y,$ die eine $rrr$-Struktur bildet. Seien nun $F_1$ und $F_2$ zwei beliebige benachbarte Flächen in $Y$, die $Y_1(F_1)=\{e_a,e_b,e_c\}$ und $Y_1=\{e_a,e_b',e_c'\}$ und $Y_0(e_b)\cap Y_0(e_c')\neq \emptyset$ für geeignete Kanten erfüllen.
Da $Y$ eine zahme Färbung besitzt und diese eine $rrr$-Struktur bildet, gilt ohne Einschränkung
\begin{align*}
&\omega(e_a)=a\\
&\omega(e_b)=b=\omega(e_b')\\
&\omega(e_c)=c=\omega(e_c')
\end{align*}
\begin{figure}[H]
\begin{center}
\includegraphics[scale=0.8,viewport=0cm 21.6cm 13cm 27cm]{tamcol2}
\end{center}
\caption{Ausschnitt der wild-gefärbten Sphäre $Y$}
\end{figure}
In $X$ werden die Flächen $F_1$ und $F_2$ durch Tetraeder ersetzt. Und die wilde Färbung auf $Y$ muss nun durch Ergänzen der fehlenden Farben der neuen Kanten zu einer wilden Färbung auf $X$ ergänzt werden. Seien nun also $f_1,f_2,f_3$ die Flächen des Tetraeders, der $F_1$ ersetzt und $f'_1,f'_2,f'_3$ die Flächen des Tetraeders der $F_2$ ersetzt, sodass für geeignete Kanten folgende Relationen erfüllt sind:
\begin{align*}
&X_1(f_1)=\{e_a,e_1,e_2\}\\
&X_1(f_2)=\{e_b,e_2,e_3\}\\
&X_1(f_3)=\{e_c,e_1,e_3\}\\
&X_1(f'_1)=\{e_a,e_1',e_2'\}\\
&X_1(f'_2)=\{e_b',e_2',e'_3\}\\
&X_1(f'_3)=\{e_c',e'_1,e'_3\}
\end{align*}
\begin{figure}[H]
\begin{center}
\includegraphics[scale=0.8,viewport=0cm 21.6cm 13cm 27cm]{tamcol3}
\end{center}
\caption{Ausschnitt der wild-gefärbten Sphäre $X$}
\end{figure}
Dann erhalten wir durch analoges Vorgehen wie im Beweis von \Cref{wild}, eine wilde Färbung $\omega_X$ auf $X$ als eine eindeutige Erweiterung der Färbung auf $Y,$ welche insbesondere die $rrr-$Struktur auf dem Multi-Tetraeder $X$ fortfuhrt. Für $e\in X_1$ ergibt sich diese durch 
\begin{align*}
\omega_X(e)=\Biggl\{\begin{tabular}[l]{lcr}
a, falls $e=e_3,e_3'$\\
b, falls $e=e_1,e_1'$\\
c, falls $e=e_2,e_2'$\\
$\omega(e)$, sonst
\end{tabular}.
\end{align*}

\begin{figure}[H]
\begin{center}
\includegraphics[scale=0.8,viewport=0cm 21.6cm 13cm 27cm]{tamcol}
\end{center}
\caption{Ausschnitt der wild-gefärbten Sphäre $X$}
\end{figure}
\end{proof}
Wenn man den Beweis genauer betrachtet, kann man folgende Folgerung formulieren.
\begin{folgerung}
Sei $X$ ein Multi-Tetraeder mit einer zahmen Färbung. Sei $Y$ der Mlti-Tetraeder, der dadurch entsteht, dass an allen Flächen von $Y$ Tetraedererweiterungen durchgeführt werden. Dann existiert eine zahme Färbung auf $Y.$
\end{folgerung}
Die Rückrichtung von \Cref{max} gilt jedoch nicht, dies halten wir mit folgendem Beispiel fest:\\
Die Sphäre, die wir betrachten ist der Multi-Tetraeder mit 20 Flächen, der durch das Symbol $1_21_12_42_32_13_43_33_2$ beschrieben wird.
\begin{center}
$\fbox{
\parbox{14cm}{
\textcolor{red}{gap$>$} \textcolor{blue}{s;}\newline
simplicial surface (12 vertices, 30 edges, and 20 faces)\newline
\textcolor{red}{gap$>$} \textcolor{blue}{GetSymbol(s);}\newline
[ [ 1, 2 ], [ 1, 1 ], [ 2, 4 ], [ 2, 3 ], [ 2, 1 ], [ 3, 4 ], [ 3, 3 ],
  [ 3, 2 ] ]\newline
\textcolor{red}{gap$>$}\textcolor{blue}{ Length(AllTameColouredSurfaces(s));}\newline
1
}}$
\end{center}
Dieser besitzt also eine zahme Färbung. Jedoch ist $s$ nicht maximal wie durch folgende Rechnung gezeigt werden kann.
\begin{center}
$\fbox{
\parbox{14cm}{
\textcolor{red}{gap$>$} \textcolor{blue}{RemoveAllTetra(s);}\newline
simplicial surface (6 vertices, 12 edges, and 8 faces)\newline
\textcolor{red}{gap$>$}\textcolor{blue}{ Length(Filtered(Vertices(s),g->FaceDegreeOfVertex(s,g)=3));}\newline
6
}}$
\end{center}
\end{document}